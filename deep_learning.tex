%\title{深度学习中文翻译版}
\documentclass[a4paper,11pt]{book}

\usepackage{xeCJK}
%\setCJKmainfont[BoldFont=STSong, ItalicFont=STKaiti]{STSong}
%\setCJKsansfont[BoldFont=STHeiti]{STXihei}
%\setCJKmonofont{STFangsong}
\usepackage{enumerate}
\usepackage{caption}
\usepackage{bm}
\setlength{\parskip}{1em}

\usepackage[T1]{fontenc}
\usepackage[utf8]{inputenc}
\usepackage{lmodern}
%%%%%%%%%%%%%%%%%%%%%%%%%%%%%%%%%%%%%%%%%%%%%%%%%%%%%%%%%
% Source: http://en.wikibooks.org/wiki/LaTeX/Hyperlinks %
%%%%%%%%%%%%%%%%%%%%%%%%%%%%%%%%%%%%%%%%%%%%%%%%%%%%%%%%%
\usepackage{hyperref}
\usepackage{graphicx}
\usepackage[english]{babel}

% *** Editing Commands ***
\usepackage{xcolor}
\usepackage[normalem]{ulem} % use normalem to protect \emph
\newcommand\add{\bgroup\markoverwith
  {\textcolor{green}{\rule[-.5ex]{.1pt}{2.5ex}}}\ULon}
\newcommand\remove{\bgroup\markoverwith
  {\textcolor{red}{\rule[-.5ex]{.1pt}{2.5ex}}}\ULon}
\newcommand{\consider}{\bgroup\markoverwith
  {\textcolor{yellow}{\rule[-.5ex]{.1pt}{2.5ex}}}\ULon}
  
% *** URL Support ***
\usepackage{url}

% *** Math Symbols Support ***
\usepackage{amsfonts}
\usepackage{amssymb}

% *** Equation Number ***
\usepackage{amsmath}
\numberwithin{equation}{chapter}

% *** Algorithm Support ***
\usepackage[ruled,lined,algochapter]{algorithm2e}

\newenvironment{dedication}
{
   \cleardoublepage
   \thispagestyle{empty}
   \vspace*{\stretch{1}}
   \hfill\begin{minipage}[t]{0.66\textwidth}
   \raggedright
}
{
   \end{minipage}
   \vspace*{\stretch{3}}
   \clearpage
}

%%%%%%%%%%%%%%%%%%%%%%%%%%%%%%%%%%%%%%%%%%%%%%%%
% Chapter quote at the start of chapter        %
% Source: http://tex.stackexchange.com/a/53380 %
%%%%%%%%%%%%%%%%%%%%%%%%%%%%%%%%%%%%%%%%%%%%%%%%
\makeatletter
\renewcommand{\@chapapp}{}% Not necessary...
\newenvironment{chapquote}[2][2em]
  {\setlength{\@tempdima}{#1}%
   \def\chapquote@author{#2}%
   \parshape 1 \@tempdima \dimexpr\textwidth-2\@tempdima\relax%
   \itshape}
  {\par\normalfont\hfill--\ \chapquote@author\hspace*{\@tempdima}\par\bigskip}
\makeatother

%%%%%%%%%%%%%%%%%%%%%%%%%%%%%%%%%%%%%%%%%%%%%%%%%%%
% First page of book which contains 'stuff' like: %
%  - Book title, subtitle                         %
%  - Book author name                             %
%%%%%%%%%%%%%%%%%%%%%%%%%%%%%%%%%%%%%%%%%%%%%%%%%%%



\title{\Huge \textbf{深度学习} }
% Author
\author{\textsc{Ian Goodfellow} \\ \textsc{Yoshua Bengio} \\ \textsc{Aaron Courville}}
\begin{document}

\frontmatter
\maketitle


\tableofcontents
\listoffigures
\listoftables

\mainmatter

%%%%%%%%%%%
% Preface %
%%%%%%%%%%%

\input{chap1.tex}


\part{应用数学与机器学习基础}
\label{part:1}

本部分介绍一些用于理解深度学习的基础数学概念。我们从定义函数和一些变量的应用数学开始,然后找到这些函数的最高点和最低点并量化置信度。


接着,我们会介绍机器学习的基本目标,并介绍如何使用特定的模型来模拟表示并完成这些目标。比如设计一个损失函数来描述模拟值和真值的差距,并使用训练算法来最小化损失函数。


这个基本的框架是许多机器学习算法的基础,包括一些不那么深的机器学习模型都有用到。在本书后续部分,我们会使用这个框架来建立深度学习算法。

\chapter{线性代数}
\label{chap:2}
%%%%%%%%%%%%%%%%%%%%%%%%%%%%%%%%%%%%%%%%%%%%%%%%%%%%%%%%%
%%%%%%%%%%% author:pascal_meng@outlook.com %%%%%%%%%%%%%%
%%%%%%%%%%%%%%%%%%%%%%%%%%%%%%%%%%%%%%%%%%%%%%%%%%%%%%%%%

\section{标量、向量、矩阵和张量}
\label{sec:2.1}

\section{矩阵和向量乘法}
\label{sec:2.2}

\section{单位矩阵和逆矩阵}
\label{sec:2.3}

\section{线性相关和和线性空间}
\label{sec:2.4}

\section{秩}
\label{sec:2.5}

\section{特殊的矩阵和向量}
\label{sec:2.6}

\section{特征分解}
\label{sec:2.7}

\section{奇异值分解}
\label{sec:2.8}

\section{摩尔-彭若斯广义逆}
\label{sec:2.9}

\section{求迹}
\label{sec:2.10}

\section{行列式}
\label{sec:2.11}

\section{示例:主成分分析}
\label{sec:2.12}
%%%%%%%%%%%%%%%%%%%%%%%%%%%%%%%%%%%%%%%%%%%%%%%%%%%%%%%%%
%%%%%%%%%%%%%%% author:wulemilly、msnh %%%%%%%%%%%%%%%%%%
%%%%%%%%%%%%%%%%%%%%%%%%%%%%%%%%%%%%%%%%%%%%%%%%%%%%%%%%%


\documentclass{article}
\usepackage{CJK}
\usepackage{amsmath}
\usepackage{graphicx}
\usepackage{amssymb}
\begin{CJK*}{GBK}{song}
  \title{第三章    概率论与信息论}
\begin{document}
  \maketitle
  \paragraph{}
  在这一章节,我们介绍概率论与信息论。
  \paragraph{}
  概率论是一个表示不确定性现象的数学框架。它将不确定性问题进行量化并为得到新的不确定性结论做出了公式推导。在人工智能应用领域,我们运用了概率论的两种主要方法。首先,概率法则告诉我们如何解决AI (人工智能)领域的问题,因此我们利用概率论设计相关算法来计算或运用多种表达式来近似解决该问题。其次,我们可以利用概率论和信息论从理论上来分析人工智能系统的行为。
 \paragraph{}
  概率论是科学与工程多学科交叉的一种基本工具。我们提供本章主要为致力于软件工程领域的读者提供相关的概率论知识,以确保读者可以理解本书的内容。
  \paragraph{}
  概率论为我们提供不确定性的思想并解释不确定性的原因,信息论使我们能够量化一个不确定性的概率分布。
  \paragraph{}
  如果您已经熟悉概率论与信息论,您可能希望跳过这一章,但是3.14节是我们需要关注的重点,它描述了我们用来描述机器学习的结构化概率模型的图表。如果您对这些科目没有深入的研究,本章将为您高效顺利开展深度学习项目的研究提供帮助,但是除此之外我们还是推荐您去阅读其他的相关内容	,比如Jaynes(2003)。

   %--section 1--%

  \section*{3.1 什么是概率论?}
    \paragraph{}
    计算机科学的许多分支主要是处理那些完全是确定性的实体。程序员通常可以安全地假定CPU可以完美无瑕的执行每条机器指令。在硬件中的错误确实发生,但是它还不足以在设计大多数软件应用程序的时候考虑它的影响。由于许多计算机科学家和软件工程师一般是处理比较确定性的问题,但是机器学习中大量使用概率论知识可能会令他们感到吃惊。
    \paragraph{}
    这是因为机器学习必须处理不确定的数据,有时也可能需要处理的随机数(不确定的),不确定性和随机性的产生来源于很多方面。从20 世纪80年代以来,研究人员已经为使用概率论来量化不确定性因素提供了非常可信的论据,这里提出的许多总结性的结论或启发参看Pearl (1988)。
    \paragraph{}
    许多事情都需要去推理它们存在的不确定性。事实上,除了数学可以通过定义来确定其真实性之外,很难保证任何一个绝对真实的命题,或任何一个绝对的事件都会发生。
    \paragraph{}以下是3种不确定性因素的来源:
    \subparagraph{}
    1.	在系统形成之时就存在固有的随机性。例如,在量子力学领域,亚原子的运动学都是用概率来描述的。我们也可以想象一个场景,我们假定一个动态随机的一个场景。比如一种卡牌游戏,所有的卡牌都是按照随机顺序排列的。
     \subparagraph{}
     2.	不完全的可观测性。当我们无法观测到所有的变量是如何驱动这个系统的时候,即便是个确定的系统也存在不确定性。例如一个蒙提霍尔问题。 你参加电视台的一个抽奖节目。台上有三个门,一个后边有汽车,其余后边是山羊。主持人让你任意选择其一。然后他打开其余两个门中的一个,你看到是山羊。这时,他给你机会让你可以重选,也就是你可以换选另一个剩下的门。那么,你换不换?这样的问题选手选择的结果是确定的,但是站在选手的角度来说,选择又是随机的。
     \subparagraph{}
     3.	模型的不完整性。当我们使用一个模型,这个模型里面有些我们观测到的一些信息必须舍弃的时候,那么,舍弃这些信息就会对这个模型的预测产生不确定性的影响。例如,假设我们造了一个机器人,它可以精确地观察到它周围的每一个物体的位置。当这个机器人对物体位置进行预测时,把空间离散化了,那么离散化就会给机器人带来许多不确定性因素。每个物体都可能出现在在它被观察到的离散单元内的任何位置。
    \paragraph{}
    在许多情况下,使用一个比较简单,但是不确定的规则比使用一个复杂但是确定的规则更实际。即便真是的规则是确定的,但是我们可以让我们的模型非常接近真是但是非常复杂的规则。比如一个非常简单的一个规则,我开一枪,所有的鸟都会飞这样一个模型,这个模型很好建立,也很好用。但是往深在想一下,如果那一堆鸟中,有还没有学会飞的鸟,有生病了飞不了的鸟,有受伤了飞不了的鸟等等等等…要把所有这些问题考虑进去,去建立这样一个确定的模型,这是不切实际的。
    \paragraph{}
    所以,我们需要一个表示和推理的不确定性的手段,概率论可以为我们提供一些我们想用在人工智能上的一些工具,但是可能它给我们呈现出的结果并不是立马就能观测到的。概率论的出现本来是用于分析事件的频率。很容易可以概率论是如何使用的,比如去描述在一个扑克游戏中的某一手牌。这些事件通常是可重复的。假设某个事件发生的概率为p,当我们重复这个事件的时候,他的结果发生的概率始终是p。这种推理似乎并不适用于不可重复的命题。如果一个医生分析一个病人,并说,病人有40\%的机会患流感,我们不可能说是拥有无限个这样的病人也没有任何理由相信同样的病人会得同样的病。但有不同的基本条件。在医生诊断病人的时候,我们使用概率来表示一件事情的可信度,用1 表示这个病人一定会得流感,用0表示这个病人一定不会得流感。前一种概率,直接关系到事件发生的概率,被称为古典概型。而后者,与确定性的定性水平有关,被称为贝叶斯概率。
    \paragraph{}
    如果我们希望常识性的东西具有不确定性,那么唯一的方法就是,使用贝叶斯概率来当成我们所熟知的古典概型。例如,当我们知道一个玩家手上的牌之后,去计算他能赢的概率的时候,我们会使用和计算某个具有确定症状的病人去计算他的犯病概率的一样的公式。关于为什么一小部分常识性的假设意味着两种概率类型必须受到同一公理约束的更多细节性问题可以参看(Ramsey(1926))。
    \paragraph{}
    概率可以看作是一种处理不确定性逻辑的扩展,逻辑提供了一套正式的规则用于确定什么样的命题是隐含的真或假的假设,概率论提供了一套正式的规则,用于确定一个命题为真的可能性及命题的其他可能性。

    %--section 2--%

    \section*{3.2 随机变量}
    \paragraph{}
    一个随机变量是一个可以随机抽取不同的值的变量。我们通常以大写字母如$X,Y,Z,W…$表示随机变量,而以小写字母如$x,y,z,w…$表示实数。例如,$x_1$ 和$x_2$都是随机变量X 的取值。对于随机向量,我们用X表示随机变量和用x表示该随机变量的一个值。对其本身而言,一个随机变量只是描述一种状态的可能性,它必须加上一个概率分布,指定每个状态的可能性大小。
    \paragraph{}
    随机变量可以是离散的或连续的。离散型随机变量是一个具有有限个或可列无限多个状态。需要注意的是,这些状态不一定是整数,它们也可以只是被命名为不考虑有任何数值的状态。一个连续的随机变量与一个真实值相关。

    %--section 3--%

   \section*{3.3 概率分布}
    \paragraph{}
    概率分布是描述一个随机变量的可能性或一个随机变量集合中每一个变量可能的状态。我们描述的概率分布的方式取决于变量是离散的或连续的。
    \subsection*{3.3.1 离散变量及概率质量函数}
    \paragraph{}
    离散变量的概率分布可以用概率质量函数描述(PMF)。我们通常表示概率质量函数用大写字母P。 通常我们将每个随机变量对应不同的概率质量函数,读者必须推断出的哪一个概率质量函数对应该随机变量的结果,而不是通过函数的名字判断;$P(\mathrm{x})$ 和$P(\mathrm{y})$通常是不相同的。
    \paragraph{}
    概率质量函数映射是指随机变量的一个状态到随机变量在该状态下的概率。概率$\mathrm{x} = x$被记为$P(x)$,当概率为1时表明$\mathrm{x} = x$是必然的,当概率为0 时表明$\mathrm{x}=x$是不可能的,有时为了更好的理解概率质量函数,我们可以明确的表示随机变量的名称:$P(\mathrm{x} = x)$。 有时,我们先定义一个变量,然后使用$\sim$符号指定其分布如下:$\mathrm{x}\sim{P(x)}$。
    \paragraph{}
    概率质量函数可以同时作用于多个变量。这种具有多个变量的概率分布被称为联合概率分布。$P(\mathrm{x} = x, \mathrm{y} = y)$同时表示概率$\mathrm{x} = x$ 和$\mathrm{y} = y$。我们还可以简写为$P(x, y)$。
    \paragraph{}
    为了计算一个随机变量x的概率质量函数,函数P必须满足下列性质:
    \subparagraph{}
     1.$P$的取值必须是x的所有可能状态的集合。
    \subparagraph{}
     $2.\forall\mathrm{x} \epsilon x,0 \leq P(x) \leq 1.$ 一个不可能事件的概率为0,没有任何一种状态比0还小。同样,确定发生的事件的概率为1,没有任何一种状态比1还大。
    \subparagraph{}
     $3.\sum _{\mathrm{x} \epsilon x} P(x) = 1.$我们将此属性称为归一化。如果没有这个属性,通过计算许多事件发生的概率我们可能获得一个大于1 概率。
    \paragraph{}
    例如,考虑一个单一的离散随机变量x 有k种不同的状态。我们可以通过计算它的概率密度函数得到x 是均匀分布的(也就是说,每一种状态的可能性相等)。

    \begin{equation}
    P(\mathrm{x}=x_i )=\frac{1}{k} \tag{3.1}
    \end{equation}
    \paragraph{}

    对于所有的i,我们可以看出这符合概率质量函数的要求。因为k 是一个正整数,所以值$\frac{1}{k} $是正的。我们也看出
    \begin{equation}
    \sum_i P(\mathrm{x}=x_i )=\sum_i\frac{1}{k}=\frac{k}{k}=1 \tag{3.2}
    \end{equation}
    \paragraph{}
    因此,概率分布是适当的归一化。
    \subsection*{3.3.2 连续变量及概率密度函数}
    \paragraph{}
    当计算连续型随机变量的概率分布时,我们使用概率密度函数(PDF)而不是概率质量函数。要成为一个概率密度函数,函数P必须满足下列性质:
    \subparagraph{}
     1.$P$的取值必须是x的所有可能状态的集合。
    \subparagraph{}
     $2.\forall \mathrm{x} \epsilon x,P(x)\geq0.$ 需要注意的是,我们不要求$p(x) \leq 1$。
    \subparagraph{}
     $3.\int p(x)dx = 1$ 。
    \paragraph{}
    一个概率密度函数不给它特定的概率,而是落在一个无穷小的区域的概率,可以近似的表示为$p(x)\delta x$。
    \paragraph{}
    我们通过整合密度函数来找到点集的实际概率。具体而言,x 位于某一集合S的概率是由$P(x)$在该集合上的积分得到的。举一个单变量的例子,x在区间[a,b]的概率是通过$\int_{[a,b]}p(x)dx$得到的。
    \paragraph{}
    举一个例子,概率密度函数相当于一个连续随机变量考虑实数区间上的均匀分布的一个特定的概率密度。我们可以用一个函数$u(x;a,b)$表示,其中a和b是区间的两个端点,且$b> a$。符号“;”表示参数化;我们认为x是函数的参数,而a和b 是定义函数的参数。为了确保概率密度都在区间范围内,我们可以说对于所有的$x\not\in[a,b]$,其$u(x;a,b)=0$。在[a,b]区间内的$u(x;a,b)=\frac{1}{(b-a)}$。我们可以看到结果是非负的,其值最大为1。我们经常将$x$在区间[a,b]的概率分布表示为$x \sim U(a, b)$。

    %--section 4--%

    \section*{3.4 边缘概率}
    \paragraph{}
    有时我们知道一组变量的边缘概率,同时也想得到子变量的概率分布。我们把子变量的概率分布叫做边缘概率分布。
    \paragraph{}
    例如,假设我们有离散随机变量x和y,我们知道P(x,y),则可以计算P(x)为:
    \begin{equation}
     \forall x \epsilon X,P(x=\textbf{X})=\Sigma_yP(x=X,y=Y)  \tag{3.3}
    \end{equation}
    “边缘概率”的名字来源于边缘概率的计算过程。当P(x,y)的值以x为行,y为列写在表格中,将行的数值求即为P(x)的值。
    \paragraph{}
    对于连续型变量,我们需要积分而不是求和:
    \begin{equation}
      p(x)=\int p(x,y)dy  \tag{3.4}
    \end{equation}

    %--section 5--%

     \section*{3.5 条件概率}
    \paragraph{}
    很多情况下,我们更感兴趣的是考虑事件A已经发生的条件下事件B的概率,这就叫条件概率。我们可以将$\mathrm{x}=x$在$\mathrm{y}=y$ 的条件下的概率表示为$P(\mathrm{y} = y | \mathrm{x} = x)$。该条件概率可以用公式计算:
    \begin{equation}
      P(\mathrm{y} = y | \mathrm{x} = x)=\frac{P(\mathrm{y} = y | \mathrm{x} = x)}{P( \mathrm{x} = x)} \tag{3.5}
    \end{equation}
    \paragraph{}
    条件概率只有当$P(\mathrm{x} = x)> 0$时才能定义。我们不能计算一个从未发生过的事件的条件概率。
    \paragraph{}
    重要的是不要将条件概率与通过采取一定措施计算将会发生事情的概率混淆,条件概率可以形象的解释为一个德国人的德语说的相当好,但如果一个随机选择的人教他学德语,但他们的原国籍不会改变。计算一个动作的因果被称为做干预查询。干预查询是因果模型的领域,我们不在这本书中探索。


    %--section 6--%

 \section*{3.6 条件概率链式法则}
    \paragraph{}
    在许多随机变量的联合概率分布可以分解为一个变量的条件分布:
    \begin{equation}
      P(x^{(1)},\cdots,x^{(n)})=P(x^{(1)})\prod^{n}_{i=2} P(x^{(i)}|x^{(1)},\cdots,x^{(i-1)}) \tag{3.6}
    \end{equation}
    \paragraph{}
    这就叫做链式法则或乘积法则,可由公式(3.5)中的条件概率的定义来推导。例如,将其定义两次我们可以得到:
    $$ P (a, b, c) = P (a | b, c)P (b, c)$$
    $$ P (b, c) = P (b | c)P (c)$$
    $$ P (a, b, c) = P (a | b, c)P (b | c)P (c)$$

    %--section 7--%

    \section*{3.7 独立性和条件独立性}
    \paragraph{}
    两个随机变量x和y是独立的,他们的概率分布可以表示为一个事件的两个因素,一个只与x相关,另一个只与y相关:
    \begin{equation}
      \forall \mathrm{x}\epsilon x,\mathrm{y}\epsilon y,p( \mathrm{x} = x,\mathrm{y} = y )=p( \mathrm{x} = x)p(\mathrm{y} = y )\tag{3.7}
    \end{equation}
    \paragraph{}
    两个随机变量x和y是相互独立的,若有一个随机变量z,在变量z的条件下x和y的条件概率分布为:
    \begin{equation}
    \forall \mathrm{x}\epsilon x,\mathrm{y}\epsilon y,\mathrm{z}\epsilon z,p( \mathrm{x}=x,\mathrm{y}=y|\mathrm{z}=z )=p( \mathrm{x}=x|\mathrm{z}=z)p(\mathrm{y}=y|\mathrm{z}=z ) \tag{3.8}
    \end{equation}
    \paragraph{}
    我们可以用更简便的方式表达独立性与条件独立性:$x\bot y $意味着x和y是相互独立的,而$x\bot y | z$则表示x和y 在z的条件下是相互独立的。

    %--section 8--%

     \section*{3.8 期望,方差和协方差}
    \paragraph{}
    函数$f(x)$的期望或期望值是x与其相对应的概率分布$P(x)$的乘积的平均值。对于离散型变量可以做一个求和的计算:
    \begin{equation}
     E_{x\sim P}[f(x)]=\sum _x P(x)f(x)  \tag{3.9}
    \end{equation}
    \paragraph{}
    对于连续型变量,它是用一个积分计算的:
    \begin{equation}
       E_{x\sim P}[f(x)]=\int P(x)f(x)dx   \tag{3.10}
    \end{equation}
    \paragraph{}
    当需要计算某一个变量时,我们可以在计算期望用其变量名作为下标,例如$E_x[f(x)]$。在所求的随机变量确定的情况下,也可以不用变量名作为下标,可将其省略,如$E[f(x)]$。 默认情况下,可以用$E[?]$表示所有括号内的随机变量的平均值。同样地,当没有歧义时,我们可以省略方括号。
    \paragraph{}
    期望是线性的,例如:
    \begin{equation}
      E_x [\alpha f(x)+\beta g(x)]=\alpha E_x[f(x)]+\beta E_x[g(x)]  \tag{3.11}
    \end{equation}
    \paragraph{}
    $\alpha$和$\beta$不依赖与$x$。
    \paragraph{}
    方差就是随机变量x的函数的数学期望,表达式如下:
    \begin{equation}
      Var(f(x))=E[(f(x)-E[f(x)])^2] \tag{3.12}
    \end{equation}
    \paragraph{}
    当方差很小时,$F(x)$的值接近他们的期望值。方差的平方根被称为标准偏差。
    \paragraph{}
    协方差表示两个变量之间的线性关系,以及变量之间的线性程度:
    \begin{equation}
      Cov(f(x),g(y))=E[(f(x)-E[f(x)])(g(y)-E[g(y)])] \tag{3.13}
    \end{equation}
    \paragraph{}
    协方差的绝对值较大意味着两个变量总体误差的期望越大。如果协方差的值是正值,说明其中一个变量大于自身的期望值时另外一个变量也大于自身的期望值;如果协方差的值是负值,说明其中一个变量大于自身的期望值的同时另外一个却小于自身的期望值,反之亦然。其他测量方法如相关系数,用来表示两个变量之间线性关系的紧密程度,而不单单是一个独立变量的受影响程度。
    \paragraph{}
    协方差和相关是有联系的,但实际上是不同的概念。两个变量的协方差为0表示不相关,两个变量的协方差不为0则表示相关。然而,协方差的独立性具有不同的属性,对于两个变量,如果协方差为0,则表示它们之间一定没有线性关系。独立性比协方差的要求更严格,因为独立性也不包括非线性关系,也就是说两个变量之间有关系,但是它的协方差为0。例如,假设我们的第一个样本的一个实数$x$在区间[-1,1]之间均匀分布,下一个样本是一个随机变量s。当概率为1/2,$s$的值为1。否则,$s$的值为-1。我们生成一个随机变量$y$且$y=sx$。 显然,$x$和$y$是不独立的,因为x完全决定了y的大小,但是$ Cov(x, y) = 0$。
    \paragraph{}
    一个随机向量$x\in R_{n}$的协方差矩阵是一个n×n矩阵,这样:
    \begin{equation}
      Cov(x)_{i,j}= Cov(x_i,x_j) \tag{3.14}
    \end{equation}
    \paragraph{}
    协方差的对角线元素给出方差:
    \begin{equation}
       Cov(x_i,x_j)=Var(x_i) \tag{3.15}
    \end{equation}

    %--section 9--%

    \section*{3.9 常用概率分布}
    \paragraph{}
    这里介绍几种在机器学习领域比较常用的概率分布。
    \subsection*{3.9.1 伯努利分布}
    \paragraph{}
    伯努利分布是一种是离散型的随机变量,变量只能取两个值,非0即1。它用一个参数 ,来衡量这个随机变量等于1的可能性。它有以下几项特征:
    \begin{equation}
     P(X=1)=\phi   \tag{3.16}
    \end{equation}
    \begin{equation}
     P(X=0)=1-\phi   \tag{3.17}
    \end{equation}
    概率公式
    \begin{equation}
     P(X=x)=\phi^{x} (1-\phi)^{1-x}  \tag{3.18}
    \end{equation}
    数学期望
    \begin{equation}
     E_{x}(X)=\phi  \tag{3.19}
    \end{equation}
    方差
    \begin{equation}
     Var_{x}(X)=\phi(1-\phi)  \tag{3.20}
    \end{equation}
    \subsection*{3.9.2 多项分布}
    \paragraph{}
    多项分布是一种是离散型的随机变量,变量取$k,k$为有限个数。它通过一个向量$ p_{i}\in [0,1]^{k-1} $ 来描述, 说明$ p_{i} $ 了第i次此事件发生的可能性。最终为$k$。多项分布是通常用于事件发生的情况不止一种的状况下。因此,我们通常不认为状态1和数值1等价,等等。所以,我们通常不需要计算多项分布的期望值和方差。
    \paragraph{}
    伯努利分布和多项分布足以描述他们的领域的任何分布。因为离散模型使用枚举方法列出所有的状态,是可行的。然而当处理连续型变量时,就会出现不可数的情况出现。所以少量参数的分布必须严格限制在这两种分布内。
    \subsection*{3.9.3 正太(高斯)分布}
    \paragraph{}
    最常用的用于描述实数的分布规律的分布为正态分布,也被称为高斯分布:
    \begin{equation}
      N(x,\mu,\sigma^{2})=\sqrt{\frac{1}{2\pi\sigma^{2}}}\mathrm{exp}(-\frac{1}{2\pi\sigma^{2}}(x-\mu)^{2})  \tag{3.21}
    \end{equation}
   \begin{figure}[!htb]
    \centering
   \centerline{\includegraphics[width=3.5in]{fig/chap3/3_1.jpg}}
   \label*{图:3.1}
   \end{figure}
   \paragraph{}
    图3.1为正太分布密度函数的一个描述。正态分布是典型的钟形曲线,$\mu$为x轴的中间峰值。宽度由$\sigma$控制。此图中的例子,是一个$\mu$ 为0,$\sigma$为1的正太分布。
    \paragraph{}
    正太分布用两个参数$\mu\in \mathrm{R}$和$ \sigma\in (0,\infty)$ 来描述。参数μ 为中心峰值坐标。同样它的方差$E(\mathrm{x})= \mu$,用$\sigma$表示标准差, $\sigma^{2}$表示方差。
    \paragraph{}
    当我们计算概率密度函数的时候,我们需要对$\sigma$进行平方和取倒数。当我们需要频繁的使用不同的参数去计算概率密度函数的时候,我们可以引入一个参数$\beta\in (0,\infty)$ 控制正太分布的精度和逆方差来更有效的参数化正太分布。
   \begin{equation}
      N(x,\mu,\beta^{-1})=\sqrt{\frac{\beta}{2\pi}}\mathrm{exp}(-\frac{\beta}{2\pi}(x-\mu)^{2})  \tag{3.22}
    \end{equation}
    \paragraph{}
    许多情况下使用正太分布是一种分成明智的选择。在没有先验知识的情况下,采用正太分布是非常好的选择有以下两个原因:
    \subparagraph{}
    第一,我们希望模型的许多分布能真正接近正常分布。中心极限定理表明,许多独立的随机变量的总和是近似正态分布的。这就意味着在实际实践中,许多复杂的系统可以成功地建模为正态分布,即使该系统可以被更结构化的去分解。
    \subparagraph{}
    第二,可能的大部分概率分布具有相同的方差,而正态分布对实数的最大不确定度进行编码。因此,我们可以认为正态分布是一个最少将先验知识插入到模型中的一种分布。充分开发和证明这个想法需要更多的数学工具。此部分见章节19.4.2。
    \paragraph{}
    正太分布推广到$\mathrm{R}^{n}$,在这种情况下,它被称为多元正态分布。可以被一个正定对称矩阵$\sum$进行参数化。
    \begin{equation}
      N(x,\mu,\sum)=\sqrt{\frac{1}{2\pi^{n}\mathrm{det}(\sum)}}\mathrm{exp}(-\frac{1}{2}(x-\mu)^{T}\sum^{-1}(x-\mu))  \tag{3.23}
    \end{equation}
    \paragraph{}
    此处的$\mu$的意思和前面的是一样的,只不过现在是个向量。参数$\sum$给出了次分布的协方差矩阵。对于单因素情况下,当我们希望去计算不同参数值的概率密度函数的时候,协方差是不是一个有效的方式计算的参数化的分布。因此我们需要对$\sum$进行转置去评估概率密度函数。我们可以用一个精度矩阵$\beta$来替换。
    \begin{equation}
      N(x,\mu,\beta^{-1})=\sqrt{\frac{\mathrm{det}(\beta)}{2\pi}}\mathrm{exp}(-\frac{\beta}{2\pi}(x-\mu)^{T}(x-\mu))  \tag{3.24}
    \end{equation}
    \paragraph{}
    我们经常将协方差矩阵修正伪对角矩阵。更简单的一个例子,各向同性的高斯分布,其协方差矩阵是系数位常数的单位矩阵。
    \subsection*{3.9.4 指数分布和拉普拉斯分布}
    \paragraph{}
    在深度学习中,我们经常希望在x = 0的有一个尖锐点的概率分布。要做到这一点,我们可以使用指数分布:
    \begin{equation}
      p(x;\lambda)=\lambda1_{x\geq0}\mathrm{exp}(-\lambda x)  \tag{3.25}
    \end{equation}
    \paragraph{}
    指数分布采用函数$1_{x\geq0}$ 表示分配概率为零的X的所有负值.而拉普拉斯分布则是可以让概率在点$\mu$处取得峰值。
    \begin{equation}
     \mathrm{Laplace}(x;\mu;\gamma)=\frac{1}{2\gamma} \mathrm{exp}(-\frac{|x-\mu|}{\gamma}) \tag{3.26}
    \end{equation}
    \subsection*{3.9.5 狄拉克分布和经验分布}
    \paragraph{}
    在某些情况下,我们希望指定的概率分布在围绕某个点的集群中。那么呢,我们可以通过定义一个狄拉克$\delta$函数实现。
    \begin{equation}
      p(x)=\delta(x-\mu)  \tag{3.27}
    \end{equation}
    \paragraph{}
    狄拉克$\delta$函数定义为在除了零以外的点函数值都等于零,而其在整个定义域上的积分等于1。狄拉克$\delta$函数不像一个普通的函数,一个输入得到一个真正的输出。相反狄拉克$\delta$函数它是一种不同类型的数学对象,称为广义函数,他是根据属性的集成定义的。我们可以认为狄拉克$\delta$函数的目的就是把除$\mu$点以外的值都无限接近于0。
    \paragraph{}
    通过定义$P(x)$是$\delta$函数平移$-\mu$所得到的无限窄的一个函数,在$x=\mu$ 时值取无限大。狄拉克分布是经验分布的一部分,把概率的$1/m$分布在每个$m$点$x^{(1)}...x^{(m)}$上形成一个给定的数据集或样本集合。
    \begin{equation}
     \hat{p}(x)=\frac{1}{m}\sum_{i=1}^{m}\delta(x-x^{(i)}) \tag{3.28}
    \end{equation}
    \paragraph{}
    在经验分布的连续型变量中使用狄拉克分布是有必要的,但是在离散型变量中,情况就简单许多。在离散型变量中,可以慨括为多项分布,与每个可能的输入值相关联的概率,简单地等于在训练集的该值的经验频率。我们可以把从训练样本数据集形成的经验分布看做成是对我们指定数据模型的训练。另外一种可以最大限度的提高训练数据的概率的可能性的经验分布参见5.5小节。
    \subsection*{3.9.6 混合分布}
    \paragraph{}
    结合其他简单分布去定义一种新的分布是挺常见的。组合分布的一个常见的方法是构造一个混合分布。混合分布是由几项分布组成的。在每个试验中,选择哪种成分的分布产生的样本,是从多项分布中抽样单位分量所决定的。
    \begin{equation}
      p(\mathrm{x})=\sum_{i}P(c=i)P(\mathrm{x}|c=i)  \tag{3.29}
    \end{equation}
    其中$P(c)$是对成员的多项分布。
    \paragraph{}
    在前面我们已经接触到了混合分布的一个例子。经验分布的实数部分就是使用狄拉克的一个混合分布。
    \paragraph{}
    混合模型是一种用来创建一个更复杂分布的简单策略之一。在16 章我们会更加详细的去介绍简单分布到混合分布的创建。混合模型使我们能分析出一个非常重要的概念,潜在变量。潜在变量是用在我们无法观测的一种随机变量。从混合分布的单位变量$c$就可以看出。潜在变量可以通过联合分布与$x$相关。在这种情况下,$P(\mathrm{x},c)=P(\mathrm{x}|c)P(c)$.其中$P(c)$是基于潜在变量的,而$P(\mathrm{x}|c)$ 可以根据可见变量和潜在变量去决定$P(\mathrm{x})$的形状,虽然没有参考隐藏变量,但是也能描述$P(\mathrm{x})$。
    \paragraph{}
    非常重要和常用的联合模型是高斯混合模型,其中$p(\mathrm{x}|c=i)$是高斯的。每个组成部分都有在自己单独的均值$\mu^{(i)}$和方差$\ sum^{(i)}$. 有些混合模型可能有更多的约束。例如:协方差可以通过约束$\sum^{(i)}=\sum\forall i$来共享成员。作为一个单一的高斯分布, 混合高斯模型可以为每个组件是对角或各向同性的协方差矩阵的约束。
    \paragraph{}
    除了均值和方差,高斯混合分布指定了每个$i$的先验分布$\alpha_{i}=P(c=i)$。先验是指已经知道$c$的情况下观察到$x$。 做一下对比,$P(c|x)$是一个后验概率,因为它在观察到$x$ 后才计算。高斯混合模型是常用的近似密度,因为任何平滑的密度都可以被多变量高斯混合模型近似。
    \begin{figure}[!htb]
    \centering
   \centerline{\includegraphics[width=3.0in]{fig/chap3/3_2.jpg}}
   \label*{图:3.2}
   \end{figure}
   \paragraph{}
   图3.2是高斯混合模型的样本。在这个例子中,有三个部分,从左到右,第一部分有一个各向同性的协方差矩阵,这意味着在每个方向上具有相同的方差。第二部分有一个对角协方差矩阵,这意味着它可以分别控制沿各轴方向的方差。从这个例子可以看出沿$X_{2}$轴的方差比$X_{1}$轴要大。第三部分有一个满秩的协方差矩阵,它能够单独控制任意方向上的方差。

    %--section 10--%

    \section*{3.10 常用函数的有用性质}
    \paragraph{}
    概率分布在很多函数的计算中应用广泛,特别是在深度学习模型中使用的概率分布。
    \paragraph{}
    这些函数之一就是描述S型曲线的logistic模型:
    \begin{equation}
    \sigma(x)=\frac{1}{1+\mathrm{exp}(-x)}   \tag{3.30}
    \end{equation}
    \paragraph{}
    S型曲线的logistic模型通常用于产生一个伯努利分布的参数$\varphi$,$\varphi$值的有效区间在(0,1)之间。图3.3是描述S型函数的图表,S型函数的参数过大或过小,都意味着函数变得平滑,且对其输入的微小变化不敏感。
    \paragraph{}
    另一个常见的函数是softplus 函数,参看(Dugas et al.,2001):
    \begin{equation}
    \zeta(x)=log(1+\mathrm{exp}(x))   \tag{3.31}
    \end{equation}
    \paragraph{}
    softplus 函数可以计算出正态分布的$\beta$和$\sigma$参数,其范围在$(0,\infty)$之间。当涉及S型曲线时常用该表达式。softplus 函数的名字来自于一个平滑的或"软化"版的如图3.4所示的softplus 函数的图表。
    \begin{figure}[!htb]
    \centering
   \centerline{\includegraphics[width=3.5in]{fig/chap3/3_3.png}}
   \label*{图:3.3}
   \end{figure}
    \begin{figure}[!htb]
    \centering
   \centerline{\includegraphics[width=3.5in]{fig/chap3/3_4.png}}
   \label*{图:3.4}
   \end{figure}
   \paragraph{}
   该函数具有以下性质:
   \begin{equation}
    \sigma(x)=\frac{\mathrm{exp}(x)}{\mathrm{exp}(x)+\mathrm{exp}(0)}   \tag{3.33}
   \end{equation}
   \begin{equation}
    \frac{d}{dx}\sigma(x)=\sigma(x)(1-\sigma(x))   \tag{3.34}
   \end{equation}
   \begin{equation}
    1-\sigma(x)=\sigma(-x)   \tag{3.35}
   \end{equation}
   \begin{equation}
    log\sigma(x)=-\zeta(-x)   \tag{3.36}
   \end{equation}
    \begin{equation}
    \frac{d}{dx}\zeta(x)=\sigma(x)  \tag{3.37}
   \end{equation}
   \begin{equation}
   \forall x\in(0,1),\sigma^{-1}(x)=log(\frac{x}{1-x})  \tag{3.38}
   \end{equation}
   \begin{equation}
   \forall x>0,\zeta^{-1}(x)=log(\mathrm{exp}(x)-1)  \tag{3.39}
   \end{equation}
   \begin{equation}
   \zeta(x)=\int_{-\infty}^{x}\sigma(y)dy  \tag{3.40}
   \end{equation}
   \begin{equation}
   \zeta(x)-\zeta(-x)=x \tag{3.41}
   \end{equation}
   \paragraph{}
   函数$\sigma^{-1}(x)$被称为统计的Logit模型,但是该术语很少用在机器学习中。
   \paragraph{}
   公式3.41对"softplus"给出了新的解释。Softplus函数是一个正部函数的光滑版本,$x^{+} =max{0, x}$。正部函数是负部函数的对应,$x^{-} = max{0,-x}$。为了获得一个平滑的函数,类似负部函数可以使用$\zeta(-x)$。$x$可以通过正部函数和负部函数通过$x^{+}- x^{-} = x$ 来抵消,也可以利用$\zeta(x)$和$\zeta(-x)$之间的关系来抵消,如公式3.41。

    %--section 11--%

    \section*{3.11 贝叶斯法则}
    \paragraph{}
    通常我们知道$P(y | x)$的情况下想计算$P(x | y)$,这时,如果我们还知道$P(x)$,我们就可以通过贝叶斯法则得到结果:
    \begin{equation}
   P(x|y)=\frac{P(x)P(y|x)}{P(y)} \tag{3.42}
   \end{equation}
   \paragraph{}
   需要注意的是,当公式中出现$P (y)$时,通常可以用$P(y)=\sum_{x}P (y|x)P(x)$计算,因此我们不需要去了解$P(y)$。贝叶斯规则是直接从条件概率的定义中得出的,但是我们熟悉这个名字是因为许多文本中提及才知道的。Reverend Thomas Bayes在公式中发现了一个特殊的现象,因此将其命名为贝叶斯。这里所提到的贝叶斯公式是由Pierre-Simon Laplace发现的。

    %--section 12--%

    \section*{3.12 连续变量的技术细节}
   \paragraph{}
  以恰当的形式理解连续随机变量和概率密度函数需要概率论的一个被称为测量理论数学分支的发展。测量理论是超出了这本教科书的范围,但我们可以简要的勾勒出一些测量理论的问题来解决。
   \paragraph{}
   在3.3.2节,我们看到,一个连续向量值$x$在集合$S$的概率是$P(x)$的积分。而一些集合$S$的选择可能会产生矛盾。例如,它可以构建两个集合$S1$和$S2$,即$P(x\in S1)+ P(x\in S2)> 1$但$S1 \cap S2 = \phi$.。这些集合通常是由大量使用的实数的无限精度的构造,例如分形集合或将有理数集合定义的集合。测量理论的一个重要贡献是提供一个表征的集合,我们可以计算出没有遇到矛盾的概率。在这本书中,我们只集成了相对简单的描述,所以这方面的测量理论不需要特别关注。
   \paragraph{}
   对于我们而言,测量理论在描述定理方面是非常有用的,其应用于$R_{n}$最点但不适用于一些边缘事件。测度理论提供了一个严格的方式来描述一个点集小到可以忽略不计。这个点集被称为“测量零”,我们在这本书中没有正式定义这个概念。然而,对它的理解是非常有用的,一组测量零点在我们测量的空间中没有体积,例如,当填充的多边形有正测量时,在$R_{2}$的线上有一个测量零点。同样,一个单独的点有测量零点。任何一个可数集合,每个都会有测量零点(因此所有的有理数集有测量零点)。
   \paragraph{}
   测量理论的另一个有用的术语是“几乎无处不在”,也就是除了集合的测量零点,几乎拥有其他所有的空间的属性。因为异常占据了可以忽略的空间,所以它们可以被安全地忽略许多应用程序。概率论中的一些重要结果持有所有的离散值,但只持有“几乎无处不在”的连续值。
   \paragraph{}
   连续变量的另一个技术细节是处理连续的随机变量,这是一个相互确定的函数。假设我们有两个随机变量$x$和$y$,使得$y = g (x)$,其中$g$是一个可逆的、连续的、可微变换。人们可能会想到$p_{y}(y)=p_{x}(g^{-1}(y))$.其实不是这样的。
   \paragraph{}
   举一个简单的例子,假设我们有一个随机变量$x$和$y$,均为标量,假设$y = x /2,x \in U(0,1)$。如果我们使用规则的$p_{y}(y)=p_{x}(2y)$,除了区间[0,1/2],$P_{y}$为0,在这个区间为1。这意味着下式违反了概率分布的定义。
   \begin{equation}
     \int p_{y}(y)dy=\frac{1}{2}.\tag{3.43}
   \end{equation}
   \paragraph{}
   这种常见的错误是因为它没有考虑函数$g$所引入的空间的变形。回忆一下$x$在一个体积无穷小的区域$\sigma x$的概率是$p(x)\sigma x$。由于$g$可以扩大或收缩空间,$x$空间中的无穷小体积在$y$空间中可能有不同的体积。
   \paragraph{}
   如何纠正这个问题,我们返回标量的情况下。我们需要保证其属性:
   \begin{equation}
      |p_{y}(g(x))dy|=|p_{x}(x)dx|.\tag{3.44}
   \end{equation}
   \paragraph{}
   从这一点上解决,可以得到:
    \begin{equation}
      p_{y}(y)=p_{x}(g^{-1}(y))|\frac{\partial{(x)}}{\partial(y)}|.\tag{3.45}
   \end{equation}
   \paragraph{}
   也可以用:
    \begin{equation}
      p_{x}(x)=p_{y}(g(x))|\frac{\partial{g(x)}}{\partial(x)}|.\tag{3.46}
   \end{equation}
   \paragraph{}
   在更高的维度,衍生出雅可比矩阵的行列式—矩阵的每个元素为$J_{i,j}=\frac{\partial{x_{i}}}{\partial(y_{j})}$.因此,对于实数向量$x$和$y$,
   \begin{equation}
      p_{x}(x)=p_{y}(g(x))|\det{(\frac{\partial{g(x)}}{\partial(x)})}|.\tag{3.47}
   \end{equation}

    %--section 13--%

    \section*{3.13 信息论}
   \paragraph{}
  信息理论是应用数学的一个分支,涉及如何量化信号中存在多少信息的问题。它最初被发明研究的是离散的字母在嘈杂的通道发送信息,如通过无线传输的通信。在这种情况下,信息理论讲述了如何设计最佳的代码并使用不同的编码方案计算来自特定概率分布的消息的期望长度。在机器学习的背景下,我们也可以应用信息论对一些不适用连续变量的信息长度进行解释。这一领域是电气工程和计算机科学的许多领域的基础。在这本教科书中,我们主要是使用一些关键的思想,从信息论来描述概率分布或量化概率分布之间的相似性。关于信息理论更详细内容参考Cover and Thomas (2006)或者MacKay(2003)。:
   \paragraph{}
   信息论背后的基础是学习一个不太可能的事件发生的信息量远远多于学习一个可能的事件发生的信息量。一个信息说:“今天早上太阳升起”是一个无用的信息是没有必要发送的,但是一个信息说:“今天早上发生了一次日食”,这个信息非常丰富。
   \paragraph{}
   我们想用直观的形式量化信息,特别的:
   \paragraph{}
   1 可能事件所含的信息量较少,在极端情况下,一定发生的事件没有任何信息量。
   \paragraph{}
  2 不太可能的事件其信息量越大。
  \paragraph{}
  3 独立事件的信息量可以相加。例如,投掷硬币时发现硬币两次朝
   \paragraph{}
   为了满足这三个属性,我们定义了一个事件$\mathrm{X} = x$的自信息:
   \begin{equation}
     I(x)=-\log{P(x)}\tag{3.48}
   \end{equation}
   \paragraph{}
   在这本书中,我们用$log$表示自然对数,基为$e$。我们定义的$I(x)$单位是$nats$。一个$nat$是通过观测概率为$1 /e$的事件得到的信息量。其他书中可能使用底数为2的对数,单位为比特或香农;测量信息位只是一个$nats$的标度测量信息。
   \paragraph{}
   当$x$是连续的,我们使用相同定义的信息通过类比,但一些属性在离散的情况下丢失。例如,尽管不是一个保证发生的事件,其单位密度的事件仍然有零的信息。
   \paragraph{}
   如图3.5所示,这个图显示了如何分类相近的确定性的较低的香农熵(Shannon entropy),如何分类接近均匀分布的有高的香农熵。在水平轴上,我们绘制$P$,一个二进制随机变量的概率等于1。熵为$(p- 1) log(1- p) - p log p$。当$p$接近0时,分布几乎是确定性的,因为随机变量几乎总是0。当$P$接近1时,分布几乎是确定性的,因为随机变量几乎总是1。当$P$ = 0.5时,熵是最大的,因为分布是均匀的两个结果。
   \begin{figure}[!htb]
   \centering
   \centerline{\includegraphics[width=3.5in]{fig/chap3/3_4.png}}
   \label*{图:3.5}
   \end{figure}
   \paragraph{}
   自我信息只处理一个单一的结果。我们可以使用香农熵在一个全概率分布下量化其不确定度。
   \begin{equation}
     H(x)=E_{(x\sim P)}[I(x)]=-E_{(x\sim P)}[\log{P(x)}]. \tag{3.49}
   \end{equation}
   \paragraph{}
   可以采用$H[P]$。换言之,一个分布的信息熵是来自该分布的事件的预期数量的信息。它给出了一个下界的比特数(对数以2为基础,否则单位是不同的)所需的平均编码符号服从一个分布$P$。分布几乎是确定性的(其中的结果是几乎肯定的)有低熵;有高熵的分布接近均匀。如图3.5。当$x$是连续的,其香农熵被称为微分熵。
   \paragraph{}
   如果我们有两个关于$x$的独立的概率分布$P(x)$和$Q(x)$,我们可以使用相对熵(Kullback-Leibler divergence,也称为KL散度)测量这两个分布的不同程度:
   \begin{equation}
     D_{KL}(P||Q)=E_{x\sim P}[\log{\frac{P(x)}{Q(x)}}]=E_{x\sim P}[\log{P(x)-\log{Q(x)}}].\tag{3.50}
   \end{equation}
   \paragraph*{}
   在离散型变量的情况下,它是一个额外的信息量(位测量时我们用以2为底的对数,但机器学习中我们通常使用$nats$和自然对数)需要发送一个包含来自概率分布$P$的符号的消息,我们可以使用一个准则最大限度地减少来自概率分布的消息的长度。
   \paragraph{}
   KL散度有许多有用的特性,最主要的是,它是非负的。KL散度为0表示在离散变量的情况下$p$和$q$的分布是相同的,等于一个无处不在的连续变量。因为KL散度是非负所以测量两个分布之间的差异,经常被理解为测量某种距离之间的分布。然而,它不是一个真正的距离测量,因为它不是对称的:对于$P$和$Q$,$D_{KL}(P||Q)\neq D_{KL}(Q||P)$。这种不对称性意味着对使用$D_{KL}$时选择$D_{KL}(P||Q)$ 和 $D_{KL}(Q||P)$ 哪一种对其结果影响很大。更多细节请参见图3.6。
   \paragraph{}
   和KL散度密切相关的是交叉熵,即$H (P,Q) = H(P) + D_{KL}(P||Q)$,这类似于KL散度,但如果缺少左边部分则:
   \begin{equation}
     H(P,Q)=-E_{x\sim P}\log{Q(x)}.\tag{3.51}
   \end{equation}
  \paragraph{}
  最小化相对于$Q$的交叉熵等价于最小化KL散度,因为$Q$和省略的$H(P)$不相关。
  \paragraph{}
  在计算这些量时,经常遇到0log 0这种表达形式。按照惯例,在信息论中,我们把这些表达式记为:$\lim_{x\rightarrow 0}x\log{x}=0$.
  \begin{figure}[!htb]
  \centering
   \centerline{\includegraphics[width=5in]{fig/chap3/3_6.png}}
   \label*{图:3.6}
   \end{figure}
   \paragraph{}
   图3.6的KL散度是不对称的。假设我们有一个分布$P(x)$,并希望近似它与另一个分布$Q(x)$。我们要么选择最小化$D_{KL}(P ||Q)$ 或$D_{KL}(Q||P)$ 我们举例说明对$P$函数使用高斯混合计算和对$q$函数进行的高斯求解,这两种选择的不同影响。选择使用哪个方向的KL 散度是根据具体问题而言。一些应用需要一个近似,通常是高概率的任何地方,真正的分布位置的概率高,而其他应用程序需要一个近似,很少有地方高概率的真实分布的地方低概率。

    %--section 14--%

    \section*{3.14 构造概率模型}
  \paragraph{}
  在机器学习中使用概率分布通常会用到一大堆变量。通常情况下,这些概率分布只会与一部分变量直接相关。而使用一个单一的函数来描述整个联合概率分布是效率不高(无论是计算或统计方面)。
  \paragraph{}
  事实上,我们并不是使用一个单一的函数来表示一个概率分布,我们可以将一个概率分布分解成许多因素,然后相乘。例如,假设有三个随机变量:$A,B$和$C$。假设$a$影响$b$的值,$b$的影响$c$的值,但$a$和$c$相对$b$是独立的。我们可以这样来描述:
  \begin{equation}
   p(a,b,c)= p(a)p(b|a)p(c|b).\tag{3.52}
   \end{equation}
   \paragraph{}
   这样分解可以大大减少描述分布的参数。每个因数都在变量内引入了指数。这就意味着,如果我们能找到一个分解能用非常少的变量描述这个分布,那就可以大大减少描述一个分布的成本。
   \paragraph{}
   我们可以用图形来描述这些分解。这里我们用一个图形的理论,点的集合可用通过线来互联。当用图来表示因式分解的时候,我们把这个叫做结构化概率模型或图形模型。
   \paragraph{}
   有两种构造概率的模型:有方向的和无方向的。这两种图形模型使用一个图,用圆圈表示一个随机变量,一条线连接两个随机变量。用这样的方式来表示两个变量在概率分布中有直接的关系。特别的,一个有方向模型包含了分布所需的所有变量$x_{i}$,关联点的概率和它父节点的变量相关,父节点用$Pag(x_{i})$表示。
   \begin{equation}
   p(x)=\prod_{i}p(x_{i}|Pag(x_{i})).\tag{3.53}
   \end{equation}

   \paragraph{}
   无方向模型使用无方向的现将两个变量连在一块,它表示因式分解中的一系列函数。这些函数和有方向模型不同,它不是任何形式的概率分布。几个顶点连接起来叫做集合。每个集合$\mathcal{C}^{i}$在无方向模型中都会和一个因子建立关系$\phi^{(i)}(\mathcal{C}^{(i)})$,这些因子只是函数而非分布。要保证每个因子的输出都是非负的,不过不一定需要向概率分布那样,他们的和或积分为1.

   \paragraph{}
   随机变量的联合概率与所有这些因子的乘积成正比—对影响较大的因子,对其结果的影响也很大。当然,不能保证这种乘积的和一定为1。因此我们将通过除以常数$Z$进行归一化,为了获得一个归一化定义$\phi$函数产生的所有状态进行求和或积分。概率分布为:
   \begin{equation}
   p(x)=\frac{1}{Z}\prod_{i}{\o}^{(i)}(c^{(i)}) \tag{3.54}
   \end{equation}
   \begin{figure}[!htb]
   \centering
   \centerline{\includegraphics[width=2.5in]{fig/chap3/3_7.png}}
   \label*{图:3.7}
   \end{figure}
 \paragraph{}
 图3.7是一个有向模型的随机变量$a,b,c,d$和$e$。此图对应的概率分布可以分解为:
\begin{equation}
   p(a,b,c,d,e)=p(a)p(b|a)p(c|a,b)p(d|b)p(e|c).\tag{3.55}
 \end{equation}
\paragraph{}
 这个图可以让我们很快的看到一些分布的属性。例如,$a$和$c$之间直接影响,而$a$和$e$之间通过$c$间接影响。
 \begin{figure}[!htb]
 \centering
   \centerline{\includegraphics[width=2in]{fig/chap3/3_8.png}}
   \label*{图:3.8}
   \end{figure}
   \paragraph{}
   图3.8是一个无向模型的随机变量$a,b,c,d$和$e$。此图对应的概率分布可以分解为:
   \begin{equation}\
     p(a,b,c,d,e)=\frac{1}{Z}(a,b,c){\o}^{(2)}|(b,d){\o}^{(3)}(c,e).\tag{3.56}
   \end{equation}
   \paragraph{}
   这个图可以让我们很快的看到一些分布的属性。例如,a和c之间直接影响,而a和e之间通过c间接影响。
   \paragraph{}
   用这些图形表示分解是描述概率分布的一种语言。他们不是相互排斥的概率分布的成员。有向或无向不是一个概率分布的属性,它是一个概率分布的一个特定的描述,但任何概率分布都可以用这两种方式描述。
   \paragraph{}
   在这本书的第一部分和第二部分,我们将使用结构化的概率模型仅仅作为一种语言来描述直接概率关系和机器学习算法中的不同之处,没有进一步了解结构化的概率模型。直到讨论课题的第三部分中,我们将探索结构化概率模型中的更多细节。
   \paragraph{}
   本章回顾了深度学习相关的概率论的基本概念。其基本的数学工具仍然是:数值方法。

\end{CJK*}
\end{document}

%%%%%%%%%%%%%%%%%%%%%%%%%%%%%%%%%%%%%%%%%%%%%%%%%%%%%%%%%
%%%%%%%%%%%%%%%%%%%%% author:BrowningWan %%%%%%%%%%%%%%%%
%%%%%%%%%%%%%%%%%%%%% part:4.0-4.3       %%%%%%%%%%%%%%%%
%%%%%%%%%%%%%%%%%%%%%%%%%%%%%%%%%%%%%%%%%%%%%%%%%%%%%%%%%

\chapter{数值优化}
\label{chap:4}



\section{4.0}
%%%%%%%%%%%%%%%%%%%%%%%%%%%%%%%%%%%%%%%%%%%%%%%%%%%%%%%%
%%%%%%%%%%%%%%%%  author:cypress1010@sina.com %%%%%%%%%%
%%%%%%%%%%%%%%%%  part:4.4-4.5                %%%%%%%%%%
%%%%%%%%%%%%%%%%%%%%%%%%%%%%%%%%%%%%%%%%%%%%%%%%%%%%%%%%


\section{4.4}
\chapter{机器学习基础}
\label{chap:5}
%%%%%%%%%%%%%%%%%%%%%%%%%%%%%%%%%%%%%%%%%%%%%%%%%%%%%%%%%
%%%%%%%%%%%%%%%%% author:dormir_yin %%%%%%%%%%%%%%%%%%%%%
%%%%%%%%%%%%%%%%%%%%%%%%%%%%%%%%%%%%%%%%%%%%%%%%%%%%%%%%%

深度学习其实也是机器学习的一种方法。为了更好的理解深度学习,我们必须也要了解一下机器学习的基础知识。这章给大家简要介绍了机器学习中一些重要的概念,这些内容在本书的后续章节会涉及到。如果你是初学者或者想对机器学习有更广更详细的了解,推荐你找一本机器学习的教科书。比如说Murphy(2012)或者Bishop(2006)。如果你已经对机器学习的基础比较熟悉了,你可以直接开始看5.11。那部分内容介绍的传统机器学习技巧对深度学习算法的发展有较大的影响。

我们首先会先定义一下什么是机器学习算法。随后我们会给一个例子 线性回归算法。接着我们介绍拟合训练数据和将我们学习到的模型运用到新的数据集的不同之处。大部分机器学习算法会让我们设置超参数。这些超参数是需要我们自己定义的,而不能通过训练过程自动更新。我们也会讨论如何设置超参数。机器学习本质上就是应用统计学。和应用统计学不同的是它强调了了利用计算机来对一些复杂的函数进行统计估计。但是减弱了对估计出来的函数计算置信区间。就是说它利用统计学方法的得出一个模型,但它不强调用传统的假设检验的方式来对该模型进行评估。接着我们会介绍两个核心的统计算法 频率估计 和 贝叶斯推断。频率学派和贝叶斯学派也是统计学的两大流派。大部分机器学习任务可以分为监督学习和无监督学习两类。我们会对这两类做一个介绍,同时会介绍一些它们用到的机器学习算法。大部分深度学习算法是基于一种叫随机梯度下降法的优化算法。我们会介绍如何构建一个完整的机器学习算法。这些算法包含多个部分。优化算法,损失函数,模型和数据集。最后,在\ref{sec:5.11}我们介绍了传统机器学习算法遇到的一些困难,这些困难促使我们发展深度学习来解决传统机器学习算法所不能解决或者很难解决的问题。

\section{机器学习算法}
\label{sec:5.1}

所谓机器学习算法,就是能够从数据中学习的算法。但是学习到底是什么含义。Mitchell 提供了一个定义。一个计算机程序从关于不同类型的任务的经验中进行学习。

我们可以想象,经验 任务 还有评估策略可以有很多,在这本书中不会给这些名词一个正式的定义。但是我们在接下来的章节中会对他们进行大致的描述,并给一些例子来帮助大家这些名词,已经如何用它们来组成一个机器学习算法。

\subsection{任务 $T$}
机器学习可以解决一些传统的算法很难解决的问题,之前我们总会用人为设计的固定的算法来解决一些问题。从一个科学或者哲学的观点来看,机器学习之所以有趣,是因为在发展理解机器学习算法的同时,我们也会思考发现人类智能的本质。人是如何思考的。

在这一段,我们会给任务 T 一个相对正式的定义。学习本身的过程不能称之为任务,学习的目的是为了获得完成任务的能力。举个例子: 如果我们制作出了一个机器人,想让它具有走路能力,那么走路就是任务T。我们可以写一个程序让机器人自己学习如何走路,也可以直接写一个程序直接控制机器人走路。

机器学习任务通常机器学习系统如何处理一个个实例 Example。实例是特征的集合。这些特征来自于我们设计的机器学习系统需要处理的事件或者物体,而且都已经被量化了。我们通常把实例表示成一个向量,向量的每个元素表示不同的特征。比如说图像的特征通常就是图像的像素值。

机器学习可以解决很多不同的任务。下面我列举了最常见的机器学习任务:
\begin{itemize}
\item \textbf{分类:} 在这类任务中,总共有k类,计算机算法需要判断每个输入属于哪个类别。为了完成这个任务,学习算法通常需要产生一个函数,或者叫对应关系。模型会将每个输入x对应到某个类别y。当然也有很多类似的分类任务,比如说f输出一个概率分布,每个类别对应一个概率。物体识别就是分类任务的一个例子。这个任务重输入时一幅图片 通常可以描述成像素值的集合。输出是图片中物体对应的编码。比如说机器人 就可以识别不同的饮料,并把他们送给点单的客人。深度学习在完成目标检测都有较好的效果。同时,目标检测的算法也可以用于人脸识别,这样我们就可以对照片里的每个人进行自动标注了,而且也可以帮助计算机更好的和人类进行交互。

\item \textbf{缺失特征情况下的分类:} 如果一些数据缺失某些特征的化,分类问题会变得比较难。也就是说不能保证输入向量里每个对应的特征都可以提供。不同的x缺德特征也有可能不一样。传统的分类任务中,学习算法需要定义一个函数,能够将输入向量映射到一个输出类别。但如果一些输入特征缺失了,学习算法就需要定义一系列函数,

\item \textbf{回归:}  在这类任务重,计算机程序需要根据输入预测一下数值。为了解决这个问题,学习算法需要得到一个函数 这类人物和分类任务很像,唯一的不同就是输出的格式不一样,分类任务要求输出的是输入对应的类别,是离散的,而回归任务要求的输出是连续的数值。

\item \textbf{描述:}  在这类任务中,机器学习系统需要观察一些非结构化数据,然后用文字或者离散化的符号描述它们。比如说文字识别,计算机程序需要识别出图片中所包含的文字,然后将识别出来的文字返回。谷歌的 就是利用深度学习来街道号码的。另一个例子是语音识别,计算机程序会根据输入的波形来判断我们说的话。然后将它们转成文字或者文字对应的编码。现在深度学习是语音识别系统一个非常重要的组成部分,已经被好多大公司使用,包括微软,IBM,谷歌。

\item \textbf{机器翻译:}  在机器翻译任务中,我们需要将某种语言的序列翻译成另一种语言的对应的序列。这是自然语言处理的的一部分,比如说将音乐翻译成法语。深度学习在这些任务中,表现的越来越出色。
\item \textbf{结构化输出:} 这类任务一般都要求输出是一个向量,或者其他能够存储多个值的数据结构。这样的任务太多了,包括我们上面提到的描述和机器翻译。当然还有一些其他的任务,比如说将

\item \textbf{异常检测:}  在这类任务中,计算机程序会对输入的事件和目标进行筛选,找出那些异常的的事件或者事物。信用卡欺诈检测就是这类任务的一个例子。通过对你的消费习惯进行建模,当你的卡被盗刷的时候,信用卡公司可以检测出来这个异常事件。当小偷盗取了你的行用卡或者信用卡信息,他进行消费的时候,这些消费记录会和你的消费习惯的概率分布不一致。这样信用卡公司在检测到你的信用卡账号有异常购买记录的时候就会直接把你的信用卡给锁了。可以看一下,里面介绍了一些异常检测的方法。

\item \textbf{模仿合成及采样:}  在这类任务中,机器学习算法或生成一些和训练数据相似的数据。这类任务在艺术和多媒体领域很有用,艺术家进行创造的过程是很枯燥的,而且要花费大量的经历。比如说,在电脑游戏中,我们可以自动生成一些物品和风景。而不是依靠艺术家一点一点画出来。当然我们也可以接收一些比较特殊的输入。比如说语音合成,我们输入一个句子,然后程序就会合成这个句子对应的语音波形。这也是上文提到的结构化输出任务,但是在这个任务中每个输入并没有对应的唯一的正确的输出。我们期望输出会有多一点的变化,这样就会让人感觉更加自然和真实。

\item \textbf{缺失值预测:} 在这类任务中,机器学习算法我们会输入给机器学习算法一个数据实例,但是它的某些元素是缺失的。这个算法必须对这些缺失值作出预测。

\item \textbf{去噪:} 在这类任务中,我们会给学习器一个被污染的数据,这个被污染的数据是原始的纯净数据经过一个未知的污染过程所得到的。学习器必须从这个被污染的数据中预测出原始的那个纯净的数据。或者说,预测出条件概率分布。

\item \textbf{密度估计和概率质量函数估计:} 在密度估计的问题中,机器学习算法需要学习一个函数。x从某个概率分布空间采样出来的数据。当x是连续变量的时候, 可以被看做是密度函数。当x是离散的时候,这个可以被看成一个概率质量函数。为了完美的完成这个任务 当然如何评价模型的好坏我们会在下面一个小结来讨论。这个算法需要我们学习数据的分布,我们需要知道哪些地方出现的数据多,哪些范围数据很少出现。这些任务需要学习算法可以得到数据分布的信息。我们可以通过对这些分布信息进行计算处理来完成其他任务。比如说我们通过概率估计获得了概率分布px,我们可以使用这个结果来解决上面提到的缺失值预测任务。如果缺失了,但是我们有其他的值。这样我们就知道缺失值得条件概率分布式 。在实际应用中,密度分布估计不能总是帮助我们解决这些相关的问题。因为很多情况我们需要通过px来做的一些计算是很难的。
\end{itemize}

当然,还有很多其他类型的任务。我们在这列了这么多只是想告诉大家机器学习能做的一些例子。这些并不是一个严格的任务的分类。


\subsection{评估准则 $P$}
为了能够对机器学习算法进行评估,我们需要针对模型的表现设计一些可以量化的评价准则。通常针对不同的任务我们会设计不同的评价准则。
对于分类任务,有缺失值得分类任务,识别任务我们使用精度来评价模型。精度就是那些模型给出正确输出的数据所占的比例。我们也可以通过计算错误率来获得同样的信息。错误率和模型相反,是模型给出错误输出的数据所占的比重。这个错误率也可以看做是01损失。在某个特定的数据例子上,如果被正确分类了,那么损失是0,否则损失是1. 但是对于概率密度估计这类任务,估计精度就,错误率或者其他类似01损失就没有意义。我们需要找一个新的评估准则来对每个数据点一个分数,这个分数的取值范围是连续的。最常用的方法就是通过模型来给每个数据点赋予一个平均对数概率。
通常我们比较重视模型的泛化性,也就是说看这个模型在处理它之前没见过的数据上面的表现。这个表现可以看出它在实际使用中是否可以工作的很好。因此,我们会通过一个测试集来评估模型,这个测试集是从我们之前用于训练模型的数据集里面分离出来的。
我们的评估准则可能看上去很直接。但是我们其实很难给系统选择一个评估准则可以真正符合我们的要求。
在一些情况下,我们很难决定我们需要评估什么。举个例子,在描述任务中,我们评估精度的时候,我们应该考虑整个序列,当整个序列都正确我们才算对,还是使用一个更细致的评估准则,当序列中某一部分对的时候,我们也给一点分,而不是直接给零分。在回归任务中,对于两个系统,一个会经常发生一些小错误,一个偶尔会发生一些大错误。我们会更偏向哪个呢,或者说准备给那个系统一个更大的惩罚。我们要根据我们的实际需求来作出选择。
在其他一些情况下,我们可以明确需要评估的对象。但是评估的过程不是很方便。举个例子,在概率密度估计的任务中经常会遇到这些问题。很多最好的概率模型表示的概率分布是暗含的,如果你想计算空间中某个特定点的概率值在这些模型中是不可以的。在这种情况下我们需要重新设计一些规则,使我们目标。或者设计一个和理想规则的近似规则。



\subsection{经验 $E$}
根据在学习过程中可以允许获得的经验的类型,机器学习算法可以直接分成监督学习和无监督学习。

这本书介绍的大多数机器学习算法可以被理解成能够接触整个数据集。在511 ,我们已经定义了一个数据集就是很多数据实例的集合。数据实例我们有时候也叫他数据点。

鸢尾花数据集就是一个相当古老的数据集,很早就被统计学家和机器学习研究者来使用。这个数据集记录了鸢尾花不同部分的数据。总共记录了150株鸢尾花,每一株就是一个数据实例。数据实例里面的特征对应的就是鸢尾花不同的部分。萼片长度,萼片宽度,花瓣长度,花瓣宽度。数据集还记录了每个鸢尾花对应的类别。总共有3种不同的鸢尾花类别。

无监督学习算法会遍历整个数据集,一般这个数据集包含很多特征。算法会从数据集的结构中学习其中有用的属性。在深度学习中,我们通常想学习产生数据集的概率分布。判断这个目标函数是不是很直接,类似于概率密度估计。或者是类似于合成,去噪之类的任务。目标不是很明显。还有一些其他的无监督学习算法,比如说聚类,就是根据数据实例的相似度将数据集分成几类。

监督学习算法的数据集也有很多特征,但是每个数据实例都会对应一个标签。比如说鸢尾花集里的每个数据实例就对应某种鸢尾花的类别。一个监督学习算法可以学习整个鸢尾花集,随后就可以将新的数据分成三种不同的鸢尾花类别。

简单的来说,无监督学习会观察几个实例$x$,这几个实例一般都是随机向量。并且会从这些数据集中来学习概率分布$p(x)$,或者这个分布中其它的一些性质。在监督学习里,也会观察一些实例x,但是,每个实例都有一个对应的值或者向量。我们需要通过通过$x$来预测$y$。通常我们会使用$p(y,x)$来进行估计。我们之所以叫监督学习是因为我们会提供一个目标值$y$。就好像是有一个老师来教机器学习系统该如何去做。在无监督学习中,并没有什么老师,机器学习算法需要独自理解数据的内容,或者从中发现一些有趣的东西。
监督学习和非监督学习并没有一个正式的定义。它们之间的界限也比较模糊。很多机器学习算法既可以做监督学习任务,也可以做非监督学习任务。举个例子,在概率图中有一个链式法则:给定一个向量$x$,它的联合分布可以被分解成:

表面上看求联合概率密度是一个非监督学习任务,但是通过这个分解我们把求联合概率px的任务分解成了n个监督学习的问题。相反,当我们想解决监督学习问题$p$ 我们可以通过传统的无监督学习的的技术来学习$x$,$y$的联合分布然后我们就可以用贝叶斯公式来解决这个问题:

%%%%%%%%%%%%%%%%%%%%%%%%%%%%%%%%%%%%%%%%%%%%%%%%%%%%%%%%%
%%%%%%%%%%%%%%%%%%% author:kakaguotao %%%%%%%%%%%%%%%%%%%
%%%%%%%%%%%%%%%%%%% part:5.2-5.6      %%%%%%%%%%%%%%%%%%%
%%%%%%%%%%%%%%%%%%%%%%%%%%%%%%%%%%%%%%%%%%%%%%%%%%%%%%%%%


\section{算法容量,过拟合,欠拟合}
\label{sec:5.2}

\section{超参数和验证集}
\label{sec:5.3}

\section{估计,偏差,方差}
\label{sec:5.4}

\section{最大似然估计}
\label{sec:5.5}

\section{贝叶斯统计}
\label{sec:5.6}
%%%%%%%%%%%%%%%%%%%%%%%%%%%%%%%%%%%%%%%%%%%%%%%%%%%%%%%%%
%%%%%%%%%%%%%%%%%%% author:KaiserW %%%%%%%%%%%%%%%%%%%%%%
%%%%%%%%%%%%%%%%%%% part:5.7-5.11  %%%%%%%%%%%%%%%%%%%%%%
%%%%%%%%%%%%%%%%%%%%%%%%%%%%%%%%%%%%%%%%%%%%%%%%%%%%%%%%%

\section{监督学习算法}
\label{sec:5.7}
前承\ref{sec:5.1.3}节,有监督学习(Supervised Learning)简单来讲就是一种学习算法,它会学着在某些输入和某些输出之间建立关联,这些输入\textbf{x}和输出\textbf{y}来自于训练集中的样本。很多时候,输出\textbf{y}很难自动采集,而必须由一位人工“监督者”(supervisor)提供,当然即便训练集的拟合目标已经自动采集完成,“监督学习”的名称仍然适用。
\subsection{概率监督学习}
\label{sec.5.7.1}
本书提到的大多数监督学习算法都是基于对概率分布$p(y|x)$的预测。我们可以简单地应用最大似然估计(maximum likelihood estimation)来找到分布$p(y|x;\theta)$参数族的最佳参数向量$\theta$。

已知线性回归(linear regression)对应参数族
\begin{equation}
	p(y|x;\theta) = \mathbb{N} (y;\theta^{T}, \textbf{\textit{I}})
  	\label{form:5.80}
\end{equation}

通过定义不同族的概率分布,我们可以将线性回归推广到分类(classification)情景。如果我们有两个类别,类0和类1,那么接下来只需确定其中一个类的的概率就可以了。类1的概率自然也就决定了类0的概率,因为两个概率值相加必然为1.
基于平均值将实数域上的正态分布进行参数化,这一分布我们也用于线性回归,这里的平均值可以是任意值。但是二元变量的分布则更加复杂一些,因为其平均值必然始终落在0和1之间。一种解决方案是应用逻辑函数(logistic function, 也称sigmoid函数)将线性函数的输出值挤压到(0,1)区间,转换后的值可以理解为是一个概率:
\begin{equation}
	p(y=1|x;\theta) = \sigma (\theta^{T}x)
  	\label{form:5.80}
\end{equation}

这一方法即是逻辑回归(logistic regression),这名字有些古怪,因为我们实际上用这个模型做分类而不是回归。
对于线性回归,我们可以解正规方程(normal equations)以求得最优权重。而逻辑回归就要复杂一些,它的最优权重没有解析解。我们只能通过最大化对数似然率(log-likelihood)来逼近最优解,具体的策略是,应用梯度下降法(gradient descent)使负对数似然率(negative log-likelihood, NLL)最小化。

这一策略基本可以应用在任何监督学习问题中:对于正确类型的输入/输出变量,写下其条件概率分布的参数族。

\subsection{支持向量机}
\label{sec:5.7.2}

支持向量机(Boser et al., 1992; Cortes and Vapnik, 1995)是最具影响力的监督学习方法之一。该方法与逻辑回归很相似,因为都是由线性函数$\omega^{T}x + b$所驱动。不同于逻辑回归,支持向量机(Support Vector Machine, SVM)并不提供概率值,只有分类结果。当$\omega^{T}x + b$为正,SVM预测为正类;同理当$\omega^{T}x + b$为负,则预测为负类。

支持向量机的关键创新点是\textbf{核技巧}(kernel trick)。核技巧观察到很多机器学习算法可以写作样本的点乘积。例如,支持向量机所用的线性函数可以写作形如
\begin{equation}
	\omega^{T}x + b = b + \sum_{i=1}^{m}{\alpha_{i}x^{T} x^{(i)}} 
    \label{form:5.81}
\end{equation}

这里$x^{(i)}$是一个训练样本,$\alpha$是系数矢量。

以这种方式重写学习算法之后,我们便可以用特征函数$\phi(x)$的输出和函数$k(\textbf{x}, \textbf{x}^{(i)})=\phi(\textbf{x})\cdot\phi(\textbf{x}^{(i)})$替代$\textbf{x}$,其中的$k(\textbf{x}, \textbf{x}^{(i)})=\phi(x)\cdot\phi(\textbf{x}^{(i)})$就叫做\textbf{核}(kernel)。$\cdot$操作符表示与$\phi(x)^{T}\phi(\textbf{x}^{(i)})$类似的内积。在有些特征空间中,我们可能无法使用真正的矢量内积;在有些无限多维的空间中,我们需要使用其他类型的内积,比如基于积分而不是加法的内积。此类内积的完整推导已经超出了本书的范畴。

用核替代了点积之后,我们可以用以下函数做预测
\begin{equation}
	f(x) = b + \sum{i}^{}{\alpha_{i}k(x,x^{(i}}
	\label{form:5.82}
\end{equation}

此函数对$textbf{x}$是非线性的,但是$\phi(\textbf{x})$和$f(\textbf{x}$之间是线性关系。并且$\alpha$和$f(\textbf{x}$也是线性关系。以下过程与基于核的方程都是严格等效的:对所有输入应用$\phi(\textbf{x}$,然后在新的变换空间中学习线性模型。

核技巧的强大有两重原因。首先,它允许我们并使用保证有效收敛的凸优化(convex optimization)技术,把对$x$的非线性函数当作线性的来学习。这是因为我们认为$\phi$是不变的,只优化$\alpha$,换言之,优化算法可以把决策方程在另一个空间中看作线性的。其次,相比于直接构建两个$\phi(\textbf{x})$矢量并显式求点积,核函数$k$的计算效率往往更高。

某些情况下,$\phi(\textbf{x})$甚至可以是无限维的,直接的显式求解将导致无穷的计算消耗。多数情况下,$k(\textbf{x}, \textbf{x'})$是$\textbf{x}$的非线性可解函数,即使$\phi(\textbf{x})$不可解。作为无限维特征空间中可解核的例子,我们构建一个特征映射,从非负整数x到$\phi(\textbf{x})$,设想该映射返回一个包含x个1及无穷多个0的矢量。我们可以写一个核函数$k(\textbf{x}, \textbf{x'}) = min(x, x^{i})$,这与无限维的点积严格等价。

最常用的核是\textbf{高斯核}(Gaussian kernel)
\begin{equation}
	k(u, v) = \mathbb{N}(u-v;0, \sigma^{2}I)
\end{equation}
$\mathbb{N}(x;\mu,\Sigma)$是标准正态密度。这个核也被称为径向基函数(radius basis function, RBF)核,因为其值在$v$空间中沿着$u$向外辐射而减小。高斯核对应着无限维空间里的点积,但是这一空间中的推导不像之前整数核的例子那样直观。

我们可以认为高斯核实现的是一种模板匹配(template matching)。一个与训练标签$y$相关的训练样本$x$构成了类$y$的一个模板。当测试点$x'$与$x$的欧几里得距离(Euclidean distance)很近的时候,高斯核有一个很大的响应,表示$x'$与$x$模板很相似。这一模型给相关训练标签$y$的权重很高。总体上看,预测是把基于对应模板样本(training example)进行过加权的训练标签(training label)组合了起来。

支持向量机并非唯一经由核技巧加强的算法,很多其他的线性模型都可以通过这种方式加强。这类搭载了核的算法也被称作核机器(kernel machine)或核方法(kernel method)。

核机器的最大缺在于,评估决策函数的计算量与训练样本数量呈线性关系,因为第$i$个样本向决策函数提供了$\alpha_{i}k(x,x^{(i)})$。支持向量机可以通过学习一个主要由0构成的矢量$alpha$来缓解这一弊端,对一个新样本做分类,只需要评估$\alpha_{i}\neq0$的样本,这些训练样本就是\textbf{支持向量}(support vector)。

核机器面临的另一大困难就是处理大数据时的超高计算资源消耗,我们将在\ref{sec:5.9}节重新审视该问题。普通的核机器很难提高适用性,我们将在\ref{sec:5.11}节重点讨论。现代深度学习的诞生正是为了突破这些局限,而当下的深度学习“复兴”正是始自Hinton et al(2006)展现了在MNIST数据集上,神经网络比RBF核支持向量机表现的更有力。

\subsection{其他简单的监督学习算法}
\label{sec:5.7.3}

我们已经简单了解过另一非概率的(non-probabilistic)监督学习算法,\textbf{近邻回归}(nearest neighbor regression)。更一般地来讲,k近邻(k-nearest neighbors)是一系列可用于分类和回归的技术。作为无参数学习算法,k近邻不为固定的参数量所限。我们通常认为k近邻算法没有任何参数,而是对训练数据施加了一个简单的函数。实际上k近邻甚至没有真正的训练或学习过程,在测试过程中,当我们想要对一个新测试输入$x$产生新输出$y$的时候,我们直接从训练数据$X$里找到离$x$最近的点,然后返回训练集对应$y$的平均值。在一个监督学习算法里,只要我们能定义出$y$的平均值,这个方法就是好用的。

在分类问题中,我们可以对one-hot编码矢量$\textbf{c}$做平均,其中$c_{y}=1$且其他i值的$c_{i}=0$。对于one-hot编码的平均可以理解为关于类的概率分布。作为无参数学习算法,k近邻的性能属于非常高的。比如当我们有一个多分类任务,并且用0-1的损失值衡量分类表现的时候,随着样本数量趋向无穷多,1近邻(k=1)将收敛至2倍贝叶斯误差(Bayes error)。超过贝叶斯误差的部分是因为,当存在两个距离相同的近邻时,我们只能随机选择其中一个,如此就割裂了两个点之间的关联。如果训练数据无穷多,每个测试点x周围都有无穷多个训练集近邻与之0距离。若是让算法纳入所有这些近邻点而不是随机选取的话,最终就能收敛至贝叶斯误差。当训练集很大的情况下,k近邻的高性能特性使其能够达到极高精度。但是随之而来的计算消耗也是巨大的,且如果训练集较小,预测的泛用性会很差。

k近邻算法的一大缺点就是,它不能自主发现有的特征比其他特征的判别力度更强。想象我们在有$x\subset\mathbb{R}^{100}$空间中有一个回归任务,x来自于各向同性的高斯分布,但是只有1个变量$x_{1}$与输出有关。更进一步地,假设这一特征变量直接决定输出,比如$y=x_{1}$恒成立。但近邻回归却无法探索出这一简单的模式,多数点$x$的最近邻判别仍将受到$x_{2}$到$x_{100}$的众多特征影响,而不是仅有$x_{1}$单独决定,于是小规模训练集的输出基本会是个随机变量。

\begin{figure}[htbp]
   \centering
   \includegraphics[width=6in]{fig/chap5/5_7.png} 
   \caption{决策树的运作机理。(上图)树的每个节点选择将输入样本送至左子节点(0)或右子节点(1)。中间节点画作圆圈,叶子节点画作方块。每个节点都以二级制字符串表示其在树中的位置,在其父节点的识别编码后增补字节(0=左或上,1=右或下)。(下图)树将输入空间划分成不同区域。这个2D平面显示了决策树如何划分$R^{2}$,中间节点画在给样本分类的分割线上,叶子节点画在样本对应区域的中心。结果是一个分段等值函数,每个叶子节点都相当于一段。每个叶子节点都需要至少一个样本才能定义,所以决策树不能学习到局部最大值比样本数量还多的函数。}
   \label{fig:5_7}
\end{figure}

另有一类学习算法也将输入孔建分成了不同的区域,并且每个域自有独立的参数,这就是\textbf{决策树}(decision tree, Breiman et al.)及其变种。如\ref{fig:5_7}所示,决策树的每个节点都对应输入空间里的不同区域,每个区域再被中间节点细分为子区域,并形成子节点(通常基于轴向切分)。空间由此继续细分为不重叠的区域,建立起叶子节点(leaf node)与输入区域(input region)之间的一一对应。每个叶子节点通常将来自同一输入区域的点映射到相同的输出。

决策树一般通过特殊的算法训练,这些算法超出了本书的范畴。如果一个学习算法可以学习任意规模的树,那么我们就可以认为该算法是无参数的,尽管在实际应用中决策树通常都经过基于规模限制的正则化(regularization)以转换为有参数模型。

决策树在使用中经常经过轴向切分(axis-aligned split),每个节点对应一个不变的输出,很多逻辑回归都能轻易解决的问题对决策树却可能很困难。例如,如果我们有一个二分类问题,当$x_{2} > x_{1}$时为正类,那么决策边界可能就不是轴向的。于是决策树就需要结合多个节点来估测决策边界,估测过程是调用一个阶跃函数(step function)在正确的决策函数边界上做等值、轴向的反复走步。

如前所述,近邻预测器和决策树存在很多限制,然而当计算资源有限的时候它们还是非常实用的。通过思考更复杂的高级算法与k近邻、决策树之间的相似与差异,我们可以获得发现更精巧算法的直觉。

关于传统监督学习算法的材料,可查看Murphy(2012), Bishop(2006), Hastie et al.(2001)或者其他机器学习课本。

\section{非监督学习算法}
\label{sec:5.8}

前承\ref{sec:5.1.3}节,无监督算法只有“特征”而没有监督信号。监督算法和非监督算法之间并没有明确严苛的区别,因为一个值属于特征还是监督者提供的标签,这本来也不存在客观定义。一般来讲,无监督学习指的是那些不借助人为干预或外部信息,直接从分布中提取信息的算法。与其高度相关的概念有:密度估计,样例提取,降噪,流形,聚类等。

有个经典的无监督学习任务,就是寻找数据的“最佳”表征。这里的“最佳”指的是尽可能多的保留$x$的信息,同时保证表征的形式比$x$本身更加简单。

定义更简单的表征可以有多种方式。三种最常见的是低维表征(low-dimensional representation)、稀疏表征(sparse representation)和独立(independent representation)表征。低维表征试图尽量压缩信息;稀疏表征(Barlow, 1989;Olshausen and Field, 1996;Hinton and Ghahramani, 1997)将数据集嵌入到另一个主要由0构成的表征里。使用稀疏表征往往要升维,整体结构是将数据分发至表征空间的各个轴上;独立表征则试图理清数据变化背后的来源,以使表征的每个维度之间都服从统计独立。

当然这三个标准并非互斥关系。低维表征常常降低元素之间的依赖性,这是因为压缩信息量的一种常用方法就是识别并剔除数据中的冗余关系。冗余的降低让降维算法可以直接忽略少部分信息以达到更大的压缩程度。

表征(representation)是深度学习的核心命题,也是本书的核心命题。在这一节中,我们开发了几个表征学习的例子,这些示例算法将展示如何实现上面所有的三个判据。之后的章节也将介绍更多从不同角度演绎这三大判据的表征学习算法以及其他判据。

\subsection{主成分分析}
\label{sec:5.8.1}

在\ref{sec:2.12}中,我们已经了解到,主成分分析(PCA, principal component analysis)算法提供了一种压缩数据的方法。我们同样把PCA看作是一种无监督学习算法,它从数据中学习表征,这样的表征基于以上提到的两种判据。PCA学习维数低于初始输入的表征,也学习一个元素与其他元素线性无关的表征。这正是统计独立表征学习的第一步。为了获得完全的独立性,表征学习算法必须将变量之间的非线性关系也识别并剔除掉。

\begin{figure}[htbp]
   \centering
   \includegraphics[width=6in]{fig/chap5/5_8.png} 
   \caption{PCA学习一种线性投影方式,使变化最大的方向与新空间的轴平行。(左图)初始数据由样本$x$构成,在这一空间中,数据的变化可能与坐标轴不平行。(右图)经过变换的数据$z=x^{T}W$现在基上是沿着轴$z_1$变化。第二大的变化方向则沿着$z_2$.}
   \label{fig:5_8}
\end{figure}

PCA学习数据的正交线性变换,将输入$x$投影到表征$z$,如\ref{fig:5_8}所示。在\ref{sec:2.12}中已经介绍过,我们可以学习构建原始数据的“最佳”一维表征(最小二乘法),这一表征其实就对应数据的第一个主成分。接着我们可以把PCA用作一种简单有效的降维方法来尽可能多的保留数据中的信息(同样根据最小二乘法)。接下来,我们将学习PCA表征如何与原始数据表征\textit{X}“去相关”。

考虑一个$m\times n$的设计矩阵$X$。假设数据的平均值为0,$E[x]=0$。如果非零,只需在预处理中将所有数据同时减去现有均值即可。

$X$的无偏协方差矩阵写作:
\begin{equation}
	Var[x] = \frac{1}{m-1}X^{T}X
   \label{form:5.85}
\end{equation}

PCA找到一个表征(通过线性变换)$z=x^{T}W$使$Var[z]$为对角阵。
在\ref{sec:2.12}中,我们已知设计矩阵$X$的主成分由$X^{T}X$的特征向量给出:
\begin{equation}
	X^{T}X = W\Lambda W^{T}
   \label{form:5.86}
\end{equation}

本节,我们将挖掘另一种主成分的推导方法。主成分也可以由奇异值分解(SVD)得到。特别地,他们就是$X$的右奇异向量。为证此理,设$W$为分解$X=U\Sigma W^{T}$的右奇异向量。再将$W$的原始特征向量还原为特征向量基。
\begin{equation}
	X^{T}X = (U\Sigma W^{T})^{T} U\Sigma W^{T} = W\Sigma^{2}W^{T}
   \label{form:5.87}
\end{equation}

SVD可以帮助我们证明PCA可以得到对角的$Var[z]$。对$X$做SVD,我们可以将$X$的方差写作:
\begin{align}
	Var[x] & = \frac{1}{m-1}X^{T}X\\
	&= \frac{1}{m-1}(U\Sigma W^T)^T U\Sigma W^T\\
	&=\frac{1}{m-1}W\Sigma ^T U^T U \Sigma W^T\\
	&=\frac{1}{m-1}W \Sigma ^2 W^T\\
\end{align}

上面的推导中我们用到了$U^T U = I$因为奇异值分解的矩阵$U$被定义为正交矩阵。下面证明如果我们取$z=x^T W$,可以确定$z$的协方差是对角矩阵:
\begin{align}
	Var[x] & = \frac{1}{m-1}Z^{T}Z\\
	&= \frac{1}{m-1}W^T X^T X W\\
	&=\frac{1}{m-1}W^T W \Sigma ^2 W W^T\\
	&=\frac{1}{m-1}\Sigma ^2\\
\end{align}
这次我们利用了$W^T W = I$,同样来自SVD的定义。

以上分析展示了当我们通过线性变换$W$将数据$x$投影到$z$的时候,所生成的表征拥有对角协方差矩阵($\Sigma ^2$),也就说明了$z$的每个元素都是相互独立的。

将数据变换成为元素相互独立的表征,这种能力是PCA的重要属性,也展现了表征的作用在于试图理清数据背后未知的变化因子。在PCA当中,这种理顺工作的形式是旋转输入空间使变化的主轴与新的表征空间$z$的基向量平行。

相关性(correlation)是数据元素之间依赖性(dependency)的重要组成部分,我们也同样对表征学习如何理顺更复杂的特征依赖感兴趣。为此,除了简单的线性变换,我们还有更多工作要做。

\subsection{k平均聚类}
\label{sec:5.8.2}

另一个简单的表征学习实例是k平均聚类(k-means clustering)。k平均算法将训练集分为k个组,每个组内部的样本相互都很接近。我们可以进一步认为该算法提供了一个k维的o独热码向量$h$来表征输入$x$。如果$x$属于一个类$i$,则$h_i=1$且$h$的其他元均为0。

k平均聚类提供的独热码正是稀疏表征的一个实例,因为每个输入的绝大多数元都是0。之后我们会接触到其他算法,它们能够学习更灵活的稀疏表征——每个输入不止有一个元为1。独热码是稀疏矩阵的极端情况,因为它丢失了分布表征的很多优点。但同时独热码也赋予了算法统计上的优势(天然地表达出同类样本相互接近)而且也带来了计算优势,整个表征可由一个整数代表。

k平均算法首先对不同的值生成k个不同的“质心”${\mu^{(1)},...,\mu^{(k)}}$,然后交替进行以下两个步骤直至收敛。步骤一:每个训练样本都被分配到某个组i当中,i就是与之最近的质心$\mu^{(i)}$的索引。步骤二:每个质心$\mu^{(i)}$更新为本组内所有训练样本$x^{(j)}$的平均值。

聚类问题的一大困难就是其天然具有病态性质,也就是并不存在一个绝对判据来衡量数据的聚类结果与真实世界多么符合。我们可以聚类的某些性质诸如分组质心到各成员的平均欧几里得距离。这可以表示我们能够在多大程度上从聚类重建原始训练数据,但是我们仍未知道聚类与真实世界的性质相去几何。

更有甚者,可能存在多种聚类结果都与真实世界的同一性质吻合良好,我们找到的可能是一个与某种性质相关、同样有效却与聚类任务本质无关的另一聚类。比如,假设我们正在一个由红色卡车、红色汽车、灰色卡车和灰色汽车组成的数据集上同时运行两个聚类算法。如果我们让两个算法各自进行二元聚类,一个算法可能按照外形分为汽车和卡车;另一算法可能按照颜色分为红色交通工具和灰色交通工具。如果再来第三个算法自动指定聚类数量,又可能分为四组——红色卡车、红色汽车、灰色卡车和灰色汽车。新算法至少同时捕捉到了两种性质的差异,但是却失去了相似性的信息。红色汽车与灰色汽车分属不同组别。聚类算法的输出并没有告诉我们,相比于灰色卡车,红色汽车与灰色汽车更接近,因为两者在颜色/外形上都不同,虽然我们都知道但算法却无法识别。

比起独热码,我们更倾向于分布表征,以上这一问题正表现出了部分原因。分布表征可以使每个交通工具具有两个性质——一个表征颜色一个表征外形(汽车/卡车)。这还未必就是最佳表征(学习算法怎么知道我们感兴趣的是颜色和外形,而非制造商和车龄呢?),但是已经减轻了算法猜测应该关照哪些属性的计算负担。并且由此我们将能够以高细粒度的方式来衡量物体之间的相似度——对比多种属性而非匹配单一属性。

\section{随机梯度下降法}
\label{sec:5.9}

\section{构建机器学习算法}
\label{sec:5.10}

\section{深度学习算法的动力}
\label{sec:5.11}

\part{深度学习:实战}
\label{part:2}

本部分将介绍可以用于解决实际问题的现代深度学习方法。


深度学习拥有着漫长历史和许多的应用,一些尝试至今已硕果累累。一些看起来野心十足的目标已经变为现实。深度学习中仍需探索的分支我们将留到最后一部分进行介绍。


本部分只介绍那些在工业中进行应用实践并获得成功的方法。


现代深度学习为有监督学习提供了一个十分强大的框架。通过增加层数或层间的单元数,网络可模拟更为复杂的函数。许多任务中都有把一个向量映射到另一个向量的工作,人类对这类工作翻译十分迅速,如果给出足够大的模型和大量的标签数据,深度学习也可以很好的完成这项工作。另外那些非向量映射可以描述的困难工作,甚至人类都需要时间思考和反应的,目前是不在深度学习讨论的范畴内。


本部分将讨论参数方程近似技术,这几乎要被所有的现代深度学习实践所使用。我们先介绍用于描述这些方程的前馈深度网络模型,然后我们会介绍正则化和优化等进一步的技术。缩放这些模型,使之适应于高分辨率的图像和长时间序列则需要进一步的讲述。我们将会介绍可用于缩放图像的卷积网络和处理时间序列的循环神经网络。最后,我们将总结一些可用于具体实践的指导,包括设计、构建、配置深度学习等,并对一些深度学习应用进行回顾。


本部分的章节对实践者来说最为重要,实践者们将开始使用深度学习技术解决真实世界中的问题。

\chapter{深度前馈网络}
\label{chap:6}
%%%%%%%%%%%%%%%%%%%%%%%%%%%%%%%%%%%%%%%%%%%%%%%%%%%%%%%%%
%%%%%%%%%%%%%%%%%% author:jim1949@163.com %%%%%%%%%%%%%%%
%%%%%%%%%%%%%%%%%% part:6.0-6.2           %%%%%%%%%%%%%%%
%%%%%%%%%%%%%%%%%%%%%%%%%%%%%%%%%%%%%%%%%%%%%%%%%%%%%%%%%

%需要加4张图,然后把6_0_1,2,3,4换成6_1,2,3,4%
\emph{正则度前馈网络}又被称为\emph{前馈神经网络}或\emph{多层感知机}(MLP), 是一种典型的深度学习模型。
一个前馈神经网络的目标是近似一个函数$f^*$。
例如,对于一个分类器而言,$y=f^*(\bm{x})$将x映射到类别$y$。一个前馈网络定义了一个映射$y=f(\bm{x};\bm{\theta})$, 并且学习了导致产生最好的近似函数的参数$\bm{\theta}$的值。

这些模型被叫做是\emph{前馈},因为信息通过从输入$\bm{x}$产生的方程,到用来定义f的中间计算,最后流向输出$\bm{y}$。这里没有那种模型的输出最后输入给自己的\emph{反馈}联系。当扩展了的反馈神经网络包含了反馈节时,他们被叫做\emph{前馈神经网络}(\emph{recurrent neural networks}),这会在第10章将会提到。

前馈网络对于机器学习的从业者及其重要。他们是形成很多重要商业应用的基础。例如,用于图像中的物体识别的卷积网络就是一种特别的反馈网络。前馈网络是在反复网络的道路上一个概念上的里程碑,它推动了很多自然语言的应用。

前馈神经网络被叫做/emph{网络}的原因是因为一般来说,它们被很多不同的方程组合所代表。这个模型与描述函数如何结合的直接循环图有关。例如,我们有3个函数$f^{(1)}$,$f^{(2)}$, 和$f^{(3)}$在链中相连,以形成$f^{(\bm{x})}=f^{(3)}(f^{(2)}(f^{(1)}(\bm{x})))$。
这些链式结构是最经常使用的神经网络结构。在这个例子里面,$f^{(1)}$被叫做神经网络的\emph{第一层},$f^{(2)}$被叫做\emph{第二层},以此类推。链条的总长给了模型的\emph{深度}。“深度学习”的名字就是从这个术语里面来的。前馈网络的最后一层叫做\emph{输出层}。在神经网络的训练中,我们训练$f^{(\bm{x})}$去匹配$f^*{(\bm{x})}$. 训练数据提供给我们有噪声的,估计的$f^*{(\bm{x})}$的例子,并且在不同的训练点被评估。每个例子$\bm{x}$都伴随一个标签$y\approx f^*(\bm{x})$。这些训练的例子直接给定了在每个点$x$输出层应该怎么做;它必须产生近似于y的值。其它层的表现没有被训练数据直接限定。学习算法必须决定如何水用那些层来得到理想的输出,但是训练数据没有说明每个独立的层应该做什么。相反,学习算法必须决定如何使用这些层来最好实现$f^*$的近似。因为训练数据没有展现每层的理想输出,这些层叫做\emph{隐藏层}(\emph{hidden layers})。

最后,这些网络被称作是\emph{神经}的原因是因为它们是基于神经科学基础的。网络的每个隐藏层一般是引入向量值的。那些隐藏层的维数决定了模型的\emph{宽度}。向量的每个元素可以解释成扮演一个类似于神经节的工作。不把层认为是简单的向量到向量的函数,我们也可以认为层由很多平行工作的\emph{单元}组成的,每个单元代表一个向量到标量的函数。每个单元像一个神经节一样,它从很多别的单元接受输入并且计算他自己的启动值。使用很多向量值代表层的想法来自于神经科学。用来计算这些代表的函数$f^(i)(\bm{x})$的选择,也被神经科学关于生物神经能计算的函数发现所引导。然而,现代的神经网络研究被很多数学和工程规则所指导,并且神经网络的目标也不是完美建模大脑。对于前馈网络的最好的认识是函数近似机器,它被设计成实现统计拟合,偶尔从我们对于大脑知道的部分获得一些灵感,而不是作为大脑函数模型。

一种理解前馈网络的方式,是从线性模型开始然后考虑怎样克服它们的缺陷。线性模型,例如逻辑回归和线性回归,就很有火因为它们可能会拟合地很有效和很可靠,要么在闭环方式要么是用凸优化的方式。线性模型也有一个明显的瑕疵:模型容量被线性函数所显,所以模型不能理解任意两个输入变量之间的交流。

为了将线性模型延伸到代表$\bm{x}$的非线性函数,我们可以应用线性模型不仅仅是$\bm{x}$自身,也可以是转换的输入$\phi(\bm{x})$,这里φ是非线性转换。相似地,我们可以应用5.7.2节描述的核心技巧,以得到基于含蓄应用$\phi(\bm{x})$映射的非线性学习算法。我们可以认为$\phi$作为提供一系列形容x的特征值,或者提供x$\bm{x}$的新代表。
问题是之后如何选择映射$\phi$。
\begin{enumerate}
\item 一种选择是使用一个非常通用的$\phi$,例如无限维的$\bm{\phi}$被隐含地用于基于$RBF$核的\emph{kernel machines}。如果$\phi(\bm{x})$有足够高的纬度,我们可以总是有足够的容量来拟合训练集,但是测试集的正则化经常保持不足。非常普通的特征映射,经常只是基于局部平滑,并且不对足够的先验信息编码解决高级的问题。
\item 还有一个方法是手工调制$\bm{\phi}$。这曾是主要的方法,直到深度学习的出现。这个方法要求对于不同的任务需要从业者人类的数十年的努力,需要参与者擅长不同领域例如语音识别或者机器视觉,并且在不同领域之间转换自如。
\item 深度学习的策略是学习$\phi$。在这种方法中,我们有模型$y=f(\bm{x};\bm{\theta}, \bm{\omega})=\phi(\bm{x}; \bm{\theta})^\top \bm{\omega}$。我们现在有参数$\bm{\theta}$去用来从一个很广义的函数类别里面去学习$\bm{\phi}$,和参数$\bm{\omega}$来从$\phi(\bm{x})$映射到期待的输出。这是一个深度前馈网络的例子,其中$\bm{\phi}$定义了隐含层。这个是唯有的三种方法之一,放弃了训练问题的凸性,但是好处更大。在这种方法中,我们把代表参数化为$\phi(\bm{x};\bm{\theta})$并且用优化算法来寻找$\bm{\theta}$以对应好的代表。如果我们希望,这个方法可以通过变得很通用来获得第一种方法的优势——我们通过使用一个很广泛集来做$\phi(\bm{x};\bm{\theta})$ 。这个方法也可以获得第二种方法的优势。人类从业人员可以将他们的知识编码来通过设计他们预计会表现良好的广泛集来帮助正则化。优势在于人类设计师只需要找出一般的合适函数集,而不用精确地找到正确的函数。
\end{enumerate}

这个通用的通过学习特征来改善模型的规则拓展了这章里面描述的前馈网络。这是个深度学习的复现的主题,适用于这本书里面描述的所有种类的模型。前馈网络是这种去学习决定性从缺乏反馈联系的$x$到$y$的映射的规则的应用。之后说的其它模型将会应用这些规则来学习随机映射,学习有反馈的函数,并且学习在一个向量上的概率分布。

我们从一个简单的前馈网络的例子来开始这章。然后,我们将会阐述每个设计决定来部署前馈网络。首先,训练前馈网络需要做出大量的和线性模型相同重要的设计决定:选择优化,成本函数,和输出形式。我们回顾这些基于梯度的学习的基础,然后进一步学习一些对于前馈网络特别的设计决定。前馈网络介绍了隐藏层的概念,并且这个要求我们去选择可以用来计算隐藏层值的\emph{激活函数}(\emph{activation functions})。我们也必须设计网络结构,包括网络里面有多少层,这些层应该怎么相互连接,以及每层里面有多少单位。学习深度神经网络需要计算复杂函数的梯度。我们提出\emph{反向传播}(\emph{back-propagation})算法和它的可以被用来有效计算梯度的现代正则化。最后,我们以一些历史层面来结尾。

\section{实例:学习XOR}
\label{sec:6.1}

为了让前馈网络的想法更加实际,我们先从一个例子开始,这个例子是用完全函数化的前馈网络来实现一个简单的任务:学习$\bf{XOR}$函数。

$\bf{XOR}$函数(“异或”)是两个二进制数$x_1$和$x_2$的运算。当正好这些二进制其中的一个数等于1,$\bf{XOR}$函数会返回1.否则,它会返回0。$\bf{XOR}$函数提供我们想要学习的目标函数$y=f^*(\bf{x})$。我们的模型提供函数$y=f(bf{x};bf{\theta})$并且我们学习的算法也会修改参数$\theta$,以使f尽可能接近$f^*$。

在这个简单的实例里面,我们不会担心数据正则化。我们想要我们的网络在四个点$\SetX=\{[0,0]^\top,[0,1]^\top,[1,0]^\top,[1,1]^\top\}$上正确地表现。我们将会在这四个点上面训练网络。唯一的挑战是拟合训练集。

我们可以把这个问题处理成回归问题,并且使用一个均方误差损失函数。我们用这个损失函数尽可能地来简化这个数学。我们将会在后面看的有对二进制数据建模的更合适的其它方法。

评估我们的全部训练集,$MSE$损失函数是
\begin{equation}
J(\theta)=\frac{1}{4} \sum_{\bf{x}\in \SetX} (f^*(\bf{x})-f(\bf{x};\bf{\theta}))^2
\end{equation}
现在我们必须选择我们的模型的形式,$f(\bf{x};\bf{\theta})$。假设我们选择一个线性模型,其中$\bf{\theta}$由$\bf{\omega}$和$b$组成。我们的模型定义成
\begin{equation}
J(\bf{x};\bf{\omega},b)=\bf{x}^\top \bf{\omega} + b 
\end{equation}

我们可以在封闭模式下最小化$J(\bf{\theta})$,对$\omega$和b使用正规方程。

在解完正常方程以后,我们获得$\bf{\omega}$=$\bf{0}$和$b=\frac{1}{2}$。每个地方线性模型都简单输出0.5。这个为什么发生呢?图 \ref{fig:6_1}展示了一个线性模型如何不能表示$\bf{XOR}$函数。一种方式去解决这个问题是使用一个模型,这个模型学习一个不同的特征空间,在里面一个线性模型能够体现解决方法。

特别地,我们将会介绍一个非常简单的前馈网络,其中有一个隐藏层包含两个隐藏单元。在图\ref{fig:6_2}中有这个模型的解释。这个前馈网络有一个隐藏单元$\bf{h}$,它是被函数$f^{(1)}(\bf{x};\bf{W},\bf{c})$。这些隐藏单元的值之后被用作第二层的输入。第二层是网络的输出层。输出层仍然只是一个线性回归模型,但是现在他将要应用到h上面而不是x。这个网络现在总共包含两个函数链:$\emph{h}=f^{(1)}(\bf{x};\bf{W},\bf{c})$和$y=f^{(2)}(\bf{h};\bf{\omega},b)$,其中有完整的模型是$f(\bf{x};\bf{W},\bf{c},\bf{\omega},b) = f^{(2)}(f^{(1)}(\bf{x}))$。

$f^{(1)}$应该计算什么样的函数呢?线性模型目前我们已经应用的很好了,并且可能最好也让$f^{(1)}$变成线性的。不幸的是如果$f^{(1)}$是线性的,那么前馈网络作为整体应该保持作为一个输入的线性函数。忽略掉目前的截距项,假设$f^{(1)}(\bf{x})=\bf{W}^\top \bf{x}$,并且$f^{(2)}(\bf{h})=\bf{h}^\top \bf{\omega}$。然后$f(\bf{\omega})=\bf{\omega}^\top \bf{W}^\top \bf{x}$。我们可以代表这个函数为$f(\bf{x})=\bf{x}^\top \bf{\omega’}$ 其中$\bf{\omega’}=\bf{W} \bf{\omega}$。


\begin{figure}[htbp] %  figure placement: here, top, bottom, or page
   \centering
   \includegraphics[width=6in]{fig/chap6/6_1.png} 
   \centering
   \caption{学习代表来解决$\bf{XOR}$问题。图像上打印的黑体字指出学习函数在每个点必须输出的值。(\emph{左边})一个线性模型直接应用到最开始的输入不能实现$\bf{XOR}$函数。当$x_1=0$,模型的输出必须随着$x_2$增长而增长。一个线性模型必须有适用于$x_2$一个固定的系数$w_2$。线性模型因此不能使用$x_1$的值来改变$x_2$上的系数,并且不能解决这个问题。(\emph{右边})在变换空间中,从神经网络中提取出由特征代表,一个线性模型现在可以解决这个问题。在我们的解决实例的方法中,必须有输出1的两点已经收缩成特征空间里面一个单独的点。换句话来说,非线性特征已经同时映射$\bf{x}=[0,1]^\top$和$\bf{x}=[0,1]^\top$到一个特征空间里面的一个单独的点,$\bf{h}=[1,0]^\top$。线性模型现在可以描述函数为在$h_1$中增长和在$h_2$中减少。在这个例子中,学习特征空间的动力只是为了使模型容量更大,这样它可以拟合训练集。在更现实的应用中,学习的代表也可以帮助模型正则化。}
   \label{fig:6_1}
\end{figure}


\begin{figure}[htbp] %  figure placement: here, top, bottom, or page
   \centering
   \includegraphics[width=6in]{fig/chap6/6_2.png} 
   \centering
   \caption{使用两种不同方式画出的,一个前馈网络的一个例子。特别地,这是我们用来解决$\bf{XOR}$例子的前馈网络。它有一个包含了两个单元的单隐藏层。(\emph{左边})用这种形式,我们在图中画出每个单元作为一个节点。这个形式是非常明确的和不含糊的,但是对于比这个例子大的网络,就会消耗太多的空间。(\emph{右边})在这个形式中,我们在图中画出一个节点对于每个整个的代表一层的激活部分的矢量。这个形式是更坚实的。有时候我们用描述两层关系的参数名称来注释图中的边缘。这里,我们指出矩阵$\bf{W}$描述着$\bf{x}$到$\bf{h}$的映射,然后向量$\bf{\omega}$描述着$\bf{h}$到$y$的映射。当给这种画图方法贴标签的时候,我们一般删除与各层相关的截距参数。}
   \label{fig:6_2}
\end{figure}

明显地,我们必须使用一个非线性的函数来表示特征。大多数神经网络通过已经学过的参数控制的仿射转换,以及叫作启动函数的一个固定的,非线性的函数做到这样。我们用这里的那个逻辑,定义$\bf{h}=g(\bf{W}^\top \bf{x}+\bf{c})$,这里$\bf{W}$提供了线性变换的权重以及$\bf{c}$作为偏置。之前,为了描述一个线性回归模型,我们用一个权重向量和一个标量偏移参数来描述一个从输入向量到一个输出向量的仿射变换。现在,我们描述一个从向量$\bf{x}$到向量$\bf{h}$的仿射变换,因此我们需要一整个向量的偏置参数。激活函数g一般被选作一个应用来元素加密的函数,有$h_i=g(\bf{x}^\top \bf{W}_{:,i}+c_i)$。在现代神经网络中,默认的推荐是使用在图\ref{fig:6_3}描述的由激活函数$g(z)=\max\{0,z\}$定义的\emph{rectified linear unit}或者ReLU(Jarrett et al., 2009; Nair and Hinton, 2010; Glorot et al., 2011a)。


我们现在可以明确完整的网络为
\begin{equation}
f(\bf{x};\bf{W},bf{c},bf{\omega},\bf{b})=\bf{\omega}^\top \max\{0,\bf{W}^\top \bf{x}+\bf{c}\}+b
\end{equation}

我们可以现在明确对于$\bf{XOR}$问题的解。设
\begin{align}
\bm{W} &= \begin{bmatrix}
1 & 1\\
1 & 1
\end{bmatrix},\\
\bm{c} &= \begin{bmatrix}
0\\
-1
\end{bmatrix},\\
\bm{w} &= \begin{bmatrix}
1\\
-2
\end{bmatrix},
\end{align}
并且b=0。

\begin{figure}[htbp] %  figure placement: here, top, bottom, or page
   \centering
   \includegraphics[width=6in]{fig/chap6/6_3.png} 
   \centering
   \caption{修正线性激活函数。这个激活函数是默认的推荐给大多数前馈神经网络使用的激活函数。应用这个函数到线性转换的输出,产生一个非线性转换。然而,这个函数仍然很接近于线性,也就是说他是有两段的分段线性函数。因为修复线性单元几乎是线性的,他们保护了很多的性能使线性模型正则好。一个通用的计算机的原则是我们可以从最小的部分里面构建复杂系统。就像图灵机的记忆一样,只需要能存储0或者1,我们可以从修正线性激活函数中,建造一个通用的函数近似器。}
   \label{fig:6_3}
\end{figure}

我们现在可以考虑让模型处理一批输入。让$\bf{X}$成为包含了二进制输入空间中所有4个点的设计矩阵,每个样例一行:
\begin{equation}
\bf{X}=\begin{bmatrix}
0 & 0\\
0 & 1\\
1 & 0\\
1 & 1
\end{bmatrix}
\end{equation}

在神经网络中第一步是将输入矩阵乘上第一层的权重矩阵:
\begin{equation}
\bf{X} \bf{W}=\begin{bmatrix}
0 & 0\\
1 & 1\\
1 & 1\\
2 & 2
\end{bmatrix}
\end{equation}

下一步,我们加上偏置向量$\bf{c}$,来获得
\begin{equation}
\begin{bmatrix}
0 & -1\\
1 & 0\\
1 & 0\\
2 & 1
\end{bmatrix}
\end{equation}

在这个空间里,所有的样例都在斜率为1的直线上。当我们在这条线上移动的时候,输出需要从0开始, 然后升到1,之后再降到0。一个线性模型不能实现这个函数。为了计算完每个样例的$\bf{h}$的值,我们应用修正线性变换:
\begin{equation}
\begin{bmatrix}
0 & 0\\
1 & 0\\
1 & 0\\
2 & 1
\end{bmatrix}
\end{equation}
这个变换已经改变了样例之间的关系。他们不再在单个线上了。就像图\ref{fig:6_1}所示,它们现在在一个空间中,这里线性模型可以解决问题。

我们通过乘上权重向量$\bf{w}$完成:

\begin{equation}
\begin{bmatrix}
0 \\
1 \\
1 \\
0
\end{bmatrix}
\end{equation}

神经网络已经从这批的每个样例中获得了正确的答案。

在这个样例中,我们简单地简化解,然后显示了它没有错误。在实际情况下,就会有很多的模型参数和很多训练样例,所以人们不能简单像我们这样猜到解。相反,一个基于梯度的优化算法可以找到会制造很少错误的参数。我们对于$\bf{XOR}$描述的解是在损失函数的一个全局最小值,所以梯度下降可以收敛到这个点。也有别的$\bf{XOR}$相同的解,梯度下降也可以找到。梯度下降的收敛点依赖于参数的起始值。在实际中,梯度下降一般不会找到像我们这里展示的一样的干净的,容易理解的整数解。

\section{基于梯度的学习}
\label{sec:6.2}

设计和训练一个神经网络和用梯度下降法训练任何其它的机器学习模型没有太大的区别。在\ref{sec:5.10}节里,我们描述了怎样去通过明确一个优化程序,一个代价函数和一个模型族来构建一个机器学习算法。

我们目前的线性模型和神经网络的最大的区别是神经网络的非线性造成大多数我们感兴趣的损失函数变成了非凸的。这意味着神经网络一般通过迭代的,基于梯度的仅仅驱使损失函数到一个很小的值的优化器来训练,而不是用来训练线性回归模型的线性方程解,或者用来训练逻辑回归或者SVMs的全局收敛保障的凸优化算法。凸优化收敛始于任何起始参数(理论上——实际上它有很强的健壮性但是会遇到数字问题)。随机的梯度下降应用到非凸损失函数没有收敛保障,并且是对于起始参数的值敏感的。对于前馈神经网络来说,初始化小的随机值的所有权重是重要的。偏置也许被起始化为0或者是小的正值。用来训练前馈网络和几乎所有其它的深度网络的迭代的基于梯度的优化算法将会在第八章被详细地描述,其中参数初始化将会被特别地在\ref{sec:8.4}节讨论。目前,不难理解训练算法几乎总是基于用梯度用各种方式来降低代价函数。这个特别的算法是梯度下降方法思想的提升和精炼,在\ref{sec:4.3}节有描述,并且,更具体的,通常是随机梯度下降算法的提高,在\ref{sec:5.9}节有介绍。

我们当然可以训练模型例如线性回归和有梯度下降的支持向量机,事实上当训练集极其大时这是很常见的。从这个观点来看,训练一个神经网络和训练任何其它的模型没有什么太大的区别。计算梯度对于神经网络来说有稍微有点复杂,但是仍然可以被有效和精确地完成。\ref{sec:6.5}节将会描述如何通过方向传播和方向传播算法的现代正则来得到下降。

正如其它机器学习模型一样,为了应用基于梯度的学习我们必须选择一个代价函数,并且我们必须选择如何代表模型的输出。我们现在重新回到这些在神经网络情景有特殊重点的设计考虑上来。

\subsection{代价函数}
\label{sec:6.2.1}

一个深度神经网络设计的重要方面是代价函数的选择。幸运的,神经网络的代价函数或多或少和其它参数模型,例如线性模型的代价函数一样。

在大多数情况,我们的参数模型定义了一个分布$p(\bf{y}|\bf{x};\bf{\theta})$并且我们简单使用最大可能性的原则。这意味着我们在训练数据和如代价函数一样的模型的预计之间使用交互熵。

有时候,我们使用更简单的方法,而不是预测一个完全在$\bf{y}$上的概率分布,我们仅仅预测一些条件为$\bf{x}$的$\bf{y}$的一些统计数字。特殊的代价函数允许我们去训练这些估计的预测值。

用来训练神经网络的整体的代价函数将会经常使这里描述的初始的代价函数和一个规则化因素结合。在第\ref{sec:5.2.2}节,我们已经见到了一些简单的用于线性模型的规则化的例子。用于线性模型的权值衰减方法也是直接适用在深度神经网络,并且是最受欢迎的正则化策略。更加高级的神经网络的正则化策略将在第7章中描述。

\subsubsection{使用最大似然法学习条件分布}
\label{sec:6.2.1.1}

大多数现代神经网络用最大似然来训练。这意味着代价函数是简单的负数对数似然,或者描述成训练数据和模型分布之间的交互熵。这个代价函数表示为

\begin{equation}
J(\bm{\theta})=-\SetE_{\mathbf{x}, \mathbf{y} \sim \hat{p}_\text{data}} \log p_\text{model} (\bm{y} \mid \bm{x})
\end{equation} 
代价函数的特殊形式依据$\log p_\text{model}$的特殊形式在模型之间中改变。上述方程的拓展通常产生不依靠模型参数的一些因子并且可能会被放弃。例如,正如我们在\ref{sec: 5.5.1}节看到的,如果$p_\text{model}(\bm{y}\mid\bm{x}) = \CalN(\bm{y};f(\bm{x};\bm{\theta}), \bm{I})$, 那么我们恢复均方误差代价,
\begin{equation}
J(\theta) = \frac{1}{2} \SetE_{\RVx, \RVy \sim  \hat{p}_\text{data}} || \bm{y} - f(\bm{x}; \bm{\theta}) ||^2 + \text{const},
\end{equation}

至少换算系数是$\frac{1}{2}$并且一个因子不依赖于$\bm{\theta}$。放弃的常数是记忆高斯分布的分歧,在这种情况下我们不选择参数化。之前,我们看到对输出分布的极大似然估计值和线性模型的均协方差最小值的等价性,但是事实上,等价性成立不考虑$f(\bm{x};\bm{\theta})$预测高斯的均值。

从来自最大可能性的代价函数的这种方法的一个优势是,它移去了为每个模型设计代价函数的负担。明确一个模型$p(\bm{y}\mid\bm{x})$自动决定一个代价函数$\log p(\bm{y}\mid\bm{x})$.
一个贯穿神经网络设计的反复出现的主题是代价函数的梯度必须是足够大的河可预测的,来作为一个对于学习算法的好的指导。饱和函数(变得非常平)低估了这个对象因为她们使得梯度变得非常小。在很多情况下它发生因为激活函数曾经产生隐藏单元的输出或者输出层会饱和。负的对数似然帮助我们在很多问题上避免这种问题。很多输出单元包括一个指数函数,当它的变量是绝对值很大的负值时,是会饱和的。在负的对数似然代价代价函数消除了一些输出单元的指数效果。我们仍然将讨论代价函数之间的交互以及在节\ref{sec: 6.2.2}中输出单元的选择。

用于展现最大似然估计值的交互熵代价的不寻常特性是,当它应用到实际中的模型中时,它通常没有最小值。对于离散输出值,大多数模型是用这种方式参数化地以至于它们不能代表概率0或者1,但是可以无限接近。逻辑回归是这个模型的一个例子。对于实值的输出变量,如果模型可以控制输出分布的密度(例如,通过学习高斯输出分布的反差参数)那么它可能会分配特别大密度到正确的训练集输出,导致互熵逼近负无穷。第7章中描述的正则化技巧提供了很多不同方法来修改学习问题,以至于模型不能在这种情况下获得无限制的反馈。

\subsubsection{学习有条件的统计量 }
\label{sec:6.2.1.2}

与学习一个完全的概率分布$p(\bm{y}\mid\bm{x};\bm{theta})$不同,我们经常想要只学习一个基于x的y的条件统计量。

例如,我们也许有一个预测器$f(\bm{x};\bm{\theta})$,我们想用它来预测$y$的平均值。

如果我们用一个足够有效的神经网络,我们可以把神经网络当作是能够代替任何来自一个广泛的函数集的函数$f$,这个类只被特征所限,例如延续性和无界性而不是有一个特别的参数形式。从这点来看,我们可以视代价函数为一个\emph{functional}而不是一个函数。一个\emph{functional}是一个从函数到实数的映射。我们因此可以认为学习作为选择一个函数而不是仅仅选择一些常识。我们可以设计我们的代价\emph{functional}来在一些我们期待的特定的函数得到最小值。例如,我们可以设计代价\emph{funcitonal},使它的最小值在一个函数上,这个函数映射$\bm{x}$到基于$\bm{x}$的$\bm{y}$的预测值。解决一个关于一个函数的优化问题需要一个数学工具叫作\emph{calculus of variations}(变分法),在第\ref{sec:19.4.2}节已经介绍。这不是很必要去为了理解\emph{calculus of variations}而去理解这章的内容。 目前,唯一必要的是理解\emph{calculus of variations}也许会被用来推导出下面的两个结果。
我们第一个用\emph{calculus of variations}推导的结果是解决优化问题
\begin{equation}
f^* = \underset{f}{\argmin}  \ \SetE_{\RVx, \RVy \sim  p_\text{data}} ||\bm{y}-f(\bm{x})||^2
\end{equation}
得到
\begin{equation}
f^*(\bm{x}) = \SetE_{\RVy\sim p_\text{data}(\bm{y}|\bm{x})} [\bm{y}],
\end{equation}
要求是这个函数在我们要优化的类里。换句话说,如果我们可以从产生分布的真实的数据中训练无穷多的样例,最小化平协方差代价函数产生函数,这个函数对于x的每个值预测y的平均值。
不同的代价函数产生不同的统计量。第二个使用\emph{calculus of variations}推导的结果是
\begin{equation}
f^* = \underset{f}{\argmin} \ \SetE_{\RVx, \RVy \sim  p_\text{data}} ||\bm{y} - f(\bm{x})||_1
\end{equation}
产生一个函数预测对于每个$\bm{x}$的$\bm{y}$的值的\emph{中位数},前提是这样一个函数也许会被我们优化过的函数族所描述。这个代价函数一般被称作\emph{mean absolute error}。

不幸的是,当和基于梯度的优化一起使用时,平均协方差误差和平均绝对误差通常导致差的结果。当和哲学代价函数结合的时候,一些饱和输出单元产生非常小的梯度。这是一个原因为什么互熵代价函数比平均协方差误差或者平均绝对误差更受欢迎,甚至当它没有必要去预测一个整个分布$p(\bm{y}\mid\b{x})$的时候也是如此。

\subsection{输出单元}
\label{sec:6.2.2}

代价函数的选择时和输出单元的选择紧密相关的。大多数情况下,我们简单地利用数据分布和模型分布之间的互熵。如何代表输出的选择决定了互熵函数的形式。

任何可能被用来作为输出的神经网络也可以被用做隐藏单元。这里,我们主要用这些单元作为模型的输出,但是原则上它们也可以在内部使用的。在第\ref{sec:6.3}节,我们将重新看一下这些单元,并且提供关于它们作为隐藏单元使用的额外细节。

通过这个部分,我们假设反馈网络提供一系列的隐藏特征,这些特征被$\bm{h}=f(\bm{x};\bm{\theta})$定义。输出层的角色就会是提供一些额外的变换,从特征到完成网络必须展现的任务。
 
\subsubsection{用于高斯输出分布的线性单元}
\label{sec:6.2.2.1}
一种简单的输出单元是一个基于一个没有非线性存在的仿射变换的输出单元。它们通常被称作线性单元。

给定特征$\bm{h}$,线性输出单元的一层产生一个向量$\hat{\bm{y}}=\bm{W}^\top \bm{h} + \bm{b}$。

线性输出层通常被用来产生线性高斯分布的平均值:
\begin{equation}
p(\bm{y}\mid\bm{x}) = \CalN(\bm{y}; \hat{\bm{y}}, \bm{I} ).
\end{equation} 
最大化对数似然此时等价于最小化均方误差。

最大可能框架使学习高斯的反差也很直接,或者使高斯的反差成为输入的一个函数。然而,协方差对于所有输入来说,必须被限制为正定矩阵。用一个线性输出层来满足这样的限制是困难的,所以一般其它输出层被用来对协方差参数化。对协方差建模的方法马上会在第\ref{sec:6.2.2.4}节描述。
因为线性单元没有饱和,它们对于基于梯度的优化算法没有任何困难,并且可能被用在大量的优化算法中。
 
\subsubsection{用于\emph{Bernoulli}输出分布的Sigmoid单元}
\label{sec:6.2.2.2}
很多任务要求预测一个二元变量y的值。具有两个类的分类问题可以归结为这种形式。

最大似然方法是定义一个基于条件$\bm{x}$上$\bm{y}$的\emph{Bernoulli}分布。

一个\emph{Bernoulli}分布只被一个单个参数定义。神经网络需要只预测$P(y=1 \mid \bm{x})$。为了使这个数字成为一个有效的概率,它必须在区间$[0,1]$里面。

满足这个限制要求一些细致的设计工作。假设我们准备用一个线性单元,并且通过阈值来限定来获得有效的概率:
\begin{equation}
P(y=1 \mid \bm{x}) = \max \left \{ 0, \min \{1, \bm{w}^\top \bm{h}+b \} \right \}.
\end{equation}

这确实定义了一个有效的有条件分布,但是我们不能用梯度下降来有效地训练它。任何时候当$\bm{w}^\top \bm{h}+b$在单元区间以外的时候,与它的参数有关的模型的梯度输出会置$\bm{0}$。$\bm{0}$的梯度一般会有问题的,因为学习算法不再有如何提高相应参数的指导。

确实,最好用别的方法来确保无论模型有错误的答案,总有很大的梯度。这个方法是基于使用sigmoid输出单元与最大似然的结合。

一个sigmoid输出单元被定义为
\begin{equation}
\hat{y} = \sigma \left (\bm{w}^\top \bm{h} + b \right ),
\end{equation}
这里$\sigma$是在第\ref{sec:3.10}节描述的逻辑sigmoid函数。
我们可以认为sigmoid输出单元是有两个部分。首先,它使用一个线性层来计算$z=\bm{w}^\top \bm{h}+b$。然后,它使用sigmoid激活函数把$z$转化到概率中。
我们暂时忽略对于$\bm{x}$的依赖性,只讨论如何使用$z$的值来定义一个$y$上的概率分布。Sigmoid可以通过建造一个非规格化概率分布$\tilde{P}(y)$来被驱动,其中概率的和没有到1。我们之后可以用一个合适的常数来分割来得到一个有效的概率分布。如果我们从一个假设非正则化对数概率在$y$和$z$中是线性的,我们可以取幂来获得那个非正则化的概率。我们之后可以正则发现这产生一个由$z$的\emph{sigmoidal}转换控制的\emph{Bernoulli}分布:
\begin{align}
\log \tilde{P}(y) &= yz,\\
\tilde{P}(y) &= \exp(yz),\\
P(y) &= \frac{\exp(yz)}{\sum_{y' = 0}^1 \exp(y' z)},\\
P(y) &= \sigma((2y-1)z).
\end{align}

基于指数的概率分布和正则化在统计模型文献中是很常见的。定义这样一个二元变量上的分布的变量z被称作\emph{logit}。

这种在对数空间中预测概率的方法可以自然地用最大似然方法学习。因为使用最大似然的代价函数是$-\log P(y\mid\bm{x})$, 在代价函数里面的对数部分抵消了$\emph{sigmoid}$的指数部分。如果没有这种效果,$\emph{sigmoid}$的饱和度可以阻止基于梯度的学习做出更好的改进。一个$\emph{Bernoulli}$的最大似然学习的代价函数被$\emph{sigmoid}$参数化后是
\begin{align}
J(\bm{\theta}) &= -\log P(y\mid\bm{x})\\
&= -\log \sigma ((2y-1)z)\\
&= \zeta((1-2y)z).
\end{align}

推导使用了一些来从节\ref{sec:3.10}的性能。通过重新书写softplus函数的损失,我们可以看出只有当$(1 − 2y)z$取绝对值很大的负值时,它才饱和。只在模型已经有正确答案——当$y=1$或者$z$取绝对值很大的正值,或者$y=0$并且$z$取绝对值很大的负值时才发生饱和。当z有错误的符号,softplus函数的参数,$(1-2y)z$,可以简化为$|z|$。当z有错误的符号,随着$|z|$变大,softplus函数渐进趋向于返回它的参数$|z|$。关于z的导数渐进趋向于$\text{sign}(z)$,所以,对于特别不正确的$z$的限制中,softplus函数一点也没有缩小梯度。这个特性很有用,因为它意味着基于梯度的学习可以很快纠正一个错误的$z$。

当我们使用其它代价函数,例如均方差误差,任何时候$\sigma(z)$饱和的时候,损失可以饱和。当$z$变成绝对值很大的负数,sigmoid激活函数饱和到0,当z变成绝对值很大的正值时,饱和到1。无论什么时候这个发生,梯度可以为了对学习有用而变的很小,不论这个模型有正确的答案或者没有。为了这个原因,最大似然几乎总是来训练sigmoid输出单元的优先的方法。

理论上,sigmoid的对数总是确定的和有限的,因为sigmoid返回值在开区间$(0,1)$之间,而不是使用整个的有效概率$[0,1]$的闭区间。在软件应用中,未来避免数字的问题,最好写下负值的对数似然作为一个z的函数,而不是作为一个函数$\hat{y}=\sigma(z)$。如果sigmoid函数下溢到零,之后对$y$取对数会产生负无穷。

\subsubsection{用于Multinoulli输出分布的softmax单元}
\label{sec:6.2.2.3}
任何时候我们想要表示一个有n个可能值的离散变量的概率分布时,我们可能使用softmax函数。这可能被视为一个被用作表示一个用于二位变量的分布的simoid函数的扩展。

Softmax函数大多数情况下,被用于作为分类器的输出,用来代表在n个不同类上的概率分布。softmax函数比较少地,可以被用在模型自身里面,如果我们希望模型在用于一些内部变量中的n个不同选项中选择。
在二元变量中,我们希望产生一个单独的数

\begin{equation}
\hat{y} = P(y=1\mid\bm{x}).
\end{equation}

因为这个数需要处在0和1之间,并且因为我们希望这个数的对数在用于对数似然的基于梯度的优化时表现良好,我们选择去预测一个数$z=\log \hat{P}(y=1\mid\bm{x})$。对其指数化和归一化给我们一个由$sigmoid$函数控制的$Bernoulli$分布。

为了推广到一个有n个值的离散变量的情况,我们现在需要产生一个向量yˆ,其中yˆ = P(y = i | x)。我们要求不仅yˆ的每个元素处在0与1之间,而且整个向量加在一起为1以至于它代表一个有效概率分布。用于bernoulli分布的同样的方法拓展到multinoulli分布。首先,一个线性层预测非标准化对数概率:
\begin{equation}
\bm{z} = \bm{W}^\top \bm{h}+\bm{b},
\end{equation}
其中$z_i=\log \hat{P}(y=i\mid\bm{x})$。softmax函数可以之后通过对z指数化和归一化来获得预期的y。最终,softmax函数的形式是
\begin{equation}
\text{softmax}(\bm{z})_i = \frac{\exp(z_i)}{\sum_j \exp(z_j)}.
\end{equation}

和\emph{logistic sigmoid}一样,在用最大的对数似然来训练\emph{softmax}来输出一个目标值y时,指数函数表现很好。在这个样例中,我们希望最大化$\log P(\RSy =i; \bm{z})=\log \text{softmax}(\bm{z})_i$。定义关于指数的softmax是自然的因为对数似然里的对数可以抵消$softmax$的指数部分:
\begin{equation}
\log \text{softmax}(\bm{z})_i = z_i - \log \sum_j \exp(z_j).
\label{eq:6.30}
\end{equation}
公式\ref{eq:6.30}里的第一个项显示里输入$z_i$总有对于代价函数的一个直接贡献。因为这个项不饱和,我们知道学习可以处理,即使$z_i$对于第二个项的贡献变得很小。当最大化对数似然,第一项鼓励$z_i$升高,而第二项鼓励所有的$z$压低。为了对第二项有一些直接的理解,$\log\sum_j \exp(z_j)$,观察这项可以粗略地被$\max_j z_j$估计。这种近似是对任何明显少于$\max_j z_j$,$z_k$都是不重要的。我们可以从这个近似中得到的解释是负的对数似然代价函数总是强烈惩罚最积极的不正确的预测。如果正确的答案已经对于softmax有最大的输入时,那么-zi项和$\log\sum_j \exp(z_j) \approx \max_j z_j = z_i$项将大致打消。这个例子之后将要对所有的训练代价贡献很小,它将由其它还没有正确分类的样本决定。

到目前为止,我们只讨论了一个单独的例子。综上,非正规化的最大似然将会驱动模型去学习参数,这些参数会驱动softmax来预测在训练集中观察的每个结果的比率:
\begin{equation}
\text{softmax}(\bm{z}(\bm{x}; \bm{\theta}))_i \approx \frac{\sum_{j=1}^m \bm{1}_{y^{(j)}=i, \bm{x}^{(j)} = \bm{x}}  }{ \sum_{j=1}^{m} \bm{1}_{\bm{x}^{(j)} = \bm{x}} }.
\end{equation}
因为最大似然是一个连续的估计器,这就保证了只要模型簇能表示训练分布,它就一定发生。在实际中,有限的模型容量和不完美的优化将意味着模型只能近似得到这些比率。

很多客观的函数而不是对数似然用softmax函数工作地并不好。具体来说,当指数部分变得绝对值很大的负值时,不用对数来抵消softmax的指数部分的客观的函数不能学习,导致梯度消失。特别地,方差是一个用于softmax单元的损失函数的很差的损失函数,并且甚至当模型能做出高度可惜的不正确的预测时,会失败地训练模型来改变它的输出。(Bridle,1990)。为了理解为什么这些其它的损失函数会失败,我们需要测试softmax函数本身。

像sigmoid,softmax激活函数会饱和。当它的输入是绝对值很大的正值或负值,Sigmoid函数有一个会饱和的单个输出。对于softmax的情况,会有很多输出值。当输入值之间的差别变得很极端时,这些输出值会饱和。当softmax饱和时,很多基于softmax的代价函数也饱和,除非它们能颠倒饱和激活函数。

为了说明softmax函数对于它的输入之间的差异作出响应,观察到softmax输出对于所有输入加上相同的标量值时是不变的:
\begin{equation}
\text{softmax}(\bm{z}) = \text{softmax}(\bm{z}+c).
\end{equation}
使用这个性质,我们可以导出一个softmax的数值上稳定的变量:
\begin{equation}
\text{softmax}(\bm{z}) = \text{softmax}(\bm{z}- \max_i z_i).
\end{equation}
变换后的形式使即使当z包含特别大的或者绝对值极其大的负值时,我们可以只用小的数值上的误差来评估softmax。测试数值稳定的变量,我们发现softmax函数由来自变量偏移$\max_i z_i$的数量驱动。

当相应的输入时最大的($z_i = \max_i z_i$)并且$z_i$比其它所有的输入都大很多时,一个输出$\text{softmax}(\bm{z})_i$饱和到1。当$z_i$不是最大输出并且最大值大很多时,$\text{softmax}(\bm{z})_i$也可以饱和到0。这是Sigmoid单元饱和方式的一般化,并且如果损失函数没有被设计去补偿的话,会导致相似的学习困难。

Softmax函数的$\bm{z}$变量可以被用作两种不同方式来产生。最普通的事简单地让神经网络的一个较早的层输出$\bm{z}$的每个元素,正如上述描述一样使用线性层$\bm{z}={W}^\top\bm{h}+\bm{b}$。虽然很直观,这个方法实际上对于分布过度参数化。$n$个输出的常数必须加一起等于1,意味着只有$n-1$参数是必要的;第$n$个值的概率可能通过从1中减去前$n-1$概率项来得到。我们可以因此我们可以强制要求$\bm{z}$的一个元素是固定的。例如,我们可以要求$z_n=0$。确实,这正是sigmoid单元做的。定义$P(y=1\mid\bm{x})=\sigma(z)$和用一个二维的$\bm{z}$和$z_1=0$定义$P(y=1\mid\bm{x})=\text{softmax}(\bm{z})_1$是等价的。无论是softmax的$n-1$个变量还是$n$个变量的方法可以描述同一对概率分布,但是会有不同的学习机制。在实际中,使用过参数化的版本或者受限制的版本直接没有太大的区别,并且去应用过参数化的版本是更容易的。

从神经科学的观点来看,把softmax看成是创造一种参与其中的单元之间的竞争的形式:softmax输出总是加在一起等于1,所以一个单元中值的增长必然和其它值的降低有关。这被认为皮质里的临近神经元之间存在的侧抑制是类似的。在极端情况下(当最大的$a_i$和其它之间的幅度差异是大的时)它变成了一种\emph{winner-take-all}的形式(其中之一的输出几乎为1,而其它几乎为0)。

“Softmax”的名字会有一点让人产生困扰。函数是比max函数更接近于argmax函数。术语“Soft”来源于softmax函数是连续的和可微的。argmax函数,结果表示为一个\emph{one-hot}向量,是不连续,不可微的。Softmax函数因此提供一个argmax的“softened”版本。相应的最大函数的软化版本是$softmax(\bm{z})^\top \bm{z}$。这可能叫softmax函数“$softargmax$”更好,但是目前的名字是一个确立的习惯。

\subsubsection{其它输出形式}
\label{sec:6.2.2.4}
上述描述的线性的,sigmoid,和softmax输出单元是最常见的。神经网络可以拓展到几乎任何我们希望的输出层中。最大似然的原则为如何为几乎任何输出层设计一个好的代价函数提供了指导。

一般而言,如果我们定义了一个条件分布$p(\bm{y}\mid\bm{x}; \bm{\theta})$,最大似然的原则建议我们使用$-\log p(\bm{y}\mid \bm{x};\bm{\theta})$作为我们的代价函数。

一般而言,我们可以把神经网络看作是代表一个函数$f(\bm{x};\bm{\theta})$。这些函数的输出不是值$y$的直接的预测。相反,$f(\bm{x}:\bm{\theta})=\bm{\omega}$为提供了$y$上的分布参数。我们的损失函数之后可以被解释为$-\log p(\RVy; \bm{\omega}(\bm{x}))$。

例如,我们也许希望学习一个基于$\RVx$的$\RVy$的条件高斯方差。在这个简单的情况,方差$\sigma^2$是恒定不变的,这里又一个闭合形式的表达因为最大似然方差的估计器简单地是观察值$\RVy$和它们的预测值直接的方差的经验平均值。一个不需要要求写特殊情况的代码的计算花费更多的方法,是简单地包括方差作为分布$p(\RVy\mid\bm{x})$的其中一个特性,这个分布被$\bm{\omega}=f(\bm{x};\bm{\theta})$所控制。负对数似然$-\log p(\bm{y};\bm{\omega}(\bm{x}))$之后将提供一个有一个特殊项的损失函数来使我们的最优化过程渐进地学习方差。在标准差不依赖于输入的简单轻快中我们可以在网络中产生一个简单的产生一个新的参数,这个参数直接被复制到$\bm{\omega}$中。这个简单的参数也许是$\sigma$本身,或者可以是代表$\sigma^2$的参数$v$,或者它可以是一个代表$\frac{1}{\sigma^2}$的参数$\beta$,取决于我们如何选择对分布参数化。我们也许希望我们的模型去预测对于$\RVx$的不同的值$\RVy$之中的一个不同数量的方差。这个被叫做heteroscedastic模型。在heteroscedastic情况中,我们简单地让方差的具体值成为$f(\RVx;\bm{theta})$输出的其中一个值。一个典型的方式去做这个是用精度,而不是方差,正如公式\ref{sec:3.2.2}描述的那样,来构建高斯分布。在多维变量的情况中,使用哦一个对角精确矩阵是最常见的。
\begin{equation}
\text{diag}(\bm{\beta}).
\end{equation}
这个公式适用于梯度下降,因为由$\bm{\beta}$参数化的高斯分布的对数似然的公式只涉及$\beta_i$的乘法和$\log \beta_i$的加法。乘法,加法和对数运算的梯度都很好的表现。通过对比,如果我们把输出以方差形式参数化,我们将需要使用除法。除法函数在零附近变得任意陡峭。尽管大的梯度可以帮助学习,任意大的梯度通常导致不稳定。如果我们用标准差把输出参数化,对数似然仍然将会涉及除法,并且也会涉及平方。通过方差运算的梯度会在零附近消失,使学习平方的参数变得困难。
不管我们是否使用标准差,方差,或者精度,我们必须确认高斯的协方差矩阵是正定的。因为精度矩阵的特征值是协方差矩阵的特征值的倒数,这和确定精度矩阵是正定的是等价的。如果我们使用一个对角矩阵,或者一个标量乘以对角矩阵,那么我们唯一需要的条件是去强制使模型的输出都为正。如果我们假设a是被用来决定对角精度的模型的原始激活,我们可以用softplus函数来获得一个正的精度向量:$\bm{\beta}=\zeta(\bm{a})$。如果用方差或者标准差而不是精度或者如果使用一个标量乘以单位矩阵而不是对角矩阵,相同的策略同样适用。

学习一个协方差或者比对角矩阵有更丰富结构的精度矩阵是很少见的。如果协方差矩阵是满的和有条件的,那么参数化的选择保证预测方差矩阵的正定。这可以通过写成$\bm{\Sigma}(\bm{x})=\bm{B}(\bm{x})\bm{B}^\top (\bm{x})$来实现,这里$\bm{B}$是一个无约束的方阵。如果矩阵是满秩的,这个实际问题是计算可能性的代价是昂贵的,一个$d\times d$的矩阵的行列式或者$\bm{\Sigma}(\bm{x})$的逆(或者等价地,更常用地,对它的特征值的分解或者$\bm{B}(x)$的特征值的分解),要求$O(d^3)$的计算量。

我们通常想要实现\emph{multimodal regression},也就是,去预测来源于条件分布$p(\bm{y}\mid\bm{x})$的实值,这个分布对于相同的$\bm{x}$值在$\bm{y}$空间中有不同的峰值。在这种情况下,一个高斯混合是输出的自然表示(Jacobs et al., 1991; Bishop, 1994)。用高斯混合作为它们的输出的神经网络通常被叫做\emph{mixture density networks}(\emph{混合密度网络})。具有$n$的部分的高斯混合输出被条件概率分布定义为
\begin{equation}
p(\bm{y}\mid\bm{x}) = \sum_{i=1}^n p(\RSc = i \mid \bm{x}) \CalN(\bm{y}; \bm{\mu}^{(i)}(\bm{x}), \bm{\Sigma}^{(i)}(\bm{x})).
\end{equation}
神经网络必须有三个输出:一个定义$p(c=i\mid\bm{x})$的向量,一个给所有的$i$提供$\bm{\mu}^{(i)}(\bm{x})$的矩阵,和一个给所有i提供$\bm{\Sigma}^{(i)}(\bm{x})$的张量。这些输出必须满足不同的限制:
\begin{enumerate}
\item 混合部分$p(\RSc=i\mid\bm{x})$:这些形成一个关于隐变量\footnote{我们认为c是隐变量是因为我们没有在数据中观察到它:已知输入$\bm{x}$和目标$\bm{y}$,确切地知道哪个高斯部分产生了$\bm{y}$是不可能的,但是我们可以想象$\bm{y}$是通过挑选其中之一来产生的,并且使那个没有观察到的选择作为随机变量。
}$c$的n个不同部分的\emph{multinoulli distribution},并且通常可以通过一个n维的向量上的softmax来获得,来保证这些输出是正并且和为1。

\item 均值$\mu^(i)(\bm{x})$:这些指明了与第i个高斯部分相连的中心或者均值,并且是无限制的(一般对于这些输出单元根本没有非线性)。如果$\bm{y}$是一个d维向量,那么网络必须输出一个$n\times d$的矩阵,这个矩阵包涵这些n个这种d维向量。用最大似然学习这些均值比只用一个输出模式学习分布的均值复杂一点。我们只想去更新用于实际上产生观测值的部分的均值。在实际情况中,我们不知道每个观测值是哪个部分产生的。负对数似然的表达式自然地对每个样例对于每个部分的概率损失产生的贡献赋予权重,权重值大小由组成部分产生的案例的概率来决定。

\item 协方差$\bm{\Sigma}^{(i)}(\bm{x})$:这些明确了用于每个部分i的方差矩阵。当学习一个单个的高斯部分的时候,我们一般用一个对角矩阵来避免计算行列式。当学习混合的均值,最大似然是复杂的,它需要将每个点的部分责任分配到每个混合部分中。给定了在混合模型下,正确的负对数似然,梯度下降将会自动按照正确的过程。
\end{enumerate}

有报道说有条件高斯混合的基于梯度的诱惑可能是不可靠的,部分是因为设计除法,(除以方差)这可能会数据上的不稳定(当一些方差对于一个特殊的样例变得小的时候,产生很大的梯度)。一个解决方案是clip gradients(在第10.11.1节可以看到),另外一种是梯度启发式收缩。(Murray and Larochelle, 2014)。

高斯混合输出是语音生成模型(Schuster, 1999)中,或者物理物体运动(Graves, 2013)。混合密度策略为网络提供了方式去代表很多输出模型,并且去控制它的输出的方差,这对于获得一个在这些实值区间里的高质量的结果极其重要。一个混合密度网络的样例在图\ref{fig:6_4}中显示。

一般来说,我们可能希望继续去为包含更多变量的更大的向量$\bm{y}$建模,并且去在这些输出变量上施加更多更丰富的结构。例如,我们可能希望对于我们的神经网络,输出组成句子的字符序列。在这些情况中,我们可能继续使用最大似然的原则应用到我们的模型$p(\bm{y};\bm{\omega(\bm{x})})$,但是我们形容$\bm{y}$模型变得太复杂,超过了本章的范围。第10章描述里如何使用循环神经网络来定义序列上的这些模型,并且第III部分描述里对于任意概率分布的建模的高级技巧。

\begin{figure}[htbp] %  figure placement: here, top, bottom, or page
   \centering
   \includegraphics[width=6in]{fig/chap6/6_4.png} 
   \centering
   \caption{样例是用有一个混合密度输出层的神经网络画得。输入x从一个均匀分布中采样,输出y从$p_{\text{model}}(y \mid x)$中采样。神经网络能够学习从输入到输出分布的参数的非线性映射。这些参数涉及了概率,这个概率决定了三个混合部分中的哪个将产生输出和每个混合部分的参数。每个混合部分是有预测的均值和方差的高斯分布。输出分布的所有这些部分能依据输入$x$而变化,并且也在非线性方式中这样做。}
   \label{fig:6_4}
\end{figure}






%%%%%%%%%%%%%%%%%%%%%%%%%%%%%%%%%%%%%%%%%%%%%%%%%%%%%%%%%
%%%%%%%%%%%%%%%%%%% author:liviclee %%%%%%%%%%%%%%%%%%%%%
%%%%%%%%%%%%%%%%%%% part:6.3-6.7    %%%%%%%%%%%%%%%%%%%%%
%%%%%%%%%%%%%%%%%%%%%%%%%%%%%%%%%%%%%%%%%%%%%%%%%%%%%%%%%

\section{隐藏单元}
\label{sec:6.3}

\section{架构设计}
\label{sec:6.4}

\section{反向传播及其他微分算法}
\label{sec:6.5}

\section{前馈神经网络的发展史}
\label{sec:6.6}

\chapter{深度学习的正则化}
\label{chap:7}
%%%%%%%%%%%%%%%%%%%%%%%%%%%%%%%%%%%%%%%%%%%%%%%%%%%%%%%%%
%%%%%%%%%%%%%%%%%% author:ysh329 %%%%%%%%%%%%%%%%%%%%%%%%
%%%%%%%%%%%%%%%%%%%%%%%%%%%%%%%%%%%%%%%%%%%%%%%%%%%%%%%%%
机器学习的一个核心问题是如何使算法在除训练集以外的新输入数据上的性能表现更好。机器学习的不少学习策略都被用来去减少测试误差,有的策略会以牺牲训练误差为代价减少测试误差。这些学习策略都统称为正则化方法。对于深度学习从业者来说,有很多正则化的方法可以使用。实际上在深度学习领域,设计并开发更有效的正则化方法已成为一个主要的研究方向。

本书第五章介绍了泛化,欠拟合,过拟合,偏差,方差以及正则化的基本概念。在继续本章的内容之前,如果您还对这些概念不了解,可以参考第五章的具体内容。

本章节将会详细地介绍正则化方法,关注正则化方法在深度模型以及为深度模型搭建提供子模块模型上的应用。

本章的部分小节将会提到机器学习中的标准概念。如果您熟悉这些概念,可以跳过与之相关的小节。但是本章的大部分内容都是这些基本概念的延伸或扩展,尤其是有关神经网络的相关案例。

在5.2.2小节给出正则化的定义:“用于修正学习算法减少泛化误差而不是训练误差的策略”。

其中讲到一些正则化方法,某些正则化是基于机器学习模型加入额外限制实现,例如给参数加上限制。某些方法则是在目标函数中加入多余的项,多余的项可以视为对应参数的软性限制。如果正则化方法选择合理,额外的限制和惩罚可以提升模型在测试集上的性能表现。一方面,这些额外的限制和惩罚是作为编码先验知识的一种形式;另一方面来说,这些额外的限制和惩罚被设计出来用于提升模型的泛化能力。有时因为惩罚和限制的存在,可以使原本不确定的问题变得确定。此外,还有其它形式的正则化方法,例如众所周知的集成方法,基于训练数据将多个假设合成一个模型。

在深度学习的背景下,大多数正则化策略都是基于正则化估计。正则化估计通过平衡偏差和方差来实现其作用,如减少方差增大偏差。一个有效的正则可以在显著减少方差的同时不会过分增大偏差。在第五章中讨论泛化和过拟合时,我们主要关注模型族训练过程中的三个情形:(1)排除真实数据产生过程中对应的欠拟合以及偏差增大,(2)真实数据的产生过程,(3)数据生成过程中也可能伴随其它数据的生成过程——导致模型步入过拟合阶段,其中相比偏差,方差是造成了估计误差的主要原因。正则化的目标好比是让模型从第三阶段到第二阶段。

在实际过程中,一个过度复杂的模型族并不需要包括目标函数或者真实数据的产生过程,或与之近似的过程。我们基本上从来都不知道真正数据的产生过程是怎么的,所以我们也就无法确定模型族里是否包含数据的真实生成过程。但是深度学习算法特别地被应用在复杂的领域,如图像、语音序列、文本,因为这些数据的真实产生过程好比模拟整个宇宙。从某种程度上来说,我们总在试图用一些有棱角的零件(即真实数据的产生过程)塞到圆孔(即模型族)中。

控制模型的复杂度不仅意味着发现模型真正的规模大小,还包括参数的规模大小。此外,我们有时也会发现在实际深度学习的应用场景中,(在最小化泛化误差的意义上)拟合最好的模型的特点,不仅是一个大模型,而且也使用了适当的正则化方法。

现在让我们回顾一下如何创建大规模、深层且使用了正则化的模型。

\section{参数范数惩罚}
在深度学习出现之前,正则化已被使用了十几年。线性模型中的线性回归模型、逻辑斯特回归模型都可以使用简单、直接且有效的正则化方法。

许多正则化方法都是基于有限的模型复杂度,如神经网络模型、线性回归模型、逻辑斯特模型,它们都是通过在目标函数$J$中增加一个参数范数惩罚项$\Omega (\theta)$,我们使用$\widetilde{J}$来表示加入正则化的目标函数:

$$
\begin{aligned}
	\widetilde{J} (\theta; X, y) = J(\theta; X, y) + \alpha \Omega (\theta)
\end{aligned}
$$

其中,$\alpha \in [0, \infty)$是一个超参数,用来平衡范数惩罚项$\Omega$的贡献度,也与标准的目标函数$J$有关。若将$\alpha$设置为$0$,那么不存在正则化项。更大的$\alpha$值对应更大的正则力度。

当训练算法正在对带正则化的目标函数$\widetilde{J}$求最小化时,基于训练数据集的原始目标函数$J$和参数$\theta$(或参数的子集)的大小的测量都将会减小。对于参数范数$\Omega$的不同选择将会导致不同的结果作为优选。在本节,我们将会讨论在模型参数上不同范数作为惩罚项的效果。

在深入研究不同范数作为正则的效果前,注意到对于神经网络通常会选择使用一个参数范数惩罚$\Omega$,它只会惩罚每层仿射变换的权重,偏置单元是不会被正则化的。偏置单元在拟合时所需要的数据量要小于权重。

每个权重明确表明两个变量之间是如何相互作用的。要将权重拟合地很好,需要在各种不同的条件下观察这些变量。每一个偏置只会控制一个单独的变量,这也就意味着在保留不被正则化的偏置时,不需要引入过多的方差。同样,对偏置参数进行正则化会引入相当程度的欠拟合可能。因此我们使用向量$w$来表明所有的权重,这些权重都会被范数惩罚所影响。其中,向量$\theta$表示所有的参数,既包括了所有的$w$以及没有被正则化的参数。

在神经网络的背景下,对于神经网络的每一层来说。因为在很多超参数的情况下搜索到正确的超参数值,代价很高。所以,在所有层中使用相同的权重衰减策略,用以减少搜索空间的规模,是情有可原的。

\subsection{$L^2$参数正则化}

在5.5.5小节中,我们已经看到了一种简单且最常见的参数范数惩罚法:$L^2$参数范数惩罚法,也是众所周知的权重衰减(weight decay)。该正则化策略通过在目标函数中加入一个正则化项$\Omega(\theta) = \frac{1}{2} ||w||_2^2$可以使权重越来越接近原本的值(一般来说,我们可以将参数正则化到空间中任何特定值的附近,令人惊讶的是,在收到正则化的作用后,可能会得到一个比之前更好的更接近真实值的结果。零是一个默认值是有意义的,当不知道正确值是正还是负时,零都会作为一个有意义的默认值。因为将模型参数正则化趋向于零是很常见的,我们也将重点关注这种特殊情况)。

在其它一些学术界,$L^2$被称为岭回归(ridge regression)或吉洪诺夫正则化(Tikhonov regularization)。

我们深入了解权重衰减的正则化方法,它的实现是通过学习正则化过的目标函数的梯度。为了简化说明,假设不存在偏置参数,所以$\theta$就相当于$w$,该模型有如下的目标函数:
$$
\begin{aligned}
	\widetilde{J} (w; X, y) = \frac{\alpha}{2} w^T w + J(w; X, y),
\end{aligned}
$$
其对应的参数梯度为:
$$
\begin{aligned}
	\triangledown_w \widetilde{J}(w;X,y) = \alpha w + \triangledown_w J(w;X,y)
\end{aligned}
$$
下面是单步梯度的权重更新:
$$
\begin{aligned}
	w \leftarrow w - \epsilon (\alpha w + \triangledown_w J(w; X, y)).
\end{aligned}
$$
也可以换个写法:
$$
\begin{aligned}
	w \leftarrow (1 - \epsilon \alpha)w - \epsilon \triangledown_w J(w; X, y).
\end{aligned}
$$
我们可以看到权重衰减相对学习做了一些修改,在执行这个不同寻常的梯度更新之前,在每一步上借助一个常数因子,对权重向量成倍的进行压缩。这个过程描述了每一步发生了什么,但对于训练的整个过程来说,这又会产生什么影响呢?

我们将通过使用一个二次函数对目标函数进一步对这个分析简化,权重值是未加入正则化项的(基于训练集)损失函数最小值时的权重值,即$\bf{w}^* = \arg \min_w J(w)$。如果目标函数的计算确实是二次形式的,好比用来拟合线性回归模型的基于最小均方误差的二次损失函数,那么这个近似就完美了。其近似$\widetilde{J}$如下:
$$
\begin{aligned}
	\widetilde{J}(\theta) = J(w^*) + \frac{1}{2} (w - w^*)^T H (w - w^*),
\end{aligned}
$$
其中$H$是损失函数$J$关于$w$值为$w^*$时的海森矩阵。在该损失函数的二次近似的式子中并没有一次项,因为$w^*$被定义为损失函数值最小时的取值,即梯度此位置处是不存在的。同样,由于$w^*$所处的位置是$J$的一个极小值点,所以我们也可以说$H$矩阵是半正定矩阵。

$\widetilde{J}$的最小值时梯度为:$\triangledown_w \widetilde{J}(w) = H (w - w^*) $等于$0$。

为了能研究权重衰减的效果,我们引入一个权重衰减梯度项来修改上面的公式。现在我们就可以求解加入正则项的损失函数$\widetilde{J}$的最小值了,使用变量$\widetilde{w}$来表示损失函数最小值时的权重。
$$
\begin{aligned}
\alpha \widetilde{w} + H (\widetilde{w} - w^*) = 0 \\
(H + \alpha I) \widetilde{w} = H w^* \\
\widetilde{w} = (H + \alpha I)^{-1} H w^*.
\end{aligned}
$$
随着超参数$\alpha$与$0$越接近,我们索要求解的$\widetilde{w}$也越接近$w^*$。但当超参数$\alpha$增大会发生什么?考虑到矩阵$H$是全对称实矩阵,我们可以使用公式$H = Q \Lambda Q^T$将其分解为对角矩阵$\Lambda$和特征向量的正交基$Q$。将此公式应用到上述最后一个等式7.10上,可得:
$$
\begin{aligned}
\widetilde{w} & = (Q \Lambda Q^T + \alpha I)^{-1} Q \Lambda Q^T w^* \\
& = \left [ Q (\Lambda + \alpha I ) Q^T \right ]^{-1} Q \Lambda Q^T w^* \\
& = Q(\Lambda + \alpha I)^{-1} \Lambda Q^T w^*.
\end{aligned}
$$
可以看到权重衰减的作用是对$w^*$沿着$H$的特征向量的轴线方向进行缩放。$w^*$的成分会被一个因子$\frac{\lambda _i}{\lambda_i + \alpha}$与$H$中的第$i$个特征向量进行对准,从而达到缩放的目的。(您也许会好奇这种缩放是如何实现的,可参考我们第一次讲到的图2.3)。

沿着$H$特征向量轴线的方向,其中$H$的特征值的相当大,有$\lambda_i >> \alpha$,但正则化的效果确实很小的。然而当第$i$个分量有$\lambda_i << \alpha$时,将会导致缩放到到零这个量级,产生的作用与图7.1描述的一样。

\begin{figure}[htbp] %  figure placement: here, top, bottom, or page
   \centering
   \includegraphics[width=3in]{fig/chap7/7_1.png} 
   \caption{该图描述了$L^2$(或权重衰减)正则化在最优值$w$时的影响。其中实心椭圆线条表示未加入正则化项的目标函数的等值线,虚线圈表示有着$L^2$正则化的目标函数的等值i线。在点$\widetilde{w}$处,实线条与虚线条相交达到等值。在$w_1$维度的轴线上,损失函数海森矩阵的特征值很小,当水平移动$w$使其距离$w^*$越来越远时,目标函数值并没有明显的增大。因为沿着$w_1$这个水平方向,目标函数没有表现出强烈的偏好,正则化项在这个轴线上有强作用。正则化项拉着$w_1$让它与零越来越近。在$w_2$维度上,目标函数对于$w$远离$w^*$的运动是非常敏感的,这个方向上对应的特征值很大,有着高曲率。这就导致权重衰减在$w_2$这个方向上的影响相当小。}
   \label{fig:7_1}
\end{figure}

只有能明显地减小目标函数值的参数的方向才能被相对完整保留,在不有助于减小目标函数的方向上,海森矩阵的小特征值告诉我们,在这个方向上的运动不会显着增加梯度。这种不重要方向的权向量分量通过在整个训练中使用正则化来衰减掉。

到目前为止,我们已经讨论了权重衰减在一个抽象的、一般形式以及二次的损失函数形式的优化上产生的作用。那这些作用是如何关系到机器学习中某些方面的呢?我们会发现研究线性回归(其损失函数形式是二次的)的过程也适用于我们目前使用的分析。就训练数据和当前得到的结果再次分析,我们将能够获得相同结果的特殊情况。 对于线性回归,损失函数是误差平方和的形式:
$$
\begin{aligned}
(Xw - y)^T (Xw - y).
\end{aligned}
$$
当加入$L^2$正则化项,目标函数变为:
$$
\begin{aligned}
	(Xw - y)^T (Xw - y) + \frac{1}{2} \alpha w^T w.
\end{aligned}
$$
这将会改变正规方程(normal aligneds)的解从
$$
\begin{aligned}
	w = (X^TX)^{-1}X^T y
\end{aligned}
$$
变为
$$
\begin{aligned}
	w =  (X^TX + \alpha I)^{-1}X^T y.
\end{aligned}
$$
在变化前的方程7.16中的矩阵$X^TX$是协方差矩阵$\frac{1}{m} X^T X$按照比例计算得到的,使用带$L^2$的正则项$(X^TX +\alpha I)^{-1}$替代原本的$X^TX$,最终得到上述最终变化后的方程7.17。新的到的矩阵除了在对角矩阵前加上了一个用于调节大小的超参数$\alpha$,与最初的矩阵是一样的,对角矩阵中每个元素与每个输入特征的方差一一对应。我们可以看到当输入数据的方差比较大时,$L^2$正则化可以使学习算法很好地适应输入数据$X$,它可以使得某些对应输出值较小的特征对应的权重得到一定程度的缩放(which makes it shrink the weights on features whose covariance with the output target is low compared to this added variance)。

\subsection{$L^1$参数正则化}

虽然$L^2$权重衰减是一种常见的权重衰减手段,但还有其它惩罚模型参数规模的方法。另一个选择就是$L^1$正则化方法。

严格地说,模型参数$w$的$L^1$正则化方法定义为:
$$
\begin{aligned}
	\Omega(\theta) = ||w||_1 = \sum_i |w_i|
\end{aligned}
$$
也就是每个参数的绝对值之和。我们首先会研究$L^1$正则化方法在没有偏执参数的线性回归模型上的作用,以及$L^1$与$L^2$两种正则化的不同。与$L^2$权重衰减类似,$L^1$权重衰减前面也有一个正的超参数$\alpha$用来控制惩罚权重衰减$\Omega$的力度。加入$L^1$正则后的目标函数$\widetilde{J} (w; X, y) $为:
$$
\begin{aligned}
	\widetilde{J} (w; X, y) = \alpha ||w||_1 + J(w;X,y),
\end{aligned}
$$
对应的梯度计算公式为(实际上是一个子梯度,不是完整梯度的计算形式):
$$
\begin{aligned}
\triangledown_w \widetilde{J} (w; X, y) = \alpha \text{sign} (w) + \triangledown_w J(X,y;w)
\end{aligned}
$$
其中$\text{sign}(w)$是对权重向量$w$逐个元素应用符号函数。

通过观察上面得到的最后一个方程7.20,我们可以很快发现$L^1$与$L^2$正则化有很大差别。具体而言,正则对梯度的贡献对于每个$w_i$不再是线性的;而是由一个常数因子$\alpha$和符号函数$\text{sign}(w_i)$组成。这样我们就不必要像研究$L^2$正则化时候那样仔细地研究其代数解做出一个损失函数$J(X, y; w)$的二次形式的近似。

线性模型的损失函数是二次形式的,也可以用泰勒级数表示。此外,可以想象用截断的泰勒级数这样的复杂模型来对这个损失函数进行近似,这样得到的梯度为:
$$
\begin{aligned}
	\triangledown_w \hat{J} (w) = H(w - w^*),
\end{aligned}
$$
其中,$H$是损失函数$J$关于$w$取值为$w^*$时的海森矩阵。

因为一般情况下,带上$L^1$惩罚的表达式里面还含有其它项,所以我们进一步假设得到的海森矩阵$H$是对角矩阵,即$H = \text{diag}([H_{1,1}, ..., H_{n,n}])$,其中$H$矩阵中的对角线元素$H_{i,i} > 0$。这条假设保证了线性回归问题的数据已做了去除输入特征之间的所有相关性的预处理,这个预处理过程可能是通过主成分分析完成的。

对于$L^1$正则化的目标函数的二次近似分解成一个基于参数加和的形式:
$$
\begin{aligned}
	\hat{J}(w;X,y)=J(w^*;X,y)+\sum_i\left [ \frac{1}{2} H_{i,i} (w_i - w_i^*)^2 + \alpha |w_i| \right ].
\end{aligned}
$$
最小化近似损失函数的问题有一个分析解(对每一个维度$i$),即如下形式:
$$
\begin{aligned}
	w_i = \text{sign}(w_i^*) \max\left \{ |w_i^*| - \frac{\alpha}{H_{i,i}}, 0 \right \}.
\end{aligned}
$$
考虑到所有的$i$有$w_i^* > 0$的情况,可能有两种结果:

当$w_i^* \leq \frac{\alpha}{H_{i,i}}$时,加入正则化的目标函数最优值时有$w_i = 0$。会导致这个的原因是因为$J(w; X, y)$给加入正则化后的目标函数$\widetilde{J}(w; X, y)$的贡献太大了,也就是在方向$i$上$L^1$正则化会将$w_i$的值推向零。

当$w_i^* > \frac{\alpha}{H_{i,i}}$时,正则化不会将最优值时的$w_i$移动到零,但扔会朝着相同的方向推进$\frac{\alpha}{H_{i,i}}$的距离。

当$w_i^* < 0$时,也会发生相似的过程,但$L^1$惩罚会使得$w_i$朝着正数方向移动$\frac{\alpha}{H_{i,i}}$的距离或者移动到$0$ 。

与$L^2$正则化方法相比,$L^1$正则化会产生一个更稀疏的解。在这种背景下产生稀疏解意味着某些参数计算出的解值为零。$L^1$的稀疏性与$L^2$正则化有本质上的不同,公式7.13即$\widetilde{w} = Q(\Lambda + \alpha I)^{-1} \Lambda Q^T w^*$给出了$L^2$正则化时$\widetilde{w}$的解,回顾该公式,有一个假设那就是海森矩阵$H$是对角矩阵且是正定矩阵,引入海森矩阵的目的是方便对$L^1$正则化的分析,可以发现$\widetilde{w_i} = \frac{H_{i,i}}{H_{i,i} + \alpha} w_i^*$。如果$w_i^*$是非零的,那么$\widetilde{w_i}$也是非零的。这就论证了即使 $L^1$ 在 $\alpha$ 比较大的情况下也会变为零,但 $L^2$ 正则化不会导致参数变得稀疏。

$L^1$正则化的稀疏性已被广泛用于特征选择。特征选择通过从原始的特征中选择出可用的特征子集来简化机器学习问题。特别是众所周知的LASSO(Tibshirani, 1995)(全称为:least absolute shrinkage and selection operator,即”最小绝对收缩与选择算子“)特征选择模型,它整合了$L^1$的线性模型惩罚和最小平方损失函数。$L^1$惩罚会使得权重向量中的一部分元素值变为零,这也表明这些权重所对应的特征可以被安全地舍弃掉。

在第5.6.1小节中,我们看到了很多可以理解为最大后验贝叶斯推断(MAP Bayesian inference)的正则化策略,特别是$L^2$正则化等价于有着高斯分布的先验权重的最大后验(MAP)贝叶斯推断。对于$L^1$正则化来说,当先验是各向同性拉普拉斯分布时(参见第3.26小节的方程)即$w \in \Re^n$,用于正则损失函数的惩罚项$\alpha \Omega (w) = \alpha \sum_i |w_i|$等价于由最大后验(MAP)贝叶斯推理最大化的对数先验项。
$$
\begin{aligned}
\log p(w) = \sum_i \log \text{Laplace}(w_i;0,\frac{1}{\alpha}) = -\alpha ||w||_1 + n \log \alpha - n \log2.
\end{aligned}
$$
从学习的角度来看相对$w$的最大化,我们可以忽略$\log \alpha - \log 2$这两项,因为它们的值不取决于权重$w$。

\section{约束优化的范数惩罚}

考虑参数范数惩罚的代价函数正则化:
$$
\begin{aligned}
	\widetilde{J}(\theta; X, y) = J(\theta; X, y) + \alpha \Omega(\theta).
\end{aligned}
$$
回顾4.4小节中,由于原来的目标函数有一套惩罚,通过构造一个广义的拉格朗日函数,可以使约束函数最小化。每个惩罚相当于一个乘积,乘积由两项构成:一项是被称为Karush–Kuhn–Tucker(KKT)乘子的系数,另一项是用来代表限制条件是否满足的函数。如果我们想要约束$\Omega(\theta)$小于某个常数$k$,我们可以构造一个广义拉格朗日函数:
$$
\begin{aligned}
	\mathcal{L} (\theta, \alpha; X, y) = J(\theta; X, y) + \alpha (\Omega(\theta) - k).
\end{aligned}
$$
约束问题的解由以下公式给出:
$$
\begin{aligned}
	\theta^* = \arg_{\theta} \min \max_{\alpha,\alpha \leq 0} \mathcal{L}(\theta, \alpha).
\end{aligned}
$$
正如第4.4小节所述,要计算得到$\theta^*$值需要同时修改$\theta$与$\alpha$的值,第4.5小节提供了一个带$L^2$正则化限制的线性回归的例子。许多不同的过程是可能的——某些情况可以使用梯度下降求解,而其它时候因为梯度为零而需要使用分析解,但是在所有过程中,当$\Omega (\theta) > k$时,$\alpha$必须增加,而当$\Omega (\theta) < k$时$\alpha$必须减小。所有的正值$\alpha$会促使$\Omega (\theta)$收缩,还有最佳值$\alpha^*$也会促使$\Omega (\theta)$收缩,但再怎么收缩减少也不会使$\Omega (\theta)$小于$k$。

为了进一步分析约束的作用,我们可以将$\alpha^*$的值固定并把问题变为$\theta$的函数的问题:
$$
\begin{aligned}
\theta^* = \arg_{\theta} \min \mathcal{L}(\theta, \alpha^*) = \arg_\theta \min J(\theta; X, y) + \alpha^* \Omega(\theta).
\end{aligned}
$$
这与使$\widetilde{J}$最小化的正则训练问题完全相同。因此,我们可以认为参数范数惩罚是对权重施加约束,如果$\Omega$是$L^2$范数,则权重被约束在$L^2$球中。如果$\Omega$是$L^1$范数,则权重被限制在位于有限$L^1$范数的区域中。通常我们不知道(通过使用系数$\alpha^*$的权重衰减的)约束区域的大小,因为得到了$\alpha^*$的值并不能直接得到$k$的值。原则上,可以求解这种交叉关系(fork),但是$k$和$\alpha^*$之间的关系取决于$J$的形式。虽然不知道约束区域的确切大小,但可以通过增加或减少$\alpha$来粗略地控制它,以便增长或缩小约束区域,较大的$\alpha$将导致较小的约束区域,而较小的$\alpha$将导致较大的约束区域。

有时我们可能希望使用明确的约束,而不是惩罚。如第4.4节所述,我们可以修改算法,如随机梯度下降沿$J(\theta)$下坡,然后投射$\theta$回到满足$\Omega (\theta) < k$的最近点。如果我们知道$k$的什么值是适当的并且不想花时间搜索与该$k$对应的$\alpha$值,这可能是有用的。

使用显式约束和重投影而不是用惩罚实施约束的另一个原因是惩罚可以导致非凸优化过程陷入对应于小$\theta$的局部最小值。当训练神经网络时,这通常表现为训练有几个“死掉的神经元单位”的神经网络。这些“死掉的神经元”是对网络学习的函数的行为没有多大贡献的单位,因为进入或离开这些神经元的权重都很小。当训练对权重的范数进行惩罚时,这些配置可以是局部最优的,即使可以通过使权重更大来显着减少$J$。通过重投影实现的显式约束在这些情况下的效果更好,因为这种方法不会促使权重逼近原点。通过重投影实现的显式约束只有当权重变大并且试图离开约束区域时才具有效果。

最后,使用重投影的显式约束可能是有用的,因为它们对优化过程施加了一些稳定性。当使用大学习率时,可能遇到正反馈回路,其中大的权重诱导大的梯度,然后引起对权重的大的更新。如果这些更新一致地增加权重的大小,则$\theta$迅速地从原点移开,直到出现数值溢流。具有重投影的显式约束可以防止此反馈循环,因为权重的持续增大没有限制。Hinton等人(2012c)建议使用约束结合高学习率,这样可允许快速探索参数空间,同时保持一些稳定性。

特别地,Hinton等人(2012c)推荐由Srebroand Shraibman(2005)引入的策略:约束神经网络层的权重矩阵的每列的范数,而不是约束整个权重矩阵的Frobenius范数。分别约束每列的范数可防止任何一个隐藏单元具有非常大的权重。如果我们在拉格朗日函数中将这个约束转换为惩罚,它将类似于$L^2$权重衰减,但是对于每个隐藏单元的权重具有单独的KKT乘子。这些KKT乘子中的每一个将被单独地动态地更新,以使每个隐藏单元服从约束。在实践中,列规范限制总是作为具有重投影的显式约束来实现。

\section{正则化和受约束问题}

在某些情况下,正则化对机器学习问题的正确定义是必要的。机器学习中的许多线性模型,包括线性回归和主成分分析(PCA)的计算都依赖计算翻转矩阵,即$X^T X$。$X^T X$是奇异的是不可能的(This is not possible whenever $X^T X$ is singular)。每当数据生成分布在一些方向上确实没有方差时,或者当在一些方向上没有观察到方差时,该矩阵可以是奇异矩阵,因为存在比输入的特征数目($X$的列)更少的样本数($X$的行)。在这种情况下,许多形式的正则化对应于反转$X^T X + \alpha I$,这个正则化矩阵保证是可逆的。

当相关矩阵可逆时,这些线性问题具有闭合形式解。有的问题也可能没有闭合解。一个例子是应用了逻辑回归的问题,问题中数据的类别是线性可分的。如果一个权重向量可以实现完美的分类,那么$2w$会有更大可能性实现完美分类。如随机梯度下降的迭代优化过程将不断更新$w$,并且在理论上将永远不会停止。在实践中,梯度下降的数值实现将最终使权重达到足够大引起数值溢出,此时要做的操作将取决于程序员如何处理不是实数的值。

大多数形式的正则化能够保证应用于不确定问题的迭代方法的收敛。例如,当似然性的斜率等于权重衰减系数时,权重衰减将导致梯度下降使权重不再增大。

使用正则化来解决不确定问题的想法超出了机器学习。相同的想法对于一些基本的线性代数问题也是有用的。

正如我们在2.9节中所看到的,我们可以使用Moore-Penrose伪逆来解决不确定的线性方程。回想矩阵$X$的伪逆$X^+$的定义是:
$$
\begin{aligned}
	X^+ = \lim_{\alpha \searrow 0} (X^T X + \alpha I)^{-1} X^T.
\end{aligned}
$$
我们现在可以将该上述公式7.29作为具有权重衰减的线性回归。特别地,上述公式7.29是方程7.17($w = (X^TX + \alpha I)^{-1}X^Ty$)在正则化系数收缩到零时候的极限情况。因此,我们可以将伪逆解释为使用正则化来稳定未确定的问题。

\section{数据集扩增}

使机器学习模型更好地泛化的最好方法基于更多的数据训练。当然,在实践中,我们所拥有的数据量有限。 解决这个问题的一种方法是创建假数据并将其添加到训练集,对于一些机器学习任务,创建新数据相当简单。

对于分类问题的数据扩增来说,这种方法是最容易的。分类器需要采用复杂的高维输入$x$,并用单个类别标识$y$来标记。这意味着分类器的主要任务是对于各种各样的数据变换其输出的结果是不变的。我们可以通过转换改变训练集中的$x$输入来容易地生成新的$(x, y)$样本对。

这种方法不是很容易适用于许多其他任务。例如,除非我们已经解决了密度估计问题,否则不可能为密度估计任务生成新的假数据。

对于特定分类问题,如物体识别,数据集扩增方法是一种特别有效的技术。图像是高维且包括各种变化因素的数据,其中许多因素可以容易地被模拟。即使模型采用第9章描述的卷积和池化技术具有局部平移不变性,诸如将训练图像在每个方向上平移几个像素的操作可以很大程度提升泛化性,许多其它操作,如旋转图像或缩放图像在数据扩增中也已被证明是非常有效的。

一个务必注意的是在平移的方法应用中,这种操作会对改变正确的类别。例如,光学字符识别任务需要识别'b'和'd'之间的差异以及'6'和'9'之间的差异,因此水平翻转和180°旋转不是扩增该任务数据集的适当方式。

还有一些平移方法是我们希望分类器能再见到平移后的数据预测其类别能不变的,但这些方法不容易实现。例如,平面外的旋转操作不能作为对输入像素的简单几何操作来实现。

数据集扩增对语音识别任务也有效(Jaitlyand Hinton,2013)。

在神经网络的输入中注入噪声(Sietsma和Dow,1991)也可以被看作是数据扩增的一种形式。即使小的随机噪声添加到输入,对于许多分类甚至一些回归任务仍然可以被解决。然而,神经网络证明对噪声的鲁棒不是非常好(Tang和Eliasmith,2010)。提高神经网络的鲁棒性的一种方法是简单地训练时将随机噪声加入到输入中。在输入处将噪声注入是一些无监督学习算法的一部分,如降噪自动编码器(Vincent等人,2008)。当噪声到隐藏单元时,噪声注入也起作用,这可以被看作在多个抽象级别处进行数据集增加。Poole等人(2014)最近表明,只要仔细调整噪声的幅度,这种方法就可以非常有效。Dropout是一个强大的正则化策略,将在7.12节中描述,可以看作是通过乘以噪声构建新输入的过程。

当比较机器学习基准测试结果时,考虑数据集增强的效果是很重要的。通常,手动设计的数据集扩充方案可以显着减少机器学习技术的泛化误差。为了比较一种机器学习算法与另一种机器学习算法的性能,有必要进行受控实验。当比较机器学习算法A和机器学习算法B时,有必要确保使用相同的手动设计的数据集增加方案来评估两种算法。假设与输入的许多合成变换组合时,没有数据集扩增的算法A执行得很差,而算法B执行良好。

在这种情况下,将不同的扩增变换方法进行组合可能提升模型分类性能,而不是使用机器学习算法B。有时决定一个实验是否被适当控制需要主观判断,例如将噪声引入输入的机器学习算法的数据集增加的形式。通常可应用的操作(诸如向输入添加高斯噪声)被认为是机器学习算法的一部分,而另一种是特定用在某应用领域的操作(诸如随机地剪切图像)被认为是单独的预处理步骤。

\section{噪声鲁棒性}

第7.4节提出使用对输入数据加入噪声的方法作为数据集增强策略。对于一些模型,在模型的输入处添加具有(译注:infinitesimal variance)方差的噪声等效于对权重的范数施加惩罚(Bishop,1995a,b)。在一般情况下,重要的是要记住噪声注入可以比简单地收缩参数更强大,特别是当噪声被添加到隐藏单元时。在隐藏单元处加入噪声是一个重要的话题,值得单独讨论;第7.12节中描述的dropout算法是该方法的延伸。

在正则化模型中添加噪声的另一种方式是将其添加到权重。这种技术主要用于循环神经网络(Jim等人,1996;Graves,2011)。这可以解释为对权重的贝叶斯推理的随机实现。贝叶斯学习的处理认为模型权重是不确定的,但可以通过反映这种不确定性的概率分布来表示。将噪声添加到权重是一种实用的且随机的方式来反映不确定性。

将噪声加入到权重上等同于(在一些假设下)更传统的正则化形式,从而增加要学习的函数的稳定性。考虑回归设置,其中我们希望训练使用在模型预测$\hat{y}(x)$和真实值$\hat{y}(x)$之间的最小二乘法成本函数将特征$x$的集合映射到标量的函数$y$:
$$
\begin{aligned}
J = E_{p(x,y)}[(\hat y(x) - y)^2].
\end{aligned}
$$
训练集包含$m$对标注样例$\{(x^{(1)}, y^{(1)}),\dots,(x^{(m)}, y^{(m)})\}$。

我们现在假设对于每个输入也包括网络去权重的随机扰动$\epsilon _W \sim N(\epsilon;0, \eta I )$。让我们想象一下有一个标准的$l$层多层感知机(MLP)。我们将扰动模型表示为$\hat{y}_{\epsilon_W} (x)$。除了注入噪声,我们关注的仍是最小化网络输出的平方误差。因此,目标函数变为
$$
\begin{aligned}
\tilde J_{W}
&= E_{p(x,y,\epsilon_{W})}[(\hat y_{\epsilon_{W}}(x) - y)^2] \\
&= E_{p(x,y,\epsilon_{W})}[\hat y_{\epsilon_{W}}^2(x) - 2y\hat y_{\epsilon_{W}} (x)+ y^2] .
\end{aligned}
$$
对于较小的$\eta$,对加入了权重噪声(具有协方差$\eta I$)的损失函数$J$的最小化等价于加入正则化项的$J$的最小化:$\eta E_{p(x,y)}[||\nabla_{W}~\hat y(x)||^2]$。这种形式的正则化会促使参数进入参数空间的区域,在该参数空间中,小的权重扰动对输出具有相对小的影响。换句话说,该形式会使模型处于对小权重的变化也相对敏感的区域,找到的不仅是最小值点,而且是周围被平坦区域包围的最小值点(Hochreiter和Schmidhuber,1995)。在简化的线性回归模型中(如$\hat{y}(x) = w^\top x + b$),这个正则化项为$\eta E_{p(x)}[||x||^2]$,因为不是参数的函数,因此相对于模型参数不会为损失函数$\widetilde{J}_W$贡献的梯度。

\subsubsection{在输出目标处注入噪声}

大多数数据集在$y$标签中有一些错误,当$y$错误时,对最大化$log p(y | x)$是有害的。防止这种情况的一种方法是对标签上的噪声建模。例如,假设训练集合标签$y$是以概率$1-\epsilon$正确的,其中$\epsilon$是一个小的常数,同时也意味着有$\epsilon$的概率标签$y$是错误的,当前样本标签$y$可能是其他的类型。这个假设很容易被解析地结合到成本函数中,而不是显式地采集噪声样本。

例如,标签平滑基于一个具有$k$个输出值的软性最大分类器(softmax)通过使用$\frac{\epsilon}{k-1}$和$1-\epsilon$分别替代$0$和$1$分类目标来对模型正则化。标准的交叉熵损失可以与软目标结合使用。 使用软性最大(softmax)分类器和硬目标的最大似然学习实际上可能不会收敛——软性最大(softmax)永远不能完全确认这个样本是$0$或$1$,因此它将继续学习到越来越大的权重,从而永远做出更极端的预测。可以使用其它正则化策略(如权重衰减)来防止此情况。标签平滑具有防止在不阻碍正确分类的情况下追求硬概率的优点。这个策略自从1980年代以来一直被使用,并继续在现代神经网络中突出特色(Szegedyet等人,2015)。

\section{半监督学习}

所谓半监督学习,是将$P(x)$产生的未标记样本和$P(x, y)$中的标记样本都用于估计$P(y | x)$或根据样本特征$x$来预测其类别$y$。

在深度学习的背景下,半监督学习通常指的是学习表示$h = f(x)$。目标是学习数据表征,使得来自相同类的样本具有类似的数据表征。无监督学习可以为如何在表示空间中分组的样本提供了有用的策略。在输入空间中聚集紧密的样本应能映射到相似的表示。新空间中的线性分类器在许多情况下可以实现更好的泛化(Belkin和Niyogi,2002;Chapelle等人,2003)。这种方法的一个变体是在应用分类器(在投影数据上)之前将主成分分析作为预处理步骤之一。

半监督模型中有着独立的无监督和监督的部分,它由$P(x)$或$P(x,y)$的生成模型与$P(y|x)$的判别模型组成且共享参数。我们可以对无监督或生成模型(例如$\log P(x)$或$-\log P(x,y)$)对监督模型的标准$-\log P(y | x)$进行平衡。生成模型对监督学习问题的解是一种特殊形式的先验知识(Lasserre等人,2006),即$P(x)$的结构通过某种共享参数的形式连接到$P(y | x)$的结构。通过控制生成模型中准则占总准则中的比例,可以找到比纯粹是生成模型或完全是判别模型准则进行训练得到更好的(判别与生成模型间的)平衡(Lasserre等人,2006;Larochelle和Bengio,2008)。

Salakhutdinov和Hinton(2008)描述了一种用于学习用于回归问题的核机器的核函数方法,其中使用未标记的样本来建模$P(x)$非常显着地改善了$P(y | x)$。

更多关于半监督学习的信息,请参阅Chapelle等人(2006)的文章。

\section{多任务学习}

多任务学习(Caruana,1993)是一种通过合并几个任务中产生的样本(可以被视为对参数施加的软约束)来改进泛化性能的方式。

以同样的方式,额外的训练样本对模型的参数施加更大的压力,使其更适用于一般化的值(译注:泛化能力得到提升),当模型中的一部分在多个任务之间共享时,模型的该部分被更多地约束为好的值(假设任务共享是合理的),这经常能为模型带来更好的泛化能力。

图7.2展示了一种非常常见的多任务学习形式,不同的监督任务(在给定$x$时输出预测类别$y^{(i)}$)共享相同的输入$x$,以及中间层的表示$h^{(shared)}$(多任务共享的),这些中间层的聚合因素池也是公共的(capturing a common pool of factors.)。该模型与其相关参数一般可以分为两类部分:

\begin{itemize}
\item 特定任务的参数 (只能通过各自特定任务中的样本中获得好的泛化能力),如图7.2中的上层。
\item 所有任务共享的通用参数(从所有不同任务汇集数据获得性能提升),如图7.2中的下层。
\end{itemize}

\begin{figure}[htbp] %  figure placement: here, top, bottom, or page
   \centering
   \includegraphics[width=3in]{fig/chap7/7_2.png} 
   \caption{多任务学习可以在深度学习框架中以多种方式进行,该图描述了一个通用情况,不同任务共享相同的输入但涉及不同的目标随机变量。深层网络的较低层(无论是前馈式的监督学习还是包括具有向下箭头的生成式学习)可以在不同的任务之间共享,而对于特定任务的参数(分别到达$h^{(1)}$和$h^{(2)}$的任务)可以在共享表示$h^{(shared)}$之上被学习。基本的假设是:存在解释输入$x$变化的大量共同权重因子,而每个特定任务只与这些权重因子的子集相关联。在该图的示例中,增加了假设:上层的隐藏层单元$h^{(1)}$和$h^{(2)}$专注于特定任务(分别预测$y^{(1)}$和$y^{(2)}$),而一些中间级的表示$h^{(shared)}$则在所有任务之间共享。在无监督学习中,一些上层权重与任何输出任务($h^{(3)}$)都不相关是有意义的:$h^{(3)}$相关的权重揭示了输入的变化但与预测$y^{(1)}$或$y^{(2)}$是不相关的。}
   \label{fig:7_2}
\end{figure}

因为共享参数,泛化性能得以提升并且可以得到泛化误差界限(Baxter,1995),(与单个任务相比,对于共享参数,样本数量的成比例增加)使共享参数的统计强度大大提高。当然,只有当不同任务间的统计关系的假设是有效时,某些不同任务间的参数才可以被共享使用。

从深度学习的角度来看,有如下的潜在先验知识:观察不同任务的数据变化权重,一些权重是可以在两个或多个任务之间共享的。

\section{提前终止}

因为大模型具有足够的代表能力,所以在训练大模型完成任务时,我们经常观察到随时间训练误差稳定下降,但是验证集误差开始再次上升。这个过程的示例请参见图7.3,该过程一般都会发生。

\begin{figure}[htbp] %  figure placement: here, top, bottom, or page
   \centering
   \includegraphics[width=5in]{fig/chap7/7_3.png} 
   \caption{显示负对数似然损失随时间变化的学习曲线(横坐标表示基于数据集的训练迭代次数或者一次完整的数据集遍历)。在本例中,我们在MNIST上训练一个maxout网络。观察到训练集的目标函数值随时间持续减小,但验证集平均损失函数值开始再次增加,形成不对称的U形曲线。}
   \label{fig:7_3}
\end{figure}

这意味着我们可以得到验证集误差最低时(并且很可能有更低的测试集误差)的模型。每当验证集上的误差降低时,我们便会存储此时的模型参数副本。当训练算法终止时,我们返回所有存储的模型参数,而不是最后一次的模型参数。当达到预先规定的迭代次数没有参数比最佳记录的验证误差更小时,该算法终止。这个过程在算法中有更正式的指定。

%%%%%%%%%%%%%%%%%%%%%%%%%%%%%%%%%%%%%%%%%%%%%%%%%%%%%
算法7.1用来确定最佳训练次数的提前终止(early stopping)元算法

这种元算法是一种通用策略,可以很好地与各种训练算法和验证集上的误差量化方法相结合。

令$n$为两次评估之间的步数。

令$p$为耐心程度(patience),即在算法停止前容忍算法出现更大验证集误差的次数。

令$\theta_{o}$为初始权重参数。

$\theta \leftarrow \theta_{o}$

$i \leftarrow 0$

$j \leftarrow 0$

$v \leftarrow \infty$

$\theta^* \leftarrow \theta$

$i^* \leftarrow i$

WHILE $j < p$

$\quad$运行训练算法$n$步,更新$\theta$ 。

$\quad$ $i \leftarrow i + n$

$\quad$ $v' \leftarrow \text{ValidationSetError}(\theta)$

$\quad$ IF $v' < v$

$\quad \quad j \leftarrow 0$

$\quad \quad \theta^* \leftarrow \theta$

$\quad \quad i^* \leftarrow i$

$\quad \quad v \leftarrow v'$

$\quad $ ELSE

$\quad \quad j \leftarrow j + 1$

$\quad$ ENDIF

ENDWHILE

最佳参数为$theta^*$,最佳训练步数为$i^*$
%%%%%%%%%%%%%%%%%%%%%%%%%%%%%%%%%%%%%%%%%%%%%%%%%%%%%

这种策略被称为提前终止。它可能是深度学习中最常用的正则化形式,该方法的普及是因为有效和简单。

提前终止(early stopping)可以被认为是非常高效的超参数选择算法。在这一个观点下,训练步数只是另一个超参数。可以在图7.3中看到,这个超参数具有U形验证集的性能曲线。大多数控制模型能力的超参数都具有如图5.3所示的U形验证集性能曲线。在使用提起终止(early stopping)的情况下,通过确定对训练集拟合所需的训练步数来控制模型能力。大多数超参数选择所需的计算和检查的代价都很大,在训练开始时设置超参数,然后运行几步训练查看效果。“训练时间”(training time)超参数是唯一的,因为它定义了训练所需的时间,并在此训练过程中尝试了其他超参数的许多值。通过提前终止(early stopping)自动选择此超参数的唯一需要的是在训练期间定期对验证集评估。理想情况下,这个过程是在单独机器,或单独CPU或单独GPU上进行的,与训练过程并行完成。如果没有这么多计算资源,则可以使用比训练集小的验证集验证,或减少评估验证集误差的次数,来获得最佳训练时间的大致估计来降低周期性估计超参数带来的成本。

提前终止(early stopping)还要做的保存一份当前得到的最佳参数的副本。该操作通常可以忽略,因为可以将这些参数存储在较慢和较大形式的存储器中(例如,在GPU存储器中训练,在主机存储器或磁盘驱动器中存储训练得到的最佳参数)。由于最佳参数很少写入并且在训练期间从不读取,所以这些偶尔的慢写入对总训练时间几乎没有影响。

提前终止(early stopping)是一种非常不明显的正则化形式,因为它对训练中的目标函数或参数空间的值的集合几乎没有改变。这意味着在不损害学习的动态过程中,提前终止(early stopping)是易于使用的。这与权重衰减形成对比,必须注意不要使用太大的权重衰减,因为这回导致网络陷入陷入不好的局部最小值,对应得到不正确的小值权重。

提前终止(early stopping)可以单独使用或与其它正则化策略结合使用。即使通过修正目标函数的方法以促进更好的泛化性能的正则化策略时,最好的泛化表现很少出现在训练目标函数的局部最小值处。

提起终止(early stopping)需要一个验证集,这意味着有一些训练数据不会用于模型训练。为了能最好地利用这一额外的数据,其一是可以在初始训练与提前终止完成之后执行额外的训练。在第二个额外训练步骤中,包括所有训练数据。有两个基本策略可以用于该第二训练过程。

一个策略(算法7.2)是再次初始化模型并基于所有数据重新训练。在第二次训练中,我们训练使用第一次训练中提前停止确定的最佳的迭代次数。在此过程中有一些细节,例如,没有一种好的方式来知道是否基于数据集重新训练相同数量的参数更新,或相同数量的传播。在第二轮训练中,每次通过数据集将需要更多的参数更新,因为训练集更大。

%%%%%%%%%%%%%%%%%%%%%%%%%%%%%%%%%%%%%%%%%%%%%%%%%%%%%
算法7.2

使用提前终止(early stopping)确定训练步数,之后在所有数据重新训练的元算法。

$X^{(train)}$和$y^{(train)}$是训练集。

将训练集$X^{(train)}$和$y^{(train)}$分别切分成$(X^{(subtrain)}, X^{(valid)})$和$(y^{(subtrain)}, y^{(valid)})$。

执行提前终止(算法7.1),随机初始化$\theta$,使用训练数据集$(X^{(subtrain)}, y^{subtrain})$以及验证数据集$X^{(valid)}$和$y^{(valid)}$。将返回$i^*$,即最佳的迭代次数。

再次对$\theta$随机初始化。


基于训练数据集$X^{train}$和$y^{(train)}$迭代$i^*$次。
%%%%%%%%%%%%%%%%%%%%%%%%%%%%%%%%%%%%%%%%%%%%%%%%%%%%%

另一个基于所有数据的策略是保持从第一轮训练中获得的参数,然后继续训练,但现在使用所有数据。在该阶段已经不再有一个何时哪次迭代停止的引导。相反,可以监视验证集上的平均损失函数,并继续训练直到它低于训练集目标函数的值,在该值处执行提前终止(early stopping)。这种策略避免了从头开始重新训练模型的高成本,但是表现并不是很好。例如,无法保证验证集的目标函数值将达到目标值,该策略甚至不能保证终止。这个过程在算法7.3中有更正式地描述。

提前终止(early stopping)也是有用的,因为它减少了训练过程的计算成本。除了由于限制训练迭代次数而明显降低计算成本之外,它还具有提供正则化的好处,而不需要在成本函数或附加项的梯度的计算中添加惩罚项。

\textbf{如何让提前终止(early stopping)发挥正则化的作用}:迄今为止,我们已经说过提前终止(early stopping)是一个正则化策略,但支持该说法的只有通过绘制出验证集误差的U型学习曲线。提前终止规范模型的实际机制是什么?Bishop(1995a)、Sjöberg和Ljung(1995)认为提前终止(early stopping)具有将优化过程限制在初始参数值$\theta$附近相对小范围参数空间的效果,如图7.4所示。具体而言,假设经过$\tau$步优化(训练迭代的次数)和使用学习速率$\epsilon$。我们可以将这二者乘积$\epsilon \tau$作为模型能力(effective capacity)的一种度量。假设梯度值是有范围限制的,那么限制迭代次数和学习速率将会限制从$\theta_0$可达到的参数空间的范围。那么可以说,$\tau$起到了权重衰减系数的倒数(reciprocal of the coefficient used for weight decay)的作用。

%%%%%%%%%%%%%%%%%%%%%%%%%%%%%%%%%%%%%%%%%%%%%%%%%%%%%
算法7.3

元算法在目标函数(即损失函数)值即将过拟合时候使用提前终止(early stopping)策略,直到目标函数值开始过拟合时之后继续训练。

$X^{(train)}$和$y^{(train)}$是训练集。

将训练集$X^{(train)}$和$y^{(train)}$分别切分成$(X^{(subtrain)}, X^{(valid)})$和$(y^{(subtrain)}, y^{(valid)})$。

执行提前终止(算法7.1),随机初始化$\theta$,使用训练数据集$(X^{(subtrain)}, y^{subtrain})$以及验证数据集$X^{(valid)}$和$y^{(valid)}$。这将会对$\theta$进行更新。

$\epsilon \leftarrow J(\theta, X^{(subtrain)}, y^{(subtrain)})$当(while)$J(\theta, X^{(valid)}, y^{(valid)}) > \epsilon$时,$\quad$基于$X^{(train)}, y^{(train)}$训练$n$步迭代,

结束(while)循环。
%%%%%%%%%%%%%%%%%%%%%%%%%%%%%%%%%%%%%%%%%%%%%%%%%%%%%

事实上,我们可以展示如何在具有二次的损失函数和梯度下降的线性模型的情况下,提前终止(early stopping)等价于$L^2$正则化。

为了与经典的$L^2$正则化比较,我们做出一个简单的设定,其中只有参数是线性权重(即$\theta = w$)。我们可以使用在权重$w^*$为最佳值的邻域中的二次近似来建模成本函数$J$:
$$
\begin{aligned}
\hat{J} (\theta) = J(w^*) + \frac{1}{2} (w - w^*)^T H(w - w^*),
\end{aligned}
$$
其中,$H$是损失函数$J$在权重参数$w$取值为$w^*$时的海森矩阵。假设$w^*$是损失函数$J(w)$最小值时的取值,我们知道$H$是半正定矩阵。在局部泰勒级数近似(local Taylor series approximation)下,给出如下梯度的计算公式:
$$
\begin{aligned}
\triangledown_w \hat{J}(w) = H(w - w^*).
\end{aligned}
$$
\begin{figure}[htbp] %  figure placement: here, top, bottom, or page
   \centering
   \includegraphics[width=5in]{fig/chap7/7_4.png} 
   \caption{提起终止(early stopping)的图像描绘。(左)图中的实线等高线表示负对数似然(negative log-likelihood)的轮廓,短划线则是随机梯度下降从起始点开始的轨迹线。最终算法停止的位置不在损失函数最小值对应权重参数$w^*$的位置处,提前终止(early stopping)会使轨迹线早于损失函数最小值对应的$w^*$处停下。(右)图是用于和$L^2$正则效果比较的图示,虚线圈表示$L^2$正则的等高线,$L^2$正则会使总损失函数的最小值比未正则化的损失函数的最小值更靠近原点。}
   \label{fig:7_4}
\end{figure}

我们将在训练期间研究参数向量遵循的轨迹。为简单起见,将初始参数向量设置为原点(对与神经网络而言,为了获得隐含单元间的对称破碎(symmetry breaking),如第6.2小节描述的,不能设定初始时的权重参数都为$0$。但该论点适用于其它任何初始值$w^{(0)}$),即$w^{(0)}=0$。让我们通过分析$\hat{J}$上的梯度下降来研究与$J$上梯度下降的近似行为:
$$
\begin{aligned}
w^{(\tau)} & = w^{(\tau - 1)} - \epsilon \bigtriangledown_w \hat{J} (w^{(\tau - 1)}) \\
& = w^{(\tau - 1)} - \epsilon H (w^{(\tau - 1)} - w^*) \\
w^{(\tau)} - w^* & = (I - \epsilon H)(w^{(\tau - 1)} - w^*).
\end{aligned}
$$
让我们现在在$H$的特征向量的空间中重写这个表达式,利用$H$的特征分解:$H = Q \Lambda Q^T$,其中$\Lambda$是对角矩阵,$Q$是特征向量的正交基。
$$
\begin{aligned}
w^{(\tau)} - w^* & = (I - \epsilon Q \Lambda Q^T) (w^{(\tau - 1)} - w^*) \\
Q^T (w^{(\tau)} - w^*) & = (I - \epsilon \Lambda ) Q^T (w^{(\tau - 1)} - w^*)
\end{aligned}
$$
假设$w^{(0)} = 0$同时$\epsilon$选择一个足够小的保证$|1 - \epsilon \lambda_i| < 1$成立的值。$\tau$参数更新后训练的参数轨迹如下:
$$
\begin{aligned}
	Q^T w^{(\tau)} = [I - (I - \epsilon \Lambda)^{\tau}] Q^T w^*.
\end{aligned}
$$
现在,用于$L^2$正则化的等式7.13(即$\widetilde{w} = Q(\Lambda + \alpha I)^{-1} \Lambda Q^T w^*.$)中的$Q^T\widetilde{w}$的表达式可以后推为:
$$
\begin{aligned}
Q^T \widetilde{w} & = (\Lambda + \alpha I)^{-1} \Lambda Q^T w^* \\
Q^T \widetilde{w} & = [I - (\Lambda + \alpha I)^{-1} \alpha] Q^T w^*
\end{aligned}
$$
比较方程7.40($Q^T w^{(\tau)} = [I - (I - \epsilon \Lambda)^{\tau}] Q^T w^*$)与7.42($Q^T \widetilde{w} = [I - (\Lambda + \alpha I)^{-1} \alpha] Q^T w^*$),如果选择超参数$\epsilon, \alpha, \tau$,那么有:
$$
\begin{aligned}
	(I - \epsilon \Lambda)^{\tau} = (\Lambda + \alpha I)^{-1} \alpha,
\end{aligned}
$$
那么可以看出$L^2$正则化和提前终止(early stopping)是等价的(至少在目标函数的二次近似下)。进一步讲,通过采用对数和使用对数$\log{(1+x)}$的序列展开,可以得出结论:当所有$\lambda_i$都很小时(即$\epsilon \lambda_i \ll 1\ and\ \lambda_i/\alpha \ll 1$),那么
$$
\begin{aligned}
\tau \approx \frac{1}{\epsilon \alpha}, \\
\alpha \approx \frac{1}{\tau \epsilon}.
\end{aligned}
$$
也就是说,在这些假设下,训练迭代次数$\tau$与$L^2$正则化参数成反比,$\tau \epsilon$具有权重衰减系数倒数的作用。

对应于(目标函数的)曲率显著方向的参数值被正则化为小于较小曲率的方向。当然,在提前终止(early stopping)中,这实际上对应于曲率显著变化方向的参数倾向学习较小曲率方向的参数(this really means that parameters that correspond to directions of significant curvature tend to learn early relative to parameters corresponding to directions of less curvature.)。

本节中的推导已经表明,长度为$\tau$的轨迹在对应于$L^2$正则化目标的最小值的点处结束。提前终止(early stopping)当然不仅仅是对轨迹长度的限制;相反,提前终止(early stopping)通常涉及监测验证集误差,以便在空间中特别好的点处停止轨迹。因此,提前终止(early stopping)具有的优点要超过权重衰减,即提前终止(early stopping)自动地确定正则化的正确量,然而权重衰减需要做很多不同的超参数取值时的训练实验。

\section{参数绑定与参数共享}

本章到目前为止,当讨论给参数加入限制或惩罚时,我们总是限定在固定的区域或点中。例如,$L^2$正则化(或权重衰减)惩罚模型参数从参数的固定值零的偏离开始。然而,有时可能需要另外能表达关于模型参数合适值的先验知识。有时可能不知道参数应该采用什么值,但是知道从领域和模型架构的知识,这其中应该有一些模型参数之间的依赖关系。

我们经常想要表达的常见类型的依赖性是某些参数应当彼此接近。考虑以下情况:我们有两个模型执行相同的分类任务(具有相同的类集合),但具有稍微不同的输入数据分布。正式地,我们有参数$w^{(A)}$的模型A和具有参数$w^{(B)}$的模型B,两个模型将输入映射到两个不同但相关的输出:$\hat{y}^{(A)} = f(w^{(A)}, x)$和$\hat{y}^{(B)} = g(w^{(B)}, x)$。

让我们假设任务是相似的(可能具有类似的输入和输出分布),我们认为模型参数应该彼此接近:$\forall i, w_i^{(A)}$应该接近$w_i^{(B)}$。我们可以通过正则化利用这些信息。具体来说,可以使用以下形式的参数规范惩罚:$\Omega (w^{(A)}, w^{(B)}) = || w^{(A)} - w^{(B)} ||_2^{2}$。这里我们使用了$L^2$惩罚,但其它选择也是可能的。

这种方法是由Lasserre等人(2006)提出,将一个模型的参数正则化,使用监督学习的范式训练为分类器,接近另一个使用无监督范式(捕获观察输入数据的分布)训练的模型的参数。构造体系结构使得分类器模型中的许多参数可以与无监督模型中的相应参数配对。

虽然参数范数惩罚是将参数正则化为(译注:让二者模型)彼此接近的一种方式,但更常用的方式是使用约束:迫使参数集相等,这种正则化方法通常被称为参数共享(parameter sharing),因为我们将各种模型或模型组件解释为共享所有参数集合中的其中一组唯一的参数。参数共享在正则化上要关闭的参数(通过范数惩罚)的显着优点(A significant advantage of parameter sharing over regularizing the parameters to be close (via a norm penalty))是只需要将所有权重参数(中唯一集合)的子集存储在存储器中,在某些模型中如卷积神经网络,参数共享可以使模型的内存占用显着减少。

\subsubsection{卷积神经网络}

到目前为止,使用最流行且最被广泛使用的参数共享是应用于计算机视觉的卷积神经网络(CNNs)。

自然图像具有对于平移不变等的许多统计特性。例如,如果猫的照片向右平移一个像素,则猫的照片仍然是猫的照片。卷积神经网络通过在图像上多个位置共享参数来考虑此属性。在输入中的不同位置计算相同的特征(具有相同权重的隐含单元)。这意味着我们可以使用相同的猫检测器找到猫,无论猫出现在图像中的第$i$列还是第$i + 1$列。

参数共享可使卷积神经网络显著减少模型参数的数量,并在不需要增加训练数据的前提下可显著增加网络规模(译注:深度或宽度)。参数共享策略仍然是将领域知识考虑到网络架构中的最好案例之一。

卷积神经网络的更多细节将在第9章讨论。

\section{稀疏表示}

权重衰减通过直接对模型参数施加惩罚来起作用。另一个策略是在神经网络中的单元激活施加惩罚,鼓励它们的激活变得稀疏的。这间接地对模型参数施加了复杂的惩罚。

我们已经讨论了(在7.1.2节中)$L^1$惩罚如何诱导稀疏参数化——这意味着许多参数变为零(或接近零)。另一方面,表示稀疏性描述了一种数据表示,这是一种许多元素为零(或接近零)的表示。这种区别的简化表示可以在线性回归中描述为:
$$
\begin{aligned}
\underset{y ~\in~ \Re^m}{
\begin{bmatrix}
18 \\ 5 \\ 15 \\ -9 \\ -3
\end{bmatrix}} =
\underset{A ~\in~ \Re^{m \times n}}{
\begin{bmatrix}
4 & 0 & 0 & -2 & 0 & 0 \\
0 & 0 & -1 & 0 & 3 & 0 \\
0 & 5 & 0 & 0 & 0 & 0 \\
1 & 0 & 0 & -1 & 0 & -4 \\
1 & 0 & 0 & 0 & -5 & 0
\end{bmatrix}}
\underset{x ~\in~ \Re^n}{
\begin{bmatrix}
2 \\ 3\\ -2\\ -5 \\ 1 \\ 4
\end{bmatrix} }\\
\underset{y ~\in~ \Re^m}{
\begin{bmatrix}
-14 \\ 1 \\ 19 \\ 2 \\ 23
\end{bmatrix}} =
\underset{B ~\in~ \Re^{m \times n}}{
\begin{bmatrix}
3 & -1 & 2 & -5 & 4 & 1 \\
4 & 2 & -3 & -1 & 1 & 3 \\
-1 & 5 & 4 & 2 & -3 & -2 \\
3 & 1 & 2 & -3 & 0 & -3 \\
-5 & 4 & -2 & 2 & -5 & -1
\end{bmatrix}}
\underset{h \in \Re^n}{
\begin{bmatrix}
0 \\ 2 \\ 0 \\ 0 \\ -3 \\ 0
\end{bmatrix} }
\end{aligned}
$$
在第一个表达式中,我们有一个稀疏参数化线性回归模型的例子。在第二个中,我们使用数据$x$的稀疏表示$h$进行线性回归。 也就是说$h$是$x$的函数,在某种意义上,$x$表示存在于$x$中的信息,但是使用稀疏向量。

表示正则化通过我们在参数正则化中使用的相同种类的机制来完成。

通过向损失函数$J$添加对表示的规范惩罚来执行表示的规范惩罚正规化。该惩罚被表示为$\Omega$。如前所述,我们用$\widetilde{J}$表示正则化损失函数:
$$
\begin{aligned}
	\widetilde{J} (\theta; X, y) = J(\theta; X, y) + \alpha \Omega (h)
\end{aligned}
$$
其中$\alpha \in [0, \infty)$表示加权范数惩罚项的相对贡献,较大的$\alpha$值对应于正则化项的加强。

正如参数上的$L^1$正则诱发参数稀疏性一样,对表示元素上的$L^1$正则也会引起表示稀疏性:$\Omega (h) = ||h||_1 = \sum_i |h_i|$。当然,$L^1$正则只是可以导致稀疏表示的正则方法的一个选择。其他的包括来自学生提出的正则(Olshausen和Field,1996;Bergstra,2011)和KL散度正则(Larochelle和Bengio,2008),对元素被限制在单元间隔的情形特别有用。Lee等人(2008)和Goodfellow等人(2009)都提出了基于几个样本$\frac{1}{m} \sum_i h^{(i)}$的平均激活进行正则化,让其接近某个目标值的策略,例如每个元素都为$0.01$的向量。

其他方法获得具有对激活值的硬约束的表示稀疏性。 例如,正交匹配追踪(orthogonal matching pursuit,Pati等人,1993)用表示h对输入x进行编码,来解决约束优化问题。
$$
\begin{aligned}
	\arg_{h, ||h||_0 < k} \min ||x - Wh||^2,
\end{aligned}
$$
其中$||h||_0$是$h$中非零项的数目。当$W$被约束为正交时,这个问题可以被有效解决。此方法通常称为OMP-k,$k$值为允许的非零特征的数量。Coates和Ng(2011)证明OMP-1可以是深层架构非常有效的特征提取器。

基本上任何具有隐藏单元的模型都可以被稀疏化。在本书中,我们将看到许多在各种情景中使用稀疏正则化的例子。

\section{Bagging和其它集成方法}

Bagging(bootstrap aggregating的简写)是一种通过组合几个模型来减少泛化误差的技术(Breiman,1994)。想法是分别训练几个不同的模型,然后让所有的模型投票输出测试示例的预测结果。 这是在机器学习称为模型平均的策略。采用这种策略的技术被称为集成方法(ensemble methods)。

模型平均能够有效的原因是不同的模型通常不会在测试集上的所有样本上产生相同的错误。

考虑一个集成$k$个回归模型的例子。假设每个模型在每个样本上的误差是$\epsilon_i$,误差服从零均值,方差为$E[\epsilon_i^2] = v$且协方差为$E[\epsilon_i \epsilon_j] = c$的多维正态分布。基于所有模型得到的集成模型的平均预测误差是$\frac{1}{k} \sum_i \epsilon_i$。集成模型得到的预测器平方误差的期望是

$$
\begin{aligned}
E \Bigg[\Bigg(\frac{1}{k} \sum_i \epsilon_i \Bigg)^2\Bigg]
& = \frac{1}{k^2} E \Bigg[\sum_i \Bigg(\epsilon_i^2 + \sum_{j \neq i} \epsilon_i \epsilon_j\Bigg)\Bigg], \\
& = \frac{1}{k} v + \frac{k-1}{k} c .
\end{aligned}
$$

在误差完全相关且$c = v$的情况下,均方误差减小到$v$,因此模型平均根本不起作用。在误差完全不相关并且$c = 0$的情况下,集成的预期平方误差仅为$\frac{1}{k}v$。这意味着模型集成的预期平方误差随集成模型的模型数目而线性减小。换句话说,平均而言,集成模型的表现将至少与其所有模型中的任何一个一样好,并且因为每个模型有各自的误差,集合模型相比成员模型的表现得到显著的改善。

不同的集成方法以不同的方式构建模型集成。例如,集成的每个成员模型可以通过使用不同的算法或目标函数训练完全不同种类的模型来形成集成。Bagging是一种允许相同类型的模型、训练算法和目标函数被重复使用多次的方法。

\begin{figure}[htbp] %  figure placement: here, top, bottom, or page
   \centering
   \includegraphics[width=5in]{fig/chap7/7_5.png} 
   \caption{该图是bagging如何工作的卡通描述。假设在上面描述的数据集上训练一个数字8检测器,原始数据集包含数字8、6和9。假设我们做出两个不同的重采样数据集。Bagging训练过程是用替换抽样来构造数据集中的每个样本。第一个数据集忽略掉数字9的数据并用数字8替换。在此数据集上,检测器获知数字顶部的圈对应于8的部分。在第二个数据集上,我们用数字9的数据来替换数字6的数据。在这种情况下,检测器获知数字底部的小圈对应于数字8的部分。每个单独的分类规则是脆弱的,但是如果平均它们的输出,则检测器是鲁棒的,只有当数字8的两个圈同时存在时才实现最大的置信度。}
   \label{fig:7_5}
\end{figure}

具体来说,bagging涉及构建$k$个不同的数据集。每个数据具有与原始数据集相同数量的样本,但每个数据集是通过从原始数据集中替换进行抽样构建的。这意味着有很大概率下,每个数据集都是原始数据集的一个子集,并且(如果在训练集中每次抽样原始数据集约$\frac{2}{3}$样本)还包含若干(译注:与先前抽样得到的数据集)重复的样本。模型$i$基于数据集$i$训练。每个数据集中包括的样本之间的差异导致训练模型之间的差异。参见图7.5的例子。

神经网络被广泛地用在各种解决方案中,即使所有的模型在同一个数据集上训练,他们通常也可从模型平均中获益。随机权重初始化,随机选择小样本批量,超参数的差异或神经网络的非确定性实现的不同结果的差异,通常足以导致集成所采用的不同模型成员产生部分独立的错误。

模型平均是一种用于减少泛化误差的强大且可靠的方法。科学论文的基准算法通常不鼓励使用它,因为任何机器学习算法都可以增加计算和内存为代价从模型平均中获益。因此,基准通常使用单个模型。

机器学习竞赛中常用的策略就是通过使用在数十个模型上的模型平均的方法来获得结果。最近一个突出的例子是Netflix大奖赛中的第一名所使用的模型集成策略(Koren,2009)。

不是所有用于构建集成的技术可使集成后的模型比单个模型更加具有正则化(译注:意味着更低的泛化误差)。例如,称为boosting的技术(Freund和Schapire,1996b,a)可构建具有比单个模型更高容量的集成模型。Boosting已经应用于构建多个神经网络的集成(Schwenk和Bengio,1998),通过向集成模型中增量地添加不同的神经网络。Boosting也被应用于将单个神经网络解释为一个集成模型(Bengio等人,2006a),模型成员是在该单个神经网络逐步添加隐藏单元向一个完整神经网络构建的过程中形成的。

\section{Dropout}

Dropout(Srivastava等人,2014)是一种计算量不大但功能强大的方法,可用来正则一大类模型。第一种近似下,dropout可以被认为是一个方法,使许多大型神经网络能够集成的更实用的Bagging方法。Bagging涉及训练多个模型,并在每个测试样本上评估多个模型。当每个模型是大型神经网络时就不切实际了,因为训练和评估这样的网络在运行时间和所需内存方面的代价大。通常使用五到十个神经网络集成——Szegedy等人(2014a)使用六个神经网络赢得ILSVRC比赛,但超过这个数目会使集成的结果迅速变得笨拙。即使是对指数级数目的神经网络做集成,dropout也是一种低成本的近似训练和评估一个bagged的集成模型的方法。

特别地,dropout训练由所有子网络组成的整体,如图7.6所示,这个整体中的子网络每个可以通过从原本的基础网络中去除某些非输出单元形成。在大多数现代神经网络中,基于一系列的变换和非线性操作,我们可以通过将其输出值乘以零来从网络中有效地移除该单元。该过程需要对模型进行一些轻微修改,如径向基函数网络中网络单元的状态和一些参考值。在这里为简单起见,我们提出了以零乘法为基础的dropout算法,但是它可以被简单地与网络中移除单元的其它操作一起工作。

回想一下,用bagging学习,我们定义k个不同的模型,通过从具有替换的训练集中抽样构建$k$个不同的数据集,然后在数据集$i$上训练模型$i$。Dropout的目的是近似这个过程,而不是构造指数数量规模的神经网络。具体来说,为了借助dropout进行训练,我们使用一个基于小批量样本的学习算法,每次更新是小步骤的,好比随机梯度下降。每次将一个样本加载到一个小批量样本中,我们随机抽样一个不同的二进制掩码,以应用于网络中的所有输入和隐藏单元。每个单元的掩码独立于所有其它单元进行采样。采样掩码值为$1$的概率(也导致一个单元会被包括进去)是在训练开始之前就被确定的超参数。它不是当前模型参数或输入样本的函数。通常以$0.8$的概率包括输入单元,以$0.5$概率包括隐含单元。然后我们照常运行正向、反向传播和学习更新。图7.7说明了如何使用dropout运行正向传播。

更正式地说,假设有掩码向量$\mu$其每个元素为指定包含的单元,并且$J(\theta, \mu)$定义了模型的损失,参数$\theta$和掩码$\mu$定义了模型。dropout训练包括最小化$E_{\mu} J(\theta, \mu)$。期望包含指数个项,但是我们可以通过对掩码向量$\mu$的采样值获得其梯度的无偏估计。

Dropout训练与bagging训练不完全相同。在bagging策略下,模型都是独立的。在dropout策略下,模型共享参数,每个模型继承来自父神经网络参数的不同子集。该参数共享使得用少量的内存就可以表示指数数量的模型。在bagging策略下,每个模型的训练都会收敛其相应的训练集。在dropout策略下,通常大多数模型没有被明确训练——通常当模型足够大时,若有足够多的时间,采样所有可能的子网络是可行的。相反,单步更新是基于采样到的子网络的训练,并且参数共享使得剩余子网络的参数达到较好值。这些是dropout和bagging的唯一区别。除此之外,dropout与bagging算法一致。例如,每个子网络的训练集合是用替换采样原始训练集合的方法得到的子集。

\begin{figure}[htbp] %  figure placement: here, top, bottom, or page
   \centering
   \includegraphics[width=5in]{fig/chap7/7_6.png} 
   \caption{dropout的训练基于所有子网络组成的集成,可以通过从底层基础网络中删除非输出单元后得到的网络来构建。这里,我们从具有两个可见单元和两个隐含单元的基本网络开始。这四个单元有十六中可能的网络子集。改图展示了从原始网络中丢弃不同的单元子集而形成的所有十六个子网络。在这个小例子中,所得到的网络的大部分没有输入单元或没有将输入连接到输出的路径。这个问题相比具有较宽层的网络变得不重要,丢弃从输入到输出的所有可能路径的对较宽层网络的概率更低。}
   \label{fig:7_6}
\end{figure}

\begin{figure}[htbp] %  figure placement: here, top, bottom, or page
   \centering
   \includegraphics[width=3in]{fig/chap7/7_7.png} 
   \caption{使用dropout策略下网络正向传播的示例。(在上图中)有两个输入单元,一个隐藏层和两个隐藏单元以及一个输出单元的前馈网络。(在下图中)是dropout策略下的正向传播,对于网络中的每个输入或隐藏单元,随机抽取矢量$\mu$中的一个元素。$\mu$的元素都是二进制的,并且彼此独立地采样。每个元素为$1$的概率是超参数,对于隐藏层该超参数通常为$0.5$,而输入层通常为$0.8$。网络中的每个单元乘以相应的掩码,然后正常传播继续通过网络的其余部分。这相当于从图7.6中随机选择一个子网络,并向前传播。}
   \label{fig:7_7}
\end{figure}

为了做出预测,一个bagging策略的集成模型的结果必须依赖所有成员的累积投票。我们将这个过程称为推断。到目前为止,我们对bagging和dropout的描述不需要模型明确的概率。现在,我们假设模型的作用是输出概率分布。 在装袋的情况下,每个模型$i$产生概率分布$p^{(i)} (y|x)$。整体的预测由所有这些分布的算术平均值给出,
$$
\begin{aligned}
	\frac{1}{k} \sum_{i=1}^{k} p^{(i)} (y | x).
\end{aligned}
$$
在dropout策略下,由掩码向量$\mu$定义的每个子模型,每个自模型定义概率分布$p(y | x, \mu)$。所有掩码的算术平均值由下式给出
$$
\begin{aligned}
	\sum_{\mu} p(\mu) p(y|x,\mu)
\end{aligned}
$$
其中$p(\mu)$是用于在训练时间采样$\mu$的概率分布。

因为这个加和包括指数数目的项,除非模型结构允许某种形式简化的情况下,否则难以评估。到目前为止,深层神经网不允许任何易于理解的简化。相反,我们可以用抽样来近似推理,通过对许多掩码的输出求平均来近似。即使是10-20个掩码,通常也足以获得良好的性能。

然而有一个更好的方法,能够仅以一个前向传播的代价获得整个集成模型预测的良好近似。为此,在集成模型成员的预测分布上我们从几何平均数改用算术平均数。Warde-Farley等人(2014)提出的论点和实证证据表明在这种情况下几何平均值与算术平均值等价。

多个概率分布的几何平均值不能保证为概率分布。为了保证结果是一个概率分布,我们强加的要求是,没有一个子模型将概率$0$分配给任何事件,并且我们对结果分布进行重新归一化。由几何平均直接定义的非规格化概率分布(unnormalized probability distribution)由下式给出
$$
\begin{aligned}
\tilde{p}_{\text{ensemble}}(y \mid x) = \sqrt[2^d]{\prod_{\mu} p(y \mid x, \mu)},
\end{aligned}
$$
其中$d$是可以丢弃的单元数。这里我们使用均匀分布$\mu$来简化表示,但非均匀分布也是可能的。为了做出预测,我们必须重新标准化集成模型(re-normalize the ensemble):
$$
\begin{aligned}
p_{\text{ensemble}}(y \mid x) = \frac{\tilde{p}_{\text{ensemble}}(y \mid x)}
{\sum_{y'}\tilde{p}_{\text{ensemble}}(y' \mid x) }.
\end{aligned}
$$
涉及dropout的一个关键的见解(Hinton等人,2012c)是,我们可以在一个模型中评估$p(y|x)$来近似$p_{\text{ensemble}}$:具有所有单元的模型,但是单元$i$的权重乘以包括单元$i$的概率。此修改的动机是捕获该单元输出的正确期望值。我们将这种方法称为权重缩放推理规则(weight scaling inference rule)。在深度非线性网络中,这个近似推理规则的精度还没有任何理论论证,但在实验中它表现很好。

因为我们通常使用$\frac{1}{2}$的包含概率,所以权重缩放规则通常等于在训练结束时将权重除以$2$,之后的过程就和往常无异。实现相同结果的另一方式是在训练期间将单元的状态乘以$2$。无论哪种方式,目标是确保在测试时单元的预期总输入与在训练时到该单元的预期总输入大致相同,即使训练中平均下来会有一半的单元会丢失。

对于没有非线性隐含单元的许多类型的模型而言,加权缩放推理规则(weight scaling inference rule)是精确的。对于一个简单的例子,考虑一个softmax回归分类器,输入变量是有着$n$个元素的向量$v$表示:
$$
\begin{aligned}
P(y = y \mid v) = \text{softmax}\big(W^\top v + b \big)_y.
\end{aligned}
$$
我们可以通过输入的二进制向量 $d$ 的逐元素乘法来索引到子模型族中:
$$
\begin{aligned}
P(y = y \mid v; d) = \text{softmax}\big(W^\top(d \odot v) + b \big)_y.
\end{aligned}
$$
集成结果的预测是根据集成的所有模型成员预测的几何平均值进行重新规范化来定义从而得到的预测值:
$$
\begin{aligned}
P_{\text{ensemble}}(y = y \mid v) = \frac{\tilde{P}_{\text{ensemble}}(y = y \mid v)}
{\sum_{y'}\tilde{P}_{\text{ensemble}}(y = y' \mid v) },
\end{aligned}
$$
其中有
$$
\begin{aligned}
\tilde{P}_{\text{ensemble}}(y=y \mid v) =
\sqrt[2^n]{\prod_{d \in \{0,1\}^n} P(y = y \mid v; d)}.
\end{aligned}
$$
为了看到权重缩放规则是精确的,我们可以简化$\widetilde{P}_{ensemble}$:
$$
\begin{aligned}
\tilde{P}_{\text{ensemble}}(y=y \mid v) =
\sqrt[2^n]{\prod_{d \in \{0,1\}^n} P(y = y \mid v; d)} \\
= \sqrt[2^n]{\prod_{d \in \{0,1\}^n} \text{softmax}(W^\top(d \odot v) + b)_y} \\
= \sqrt[2^n]{\prod_{d \in \{0,1\}^n} \frac{\exp (W_{y,:}^\top(d \odot v) + b_y)}
{\sum_{y'}\exp (W_{y',;}^\top(d \odot v) + b_{y'})}}\\
= \frac{\sqrt[2^n]{\prod_{d \in \{0,1\}^n}\exp (W_{y,:}^\top(d \odot v) + b_y)}}
{ \sqrt[2^n] \prod_{d \in \{0,1\}^n} \sum_{y'}\exp (W_{y',:}^\top(d \odot v) + b_{y'})}
\end{aligned}
$$
因为$\widetilde{P}$将被归一化,我们可以安全地忽略与因子的乘法操作,其中因子相对于$y$是常数:
$$
\begin{aligned}
\tilde{P}_{\text{ensemble}}(y=y \mid v) &\propto
\sqrt[2^n]{\prod_{d \in \{0,1\}^n} \exp (W_{y,:}^\top(d \odot v) + b_y)} \\
& = \exp \Bigg(\frac{1}{2^n} \sum_{d \in \{0,1\}^n} W_{y,;}^\top(d \odot v) + b_y \Bigg) \\
& = \exp \Big(\frac{1}{2}W_{y,:}^\top v + b_y \Big) .
\end{aligned}
$$
将其代入方程7.58($\begin{aligned} P_{\text{ensemble}}(y = y \mid v) = \frac{\tilde{P}_{\text{ensemble}}(y = y \mid v)} {\sum_{y’}\tilde{P}_{\text{ensemble}}(y = y’ \mid v) }, \end{aligned}$),我们得到一个softmax分类器,权重为$\frac{1}{2}W$。

权重缩放规则在其它场景的设定中也是精确的,包括具有条件正态输出的回归网络,以及无在隐含层没有非线性单元的深层网络。然而,权重缩放规则只是对具有非线性的深层模型的一种估计。虽然近似没有在理论上的足够证明,但在实际中却有效。Goodfellow等人(2013a)实验发现,对于集成模型,权重缩放近似可以比蒙特卡罗近似效果更好(在分类的精度方面)。即使当蒙特卡罗近似允许采样多达1,000个子网络时,这也成立。Gal和Ghahramani(2015)发现,一些模型使用二十个样本和蒙特卡罗近似可以获得更好的分类精度。这看来推理近似的最佳方法选择是和问题相关的。

Srivastava等人(2014)表明,dropout比其它代价低的标准正则化算法更有效,例如权重衰减(weight decay),过滤范数约束(filter norm constraints)和稀疏活动正则化(sparse activity regularization)。Dropout还可以与其它形式的正则化组合实现其进一步改进。

Dropout的一个优点是它在计算效率高。在训练期间使用dropout每次更新的每个样本仅需要$O(n)$时间复杂度的计算,以生成$n$个随机二进制数并将它们乘以各自神经元的状态。该过程实现后,它还可能需要$O(n)$的空间复杂度来存储这些二进制数,直到反向传播阶段。尽管在开始对样本执行推断前必须将权重除以$2$,但在训练模型时的推理具有的损失在每个样本上的计算都是相同的,就好像我们没有使用dropout策略。

Dropout的另一个重要优点是它不会很大程度地限制所使用模型的类型或训练算法。它几乎适用于任何以分布式表征样本、且可以使用随机梯度下降训练的模型。这包括前馈神经网络,概率模型如受限玻尔兹曼机(restricted Boltzmann machines,Srivastava等人,2014)和递归神经网络(recurrent neural networks,Bayer和Osendorfer,2014;Pascanu等人,2014a)。许多其它类似强有力的正则化策略对模型的架构施加了更严格的限制。

虽然将dropout应用到特定模型的每步损失值是可以忽略的,但在完整系统中使用dropout的损失值可能是明显的。因为dropout是一种正则化技术,它降低了模型对数据的有效容量。要抵消这种影响,我们必须增加模型的大小。通常当使用dropout时的最佳验证集误差要比不用时低得多,但是这是以大得多的模型和更多迭代步数的训练算法为代价的。对于非常大的数据集,正则化使得泛化误差几乎没有减少。在这些情况下,使用dropout的较大模型的计算成本的代价要超过常规正则化手段带来的好处。

当带标记的训练本数量很小时,dropout不太有效。贝叶斯神经网络(Neal,1996)在替代拼接数据集(Alternative Splicing Dataset,Xiong等人,2011)上的表现优于dropout策略,替代拼接数据集(Alternative Splicing Dataset)中的可用样本数少于5,000(Srivastava等人,2014)。当有额外的未标记数据可用时,无监督特征学习可以获得优于dropout的优点。

Wager等人(2013)表明,当应用于线性回归时,dropout相当于 $L^2$ 权重衰减的正则化法,每个输入特征具有不同的权重衰减系数。每个特征的正则化系数的大小由其方差确定。类似的结果适用于其它线性模型。对于深度模型,辍学不等于权重衰减的正则化法。

在使用dropout训练时所使用的随机性对于该方法的成功不是必需的。它只是近似所有子模型之和的一种手段。Wang和Manning(2013)衍生出了边缘化的分析近似。它们的近似由于在梯度计算中随机性会减少,被称为快速dropout,可以加快收敛时间。该方法还可在测试时使用,作为在所有子网络上做平均的、比起权重缩放近似更为原始的(但计算代价大)近似法。在小规模神经网络对问题的解决性能上,快速dropout已接近标准的性能表现,但尚未产生显著的性能提升或应用于大问题上。

正如随机性不是dropout实现正则化效果所必须的,dropout的正则能力也是不够的。为了证明这点,Warde-Farley等人(2014)在控制实验中设计了一种称为dropout boosting的方法,dropout boosting使用与dropout完全相同的掩码噪声,但缺乏正则的效果。dropout boosting训练整个集成模型联合地最大化训练集上的对数似然。在传统意义上,传统的dropout类似于bagging,这种方法类似于boosting。如预期的那样,与将整个网络作为单个模型进行训练相比,使用dropout boosting的实验几乎表现不出正则化效果。这表明,作为bagging的dropout的解释具有超过对噪声鲁棒性的dropout的解释的价值。bagged集成模型的正则化效果,仅在随机抽样集成模型中的每单个成员模型表现良好时才能实现。

Dropout已经启发了其它随机方法来训练具有指数量级数目的模型在集成时共享权重。DropConnect是一种特殊情况,其中单个标量权重和单个隐藏单元状态之间的每个乘积被认为是可以舍弃的单元(Wan等人,2013)。随机池化(stochastic pooling)是用于建立单个卷积网络集成到一起的随机池化形式(参见第9.3节),其中每个卷积网络给出每个特征图的不同空间位置。到目前为止,dropout仍然是最被广泛使用的隐式集成方法。

Dropout的一个关键想法是,训练具有随机行为的网络,并通过对多个随机决策取平均来做出预测,实现了一种具有参数共享的bagging形式。早些时候,我们描述了dropout作为对一个由可留或可丢弃单元形成的模型集合的bagging。然而,这个模型平均策略不需要基于单元的保留和舍弃。原则上,任何种类的随机修改都是允许的。在实践中,我们必须选择神经网络能够学习抵抗的一系列修正策略。理想情况下,我们还应该使用允许快速近似推理规则的模型集合。我们可以认为由向量$\mu$参数化的任一修改形式,都作为训练所有$\mu$可能取值的$p(y | x, \mu)$的集成模型(We can think of any form of modification parametrized by a vector $\mu$ as training an ensemble consisting of $p(y | x, \mu)$ for all possible values of $\mu$.)。没有要求$\mu$具有一定数量的值。例如,$\mu$可以是实值的。Srivastava等人(2014)表明,将权重乘以$\mu N(1,I)$可以优于基于二进制掩码的dropout策略。因为$E[\mu] = 1$,所以标准网络自动在集成模型中实现近似推理,而不需要任何权重缩放。

到目前为止,我们已经描述了dropout完全作为一种高效,近似bagging的策略。然而,还有另一个更加深入的dropout观点。Dropout训练不仅是一个bagging的模型集成策略,而是共享隐含单元模型的模型集合策略。这意味着每个隐含单元必须能够很好地执行,而不管模型中存在哪些其它隐含单元。隐含单元必须准备在模型之间交换和互换。Hinton等人(2012c)的灵感来自生物学的一个想法:有性生殖,涉及在两个不同的生物之间交换基因,创造进化压力的基因,变得不但好且在不同生物间变得容易交换。这样的基因和此类特征对于其环境中的变化是非常鲁棒的,因为它们不能不正确地适应任何一种生物或模型的不寻常特征。因此,dropout正则化后的每个隐含单元,不仅是一个好特征,而且该特征在许多环境下都表现良好。Warde-Farley等人(2014)将dropout训练与大的集成模型的训练进行了比较,并得出结论,dropout在泛化误差的提升上远超独立模型集成得到的改进。

重要的是要理解,dropout起作用的原因大部分来自掩码噪声被施加到隐含单元上这一事实。这可以被看作是对输入信息内容的高度智能自适应破坏的形式,而不是破坏输入的原始值。例如,如果模型通过学习隐藏单元$h_i$来确定鼻子从而检测人脸,那么当$h_i$被丢掉就相当于图像中存在鼻子的信息被擦除了。模型必须学习另一个$h_i$,或者对鼻子的存在进行冗余编码,或者通过另一个特征(例如嘴)来检测人脸。除非噪声很大以至于图像中几乎所有信息都被去除,否则在输入处添加非结构化噪声的传统噪声注入技术不能从面部图像随机擦除关于鼻子的信息。破坏提取的特征而不是原始特征,才能使得破坏过程可以利用模型到目前为止从输入数据的分布获取到的所有知识。

Dropout的另一个重要方面是噪声实现过程是乘法的。如果噪声与固定尺度相加,则具有附加噪声$\epsilon$的线性校正单元(ReLU)的激活隐含单元$h_i$可以轻易地学习这会使$h_i$值变得非常大,以便通过比较使添加的噪声$\epsilon$不明显(If the noise were additive with fixed scale, then a rectified linear hidden unit hi with added noise $\epsilon$ could simply learn to have hi become very large in order to make the added noise $\epsilon$ insignificant by comparison.)。乘法噪声不允许这种针对噪声鲁棒性问题的异常解决方案。

另一种深度学习算法,批量归一化(batch normalization),是在训练时在隐含单元上引入加性和乘性噪声的方式对模型参数进行重新参数化的方法。批量归一化的主要目的是提升优化性能,噪声也有正则化效果,并且有时使dropout变得不必要。在第8.7.1节中详细描述了批量归一化(batch normalization)。

\section{对抗训练}

在许多情况下,当在独立同分布的测试集上评估时,神经网络已经开始达到人类的表现性能。因此会自然想知道这些模型对这些任务是否达到同人类一样的理解能力。为了探索网络对底层任务的理解水平,我们可以研究模型分类错误的样本。Szegedy等人(2014b)发现,性能接近人类的神经网络在某些样本点上具有接近$100%$的误差率,这些样本点是使用优化过程搜索到的数据点,例如搜索到与输入点$x^{'}$接近的点是$x$,使得模型输出与在$x^{'}$样本点处非常不同。在许多情况下,$x^{'}$可以非常近似于$x$,人类观察者不能区分原始样本和对抗样本之间的区别,但是网络可以做出高度不同的预测,参见图7.8的例子。

对抗性样本有许多含义,例如超出本章知识范围的计算机安全方面。然而,它在正则化中很有趣,通过对抗训练训练来自训练集的对抗扰动样本,可以减少原始独立同分步的测试集上的错误率(Szegedy等人,2014b;Goodfellow等人,2014b )。

\begin{figure}[htbp] %  figure placement: here, top, bottom, or page
   \centering
   \includegraphics[width=5in]{fig/chap7/7_8.png} 
   \caption{在ImageNet上应用于GoogLeNet(Szegedy等人,2014a)生成对抗样本的演示。通过添加一个不可察觉的小向量,其元素等于损失函数相对于输入的梯度的元素的符号(the sign of the elements),我们可以改变GoogLeNet对图像的分类。经Goodfellow等人许可转载(2014b)。}
   \label{fig:7_8}
\end{figure}

Goodfellow等人(2014b)表明,这些对抗性样本的主要原因之一是过度的线性。神经网络主要由线性构建块构成。在一些实验中,它们实现的总体功能证明是高度线性的。这些线性函数易于优化。不幸的是,如果线性函数具有许多输入,它的值可以非常快速地改变。如果我们通过$\epsilon $改变每个输入,则具有权重$w$的线性函数值可以被越来越大的$\epsilon ||w||_1$改变,如果$w$是高维的,则计算出的值是非常大的。对抗训练通过鼓励网络局部恒定训练数据的邻域(locally constantin the neighborhood of the training data)来阻止这种高度敏感的局部线性行为。这可以被看作是在监督神经网络中显式地引入局部恒定性(local constancy prior)的方式。

对抗训练有助于说明结合使用大型函数家族和积极正则化力量。纯线性模型,如逻辑回归,不能抵抗对抗样本,因为它们是被强制线性的。神经网络能够表示从接近线性到几乎局部常量值的函数,并因此具有捕获训练数据中线性趋势,并同时学习抵抗局部扰动的灵活性。

对抗样本还提供了完成半监督学习的手段。在与数据集中的标签不相关联的点$x$处,模型本身分配一些标签$y$。模型的标签$y$可能不是真实的标签,但是如果模型是高质量的,则$y$具有提供真实标签的高概率。我们可以选找一个对抗样本,使得分类器输出一个$y^{'} \neq y$的标签$y$。使用不是真实标签而是由训练模型提供的标签产生的对抗性样本被称为虚拟对抗性样本(virtual adversarial examples,Miyato等人,2015)。分类器也许会把相同的标签分配给$x$和$x^{'}$。这鼓励分类器学习一个函数,该函数对于未标记数据所在的流形中的任何地方的小变化是鲁棒的。驱动这种方法的假设是不同的类通常位于断开的流形空间上,并且小的扰动不应该能够从一个类的流形空间跳到另一个类的流形空间上。

\section{切线距离与切线传播和流形切线分类器}

许多机器学习算法旨在通过假设数据位于低维流行空间附近来克服维数诅咒,如第5.11.3节所述。

早期利用流形假设的尝试之一是切线距离算法(Simard等人,1993,1998)。它是一种非参数最近邻算法,其中使用的度量不是通用欧几里得距离,而是一种从流形空间衍生出的度量方法,接近概率集中附近的流形空间。假设我们试图对实例进行分类,并且处在同一流形空间的样本具有相同的类别。由于分类器应该对流形空间局部的变化因子保持不变,因此使用点$x_1$和$x_2$之间的最近相邻距离作为它们分别属于的流形$M_1$和$M_2$之间的距离是有意义的。虽然在计算上有困难(需要解决优化问题,找到$M_1$和$M_2$上最接近的点对),但是在局部有意义的一种简便的替代方案是通过其在$x_i$处的切平面近似$M_i$,并且测量两个切线,或者在切平面和点之间策略。这可以通过解决低维线性系统(在流形空间的维度中)来实现。当然,该算法需要指定切向量。

在相关方法中,切线传播算法(Simard等人,1992)(图7.9)训练一个具有额外惩罚的神经网络分类器,使得神经网络的每个输出$f(x)$值局部地对变化因子保持不变性。这些变化因子对应于沿着流形的移动,相同类别的样本在该流形空间的附近集中。局部不变性通过要求$\triangledown_x f(x)$与$x$处的已知流形空间的切向量$v^{(i)}$正交来实现,或者等效地,在方向$v^{(i)}$位于样本$x$处的$f$的方向导数,通过增加正则化惩罚$\Omega$:
$$
\begin{aligned} \label{eq:767}
\Omega(f) = \sum_i \Big((\nabla_{x} f(x)^\top v^{(i)}) \Big)^2 .
\end{aligned}
$$
这个正则化器当然可以通过适当的超参数缩放,并且对于大多数神经网络,我们将需要对许多输出求和,而不是这里为了简单起见描述的单个输出$f(x)$。与切线距离算法一样,切向量通常是从图像中的变换效果(例如平移,旋转和缩放)的形式知识这样的先验推导出来的。切线传播不仅用于监督学习(Simard等人,1992),而且也适用在强化学习的背景下(Thrun,1995)。

切线传播与数据集扩增(data augmentation)密切相关。在这两种情况下,在不改变网络原本样本的输出情况下,算法使用者将指定一系列数据集图像扩增来获取更多图像数据,并在此过程中考虑该任务的先验知识。不同的是,在数据集增加的情况下,网络被显式地训练以正确地分类通过对原始图像变换从而创建的极少的不同输入。切线传播不需要显式访问新的输入点。相反,它分析地正则化模型以抵抗在与变换相对应方向上的扰动。虽然这种分析方法在理智上是优雅的,但它有两个主要的缺点。首先,它只是将模型正则化以抵抗极少的扰动。显式数据集赋予模型对较大扰动的抵抗力。第二,极少数方法对基于校正线性单元(ReLU)的模型造成了不利影响。这些模型只能通过关闭单元或缩小其权重来收缩其导数。它们不能通过以大的权重以高的值饱和而收缩它们的导数值,如S型或$tanh$激活函数。数据集增加可与线性校正单元(ReLU)很好地兼容,因为线性校正单元(ReLU)的不同子集可以激活每个原始输入的不同变换版本。

\begin{figure}[htbp] %  figure placement: here, top, bottom, or page
   \centering
   \includegraphics[width=3in]{fig/chap7/7_9.png} 
   \caption{该图描述了正切传播算法(Simard等人,1992)和流形正切分类器(Rifai等人,2011c)的主要思想,它们都对分类器的输出函数$f(x)$正则化。每条曲线表示不同类别的流形空间,这里显示的是嵌入在二维空间中的一维流形空间。在一个曲线中,我们选择单个点并绘制与该类相切的向量(平行于并且接触流形空间)和垂直于类流形空间(与流形空间正交)的向量。在多个维度中,可以存在许多切线方向和许多法线方向。我们期望分类函数随着其在垂直于流形的方向上移动而快速改变,并且不随着其沿类流形空间移动而改变。切向传播和流形切线分类器正则化$f(x)$不随着$x$沿流形移动而改变很多。流形切线分类器通过训练自动编码器来拟合训练数据来估计流形切线方向,切向传播需要用户手动指定计算切线方向的函数(例如在相同类的流形中指定图像的小平移保持)。使用自动编码器估计流形将在第14章中描述。}
   \label{fig:7_9}
\end{figure}

切线传播也与\textbf{双重反向传播}(Drucker和LeCun,1992)、对抗训练(Szegedy等人,2014b;Goodfellow等人,2014b)有关。双重反向传播的正则会使Jacobian矩阵变小,而对抗训练在原始输入附近找与之接近的输,并训练模型产生与原始输入一样的输出。切线传播和指定变化方法的数据集扩增这两个策略都要求模型对于输入中的某些指定的变化方向具有一定的不变性。只要变化很小,双重反向传播和对抗训练在模型对输入样本在所有变化方向都有不变性。正如数据集扩增是切线传播的非极少数版本,对抗训练是双重反向传播的非极少数版本(Just as dataset augmentation is the non-infinitesimal version of tangent propagation, adversarial training is the non-infinitesimal version of double backprop.)。

流形切线分类器(Rifai等人,2011c),消除了需要知道切向量的这一先验。如我们将在第14章中看到的,自动编码器可以估计流形切线向量。流形切线分类器使用这种技术,以避免需要用户指定切向量。如图14.10所示,这些估计的切线向量超出了由图像的几何形状(例如平移,旋转和缩放)产生的经典不变量,并且包括因为是特定对象的(例如移动身体部分),也是必须被学习的因素。因此,用流形切线分类提出的算法很简单:(1)使用自动编码器通过无监督学习来学习流形结构,(2)使用这些切线在切线传播中正则化神经网络分类器(方程7.67)。

本章描述了用于正则化神经网络的大多数策略。正则化是机器学习的中心主题,因此其余正则化策略将定期重新讨论。机器学习的另一个中心主题是优化,见下一章节。
\chapter{训练深度模型的优化方法}
\label{chap:8}
%%%%%%%%%%%%%%%%%%%%%%%%%%%%%%%%%%%%%%%%%%%%%%%%%%%%%%%%%
%%%%%%%%%%%%%%%%%%% author:wzwei1636@163.com  %%%%%%%%%%%
%%%%%%%%%%%%%%%%%%% part:8.0-8.2              %%%%%%%%%%%
%%%%%%%%%%%%%%%%%%%%%%%%%%%%%%%%%%%%%%%%%%%%%%%%%%%%%%%%%
\section{8.0}


深度学习算法在许多场景下涉及到最优化。例如,在PCA模型的性能推断上涉及解决一个最优化问题。我们经常用最优化解析方法去证明或者设计算法。在深度学习包含的所有最优化问题中,最困难的是神经网络训练。甚至https://preview.overleaf.com/public/yqhdxhwbpfpr/images/da8c750f57110730fbd2f2805ecef1d164d8f7d3.jpeg为了解决神经网络训练的一个实例,花费几天或者上月的时间在几百台机器上都是很常见的,因为这个问题非常重要且代价很大,然而我们已经开发了一类特殊的最优化技术去解决它。本章将给出训练神经网络的最优化技术。

如果你不熟悉基于梯度的最优化方法的基本理论,我们推荐复习第四章。那一章包含简短的数值最优化方法的概述。

本章集中于一个特殊的最优化场景,找到能够显著减少神经网络损失函数$J(\theta)$的参数$\theta$,通常包含在整个训练集评估的性能度量以及额外的正则化项。

刚开始,我们将描述对于一个机器学习任务,在训练算法上使用的最优化方法有别于纯粹的最优化。下一步,我们将给出一些具体的优化神经网络的难点。然后定义一些实践上的算法,包含最优化算法本身和初始化参数的一些策略。更加高级的算法在训练或者利用包含在损失函数的二阶导数来调整学习率。最后,通过回顾一系列的优化策略,我们可以将一些简单的最优化算法结合转化为更高层次的程序。

\section{学习和优化有什么不同}

训练深度模型中使用的最优化算法在某些方面不同于传统的最优化算法。机器学习经常表现的间接。在大部分机器学习的场景中,我们关注性能度量$P$,它是在测试集中定义的,也是非常棘手的。我们仅仅间接优化P,我们希望通过减少损失函数$J(\Vtheta)$提升性能$P$。相比之下,纯粹的最优化方法,其目的在于最小化$J(\Vtheta)$。训练深度模型的最优化算法通常也包括在机器学习目标函数的特殊结构上的一些特性。

通常,损失函数可写为训练集上的平均,如

\begin{equation}
J(\theta)=E_{(x,y)\thicksim\hat{P}_{data}}L(f(x;\theta),y)
\end{equation}

其中L是每个样本的损失函数,当输入为x,$f(x;\theta)$是预测的输出值,$\hat{p}_{data}$在监督学习中,y是目标值。整个这章,我们开发非正则化监督例子,其中对于L,参数是$f(x;\theta)$和y。然而,很轻易去扩展,例如,为了开发正则化的各种形式或非监督学习,在参数中包含$\theta$或x,排除y。

等式8.1定义了一个关于训练集的目标函数。我们通常对相应的目标函数最小化,其中期望 是在数据的生成分布$P_{data}$而不是有限的训练集上。

\begin{equation}
J^*(\theta)=E_{(x,y)\thicksim\hat{P}_{data}}L(f(x;\theta),y)
\end{equation}

\subsection{经验风险最小化}

机器学习算法的目标是在等式减小泛化误差,即所谓的风险。需要强调的是期望是取值在真实潜在分布$p_{data}$上的。如果我们知道真实分布$P_{data}(x,y)$,风险最小化将由最优化算法解决。然而,当我们不知道$P_{data}(x,y)$,仅仅有一些训练样本时,就遇到了一个机器学习问题。

最简单的方式是将机器学习问题转化为最优化问题,通过在训练集上最小化期望损失。也就意味着用经验分布$\hat{P}(x,y)$替换真实分布p(x,y)。最小化经验风险:

\begin{equation}
E_{(x,y)\thicksim\hat{P}_{data}}[L(f(x;\theta),y)]=\frac{1}{m}\sum_{i=1}^mL(f(x^{(i)};\theta),y^{(i)})
\end{equation}

其中m是训练样本的个数。

基于最小化平均训练误差的训练过程即为经验风险最小化。并不是直接优化风险,我们最优化经验风险,希望经验风险显著减小。基于不同的条件各种理论结果建立起来了,真实的风险是可以通过不同的变量减小。

然后,最小化经验风险易于过拟合。在许多情况下,最小化经验风险并不真正可行。大部分有效的模型最优化算法是基于梯度下降的,然而许多有用的损失函数,例如0-1损失函数,没有导数。这里两个问题意味着,在深度学习中,我们很难最小化经验损失。相反,我们必须使用略有不同的方法,我们真正优化的量甚至不同于我们真实想要优化的量。

\subsection{替化损失函数和提前终止}

有时,损失函数我们真正关心的(分类误差)并不是有效的优化。例如,严格最小化期望0-1损失函数明显棘手的(在维数上是指数的),甚至是对于一个线性分类器(Marcotte and Savard, 1992).在这种情况下,一个典型的优化是用代理损失函数替代,充当一个代理但是却非常有效。例如,正例的非负对数似然就是替代0-1损失的。非负对数似然允许模型估计类别的条件概率。给定输入,若模型确实有效,则选择最小期望分类误差。

在某些情况下,一个代理损失函数事实上导致可以学到更多。例如,当训练中用对数似然替代时,当训练集上0-1损失函数为0时,在测试集上0-1损失函数通常连续减少。这是因为尽管0-1损失是0,我们可以提高分类器的鲁棒性进一步的区分不同的类,获取更可靠的分类器,因此相比于在训练集上简单的最小化平均0-1损失,能获取更多的信息。

一般的最优化方法和深度学习中的最优化算法的一个重要差异在于,我们在使用训练算法时,通常不会再局部最小值停止。相反,当基于提前停止的收敛规则满足时,一个机器学习算法通常最小化一个代理损失函数停止。典型的,提前停止规则是基于真实的潜在的损失函数,例如在验证集上度量的0-1损失,当过拟合开始发生时,算法停止。尽管替代的损失函数依然有比较大的导数,训练能停止。这非常不同于纯粹的最优化方法,其中最优化算法当梯度非常小时,被认为是收敛的。

\subsection{批算法和小批算法}

机器学习算法不同于一般的最优化算法的一个方面在于为目标函数在训练集上被分解为为训练样本上的求和。机器学习优化算法通常使用整个损失函数中的一部分项去更新基于估计损失函数的期望值。
例如,在对数空间中,极大似然估计问题可以分解成每个样本的和:

\begin{equation}
\theta_{ML}=argmax_\theta\sum_{i=1}^m log p_{model}(x^{(i)},y^{(i)};\theta)
\end{equation}

最大化此项等价于最大化定义在训练集上的经验分布的期望:

\begin{equation}
J(\theta)=E_{(x,y)\thicksim\hat{P}_{data}} log p_{model}(x,y;\theta)
\end{equation}

目标函数的多数性质被大部分最优化算法所使用的也是训练集上的期望。例如,通常大部分使用的性质是梯度:

\begin{equation}
\nabla_{\theta}J(\theta)=E_{(x,y)\thicksim\hat{P}_{data}}\nabla_{\theta} log p_{model}(x,y;\theta)
\end{equation}

准确计算这个期望代价非常大,因为它需要在整个数据集的每个实例上评估模型。实践中,我们通过对数据集进行随机抽样来计算这些期望,然后再在实例上取均值。
    
回忆n个样本均值的标准差$\frac{\sigma}{\sqrt{n}}$,其中$\sigma$是这些样本的标准差。在100个实例和1000个实例上比较梯度的两个假设估计,后者比前者需要多出100倍的计算,但是平均能减少10倍的标准差。大多数的最优化算法当他们能快速近似计算梯度时往往比缓慢计算的严格梯度要收敛的更快。

其他有助于来自小样本的梯度统计估计在训练样本集上是多余的,最糟糕的情况,训练集中的所有m个样本,是彼此相同的。一个基于抽样的梯度估计通过m个样本能正确计算,比使用朴素方法计算次数要少。实践中,我们不可能遇到这种情况,但是我们可以使用大样本量计算梯度。

最优化算法使用整个训练集称之为批或确定梯度方法,因为它们通过一个大的批次同时计算所有的样本量。这个术语可能与 “batch”有点混淆,其经常被使用其描述由最小批处理随机梯度下降的分批处理,典型的,术语批随机梯度下降暗含着真个训练集的使用,同时术语batch的使用并不是描述着一组样本。例如,通常使用术语“bitch size”描绘批梯度的规模。

最优化算法有时一次使用一个实例称之为随机或者在线方法。术语在线有时被保留的情况下来自于,从一个不断创建的例子,而不是传输固定大小的训练集。

深度学习使用的大部分算法介于两者之间,即使用一个以上或者少于所以的训练样本。传统上被称之为批处理或者批处理随机方法,现在统一简称为随机方法。

随机方法的标准范例是随机梯度下降,在8.3.1会详细说明。

批规模由下面几个因素驱动:
1.更大的批次提供一个更加精确的梯度估计,但是不是呈线性的返回。
2.在极小的批次下,多核架构一般未被充分利用,这个激发使用一些绝对数量的最小批次,在处理一个批上时间上并没有减少。
3.如果批次中所有的示例被并行处理,则大量的内存被批规模消耗掉。许多硬件限制了批的规模。
4.各种硬件通过使用特定的数组可以获得更好的运行,特别是在使用GPUs的过程中,2次幂的批规模来提供更好的性能。明显的2的幂次方的批规模从32变化到256,为了尝试大的模型,可能是16.
5.小批次可能提供一个正则化的影响,可能因为噪声被被添加到学习过程,泛化误差对于规模为1的批次是最好的。在这个规模上进行训练可能需要小的学习速率来维持因为在梯度估计中的高方差。总的运行时间可能非常高因为需要采取更多的步数,均是因为学习速率的减小和需要采取更多的步数来观测整个训练集。

不同的算法通过不同的方式使用最小批的各种信息。一些算法对样本误差非常敏感,可能是因为他们使用非常少的样本取估计准确度,或者是因为他们放大样本误差。仅仅基于梯度g计算更新通常相对健壮,他们可以处理如100的批规模。二阶方法,通过使用Hessian矩阵H,通过计算$H^{-1}g$进行更新,典型的需要如10000的更大的批次。这些大的批次规模需要最小化估计$H^{-1}g$的波动。设想H使用非常少的条件数被完美估计。乘以H或者它的逆放大预先存在的错误,这种情况下,在g中存在估计误差,估计g中非常小的变化能引起更新$H^{-1}g$的巨大变化,甚至当H被完美估计。当然H只能被近似估计,因此更新$H^{-1}g$将比我们通过应用少量的条件去估计g来预测会包含更多的错误。

最小批被随机选择是非常重要的,计算样本的期望梯度的无偏估计需要样本是独立的,我们也希望两个随后的梯度估计是相互独立的,我们希望两个随后的梯度估计相互独立,许多数据集一般被设计成许多连续的例子是高度相关的。例如,我们可能有一系列的关于血液样本的医药数据。对第一个病人,在不同时间获取5例血液样本,然后从第二个病人那里获取3例血液样本,然后依次进行这样的操作,如果我们按照顺序从中取出实例,则每个minibatches将是极度有偏的,因为它将代表数据集中许多患者的最显著的那个。在这种情况下,数据集中实例的顺序将有重要意义,因为必须在选择minibatches之前必须打乱实例的顺序。对于大数据集而言,例如一个数据中心包含几十亿的实例,每次随机抽样取构造

minibatches是不现实的。幸运的是,实践上足够去打乱实例的顺序,并以无序的形式存储。这将强加一个固定连续的minibatches,从而在所有的模型训练中使用,当使用整个数据集时,每一个单独的模型将被迫重用这个排序。然而,来自真正随机选择的偏差看起来并没有一个显著的有害影响。不打乱实例的顺序将严重减弱算法的有效性。

机器学习中的许多最优化问题可以在训练集中进行分解,从而我们可以并行的不同的实例上计算整个单独的更新。换言之,我们可以计算更新能够同时对于每个X的minibatch最小化J(x),从而对多个minibatch计算更新。这些异步并行分布式方法在12.1.3将被讨论。

一个有趣的动机对于minibatch随机梯度下降在于只要没有实例被重复,将一直会有泛化误差的梯度。大多数minibatch随机梯度下降方法的实现会打乱数据顺序一次,然后多次遍历数据更新参数。第一次遍历时,每个minibatch样本都用来计算真实泛化误差的无偏估计。第二次遍历时,此时已经是有偏估计了,因为使用了由重新采样生成的值,而不是从数据的生成分布上获取新的样本。	

事实上随机梯度下降最小化泛化误差是在线学习最简单的情况,其中样本或者minibatch都是从数据流中抽取出来的。换句话说,并不是接收固定数量的训练集,学习其类似于生物能够在每个瞬间了解新的实例,其中每个样本(x,y)来自于生成分布$P_{data}(x,y)$。在这种情况下,实例不会重复,每个经验来自$P_{data}$都是公平的。

当x和y都是离散时,相等最易得到。此时,泛化误差被记为:

\begin{equation}
J^*(\theta)=\sum_x\sum_yp_{data}(x,y)L(f(x;\theta),y)
\end{equation}

以及梯度为:

\begin{equation}
g=\nabla_{\theta}J^*(\theta)=\sum_x\sum_yp_{data}(x,y)\nabla_{\theta}L(f(x;\theta),y)
\end{equation}

我们已经了解了等式8.5和8.6的对数似然的情况,对于除似然之外的其他函数L也是成立的。当x和y是连续的,关于$P_{data}$和L的一些假设,一个类似的结果也可以得到。

因此,通过对来自生成分布$P_{data}$的实例$\{X^{(1)},\dots X^{(m)}\}$和响应变量$y^i$进行抽样,我们就可以获取泛化误差梯度的无偏估计。计算关于参数的损失梯度:

\begin{equation}
\hat{g}=\frac{1}{m}\nabla_{\theta}\sum_iL(f(x^{(i)};\theta),y^{(i)})
\end{equation}

在泛化误差上通过沿着$\hat{g}$的方向上执行SGD来更新theta。

当然,这个理解仅适用于当实例不被重新使用时。最好是多次使用训练集,除非训练集特别大。当多次训练时,只有第一次训练泛化误差的梯度是无偏的,当然了,其他的训练足够有效,因为通过增加训练误差和测试误差之间的差距来减小训练误差,因此来抵消。

随着数据规模的快速扩大,对于机器学习应用去一次使用一个实例,甚至使用不完整的训练集,正变得越来越寻常。当使用特别大的训练集时,过拟合不是一个问题,因此欠你和和计算有效性变成了主要担忧的事情。具体可参考Bottou and Bousquet(2008)关于随着训练集规模的增长,关于在计算泛化误差上的瓶颈的讨论。

\section{神经网络最优化算法的挑战}

一般上最优化算法都是一个非常困难的工作。过去机器学习通过仔细设计目标函数和约束条件来避免常规的优化难度,以此来确保最优化问题是凸优化。当训练神经网络时,我们必须面对一般的非凸问题。在这一部分,我们总结了一些对于训练深度模型所涉及的最优化问题的显著的困难和挑战。

\subsection{病态}

当优化凸函数时,会引起一些挑战。其中最突出的就是黑塞矩阵H的病态问题。这在大多数数值最优化中是一个非常普遍的问题,凸或非凸在4.3.1有详细的描述。
	病态问题一般认为是存在于神经网络训练的问题。病态会引起SGD“卡住”也就是说甚至非常小的步长都会引起损失函数的急剧增加。
	回忆等式4.9,损失函数的二阶泰勒级数展开预测梯度减少  的步长将会给损失函数增加

\begin{equation}
\frac{1}{2} \varepsilon^2 g^THg-\varepsilon g^Tg
\end{equation}

当$\frac{1}{2} \varepsilon^2 g^THg$超出$\varepsilon g^Tg$,将会引发梯度的病态。

% \begin{figure}[htbp]
%    \centering
%    \includegraphics[width=6in]{fig/chap8/8_1.png} 
%    \caption{}
%    \label{fig:8.1}
% \end{figure}

为了判定病态是否对于神经网络的训练不利,通过监控$g^Tg$和$g^THg$的平方梯度。在许多情况下,梯度范数在学习时并不会明显减少,但是$g^THg$会数量级的增加。结果就是学习变得非常慢,尽管存在一个强大的梯度,但是因为学习率必须缩小以弥补更强曲率。图8.1展示了在成功训练神经网络梯度显著增加的例子。

尽管病态在除神经网络训练之外也会出现在其他情况下,但是一些技术已经用于实践,在许多场景下也非常少应用于神经网络。例如,牛顿法对于没有什么条件的Hessian矩阵,最小化凸函数是一个非常优秀的工具,但是在后面的部分,我们将说明牛顿法在它应用到神经网络之前需要重要的修正。

\subsection{局部极小值}

一个凸优化问题的最显著问题之一就是会找到局部最小值问题。任何局部最小值确保是一个全局最小值。一些凸优化函数在底部有一个平坦的区域而不是单个的全局最小值点,因此这个平坦区域的任何一个点都可以被视为一个可行解。当优化一个凸函数时,如果我们发现任何形式的临界点,那么我们就找到了一个好的解决方法。

对于非凸函数,例如神经网络,可能有许多的局部最小值。事实上,任何一个深度模型本质上都确保有一个非常大的局部最优解。然而,正如我们将要看到的,这将不是一个主要的问题。

神经网络和多个等效参数化的潜在模型都有多个局部极小的模型识别问题。一个模型被认为是可识别的,如果一个足够大的训练集可以排除所有除模型参数的设置。含有潜在变量的模型经常是不可识别的,因为我们可以变换潜在变量获取等价的模型。例如,我们可以取神经网络,修改第一层通过将神经元i的输入权重变量和神经元j的输入权重变量交换,输出了相同的权重向量。如果我们有m层,每一层有n个神经元,则将有$n!^{m}$种方式设计隐藏的神经元。这种非识别称之为重空间对称性。

除了重空间对称性,各种神经网络有其他引起不可识别的因素。例如,在任何修正线性或maxout网络,我们如果能度量所有的输出权重,那么可以度量所有的输入权重和单位偏差$\alpha$,意味着如果损失函数不包含例如权重衰减的项,直接基于权重而不是模型的输出——修正线性或maxout网络的每一个局部最小值位于(m*n)维的等价局部最小值的超平面。

这些模型的可识别问题意味着神经网络的损失函数中有特别大或者不可数的局部最小值。然而,由不可识别引起的所有局部最小值都等价于损失函数的的其他值。因此,这些局部最小值不是一个有问题的非凸形式。

如果局部最小值与全局最小值比较有高的代价损失,则局部最小值是有问题的。甚至在没有隐藏神经元的情况下构造一个小的神经网络,局部最小值的代价损失要比全局最小值要高(Sontag and Sussman, 1989; Brady et al., 1989; Gori and Tesi, 1992).若局部最小值有很高的代价损失且较常见,则对于基于梯度的最优化算法来说就是一个严重的问题。

对于在神经网络的实践中是否有代价损失较高的许多局部最小值,且是否最优化算法会遇到这个问题,这依然是一个开放问题。多年来,大多数的实践者相信局部最小值是困扰神经网络最优化的常见问题。现在,似乎并不是这样,这个问题仍然是研究的热门领域,但是许多专家怀疑对于足够大的神经网络,大部分的局部最小值有较小的代价损失值,去找到一个真正的全局最小值并不是很重要,宁愿在参数空间找到一个点有较小的代价损失而不是最小的代价损失。(Saxe et al., 2013; Dauphin et al., 2014; Goodfellow et al., 2015; Choromanska et al., 2014).

许多实践者将神经网络最优化中遇到的困难都归咎于局部最小值。我们鼓励实践者仔细的去测试具体问题。一个测试可以通过不断计算梯度的范数来排除局部最小值问题。如果梯度的范数并不显著减少,这个问题既不是局部最小值也不是其他类型的临界值。这种负检验可以消除局部最小值。在高维空间,明确确定局部最小值是非常困难的问题。

\subsection{高原,鞍点和平台}

对于许多高维非凸函数,局部最小或最大事实上与另外一种比较罕见的鞍点,都是梯度为0的情况。鞍点周围的点比其他点有更大的损失。在鞍点,黑塞矩阵既有整的特征值也有负的特征值。与正的特征值相关的特征向量比鞍点有更大的损失,而接近负特征值的特征向量有着更小的损失。我们可以将鞍点视为沿着损失函数某一横截面的一个局部最小值,以及损失函数另一横截面的一个局部最大值。图4.5说明了这个方面。

许多不同类别的随机函数表现出如下的行为:在低维空间,局部最小值很常见。在高维空间,局部最小值罕见,而鞍点非常常见。对于函数$f:\square^n\to \square$ ,鞍点的数量相比局部最小值随着n的增加呈现指数增长。为了直观理解这个现象,观测Heassian矩阵在局部最小值点只有正的特征值。而Hessian矩阵在鞍点既有正的又有负的特征值。想象一下,每一个特征值的符号是通过翻转一个硬币而产生的。在一维的情况下,通过投掷硬币得到正面是很容易获得一个局部极小值。在n维空间,所有n个硬币都是正面的概率呈现负指数的情况。
见Dauphin et al. (2014),回顾相关的理论工作。

许多随机函数令人惊讶的性质在于当位于代价较低的区域时,Hessian矩阵的特征值更可能是正的,类以硬币投掷,意味着如果我们在某个重要的点有着较低的损失,则硬币可能n次朝上。也意味着局部最小值更有能有更小的损失。临界点上有着更高的代价损失意味着很可能是鞍点,临界点上有点特别高的代价损失意味着很可能是局部最小值。

这种现象发生在许多不同类别的随机函数上。神经网络也会遇到这种情况吗?Baldi and Hornik (1989)理论上证明了没有非线性的浅自动编码器(前馈神经网络被训练使得输入和输出保持一致,第14章有相关说明)有全局最小值和鞍点,此外没有局部最小值比全局最小值有更高的代价损失。没有证明观测到这些结果延伸到没有非线性的深度神经网络,这些网络的输出是输入的线性函数,但是学习一个非线性神经网络模型是很有用的,因为它们的损失函数是关于参数的非凸函数。这些网络本质上仅仅是多个矩阵结合在一起。Saxe et al. (2013) 提供准确的解决方案,在这样的网络中完成动态学习,也展示了学习这些模型可以获取到在用非线性激活函数训练深度模型观测到的许多定性特征。Dauphin et al. (2014)实验上展示了真正的神经网络也有包含许多较高损失代价鞍点的损失函数。Choromanska et al. (2014)提供了额外的理论论证,证明了与神经网络相关的其他类型的高维随机函数也是如此。

对于训练算法中鞍点带来的影响是什么?对于仅仅使用梯度信息的一阶最优化算法,这种情形不是特别清晰。靠近鞍点,梯度经常变得非常小。另外一方面,梯度下降经验上看上去在许多情况下可以摆脱鞍点。Goodfellow et al. (2015) 提供了一些先进的神经网络的路径图的可视化,图8.2就是一个例子展示。这些直观上表明了靠近比较突出的鞍点损失函数平坦且权重为0,但是它们也表明梯度下降轨迹会迅速逃离这个区域。Goodfellow et al. (2015)也论证了连续时间梯度下降靠近鞍点是排斥而不是吸引,但是这种情况与许多现实中使用的梯度下降可能有许多不同。

% \begin{figure}[htbp]
%    \centering
%    \includegraphics[width=6in]{fig/chap8/8_2.png} 
%    \caption{}
%    \label{fig:8.2}
% \end{figure}

对于牛顿法,很明显鞍点构成一个问题。梯度下降被设计成是下山,并不明显是寻找一个临界点。牛顿法被设计解决某个点的梯度为0。没有合适的修改,它将落入鞍点,在高维空间中鞍点的扩散可以解释为什么二阶方法对于神经网络训练为什么没有成功取代梯度下降。Dauphin et al. (2014)引入对于二阶最优化算法的无鞍牛顿法,展示了比传统的方法显著提高。二阶方法对于度量庞大的神经网络依然非常困难,但是如果可以度量,这个无鞍的解决方法还是有很多前景的。

除了最小值和鞍点还有其他类型梯度为0的点,也有极大值,和鞍点一样来自最优化问题,许多算法并不对极大值感兴趣,除了未修改的牛顿法。

也有许多常数对应的平坦区域。在这些点,梯度和Hessian矩阵都是0.这些退化的点对于所有的数值最优化算法都是很大的问题。在凸优化问题中,一个宽且平坦的区域一定会包含所有的全局最小值,但是在一般的最优化问题中,这样的区域对应目标函数的一个比较大的值。

\subsection{悬崖和梯度爆炸}

含有多层的神经网络经常会有非常陡峭的区域类似悬崖,图8.3进行了说明。这些结果来自于许多大的权重值相乘。在面对一个非常陡峭的悬崖结构时,梯度更新随着参数的变化会移动的非常远,通常会调离悬崖结构。

% \begin{figure}[htbp]
%    \centering
%    \includegraphics[width=6in]{fig/chap8/8_3.png} 
%    \caption{}
%    \label{fig:8.3}
% \end{figure}

这个悬崖是非常危险的,无论我们从上面或是下面接近它,但是幸运的是通过使用缩短梯度可以避免这些严重的后果。基本思想是:回忆梯度并没有指定最优步长,但只是沿着在极度小的区域中的最优方向。当传统的梯度下降算法采取大的步长时,启发式的缩短梯度会干预使得步长的减小足够小,从而不太可能落在梯度沿着最速下降的方向以外的区域。悬崖结构对于递归神经网络是非常常见的,因为这些模型设计带许多因数的相乘,长时间序列因此因为乘法产生了一个极端值。

\subsection{长期依赖}

神经网络最优化算法的另外一个困难必须克服是由当计算图变得非常深时引起的。含有多层的前馈神经网络或有如此神的计算图,递归神经网络也是如此,第10章会有描述,通过一个长的时间序列上每隔一个时间段重复应用相同的操作会构造出深的计算图。重复使用相同的参数会引起明显的困难。

例如,设想一个计算图包含重复乘以W的步骤。t步之后,等价于乘以$W^t$。设想W有特征分解$W=Vdiag(\lambda)V^{-1}$。这种简单的情况下,可以直接看到如下:

\begin{equation}
W^t=(V_{diag}(\lambda)V^{-1})^t=V_{diag}(\lambda)^tV^{(-1)}
\end{equation}

任何特征值$\lambda_i$当大于1时,量级会指数扩张,当小于1时,会减小趋近于0.这个消失和梯度爆炸问题涉及到事实上图是根据$diag(\lambda)^t$度量的。梯度消失对于从哪些参数方面去改善损失函数是非常困难的,而梯度爆炸使得学习不稳定。悬崖结构早期描述了促进缩短梯度是梯度爆炸的示例。

每次重复乘以W类似于使用幂算法去找到矩阵W的最大特征值以及相应的特征向量。从这一点看来将并不奇怪,$x^TW^t$将最终舍弃所有的正定于W的特征向量的x。

递归神经网络每次使用同一个W,但是前馈神经网络不是,因此尽管非常深的前馈神经网络可以在很多程度上避免梯度消失和爆炸问题(Sussillo, 2014).

我们将推迟关于训练递归神经网络挑战的讨论,知道10.7部分,在那里递归神经网络将更详细的被描述。

\subsection{不精确梯度}

大多数优化算法都需要提取梯度或者Hessian矩阵,而现实中我们通常只能使用带噪声或有偏来估计这些量。几乎每个深度学习算法都依赖于基于采样的估计,至少使用训练样本的最小批次来计算梯度。

在其他情况下,我们想要最小化目标函数实际上是无解的。当目标函数无解时,通常梯度也是无法计算的。这些问题大多出现在第三部分中更高级的模型中。例如对比散度给出近似比较难解的玻尔兹曼机的极大似然的梯度技术。

各种神经网络最优化算法被设计去解释梯度估计上的缺陷。通过选择替代的损失函数可以避免这个问题。

\subsection{局部和全局结构的不一致性}

迄今为止我们讨论的许多问题对应于损失函数在单点的性质——如果$J(\theta)$在$\theta$处没有约束或者是$\theta$位于悬崖,又或者是$\theta$是一个鞍点,是很难取一个步长的。

克服在单点的上述问题,以及依然表现不佳,如果导致局部最大改善的方向不是朝着许多更小代价损失的区域方向,是有可能的。

Goodfellow et al. (2015)论证了训练过程的运行时间是归结于到达解决方法的轨迹长度。图8.2证明了学习轨迹花费的大部分时间是在追踪围绕山地结构的很宽的弧线。

许多关于最优化问题研究的困难集中于是否训练会到达全局最小值、局部极小值、鞍点。但是实践中神经网络不会到达任何类型的临界点,图8.1表明了神经网络不会落在小梯度的区域。事实上,这些临界点甚至没有必要存在。例如损失函数$-logp(y|x;\theta)$没有全局最小值点,相反可以渐进到一些值,此时模型也会更加准确。对于离散y和$p(y|x;\theta)$由softmax提供的分类器,非负对数似然当模型可以正确对训练集中的每个示例正确分类时,可以沿任意方向趋于0,但是不可能严格等于0.同样的,真实模型$p(y|x;\theta)=N(y;f(\theta,\beta^{-1}))$有负对数似然渐进于负无穷,若$f(\theta)$可以正确预测训练集中的目标y,学习算法将无限增加增加$\beta$。图8.4说明了局部最优化失败去寻找一个好的损失函数,即使是存在任何的局部极小值或者鞍点。

未来的研究需要进一步理解影响学习轨迹长度和更好描述结果的因素。

% \begin{figure}[htbp]
%    \centering
%    \includegraphics[width=6in]{fig/chap8/8_4.png} 
%    \caption{}
%    \label{fig:8.4}
% \end{figure}

许多当前的研究方向旨在对于复杂全局结果找到好的初始点,而不是使用非局部移动开发算法。

梯度下降和所有的学习算法对于基于通过较小、局部的移动来训练神经网络是有效的。先前的部分主要集中于这些局部移动的正确方向将非常困难去计算。可以计算这些目标函数的梯度,近似通过偏差或者方差沿着估计的正确方向近似。目标函数缺少约束条件或不连续的梯度,将导致区域非常小。在这些情况下,步长大小为$\varepsilon$的局部下降可以定义一个合理的短路径的解决方案,但我们只能够计算步长大小为$\delta\square\varepsilon$的局部下降。局部下降可能或不可能定义一个解决方案,但路径包含许多步骤,所以这些路径会引发高的计算成本。当函数有一个基表大的平坦区域时或者设法到达临界点时,局部信息没有作用。其他情况,拒不移动太贪婪将导致远离任何解,如图8.4,或者沿着解的多余的长轨迹,如图8.2.当前我们不了解这些问题对于引起神经网络优化困难,哪些是最相关的,这是一个热门的研究领域。

无论哪个问题都是非常重要的,如果存在一个区域连接相应局部下降解的路径,或者如果我们使用比较好的区域初始化学习 。最后的观点建议为传统的最优化算法研究选择好的初始点。

\subsection{理论上的优化限制}

一些理论结果表明为神经网络设计的最优化算法会有性能上的限制(Blum and Rivest,
1992; Judd, 1989; Wolpert and MacReady, 1997)。通常这些结果在实践中神经网络的使用上是不可用的。

一些理论结果只适用于神经网络的单元输出离散值。但是大多数神经网络单元输出都睡平滑的值,使得通过局部搜索的优化可行。一些理论结果表明存在一些有问题的类是难以被正确分类的,但是很难讲清楚是否存在一个特定的问题导致错误分类。其他的一些结果表明为一个特定大小的网络找到solution是很棘手的,但是在实践中我们可以简单地找出一个solution,通过使用一个较大的网络设定较多的参数,得到一个可以接受的solution。而且在神经网络训练中,通常不关心找到一个函数的严格最小值,而是在于足够去减少它的泛化误差。理论分析一个优化算法是否能完成是相当困难的,因此在机器学习研究中药更加注重优化算法的性能的现实结果














\section{训练深度模型的优化方法}

深度学习算法在许多场景下涉及到最优化。例如,在PCA模型的性能推断上涉及解决一个最优化问题。我们经常用最优化解析方法去证明或者设计算法。在深度学习包含的所有最优化问题中,最困难的是神经网络训练。甚至为了解决神经网络训练的一个实例,花费几天或者上月的时间在几百台机器上都是很常见的,因为这个问题非常重要且代价很大,然而我们已经开发了一类特殊的最优化技术去解决它。本章将给出训练神经网络的最优化技术。

如果你不熟悉基于梯度的最优化方法的基本理论,我们推荐复习第四章。那一章包含简短的数值最优化方法的概述。

本章集中于一个特殊的最优化场景,找到能够显著减少神经网络损失函数$J(\Vtheta)$的参数$\Vtheta$,通常包含在整个训练集评估的性能度量以及额外的正则化项。
	
刚开始,我们将描述对于一个机器学习任务,在训练算法上使用的最优化方法有别于纯粹的最优化。下一步,我们将给出一些具体的优化神经网络的难点。然后定义一些实践上的算法,包含最优化算法本身和初始化参数的一些策略。更加高级的算法在训练或者利用包含在损失函数的二阶导数来调整学习率。最后,通过回顾一系列的优化策略,我们可以将一些简单的最优化算法结合转化为更高层次的程序。

\section{学习和优化有什么不同}

训练深度模型中使用的最优化算法在某些方面不同于传统的最优化算法。机器学习经常表现的间接。在大部分机器学习的场景中,我们关注性能度量$P$.,它是在测试集中定义的,也是非常棘手的。我们仅仅间接优化$P$.,我们希望通过减少损失函数$J(\Vtheta)$提升性能$P$。相比之下,纯粹的最优化方法,其目的在于最小化$J(\Vtheta)$。训练深度模型的最优化算法通常也包括在机器学习目标函数的特殊结构上的一些特性。

通常,损失函数可写为训练集上的平均,如
\begin{equation}
\label{eq:8.1}
    J(\Vtheta) = \SetE_{(\RVx, \RSy) \sim\hat{p}_\text{data}} L(f(\Vx ; \Vtheta), y),
\end{equation}
其中$L$是每个样本的损失函数,$f(\Vx;\Vtheta)$是输入是$\Vx$时的预测输出,$\hat{p}_{\text{data}}$是经验分布。
监督学习中,$y$是目标输出。
在本章中,我们会介绍不带正则项的监督学习,$L$的参数是$f(\Vx;\Vtheta)$和$y$。
很容易将这种监督学习扩展成其他形式,如为了开发正则化的各种形式或非监督学习,在参数中包含$\Vtheta$或者$\Vx$,排除参数$y$。

eqnref{eq:8.1} 定义了训练集上的目标函数。我们通常对相应的目标函数最小化,其中期望是在数据的生成分布$p_{\text{data}}$而不是有限的训练集。
\begin{equation}
\label{eq:8.2}
    J^*(\Vtheta) = \SetE_{(\RVx, \RSy) \sim p_\text{data}} L(f(\Vx ;\Vtheta),y).
\end{equation}

\subsection{经验风险最小化}

机器学习算法的目标是在(公式引用出错)减小泛化误差,即所谓的风险。需要强调的是期望是取值在真实潜在分布$p_\text{data}$上的。如果我们知道真实分布$p_\text{data}(\Vx, y)$,风险最小化将由最优化算法解决。然而,当我们不知道$p_\text{data}(\Vx, y)$,仅仅有一些训练样本时,就遇到了一个机器学习问题。

最简单的方式是将机器学习问题转化为最优化问题,通过在训练集上最小化期望损失。也就意味着用经验分布$\hat{p}(\Vx,y)$替换真实分布$p(\Vx,y)$。最小化经验风险:
\begin{equation}
\label{eq:8.3}
    \SetE_{\RVx, \RSy \sim \hat{p}_\text{data}} [L(f(\Vx ; \Vtheta), y)]
    = \frac{1}{m} \sum_{i=1}^m L( f(\Vx^{(i)}; \Vtheta), y^{(i)}) ,
\end{equation}
其中$m$表示训练样本的数目。

基于最小化平均训练误差的训练过程即为经验风险最小化。并不是直接优化风险,我们最优化经验风险,希望经验风险显著减小。基于不同的条件各种理论结果建立起来了,真实的风险是可以通过不同的变量减小。

然而,最小化经验风险易于过拟合。在许多情况下,最小化经验风险并不真正可行。大部分有效的模型最优化算法是基于梯度下降的,然而许多有用的损失函数,例如0-1损失函数,没有导数。这里两个问题意味着,在深度学习中,我们很难最小化经验损失。相反,我们必须使用略有不同的方法,我们真正优化的量甚至不同于我们真实想要优化的量。

\subsection{替代损失函数和提前终止}

有时,损失函数我们真正关心的(分类误差)并不是有效的优化。例如,严格最小化期望0-1损失函数明显棘手的(在维数上是指数的),甚至是对于一个线性分类器(引用出错).在这种情况下,一个典型的优化是用代理损失函数替代,充当一个代理但是却非常有效。例如,正例的非负对数似然就是替代0-1损失的。非负对数似然允许模型估计类别的条件概率。给定输入,若模型确实有效,则选择最小期望分类误差。

在某些情况下,一个代理损失函数事实上导致可以学到更多。例如,当训练中用对数似然替代时,当训练集上0-1损失函数为0时,在测试集上0-1损失函数通常连续减少。这是因为尽管0-1损失是0,我们可以提高分类器的鲁棒性进一步的区分不同的类,获取更可靠的分类器,因此相比于在训练集上简单的最小化平均0-1损失,能获取更多的信息。

一般的最优化方法和深度学习中的最优化算法的一个重要差异在于,我们在使用训练算法时,通常不会再局部最小值停止。相反,当基于提前停止的收敛规则满足时,一个机器学习算法通常最小化一个代理损失函数停止。典型的,提前停止规则是基于真实的潜在的损失函数,例如在验证集上度量的0-1损失,当过拟合开始发生时,算法停止。尽管替代的损失函数依然有比较大的导数,训练能停止。这非常不同于纯粹的最优化方法,其中最优化算法当梯度非常小时,被认为是收敛的。

\subsection{批算法和小批算法}
机器学习算法不同于一般的最优化算法的一个方面在于为目标函数在训练集上被分解为为训练样本上的求和。
机器学习优化算法通常使用整个损失函数中的一部分项去更新基于估计损失函数的期望值。

例如,在对数空间中,极大似然估计问题可以分解成每个样本的和:
\begin{equation}
\label{eq:8.4}
    \Vtheta_{\text{ML}} = \underset{\Vtheta}{\argmax} \sum_{i=1}^m
    \log p_{\text{model}} (\Vx^{(i)}, y^{(i)}; \Vtheta) .
\end{equation}

最大化此项等价于最大化定义在训练集上的经验分布的期望:
\begin{equation}
\label{eq:8.5}
    J(\Vtheta) = \SetE_{\RVx, \RSy \sim\hat{p}_\text{data}}
    \log p_{\text{model}} (\Vx,y ; \Vtheta) .
\end{equation}

目标函数的多数性质被大部分最优化算法所使用的也是训练集上的期望。例如,通常大部分使用的性质是梯度:
\begin{equation}
\label{eq:8.6}
    \nabla_{\Vtheta} J(\Vtheta) = \SetE_{\RVx, \RSy \sim\hat{p}_{\text{data}}}
    \nabla_{\Vtheta} \log p_{\text{model}} (\Vx,y; \Vtheta) .
\end{equation}

准确计算这个期望代价非常大,因为它需要在整个数据集的每个实例上评估模型。实践中,我们通过对数据集进行随机抽样来计算这些期望,然后再在实例上取均值。

回忆$n$个样本的均值标准误差( )是$\sigma/\sqrt{n}$,其中$\sigma$是这些样本的标准差。在100个实例和1000 个实例上比较梯度的两个假设估计,后者比前者需要多出100倍的计算,但是平均能减少10倍的标准差。大多数的最优化算法当他们能快速近似计算梯度时往往比缓慢计算的严格梯度要收敛的更快。

其他有助于来自小样本的梯度统计估计在训练样本集上是多余的,最糟糕的情况,训练集中的所有m个样本,是彼此相同的。一个基于抽样的梯度估计通过m个样本能正确计算,比使用朴素方法计算次数要少。实践中,我们不可能遇到这种情况,但是我们可以使用大样本量计算梯度。

最优化算法使用整个训练集称之为批或确定梯度方法,因为它们通过一个大的批次同时计算所有的样本量。这个术语可能与 “batch”有点混淆,其经常被使用其描述由最小批处理随机梯度下降的分批处理,典型的,术语批随机梯度下降暗含着真个训练集的使用,同时术语batch的使用并不是描述着一组样本。例如,通常使用术语“bitch size”描绘批梯度的规模。

最优化算法有时一次使用一个实例称之为随机或者在线方法。术语在线有时被保留的情况下来自于,从一个不断创建的例子,而不是传输固定大小的训练集。

深度学习使用的大部分算法介于两者之间,即使用一个以上或者少于所以的训练样本。传统上被称之为批处理或者批处理随机方法,现在统一简称为随机方法。

随机方法的标准范例是随机梯度下降,在8.3.1会详细说明。

批规模由下面几个因素驱动:
\begin{itemize}
\item 更大的批次提供一个更加精确的梯度估计,但是不是呈线性的返回。
\item 在极小的批次下,多核架构一般未被充分利用,这个激发使用一些绝对数量的最小批次,在处理一个批上时间上并没有减少。
\item 如果批次中所有的示例被并行处理,则大量的内存被批规模消耗掉。许多硬件限制了批的规模。
\item 各种硬件通过使用特定的数组可以获得更好的运行,特别是在使用GPUs的过程中,2次幂的批规模来提供更好的性能。明显的2的幂次方的批规模从32变化到256,为了尝试大的模型,可能是16.
\item 小批次可能提供一个正则化的影响,可能因为噪声被被添加到学习过程,泛化误差对于规模为1的批次是最好的。在这个规模上进行训练可能需要小的学习速率来维持因为在梯度估计中的高方差。总的运行时间可能非常高因为需要采取更多的步数,均是因为学习速率的减小和需要采取更多的步数来观测整个训练集。
\end{itemize}

不同的算法通过不同的方式使用最小批的各种信息。一些算法对样本误差非常敏感,可能是因为他们使用非常少的样本取估计准确度,或者是因为他们放大样本误差。仅仅基于梯度$\Vg$计算更新通常相对健壮,他们可以处理如100的批规模。二阶方法,通过使用hessian矩阵$\MH$,通过计算$\MH^{-1}\Vg$进行更新,典型的需要如10000的更大的批次。这些大的批次规模需要最小化估计$\MH^{-1}\Vg$的波动。设想$\MH$使用非常少的条件数被完美估计。乘以H或者它的逆放大预先存在的错误,这种情况下,在$\Vg$中存在估计误差,估计$\Vg$中非常小的变化能引起更新 的巨大变化,甚至当$\MH$被完美估计。当然$\MH$只能被近似估计,因此更新 将比我们通过应用少量的条件去估计$\Vg$来预测会包含更多的错误。

最小批被随机选择是非常重要的,计算样本的期望梯度的无偏估计需要样本是独立的,我们也希望两个随后的梯度估计是相互独立的,我们希望两个随后的梯度估计相互独立,许多数据集一般被设计成许多连续的例子是高度相关的。例如,我们可能有一系列的关于血液样本的医药数据。对第一个病人,在不同时间获取5例血液样本,然后从第二个病人那里获取3例血液样本,然后依次进行这样的操作,如果我们按照顺序从中取出实例,则每个minibatches将是极度有偏的,因为它将代表数据集中许多患者的最显著的那个。在这种情况下,数据集中实例的顺序将有重要意义,因为必须在选择minibatches之前必须打乱实例的顺序。对于大数据集而言,例如一个数据中心包含几十亿的实例,每次随机抽样取构造minibatches是不现实的。幸运的是,实践上足够去打乱实例的顺序,并以无序的形式存储。这将强加一个固定连续的minibatches,从而在所有的模型训练中使用,当使用整个数据集时,每一个单独的模型将被迫重用这个排序。然而,来自真正随机选择的偏差看起来并没有一个显著的有害影响。不打乱实例的顺序将严重减弱算法的有效性。

机器学习中的许多最优化问题可以在训练集中进行分解,从而我们可以并行的不同的实例上计算整个单独的更新。换言之,我们可以计算更新能够同时对于每个$\MX$的minibatch最小化$J(\MX)$,从而对多个minibatch计算更新。这些异步并行分布式方法在12.1.3将被讨论。

一个有趣的动机对于minibatch随机梯度下降在于只要没有实例被重复,将一直会有泛化误差的梯度。大多数minibatch随机梯度下降方法的实现会打乱数据顺序一次,然后多次遍历数据更新参数。
第一次遍历时,每个minibatch样本都用来计算真实泛化误差的无偏估计。
第二次遍历时,此时已经是有偏估计了,因为使用了由重新采样生成的值,而不是从数据的生成分布上获取新的样本。

事实上随机梯度下降最小化泛化误差是在线学习最简单的情况,
其中样本或者minibatch都是从数据流中抽取出来的。
换句话说,并不是接收固定数量的训练集,学习其类似于生物能够在每个瞬间了解新的实例,
每个样本$(\Vx,y)$都来自数据生成分布$p_{\text{data}}(\Vx,y)$。
这种情况下,样本从来不会重复;每次更新也是公平地从分布$p_\text{data}$中采样获得。

在$\Vx$和$y$是离散时,以上的等价性很容易得到。此时,泛化误差被记为:
\begin{equation}
\label{eq:8.7}
    J^*(\Vtheta) = \sum_{\Vx} \sum_y p_{\text{data}}(\Vx, y) L(f(\Vx; \Vtheta),y),
\end{equation}
以及梯度:
\begin{equation}
\label{eq:8.8}
    \Vg = \nabla_{\Vtheta} J^*(\Vtheta) = \sum_{\Vx} \sum_y p_{\text{data}}
    (\Vx, y) \nabla_{\Vtheta} L(f(\Vx;\Vtheta),y) .
\end{equation}

我们已经了解了(公式引用出错)和(公式引用出错)的对数似然的情况,对于除似然之外的其他函数$L$也是成立的。当$\Vx$ 和$y$是连续的,关于$p_\text{data}$和$L$的一些假设,一个类似的结果也可以得到。

因此,通过对来自生成分布$p_\text{data}$的实例$\{ \Vx^{(1)}, \dots,\Vx^{(m)} \}$以及对应的标签$y^{(i)}$进行抽样,我们就可以获取泛化误差梯度的无偏估计。计算关于参数的损失梯度:
\begin{equation}
\label{eq:8.9}
    \hat{\Vg} = \frac{1}{m} \nabla_{\Vtheta} \sum_i L(f(\Vx^{(i)};\Vtheta),y^{(i)} ).
\end{equation}
在泛化误差上通过沿着$\hat{\Vg}$的方向上执行SGD来更新$\Vtheta$。

当然,这个理解仅适用于当实例不被重新使用时。最好是多次使用训练集,除非训练集特别大。当多次训练时,只有第一次训练泛化误差的梯度是无偏的,当然了,其他的训练足够有效,因为通过增加训练误差和测试误差之间的差距来减小训练误差,因此来抵消。

随着数据规模的快速扩大,对于机器学习应用去一次使用一个实例,甚至使用不完整的训练集,正变得越来越寻常。当使用特别大的训练集时,过拟合不是一个问题,因此欠你和和计算有效性变成了主要担忧的事情。具体可参考\cite{bottou-bousquet-2008}关于随着训练集规模的增长,关于在计算泛化误差上的瓶颈的讨论。

\section{神经网络最优化算法的挑战}
一般上最优化算法都是一个非常困难的工作。过去机器学习通过仔细设计目标函数和约束条件来避免常规的优化难度,以此来确保最优化问题是凸优化。当训练神经网络时,我们必须面对一般的非凸问题。在这一部分,我们总结了一些对于训练深度模型所涉及的最优化问题的显著的困难和挑战。

\subsection{病态}
当优化凸函数时,会引起一些挑战。其中最突出的就是hessian矩阵$\MH$的病态问题。这在大多数数值最优化中是一个非常普遍的问题,凸或非凸在4.3.1有详细的描述。

病态问题一般认为是存在于神经网络训练的问题。病态会引起SGD“卡住”也就是说甚至非常小的步长都会引起损失函数的急剧增加。

回忆(公式引用出错),损失函数的二阶泰勒级数展开预测梯度减少$-\epsilon\Vg$的步长将会给损失函数增加
\begin{equation}
    \frac{1}{2} \epsilon^2 \Vg^\top \MH\Vg - \epsilon\Vg^\top\Vg
\end{equation}
到损失中。

当$\frac{1}{2} \epsilon^2 \Vg^\top\MH\Vg$超过$\epsilon\Vg^\top\Vg$时,梯度的病态会成为问题。

为了判定病态是否对于神经网络的训练不利,通过监控$\Vg^\top\Vg$和$\Vg^\top \MH\Vg$的平方梯度。在许多情况下,梯度范数在学习时并不会明显减少,但是$\Vg^\top\MH\Vg$会数量级的增加。结果就是学习变得非常慢,尽管存在一个强大的梯度,但是因为学习率必须缩小以弥补更强曲率。图8.1展示了在成功训练神经网络梯度显著增加的例子。

\begin{center}
%\includegraphics[width=3.7in,height=2.7in]{1.jpg}\\
图~8.1
\end{center}

尽管病态在除神经网络训练之外也会出现在其他情况下,但是一些技术已经用于实践,在许多场景下也非常少应用于神经网络。例如,牛顿法对于没有什么条件的hessian矩阵,最小化凸函数是一个非常优秀的工具,但是在后面的部分,我们将说明牛顿法在它应用到神经网络之前需要重要的修正。

\subsection{局部极小值}
一个凸优化问题的最显著问题之一就是会找到局部最小值问题。任何局部最小值确保是一个全局最小值。一些凸优化函数在底部有一个平坦的区域而不是单个的全局最小值点,因此这个平坦区域的任何一个点都可以被视为一个可行解。当优化一个凸函数时,如果我们发现任何形式的临界点,那么我们就找到了一个好的解决方法。

对于非凸函数,例如神经网络,可能有许多的局部最小值。事实上,任何一个深度模型本质上都确保有一个非常大的局部最优解。然而,正如我们将要看到的,这将不是一个主要的问题。

神经网络和多个等效参数化的潜在模型都有多个局部极小的模型识别问题。一个模型被认为是可识别的,如果一个足够大的训练集可以排除所有除模型参数的设置。含有潜在变量的模型经常是不可识别的,因为我们可以变换潜在变量获取等价的模型。例如,我们可以取神经网络,修改第一层通过将神经元$i$的输入权重变量和神经元$j$的输入权重变量交换,输出了相同的权重向量。如果我们有$m$层,每一层有$n$个神经元,则将有$n!^m$种方式设计隐藏的神经元。这种非识别称之为重空间对称性。

除了重空间对称性,各种神经网络有其他引起不可识别的因素。例如,在任何修正线性或maxout网络,我们如果能度量所有的输出权重,那么可以度量所有的输入权重和单位偏差$\alpha$,意味着如果损失函数不包含例如权重衰减的项,直接基于权重而不是模型的输出——修正线性或maxout网络的每一个局部最小值位于$(m\times n)$维的等价局部最小值的超平面。

这些模型的可识别问题意味着神经网络的损失函数中有特别大或者不可数的局部最小值。然而,由不可识别引起的所有局部最小值都等价于损失函数的的其他值。因此,这些局部最小值不是一个有问题的非凸形式。

如果局部最小值与全局最小值比较有高的代价损失,则局部最小值是有问题的。甚至在没有隐藏神经元的情况下构造一个小的神经网络,局部最小值的代价损失要比全局最小值要高
(Sontag and Sussman, 1989; Brady et al., 1989; Gori and Tesi, 1992).若局部最小值有很高的代价损失且较常见,则对于基于梯度的最优化算法来说就是一个严重的问题。

对于在神经网络的实践中是否有代价损失较高的许多局部最小值,且是否最优化算法会遇到这个问题,这依然是一个开放问题。多年来,大多数的实践者相信局部最小值是困扰神经网络最优化的常见问题。现在,似乎并不是这样,这个问题仍然是研究的热门领域,但是许多专家怀疑对于足够大的神经网络,大部分的局部最小值有较小的代价损失值,去找到一个真正的全局最小值并不是很重要,宁愿在参数空间找到一个点有较小的代价损失而不是最小的代价损失。(Saxe et al., 2013; Dauphin et al., 2014; Goodfellow et al., 2015; Choromanska et al., 2014).

许多实践者将神经网络最优化中遇到的困难都归咎于局部最小值。我们鼓励实践者仔细的去测试具体问题。一个测试可以通过不断计算梯度的范数来排除局部最小值问题。如果梯度的范数并不显著减少,这个问题既不是局部最小值也不是其他类型的临界值。这种负检验可以消除局部最小值。在高维空间,明确确定局部最小值是非常困难的问题。

\subsection{高原,鞍点和平台}
对于许多高维非凸函数,局部最小或最大事实上与另外一种比较罕见的鞍点,都是梯度为0的情况。鞍点周围的点比其他点有更大的损失。在鞍点,hessian既有整的特征值也有负的特征值。与正的特征值相关的特征向量比鞍点有更大的损失,而接近负特征值的特征向量有着更小的损失。我们可以将鞍点视为沿着损失函数某一横截面的一个局部最小值,以及损失函数另一横截面的一个局部最大值。图4.5说明了这个方面。

许多不同类别的随机函数表现出如下的行为:在低维空间,局部最小值很常见。在高维空间,局部最小值罕见,而鞍点非常常见。对于函数$f:\SetR^n \to \SetR$,鞍点的数量相比局部最小值随着n的增加呈现指数增长。为了直观理解这个现象,观测Heassian矩阵在局部最小值点只有正的特征值。而Hessian矩阵在鞍点既有正的又有负的特征值。想象一下,每一个特征值的符号是通过翻转一个硬币而产生的。在一维的情况下,通过投掷硬币得到正面是很容易获得一个局部极小值。在$n$维空间,所有$n$个硬币都是正面的概率呈现负指数的情况。
见\cite{Dauphin-et-al-NIPS2014-small},回顾相关的理论工作。

许多随机函数令人惊讶的性质在于当位于代价较低的区域时,Hessian矩阵的特征值更可能是正的,类以硬币投掷,意味着如果我们在某个重要的点有着较低的损失,则硬币可能$n$次朝上。也意味着局部最小值更有能有更小的损失。临界点上有着更高的代价损失意味着很可能是鞍点,临界点上有点特别高的代价损失意味着很可能是局部最小值。

这种现象发生在许多不同类别的随机函数上。神经网络也会遇到这种情况吗?Baldi and Hornik (1989)理论上证明了没有非线性的浅自动编码器(前馈神经网络被训练使得输入和输出保持一致,第14章有相关说明)有全局最小值和鞍点,此外没有局部最小值比全局最小值有更高的代价损失。没有证明观测到这些结果延伸到没有非线性的深度神经网络,这些网络的输出是输入的线性函数,但是学习一个非线性神经网络模型是很有用的,因为它们的损失函数是关于参数的非凸函数。这些网络本质上仅仅是多个矩阵结合在一起。Saxe et al. (2013) 提供准确的解决方案,在这样的网络中完成动态学习,也展示了学习这些模型可以获取到在用非线性激活函数训练深度模型观测到的许多定性特征。Dauphin et al. (2014)实验上展示了真正的神经网络也有包含许多较高损失代价鞍点的损失函数。Choromanska et al. (2014)提供了额外的理论论证,证明了与神经网络相关的其他类型的高维随机函数也是如此。

对于训练算法中鞍点带来的影响是什么?对于仅仅使用梯度信息的一阶最优化算法,这种情形不是特别清晰。靠近鞍点,梯度经常变得非常小。另外一方面,梯度下降经验上看上去在许多情况下可以摆脱鞍点。Goodfellow et al. (2015)提供了一些先进的神经网络的路径图的可视化,图8.2就是一个例子展示。这些直观上表明了靠近比较突出的鞍点损失函数平坦且权重为0,但是它们也表明梯度下降轨迹会迅速逃离这个区域。Goodfellow et al. (2015)也论证了连续时间梯度下降靠近鞍点是排斥而不是吸引,但是这种情况与许多现实中使用的梯度下降可能有许多不同。

\begin{center}
%\includegraphics[width=3.7in,height=2.7in]{2.jpg}\\
图~8.2
\end{center}

对于牛顿法,很明显鞍点构成一个问题。梯度下降被设计成是下山,并不明显是寻找一个临界点。牛顿法被设计解决某个点的梯度为0。没有合适的修改,它将落入鞍点,在高维空间中鞍点的扩散可以解释为什么二阶方法对于神经网络训练为什么没有成功取代梯度下降。Dauphin et al. (2014)引入对于二阶最优化算法的无鞍牛顿法,展示了比传统的方法显著提高。二阶方法对于度量庞大的神经网络依然非常困难,但是如果可以度量,这个无鞍的解决方法还是有很多前景的。

除了最小值和鞍点还有其他类型梯度为0的点,也有极大值,和鞍点一样来自最优化问题,许多算法并不对极大值感兴趣,除了未修改的牛顿法。

也有许多常数对应的平坦区域。在这些点,梯度和Hessian矩阵都是0.这些退化的点对于所有的数值最优化算法都是很大的问题。在凸优化问题中,一个宽且平坦的区域一定会包含所有的全局最小值,但是在一般的最优化问题中,这样的区域对应目标函数的一个比较大的值。

\subsection{悬崖和梯度爆炸}
含有多层的神经网络经常会有非常陡峭的区域类似悬崖,图8.3进行了说明。这些结果来自于许多大的权重值相乘。在面对一个非常陡峭的悬崖结构时,梯度更新随着参数的变化会移动的非常远,通常会调离悬崖结构。

\begin{center}
%\includegraphics[width=3.7in,height=2.7in]{3.jpg}\\
图~8.3
\end{center}

这个悬崖是非常危险的,无论我们从上面或是下面接近它,但是幸运的是通过使用缩短梯度可以避免这些严重的后果。基本思想是:回忆梯度并没有指定最优步长,但只是沿着在极度小的区域中的最优方向。当传统的梯度下降算法采取大的步长时,启发式的缩短梯度会干预使得步长的减小足够小,从而不太可能落在梯度沿着最速下降的方向以外的区域。悬崖结构对于递归神经网络是非常常见的,因为这些模型设计带许多因数的相乘,长时间序列因此因为乘法产生了一个极端值。

\subsection{长期依赖}
神经网络最优化算法的另外一个困难必须克服是由当计算图变得非常深时引起的。含有多层的前馈神经网络或有如此神的计算图,递归神经网络也是如此,第10章会有描述,通过一个长的时间序列上每隔一个时间段重复应用相同的操作会构造出深的计算图。重复使用相同的参数会引起明显的困难。

例如,假设某个计算图中包含一条重复与矩阵$\MW$相乘的路径。
那么$t$步后,相当于和$\MW^t$相乘。
假设$\MW$有特征值分解$\MW = \MV \text{diag}(\Vlambda) \MV^{-1}$。
在这种简单的情况下,很容易看到
\begin{equation}
  \MW^t = (\MV \text{diag}(\Vlambda) \MV^{-1})^t = \MV\text{diag}(\Vlambda)^t  \MV^{-1}.
\end{equation}

任何特征值$\lambda_i$当大于1时,量级会指数扩张,当小于1时,会减小趋近于0.这个消失和梯度爆炸问题涉及到事实上图是根据$\text{diag}(\Vlambda)^t$度量的。梯度消失对于从哪些参数方面去改善损失函数是非常困难的,而梯度爆炸使得学习不稳定。悬崖结构早期描述了促进缩短梯度是梯度爆炸的示例。

每次重复乘以$\MW$类似于使用幂算法去找到矩阵$\MW$的最大特征值以及相应的特征向量。从这一点看来将并不奇怪,$\Vx^\top\MW^t$将最终舍弃所有的正定于$\MW$的特征向量的x。

递归神经网络每次使用同一个$\MW$,但是前馈神经网络不是,因此尽管非常深的前馈神经网络可以在很多程度上避免梯度消失和爆炸问题(Sussillo, 2014).

我们将推迟关于训练递归神经网络挑战的讨论,知道10.7部分,在那里递归神经网络将更详细的被描述。

\subsection{不精确梯度}
大多数优化算法都需要提取梯度或者Hessian矩阵,而现实中我们通常只能使用带噪声或有偏来估计这些量。几乎每个深度学习算法都依赖于基于采样的估计,至少使用训练样本的最小批次来计算梯度。

在其他情况下,我们想要最小化目标函数实际上是无解的。当目标函数无解时,通常梯度也是无法计算的。这些问题大多出现在第三部分中更高级的模型中。例如对比散度给出近似比较难解的玻尔兹曼机的极大似然的梯度技术。

各种神经网络最优化算法被设计去解释梯度估计上的缺陷。通过选择替代的损失函数可以避免这个问题。

\subsection{局部和全局结构的不一致性}
迄今为止我们讨论的许多问题对应于损失函数在单点的性质——如果$J(\Vtheta)$在$\Vtheta$处没有约束或者是$\Vtheta$位于悬崖,又或者$\Vtheta$是一个鞍点,是很难取一个步长的。

克服在单点的上述问题,以及依然表现不佳,如果导致局部最大改善的方向不是朝着许多更小代价损失的区域方向,是有可能的。

Goodfellow et al. (2015)论证了训练过程的运行时间是归结于到达解决方法的轨迹长度。图8.2证明了学习轨迹花费的大部分时间是在追踪围绕山地结构的很宽的弧线。

许多关于最优化问题研究的困难集中于是否训练会到达全局最小值、局部极小值、鞍点。但是实践中神经网络不会到达任何类型的临界点,图8.1表明了神经网络不会落在小梯度的区域。事实上,这些临界点甚至没有必要存在。例如损失函数$-\log p(y\mid\Vx;\Vtheta)$没有全局最小值点,相反可以渐进到一些值,此时模型也会更加准确。对于离散$y$和softmax分布$p(y\mid\Vx)$提供的分类器,非负对数似然当模型可以正确对训练集中的每个示例正确分类时,可以沿任意方向趋于0,但是不可能严格等于0.同样的,真实模型$p(y\mid\Vx) = \mathcal{N}(y;f(\Vtheta),\beta^{-1})$有负对数似然渐进于负无穷,若$f(\Vtheta)$可以正确预测训练集中的目标$y$,学习算法将无限增加增加$\beta$。图8.4说明了局部最优化失败去寻找一个好的损失函数,即使是存在任何的局部极小值或者鞍点。

\begin{center}
%\includegraphics[width=3.7in,height=2.7in]{4.jpg}\\
图~8.4
\end{center}

未来的研究需要进一步理解影响学习轨迹长度和更好描述结果的因素。

许多当前的研究方向旨在对于复杂全局结果找到好的初始点,而不是使用非局部移动开发算法。

梯度下降和所有的学习算法对于基于通过较小、局部的移动来训练神经网络是有效的。先前的部分主要集中于这些局部移动的正确方向将非常困难去计算。可以计算这些目标函数的梯度,近似通过偏差或者方差沿着估计的正确方向近似。目标函数缺少约束条件或不连续的梯度,将导致区域非常小。在这些情况下,步长大小为$\epsilon$的局部下降可以定义一个合理的短路径的解决方案,但我们只能够计算步长大小为$\delta \ll \epsilon$的局部下降。局部下降可能或不可能定义一个解决方案,但路径包含许多步骤,所以这些路径会引发高的计算成本。当函数有一个基表大的平坦区域时或者设法到达临界点时,局部信息没有作用。其他情况,拒不移动太贪婪将导致远离任何解,如图8.4,或者沿着解的多余的长轨迹,如图8.2.当前我们不了解这些问题对于引起神经网络优化困难,哪些是最相关的,这是一个热门的研究领域。

无论哪个问题都是非常重要的,如果存在一个区域连接相应局部下降解的路径,或者如果我们使用比较好的区域初始化学习 。最后的观点建议为传统的最优化算法研究选择好的初始点。

\subsection{理论上的优化限制}
一些理论结果表明为神经网络设计的最优化算法会有性能上的限制(Blum and Rivest,
1992; Judd, 1989; Wolpert and MacReady, 1997)。通常这些结果在实践中神经网络的使用上是不可用的。

一些理论结果只适用于神经网络的单元输出离散值。但是大多数神经网络单元输出都睡平滑的值,使得通过局部搜索的优化可行。一些理论结果表明存在一些有问题的类是难以被正确分类的,但是很难讲清楚是否存在一个特定的问题导致错误分类。其他的一些结果表明为一个特定大小的网络找到解决方案是很棘手的,但是在实践中我们可以简单地找出一个解决方案,通过使用一个较大的网络设定较多的参数,得到一个可以接受的解决方案。而且在神经网络训练中,通常不关心找到一个函数的严格最小值,而是在于足够去减少它的泛化误差。理论分析一个优化算法是否能完成是相当困难的,因此在机器学习研究中药更加注重优化算法的性能的现实结果


%%%%%%%%%%%%%%%%%%%%%%%%%%%%%%%%%%%%%%%%%%%%%%%%%%%%%%%%%
%%%%%%%%%%%%%%%%%%% author:SilentSkyWalker %%%%%%%%%%%%%%
%%%%%%%%%%%%%%%%%%% part:8.3-8.5           %%%%%%%%%%%%%%
%%%%%%%%%%%%%%%%%%%%%%%%%%%%%%%%%%%%%%%%%%%%%%%%%%%%%%%%%

\section{8.3}
%%%%%%%%%%%%%%%%%%%%%%%%%%%%%%%%%%%%%%%%%%%%%%%%%%%%%%%%%
%%%%%%%%%%%%%%%%%%% author:dimitri0802 %%%%%%%%%%%%%%%%%%
%%%%%%%%%%%%%%%%%%% part:8.6-8.7       %%%%%%%%%%%%%%%%%%
%%%%%%%%%%%%%%%%%%%%%%%%%%%%%%%%%%%%%%%%%%%%%%%%%%%%%%%%%

\section{8.6}
\chapter{卷积网络}
\label{chap:9}
%%%%%%%%%%%%%%%%%%%%%%%%%%%%%%%%%%%%%%%%%%%%%%%%%%%%%%%%%
%%%%%%%%%%%%%%%%%%% author:ifighting %%%%%%%%%%%%%%%%%%%%
%%%%%%%%%%%%%%%%%%% part:9.1-9.6     %%%%%%%%%%%%%%%%%%%%
%%%%%%%%%%%%%%%%%%%%%%%%%%%%%%%%%%%%%%%%%%%%%%%%%%%%%%%%%

Convolutional Network,也叫做CNN,是一种专门用来处理具有类似网格拓补结构的数据的神经网络。
例如时间序列数据(可以认为是在时间轴上按照特定规律地采样形成的一维网格)和图像数据(可以看作是二维的由像素组成的网格)。
Convolutional network在很多应用领域都表现优秀。
Convolutional network一词说明这种网络使用了convolution这种数学运算。
卷积是一种特殊的线性运算。convolutional network是指那些至少在网络的一层中使用卷积运算来替代一般的矩阵乘法运算的神经网络。

本章,我们首先说明什么是卷积运算。
接着,我们会解释在神经网络中使用卷积运算的动机。
然后我们会介绍一种几乎所有的CNN都会用到的操作pooling。
通常来说,CNN中用到的卷积运算和其他领域(例如工程领域以及纯数学领域)中的定义并不完全一致。
我们会对神经网络实践中用得比较多的几种卷积函数的变体进行说明。
我们也会说明如何在多种不同维数的数据上使用卷积运算。
之后我们讨论使得卷积运算更加高效的一些方法。
CNN是神经科学的原理影响深度学习的典型代表,之后我们也会讨论这些神经科学的原理,并对卷积神经网络在深度学习发展史中的作用作出评价。
本章没有介绍如何为你的卷积神经网络选择合适的结构,因为本章的目标是描述CNN提供的强大工具,第11章会对在具体环境中使用相应的工具给出一些指导。
对于convolutional network结构的研究进展得如此迅速,以至于在特定的基准线上,数月甚至几周就会产生一个新的最优的网络结构,以至于评价究竟哪种结构是最好的也是不切实际的。然而,最好的网络结构也是由本章所描述的基本部件搭建起来的。

 
\section{卷积运算}

一般而言,卷积是对两个实值函数的一种数学运算。为了给出卷积的定义,我们从两个可能会用到的函数的例子出发。

假设我们正在使用激光传感器追踪一艘宇宙飞船的位置。我们的激光传感器给出一个单独的输出$x(t)$,表示宇宙飞船在时刻$t$ 的位置。$x$和$t$都是实值的,这意味着我们可以在任意的某个时刻从传感器中获取飞船的位置。

现在假设我们的传感器的输出含有噪声。为了减少噪声对飞船位置估计的影响,我们对得到的测量结果进行平均。
显然,时间上越近的测量结果越相关,因此我们采用一种加权平均的方法,对于越近的测量结果赋予更高的权值。
我们采用一个加权函数$w(a)$ 来实现,其中$a$表示测量结果据当前时刻的时间间隔。如果我们在任意时刻都采用这种加权平均的操作,就得到了对于飞船位置的连续估计函数$s$:
\begin{equation}
s(t) = \int x(a)w(t-a)da.
\end{equation}

这种运算就叫做(卷积)convolution。
卷积运算通常用星号表示:
\begin{equation}
s(t) = (x*w)(t).
\end{equation}

在我们的例子中,$w$必须是一个有效的概率密度函数,否则输出就不再是一个加权平均。
另外,$w$在参数为负值时必须为0,否则它会涉及到未来,这不是我们能够做到的。
但这些限制条件只是针对当前这个例子。
一般而言,卷积被定义在满足上述积分式的任意函数上,并且也可能被用于加权平均以外的目的。
在CNN的术语中,第一个参数(在这个例子中,函数$x$)叫做input,第二个参数(函数$w$)叫做kernel。
输出有时被称作feature map。 
在这个例子中,激光传感器给出连续的任意时刻测量结果的想法是不现实的。
一般而言,当我们用计算机处理数据时,时间会被离散化,传感器会给出特定时间间隔的数据。
所以比较现实的的假设是传感器每秒给出一次测量结果,这样,时间$t$只能取整数值。
如果我们假设$x$和$w$都定义在整数时刻$t$上,就得到了离散形式的卷积:

\begin{equation}
s(t) = (x*w)(t) = \sum_{a = -\infty}^{\infty} x(a)w(t-a).
\end{equation}

在机器学习的应用中,输入一般是高维矩阵数据,而kernal也是由算法产生的高维矩阵数据。
我们把这种高维矩阵数据叫做张量。
因为输入与核的每一个元素都分开存储,我们经常假设在存储了数据的有限点集以外,这些函数的值都为零。
这意味着在实际操作中,我们可以统一地把无限的求和当作对有限个数组元素的求和来用。
最后,我们有时对多个维度进行卷积运算。
例如,如果把二维的图像$I$作为输入,我们也相应的需要使用二维的核$K$:

\begin{equation}
S(i,j) = (I*K)(i,j) = \sum_m \sum_n I(m,n) K(i-m, j-n).
\end{equation}

卷积是具有交换性质(commutative),我们可以等价地写作:

\begin{equation}
S(i, j) = (K*I)(i,j) = \sum_m \sum_n I(i-m, j-n) K(m, n).
\end{equation}

通常,下面的公式在机器学习库中更方便应用,因为它在$m$和$n$的有效范围内变化更少。

 
卷积运算可交换性的出现是因为我们相对输入翻转了kernal,这意味着当$m$增大时,输入的索引增大,但核的索引相应的减小。
翻转核的唯一目的就是为了得到可交换性。
尽管可交换性在证明时很有用,但在神经网络的应用中却不是一个重要的性质。
与之不同的是,许多神经网络库会实现一个相关的函数,称为cross correlation,和卷积运算几乎一样但是并不翻转核:

\begin{equation}
S(i, j) = (I*K)(i, j) = \sum_m \sum_n I(i+m, j+n) K(m, n).
\end{equation}

许多机器学习的库使用互相关函数但是叫它卷积。
在这本书中我们遵循把两种运算都叫做卷积的这个传统,只有在用到核的翻转时才会在上下文中特别指明区别。
在机器学习中,学习算法会在核合适的位置学得恰当的值, 所以一个基于核翻转的卷积运算的学习算法所学得的核,是对未进行翻转的算法学得的核的翻转。
单独使用卷积运算在机器学习中是很少见的,卷积经常和其他的函数一起使用,无论卷积运算是否翻转了它的核,这些函数的组合通常是不可交换的。

\begin{figure}[htbp] %  figure placement: here, top, bottom, or page
   \centering
   \includegraphics[width=4in]{fig/chap9/9_1.png} 
   \caption{图9.1演示了一个在2维张量上的卷积运算(核没有翻转)的例子。}
   \label{fig:9_1}
\end{figure}


离散卷积可以看作矩阵的乘法,然而,这个矩阵的一些元素被限制为必须和另一些元素相等。
例如对于单变量的离散卷积,矩阵的每一行都必须和上一行移动一个元素后相等。
这种矩阵叫做Toeplitz matrix。
对于二维情况,卷积对应着一个doubly block circulant matrix(二维矩阵)。
除了这些元素相等的限制以外,卷积通常对应着一个非常稀疏的矩阵(几乎所有的元素都为零)。
这是因为核通常要远小于输入的图像。任何一个使用矩阵乘法但是并不依赖矩阵结构的特殊性质的神经网络算法,都适用于卷积运算,并且不需要对神经网络做出大的修改。
典型的CNN为了更有效地处理大规模输入,确实使用了一些专门化的技巧,但这些在理论分析方面并不是严格必要的。

 
\section{动机}

卷积运算通过三个重要的思想来帮助改进机器学习系统:sparse interactions(局部感知)、parameter sharing(权值共享)、equivariant representations。
另外,卷积提供了一种处理大小可变的输入的方法。我们会在下文中依次介绍这些思想。

传统的神经网络使用矩阵乘法来建立输入与输出的连接关系。
其中,参数矩阵的每一个独立的参数都描述了每一个输入单元与每一个输出单元间的交互。
这意味着每一个输出单元与每一个输入单元都产生交互。
然而,CNN具有sparse interactions(也叫做sparse connectivity或者sparse weights)的特征。
这通过使得核的规模远小于输入的规模来实现。
举个例子,当进行图像处理时,输入的图像可能包含百万个像素点,但是我们可以通过只占用几十到上百个像素点的核来探测一些小的有意义的特征,例如图像的边缘。
这意味着我们需要存储的参数更少,不仅减少了模型的存储需求,而且提高了它的统计效率。
这也意味着为了得到输出我们只需要更少的计算量。
这些效率上的提高往往是很显著的。
如果有$m$个输入和$n$个输出,那么矩阵乘法需要$m \times n$个参数并且相应算法的时间复杂度为$O(m\times n)$(对于每一个例子)。
如果我们限制每一个输出拥有的连接数为$k$,那么稀疏的连接方法只需要$k\times n$个参数以及$O(k\times n)$的运行时间。
在很多应用方面,只需保持$k$的数量级远小于$m$,就能在机器学习的任务中取得好的表现。
sparse connectivity(局部感知)的图形化解释如图\ref{fig:chap9_area_of_effect}和图\ref{fig:chap9_receptive_field}所示。
在深度卷积网络中,处在深层的单元可能不直接地与绝大部分输入连接,如图\ref{fig:chap9_deep_receptive_field}所示。
这允许网络可以通过只描述sparse interactions的区域来高效地描述多个变量的复杂交互过程。


% fig 9.2
\begin{figure}[!htb]
\ifOpenSource
\centerline{\includegraphics{figure.pdf}}
\else
\centerline{\includegraphics{Chapter9/figures/area_of_effect}}
\fi
\captionsetup{singlelinecheck=off}
\caption[Caption for LOF]{\gls{sparse_connectivity},对每幅图从下往上看。我们强调了一个输入单元$x_3$以及在$\bm{s}$中受该单元影响的输出单元。\emph{(上)}当$\bm{s}$是由核宽度为3的卷积产生时,只有三个输出受到$\bm{x}$的影响\protect\footnotemark。\emph{(下)}当$\bm{s}$是由矩阵乘法产生时,连接不再是稀疏的,所以所有的输出都会受到$x_3$的影响。}
\label{fig:chap9_area_of_effect}
\end{figure}
 \footnotetext{译者注:译者认为此处应当是$x_3$。}
% fig 9.3
\begin{figure}[!htb]
\ifOpenSource
\centerline{\includegraphics{figure.pdf}}
\else
\centerline{\includegraphics{Chapter9/figures/receptive_field}}
\fi
\captionsetup{singlelinecheck=off}
\caption[.]{\gls{sparse_connectivity},对每幅图从上往下看。我们强调了一个输出单元$s_3$以及$\bm{x}$中影响该单元的输入单元。这些单元被称为$s_3$的\gls{receptive_field}。\emph{(上)}当$\bm{s}$是由核宽度为3的卷积产生时,只有三个输入影响$s_3$。\emph{(下)}当$\bm{s}$是由矩阵乘法产生时,连接不再是稀疏的,所以所有的输入都会影响$s_3$。}


\label{fig:chap9_receptive_field}
\end{figure}
% fig 9.4
\begin{figure}[!htb]
\ifOpenSource
\centerline{\includegraphics{figure.pdf}}
\else
\centerline{\includegraphics{Chapter9/figures/deep_receptive_field}}
\fi
\caption{处于\gls{convolutional_network}更深的层中的单元,它们的\gls{receptive_field}要比处在浅层的单元的\gls{receptive_field}更大。如果网络还包含类似\gls{stride}卷积(图\ref{fig:chap9_stride_conv})或者\gls{pooling}(\ref{sec:pooling}节)之类的结构特征,这种效应会加强。这意味着在\gls{convolutional_network}中即使是\emph{直接}连接都是很稀疏的,处在更深的层中的单元可以\emph{间接地}连接到全部或者大部分输入图像。}
\label{fig:chap9_deep_receptive_field}
\end{figure}

 
\firstgls{parameter_sharing}是指在一个模型的多个函数中使用相同的参数。
在传统的神经网络中,当计算一层的输出时,权值矩阵的每一个元素只使用一次,当它乘以输入的一个元素后就再也不会用到了。
作为\gls{parameter_sharing}的同义词,我们可以说一个网络含有\firstgls{tied_weights},因为用于一个输入的权值也会被绑定在其他的权值上。
在\gls{CNN}中,核的每一个元素都作用在输入的每一位置上(除了一些可能的边界像素,取决于对于边界的决策设计)。
卷积运算中的\gls{parameter_sharing}保证了我们只需要学习一个参数集合,而不是对于每一位置都需要学习一个单独的参数集合。
这虽然没有改变前向传播的时间(仍然是$O(k\times n)$),但它显著地把模型的存储需求降低至$k$个参数,并且$k$通常是远小于$m$的数量级。
因为$m$ 和$n$通常规模很接近,$k$在实际中相对于$m\times n$是很小的。
因此,卷积在存储需求和统计效率方面极大地优于稠密矩阵的乘法运算。
图\ref{fig:chap9_parameter_sharing}演示了\gls{parameter_sharing}是如何实现的。
% fig 9.5
\begin{figure}[!htb]
\ifOpenSource
\centerline{\includegraphics{figure.pdf}}
\else
\centerline{\includegraphics{Chapter9/figures/parameter_sharing}}
\fi
\caption{\gls{parameter_sharing}。黑色箭头表示在两个不同的模型中使用了特殊参数的连接。\emph{(上)}黑色箭头表示在卷积模型中3元素核的中间元素的使用。因为\gls{parameter_sharing},这单个参数被用于所有的输入位置。\emph{(下)}这单个黑色箭头表示在全连接模型中权重矩阵的中间元素的使用。这个模型没有使用\gls{parameter_sharing},所以参数只使用了一次。}
\label{fig:chap9_parameter_sharing}
\end{figure}
% -- 326 --
 
% -- 327 --
 
作为前两条原则的一个实际例子,图\ref{fig:chap9_efficiency_of_edge_detection}说明了\gls{sparse_connectivity}和\gls{parameter_sharing}是如何显著地提高用于图像边缘检测的线性函数的效率的。
% fig 9.6
\begin{figure}
\ifOpenSource
\centerline{\includegraphics{figure.pdf}}
\else
\centering    
\subfigure{ \label{fig:chap9_efficiency_of_edge_detection_a}     
\includegraphics[width=0.35\textwidth]{Chapter9/figures/sundance.png}}     
\subfigure{ \label{fig:chap9_efficiency_of_edge_detection_b}     
\includegraphics[width=0.35\textwidth]{Chapter9/figures/edges.png}}     
\fi
\captionsetup{singlelinecheck=off}
\caption{边缘检测的效率。右边的图像是通过获得原始图像中的每个像素并减去左边相邻像素的值而形成的。这给出了输入图像中所有垂直方向上的边缘的强度,这对目标检测是有用的操作。两个图像都是280像素的高度。输入图像宽320像素,而输出图像宽319像素。这个变换可以通过包含两个元素的卷积核来描述,并且需要$319\times 280\times 3 = 267,960$个浮点运算(每个输出像素需要两次乘法和一次加法)。为了用矩阵乘法描述相同的变换,需要$320\times 280\times 319\times 280$个或者说超过80亿个元素的矩阵,这使得卷积对于表示这种变换更有效40亿倍。直接运行矩阵乘法的算法将执行超过160亿个浮点运算,这使得卷积在计算上大约有60,000倍的效率。当然,矩阵的大多数元素将为零。如果我们只存储矩阵的非零元,则矩阵乘法和卷积都需要相同数量的浮点运算来计算。矩阵仍然需要包含$2\times 319\times 280=178,640$个元素。将小的局部区域上的相同线性变换应用到整个输入上,卷积是描述这种变换的极其有效的方法。照片来源:Paula Goodfellow。}   
\label{fig:chap9_efficiency_of_edge_detection}     
\end{figure}
 
对于卷积,权值的特殊形式使得神经网络层具有平移不变性。
如果一个函数满足输入改变,输出也以同样的方式改变这一性质,我们就说它是等变(equivariant)的。
特别地,如果函数$f(x)$与$g(x)$满足$f(g(x))= g(f(x))$,我们就说$f(x)$对于变换$g$具有等变性。
对于卷积来说,如果令$g$是任意的输入平移函数,那么卷积函数对于$g$具有等变性。
举个例子,令$I$表示图像的明亮度函数(取值为整数),$g$表示图像函数的变换函数(把一个图像函数映射到另一个图像函数的函数)使得$I' = g(I)$,其中$I'(x,y) = I(x-1, y)$。
这个函数把$I$中的每个像素向右移动一格。
如果我们先对$I$进行这种变换然后进行卷积操作所得到的结果,与先对$I$进行卷积然后再对输出使用平移函数$g$得到的结果是一样的。
当处理时间序列数据时,卷积产生一条用来表明输入中出现不同特征的某种时间轴。
如果我们把输入中的一个事件向后延时,在输出中也会有完全相同的表示,只是时间延时了。
图像与之类似,卷积产生了一个2维映射来表明某种属性在输入的什么位置出现了。
如果我们移动输入中的对象,它的表示也会在输出中移动同样的量。
当处理多个输入位置时,一些作用在邻居像素的函数是很有用的。
例如在处理图像时,在CNN的第一层进行图像的边缘检测是很有用的。
相同的边缘或多或少地散落在图像的各处,所以应当对整个图像有权值共享
但在某些情况下,我们并不希望对整幅图共享权值。
例如当我们在处理人脸图像(图像已经被剪裁成人脸在中心)时,我们可能会希望在不同的部位探测出不同的特征(处理人脸上部的网络需要去搜寻眉毛,处理人脸下部的网络就需要去搜寻下巴了)。

 
卷积对其他的一些变换并不是天然等变的,例如对于图像尺度或者角度的变换,需要其他的一些机制来处理这些变换。

最后,一些不能被传统的由(固定大小的)矩阵乘法定义的神经网络处理的特殊数据,可能通过卷积神经网络来处理,我们将在\ref{sec:data_types}节中进行讨论。

\section{pooling}

卷积神经网络的卷积层通常包含三级(如图\ref{fig:chap9_conv_layer}所示)。
在第一级中,卷积层并行地进行多个卷积运算来产生一组线性激活函数。
在第二级中,非线性的激活函数如ReLU函数等作用在第一级中的每一个线性输出上。
这一级有时也被称为detector stage(检测阶段?)。
在第三级中,我们使用pooling funciton函数(下采样函数)来更进一步地调整卷积层的输出。


% fig 9.7
\begin{figure}[!htb]
\ifOpenSource
\centerline{\includegraphics{figure.pdf}}
\else
\centerline{\includegraphics{Chapter9/figures/conv_layer}}
\fi
\caption{典型\gls{CNN}层的组件。有两组常用的术语用于描述这些层。\emph{(左)}在这组术语中,\gls{convolutional_network}被视为少量相对复杂的层,每层具有许多``级''。在这组术语中,核张量与网络层之间存在一一对应关系。在本书中,我们通常使用这组术语。\emph{(右)}在这组术语中,\gls{convolutional_network}被视为更大数量的简单层;每一个处理步骤都被认为是一个独立的层。这意味着不是每个``层''都有参数。}
\label{fig:chap9_conv_layer}
\end{figure}

pooling函数(下采样层)使用某一位置的相邻输出的总体统计特征来代替网络在该位置的输出。
例如,max pooling函数给出相邻矩形区域内的最大值。
其他常用的pooling函数包括相邻矩形区域内的平均值、$L^2$范数以及依靠据中心像素距离的加权平均函数。

 
不管采用什么样的pooling函数,当输入作出少量平移时,pooling能帮助我们的表示近似不变的。
对于平移的不变性是说当我们把输入平移一微小的量,大多数通过pooling函数的输出值并不会发生改变。
图\ref{fig:chap9_max_pool_invariance}用了一个例子来说明这是如何实现的。
局部平移不变性是一个很重要的性质,尤其是当我们关心某个特征是否出现而不关心它出现的具体位置时。
例如,当判定一张图像中是否包含人脸时,我们并不需要知道眼睛的具体像素位置,我们只需要知道有一只眼睛在脸的左边,有一只在右边就行了。
但在一些其他领域,保存特征的具体位置却很重要。
例如当我们想要寻找一个由两条边相交而成的拐角时,我们就需要很好地保存边的位置来判定它们是否相交。

% fig 9.8
\begin{figure}[!htb]
\ifOpenSource
\centerline{\includegraphics{figure.pdf}}
\else
\centerline{\includegraphics{Chapter9/figures/max_pool_invariance}}
\fi
\caption{\gls{max_pooling}引入不变性。\emph{(上)}卷积层中间输出的视图。下面一行显示非线性的输出。上面一行显示\gls{max_pooling}的输出,每个\gls{pool}的宽度为三个像素并且\gls{pooling}区域的\gls{stride}为一个像素。\emph{(下)}相同网络的视图,不过对输入右移了一个像素。下面一行的所有值都发生了改变,但上面一行只有一半的值发生了改变,这是因为\gls{max_pooling}单元只对周围的最大值比较敏感,而不是对精确的位置。}
\label{fig:chap9_max_pool_invariance}
\end{figure}

 
使用pooling可以认为网络结构中增加了一个无限强的先验:卷积层学得的函数必须具有对少量平移的不变性。当这个假设成立时,pooling可以极大地提高网络的统计效率。

pooling对于空间区域具有平移不变性,但当我们对于分离参数的卷积输出进行pooling操作时,特征能够学得应该对于哪种变换具有不变性(如图\ref{fig:chap9_learned_rotation}所示)。


% fig 9.9
\begin{figure}[!htb]
\ifOpenSource
\centerline{\includegraphics{figure.pdf}}
\else
\centerline{\includegraphics{Chapter9/figures/learned_rotation}}
\fi
\caption{学习不变性的示例。使用分离的参数学得多个特征,再使用\gls{pooling}单元进行\gls{pooling},可以学得对输入的某些变换的不变性。这里我们展示了用三个学得的过滤器和一个\gls{max_pooling}单元可以学得对旋转变换的不变性。这三个过滤器都旨在检测手写的数字5。每个过滤器尝试匹配稍微不同方向的5。当输入中出现5时,相应的过滤器会匹配它并且在探测单元中引起大的激活。然后,无论哪个探测单元被激活,\gls{max_pooling}单元都具有大的激活。我们在这里展示网络如何处理两个不同的输入,导致两个不同的探测单元被激活。然而对\gls{pooling}单元的影响大致相同。这个原则在\gls{maxout}网络\citep{Goodfellow-et-al-ICML2013}和其他卷积网络中使用。空间位置上的\gls{max_pooling}对于平移是天然不变的;这种多通道方法只在学习其他变换时是必要的。}
\label{fig:chap9_learned_rotation}
\end{figure}

 
因为pooling总结了全部的周围邻居的反馈,这使得pooling单元少于探测单元成为可能,我们可以通过综合pooling区域的$k$个像素的统计特征而不是单个像素来实现。
图\ref{fig:chap9_pool_downsample}给出了一个例子。
这种方法提高了网络的计算效率,因为下一层少了约$k$ 倍的输入。
当下一层的参数数目是其输入大小的函数时(例如当下一层是全连接的依赖矩阵乘法的网络层时),这种对于输入规模的减小也可以提高统计效率并且减少对于参数的存储需求。


% fig 9.10
\begin{figure}[!htb]
\ifOpenSource
\centerline{\includegraphics{figure.pdf}}
\else
\centerline{\includegraphics{Chapter9/figures/pool_downsample}}
\fi
\caption{带有\gls{downsampling}的\gls{pooling}。这里我们使用\gls{max_pooling},\gls{pool}的宽度为三并且\gls{pool}之间的\gls{stride}为二。这使得表示的大小减少了一半,减轻了下一层的计算和统计负担。注意到最右边的\gls{pooling}区域尺寸较小,但如果我们不想忽略一些探测单元的话就必须包含这个区域。}
\label{fig:chap9_pool_downsample}
\end{figure}

 
在很多任务中,pooling对于处理不同大小的输入具有重要作用。
例如我们想对不同大小的图像进行分类时,分类层的输入必须是固定的大小,而这通常通过调整pooling区域的偏置大小来实现,这样分类层总是能接收到相同数量的统计特征而不管最初的输入大小了。
例如,最终的pooling层可能会输出四组综合统计特征,每组对应着图像的一个象限,而与图像的大小无关。

一些理论工作对于在不同情况下应当使用哪种pooling函数给出了一些指导。
动态地把特征pooling在一起也是可行的,例如,通过针对具有特定性质的位置运行聚类算法。
这种方法对于每幅图像产生一个不同的pooling区域集合。
另一种方法是先学习一个单独的pooling结构,再应用到全部的图像中。

pooling可能会使得一些利用自顶向下信息的神经网络结构变得复杂,例如玻尔兹曼机和稀疏自编码器。
这些问题将在第\ref{part:deep_learning_research}部分中当我们遇到这些类型的网络时进一步讨论。
convolutional Boltzmann machines 中的\gls{pooling}出现在\ref{sec:convolutional_boltzmann_machines}节。
一些可微网络中需要的在{pooling单元中进行的类逆运算将在\ref{sec:convolutional_generative_networks}节中讨论。

图\ref{fig:chap9_cnn_classifier}给出了一些使用卷积和\gls{pooling}操作的用于分类的\gls{CNN}的完整结构的例子。
% fig 9.11
\begin{figure}[!htb]
\ifOpenSource
\centerline{\includegraphics{figure.pdf}}
\else
\centerline{\includegraphics{Chapter9/figures/cnn_classifier}}
\fi
\caption{\gls{convolutional_network}用于分类的架构示例。本图中使用的具体\gls{stride}和深度并不适合实际使用;它们被设计得非常浅以适合页面。实际的\gls{convolutional_network}也常常涉及大量的分支,不同于这里为简单起见所使用的链式结构。\emph{(左)}处理固定大小的图像的\gls{convolutional_network}。在卷积层和\gls{pooling}层几层交替之后,卷积特征映射的张量被重新整形以展平空间维度。网络的其余部分是一个普通的前馈网络分类器,如第\ref{chap:deep_feedforward_networks}章所述。\emph{(中)}处理大小可变的图像的\gls{convolutional_network},但仍保持全连接的部分。该网络使用具有可变大小但是数量固定的\gls{pool}的\gls{pooling}操作,以便向网络的全连接部分提供576个单位的固定大小的向量。 \emph{(右)}没有任何全连接权重层的\gls{convolutional_network}。相反,最后的卷积层为每个类输出一个特征映射。该模型可能学习每个类在每个空间位置出现的可能性的映射。将特征映射进行平均得到的单个值,提供了顶部softmax分类器的变量。}
\label{fig:chap9_cnn_classifier}
\end{figure}

\section{卷积与\glsentrytext{pooling}作为一种无限强的先验}


回忆一下\ref{sec:capacity_overfitting_and_underfitting}节中prior probability distribution(先验概率分布)的概念。

这是一个模型参数的概率分布,它表示在我们了解数据之前我们假设数据是处于什么分布。

 
先验被认为是强或者弱取决于先验中概率密度的集中程度。
弱先验具有较高的熵值,例如方差很大的gaussian distribution,这样的先验允许数据对于参数的改变具有或多或少的自由性。
强先验具有较低的熵值,例如方差很小的gaussian distribution,这样的先验在决定参数最终取值时起着更加积极的作用。

一个无限强的先验对一些参数的概率置零并且要求禁止对这些参数赋值,无论数据对于这些参数的值给出了多大的支持。

我们可以把CNN想成和全连接网络类似,但对于这个全连接网络的权值有一个无限强的先验。
这个无限强的先验是说一个隐藏神经元的权值必须和它邻居的权值相等,但在空间中改变。
这个先验也要求除了那些处在神经元空间连续的小的接收域以内的权值外,其余的权值都为零。

总之,我们可以把卷积的使用当作是对网络中一层的参数引入了一个无限强的先验概率分布。
这个先验是说该层应该学得的函数只包含局部连接关系并且对平移具有等变性。
类似的,使用下采样层也是一个无限强的先验:每一个单元都具有对少量平移的不变性。

当然,把CNN当作一个具有无限强先验的全连接网络来实现会导致极大的计算浪费。
但把CNN想成具有无限强先验的全连接网络可以帮助我们更好地理解CNN是如何工作的。

其中一个关键的地方是卷积和pooling可能导致欠拟合。
与任何其他先验类似,卷积和pooling只有当先验的假设合理且正确时才有用。

如果一项任务依赖于保存精确的空间信息,那么在所有的特征上使用pooling将会增大训练误差。
一些CNN\citep{Szegedy-et-al-arxiv2014}为了既获得具有较高不变性的特征又获得当平移不变性不合理时不会导致欠拟合的特征,被设计成在一些通道上使用\gls{pooling}而在另一些通道上不使用。
当一项任务涉及到要对输入中相隔较远的信息进行合并时,那么卷积所需要的先验可能就不正确了。

另一个关键洞察是当我们比较卷积模型的统计学习表现时,只能以基准中的其他卷积模型作为比较的对象。
其他不使用卷积的模型即使我们把图像中的所有像素点都置换后依然有可能进行学习。
对于许多图像数据集,还有一些分别的基准,有些是针对那些具有\firstgls{permutation_invariant}并且必须通过学习发现拓扑结构的模型,还有一些是针对设计者将空间关系的知识通过硬编码给了它们的模型。

% -- 336 --

\section{基本卷积函数的变体}
\label{sec:variants_of_the_basic_convolution_function}

当在神经网络的上下文中讨论卷积时,我们通常不是特指数学文献中使用的那种标准的离散卷积运算。
实际应用中的函数略微有些不同。
这里我们详细讨论一下这些差异,并且对神经网络中用到的函数的一些重要性质进行重点说明。

首先,当我们提到神经网络中的卷积时,我们通常是指一次特定的运算,而这种运算包含了并行地使用多个卷积。
这是因为带有单个核的卷积只能提取一种类型的特征,尽管它作用在多个空间位置上。
我们通常希望神经网络的一层能够在多个位置提取多种类型的特征。

另外,输入通常也不仅仅是实值的网格,而是由一系列向量值的观测数据构成的网格。
例如,一幅彩色图像在每一个像素点都会有红绿蓝三种颜色的亮度。
在多层的CNN中,第二层的输入是第一层的输出,通常在每个位置包含多个卷积的输出。
当用于图像时,我们通常把卷积的输入输出都看作是3维的张量,其中一个索引用于标明不同的通道(例如红绿蓝),另外两个索引标明在每个通道上的空间坐标。
软件实现通常使用批处理模式,所以它们会使用4维的张量,第四维索引用于标明批处理中不同的实例,但我们为简明起见这里忽略批处理索引。

因为CNN通常使用多通道的卷积,它们基于的线性运算并不保证一定是可交换的,即使使用了核翻转也是如此。
这些多通道的运算只有当其中的每个运算的输出和输入具有相同的通道数时才是可交换的。

假定我们有一个4维的核张量$\TSK$,它的每一个元素是$\TEK_{i,j,k,l}$,表示输出的处于通道$i$中的一个单元和输入的处于通道$j$中的一个单元的连接强度,并且在输出单元和输入单元之间有一个$k$行$l$列的偏置。

假定我们的输入由观测数据$\TSV$组成,它的每一个元素是$\TEV_{i,j,k}$,表示处在通道$i$中第$j$行第$k$列的值。

假定我们的输出$\TSZ$和输入$\TSV$具有相同的形式。如果输出$\TSZ$是通过对$\TSK$和$\TSV$进行卷积而不涉及翻转$\TSK$得到的,那么

\begin{equation}
\TEZ_{i,j,k} = \sum_{l,m,n} \TEV_{l, j+m-1, k+n-1} \TEK_{i,l,m,n},
\end{equation}

这里对所有的$l$,$m$和$n$进行求和是对所有(在求和式中)有效的张量索引的值进行求和。
在线性代数中,向量的索引通常从1开始,这就是上述公式中$-1$的由来。
但是像C或Python这类编程语言索引通常从0开始,这使得上述公式可以更加简洁。

 
我们有时会希望跳过核中的一些位置来降低计算的开销(相应的代价是提取特征没有先前那么好了)。
我们可以把这一过程看作是对卷积函数输出的下采样(downsampling)。
如果我们只想对输出的每个方向上的$s$个像素进行采样,那么我们可以定义一个下采样卷积函数$c$使得
\begin{equation}
\TEZ_{i,j,k} = c(\TSK, \TSV, s)_{i,j,k} = \sum_{l,m,n} [\TEV_{l,(j-1)\times s+m, (k-1)\times s +n,}
 \TEK_{i,l,m,n}].
 \label{eq:9.8}
\end{equation}
我们把$s$称为下采样卷积的stride(步长)。
当然也可以对每个移动方向定义不同的步幅。
图\ref{fig:chap9_stride_conv}演示了一个实例。
% fig 9.12
\begin{figure}[!htb]
\ifOpenSource
\centerline{\includegraphics{figure.pdf}}
\else
\centerline{\includegraphics{Chapter9/figures/stride_conv}}
\fi
\caption{带有\gls{stride}的卷积。在这个例子中,我们的\gls{stride}为二。\emph{(上)}在单个操作中实现的\gls{stride}为二的卷积。\emph{(下)}\gls{stride}大于一个像素的卷积在数学上等价于单位\gls{stride}的卷积随后\gls{downsampling}。显然,涉及\gls{downsampling}的两步法在计算上是浪费的,因为它计算了许多将被丢弃的值。}
\label{fig:chap9_stride_conv}
\end{figure}

在任何CNN的应用中都有一个重要性质,那就是能够隐含地对输入$\TSV$用零进行填充(pad)使得它加宽。
如果没有这个性质,表示的宽度在每一层就会缩减,缩减的幅度是比核少一个像素这么多。
对输入进行零填充允许我们对核的宽度和输出的大小进行独立的控制。
如果没有零填充,我们就被迫面临二选一的局面,要么选择网络空间宽度的快速缩减,要么选择一个小型的核——这两种情境都会极大得限制网络的表示能力。

图\ref{fig:chap9_zero_pad_shrink}给出了一个例子。
% fig 9.13
\begin{figure}[!htb]
\ifOpenSource
\centerline{\includegraphics{figure.pdf}}
\else
\centerline{\includegraphics{Chapter9/figures/zero_pad_shrink}}
\fi
\caption{零填充对网络大小的影响。考虑一个\gls{convolutional_network},每层有一个宽度为六的核。 在这个例子中,我们不使用任何\gls{pooling},所以只有卷积操作本身缩小网络的大小。\emph{(上)}在这个卷积网络中,我们不使用任何隐含的零填充。这使得表示在每层缩小五个像素。从十六个像素的输入开始,我们只能有三个卷积层,并且最后一层不能移动核,所以可以说只有两层是真正的卷积层。可以通过使用较小的核来减缓收缩速率,但是较小的核表示能力不足,并且在这种结构中一些收缩是不可避免的。\emph{(下)}通过向每层添加五个隐含的零,我们防止了表示随深度收缩。这允许我们设计一个任意深的卷积网络。}
\label{fig:chap9_zero_pad_shrink}
\end{figure}

有三种零填充设定的情况值得注意。
第一种是无论怎样都不使用零填充的极端情况,并且卷积核只允许访问那些图像中能够完全包含整个核的位置。
在MATLAB中,这称为valid卷积。

在这种情况下,输出的所有像素都是输入中相同数量像素的函数,这使得输出像素的表示更加规范。
然而,输出的大小在每一层都会缩减。
如果输入的图像宽度是$m$,核的宽度是$k$,那么输出的宽度就会变成$m-k+1$。
如果卷积核非常大的话缩减率会非常显著。
因为缩减数大于0,这限制了网络中能够包含的卷积层的层数。
当层数增加时,网络的空间维度最终会缩减到$1\times 1$,这种情况下另外的层就不可能进行有意义的卷积了。
第二种特殊的情况是只进行足够的零填充来保持输出和输入具有相同的大小
。
在MATLAB中,这称为same卷积。
在这种情况下,网络能够包含任意多的卷积层,只要硬件可以支持,这是因为卷积运算并没有改变相关的结构。
然而,输入像素中靠近边界的部分相比于中间部分对于输出像素的影响更小。
这可能会导致边界像素存在一定程度的欠表示。
这使得第三种极端情况产生了,在MATLAB中称为full卷积。

它进行了足够多的零填充使得每个像素在每个方向上恰好被访问了$k$次
这将导致学得一个在卷积特征映射的所有位置都表现不错的单核更为困难。
通常零填充的最优数量(对于测试集的分类正确率)处于``valid卷积''和``same卷积''之间的某个位置。



在一些情况下,我们并不一定真正想用卷积,而只是用一些局部连接的网络层。
在这种情况下,我们的多层感知机对应的邻接矩阵是相同的,但每一个连接都有它自己的权重,用一个6维的张量$\TSW$来表示。
$\TSW$的索引分别是:输出的通道$i$,输出的行$j$和列$k$,输入的通道$l$,输入的行偏置$m$和列偏置$n$。
局部连接层的线性部分可以表示为

\begin{equation}
\TEZ_{i,j,k} = \sum_{l,m,n} [\TEV_{l, j+m-1, k+n-1} w_{i, j, k, l, m, n}]. %这里应该是$\TEW$?,不清楚,待考证
\end{equation}
这有时也被称为unshared convolution(非权值共享的卷积?),因为它和带有一个小核的离散卷积运算很像,但并不横跨位置来共享参数。
图\ref{fig:chap9_local}比较了局部连接、卷积和全连接的区别。


% fig 9.14
\begin{figure}[!htb]
\ifOpenSource
\centerline{\includegraphics{figure.pdf}}
\else
\centerline{\includegraphics{Chapter9/figures/local}}
\fi
\caption{局部连接,卷积和全连接的比较。\emph{(上)}每一小片(接受域)有两个像素的局部连接层。每条边用唯一的字母标记,来显示每条边都有自身的权重参数。\emph{(中)}核宽度为两个像素的卷积层。该模型与局部连接层具有完全相同的连接。区别不在于哪些单元相互交互,而在于如何共享参数。局部连接层没有\gls{parameter_sharing}。卷积层在整个输入上重复使用相同的两个权重,正如用于标记每条边的字母重复出现所指示的那样。\emph{(下)}全连接层类似于局部连接层,它的每条边都有其自身的参数(在该图中用字母明确标记的话就太多了)。 然而,它不具有局部连接层的连接受限的特征。}
\label{fig:chap9_local}
\end{figure}
 

 
当我们知道每一个特征都是一小部分空间的函数而不是整个空间的特征时,局部连接层是很有用的。
例如,如果我们想要辨别一张图片是否是人脸图像时,我们只需要去寻找嘴是否在图像的下部中央部分即可。

使用那些连接被更进一步限制的卷积或者局部连接层也是有用的,例如,限制每一个输出的通道$i$仅仅是输入通道$l$的一部分的函数时。
实现这种情况的一种通用方法是使输出的前$m$个通道仅仅连接到输入的前$n$个通道,输出的接下来的$m$个通道仅仅连接到输入的接下来的$n$个通道,以此类推。


图\ref{fig:chap9_conv_groups}给出了一个例子。
对少量通道间的连接进行建模允许网络使用更少的参数,这降低了存储的消耗以及提高了统计效率,并且减少了前向和反向传播所需要的计算量。
这些目标的实现并没有减少隐藏神经元的数目。


% fig 9.15
\begin{figure}[!htb]
\ifOpenSource
\centerline{\includegraphics{figure.pdf}}
\else
\centerline{\includegraphics{Chapter9/figures/conv_groups}}
\fi
\caption{\gls{convolutional_network}的前两个输出通道只和前两个输入通道相连,随后的两个输出通道只和随后的两个输入通道相连。}
\label{fig:chap9_conv_groups}
\end{figure}

tiled convolution对卷积层和局部连接层进行了折衷。
这里并不是对每一个空间位置的权重集合进行学习,我们学习一组核使得当我们在空间移动时它们可以循环利用。
这意味着在近邻的位置上拥有不同的过滤器,就像局部连接层一样,但是对于这些参数的存储需求仅仅会增长常数倍,这个常数就是核的集合的大小,而不是整个输出的特征映射的大小。


图\ref{fig:chap9_tiled}对局部连接层、tiled convolution和标准卷积进行了比较。
% fig 9.16
\begin{figure}[!htb]
\ifOpenSource
\centerline{\includegraphics{figure.pdf}}
\else
\centerline{\includegraphics{Chapter9/figures/tiled}}
\fi
\captionsetup{singlelinecheck=off}
\caption[.]{局部连接层、\gls{tiled_convolution}和标准卷积的比较。当使用大小相同的核时,这三种方法的单元之间具有相同的连接。此图是对使用两个像素宽的核的说明。这三种方法之间的区别在于它们如何共享参数。\emph{(上)}局部连接层根本没有共享参数。我们对每个连接使用唯一的字母标记,来表明每个连接都有它自身的权重。\emph{(中)}\gls{tiled_convolution}有$t$个不同的核。这里我们说明$t=2$的情况。其中一个核具有标记为``a''和``b''的边,而另一个具有标记为``c''和``d''的边。我们每次在输出中向右移动一个像素,移动后使用不同的核。这意味着,与局部连接层类似,输出中的相邻单元具有不同的参数。与局部连接层不同的是,在我们遍历所有可用的$t$个核之后,我们循环回到了第一个核。如果两个输出单元间隔$t$个步长的倍数,则它们共享参数。\emph{(下)}传统卷积等效于$t=1$的\gls{tiled_convolution}。它只有一个核,并且被应用到各个地方,我们在图中表示为在各处使用具有标记为``a''和``b''的边的核。}
\label{fig:chap9_tiled}
\end{figure}
 

 
为了代数地定义tiled convolution},令$\TSK$是一个6维的张量
我们这里并不是使用分别的索引来表示输出映射中的每一个位置,输出的位置在每个方向上在$t$个不同的核的组成的集合中进行循环。
如果$t$等于输出的宽度,这就是局部连接层了。

\begin{equation}
\TEZ_{i, j, k} = \sum_{l, m, n} \TEV_{l, j+m-1, k+n-1} \TEK_{i, l, m, n, j\% t +1, k\% t+1},
\end{equation}
这里百分号是取模运算,其中$t\% t =0, (t+1)\% t = 1$等等。
在每一维上使用不同的$t$可以很直观地对这个公式进行扩展。
 

 
局部连接层与tiled convolution层都和max pooling有一些关联:这些层的探测单元都是由不同的滤波器产生的。
如果这些过滤器能够学会探测相同隐含特征的不同变换形式,那么max_pooling的单元对于学得的变换就具有不变性(如图\ref{fig:chap9_learned_rotation}所示)。
卷积层对于平移具有内置的不变性。
 

 
实现CNN时,采用除卷积以外一般要进行其他的操作。
为了实现学习,必须在给定输出的梯度时能够计算核的梯度。
在一些简单情况下,这种运算可以通过卷积来实现,但在很多我们感兴趣的情况下,包括stride大于1的情况,并不具有这样的性质。

由于卷积是一种线性运算,所以可以表示成矩阵乘法的形式(如果我们首先把输入张量变形为一个扁平的向量)。涉及到的矩阵是卷积核的函数。
这个矩阵是稀疏的并且核的每个元素都复制给矩阵的很多个元素。
这种观点能够帮助我们导出CNN需要的很多其他运算。

通过卷积定义的矩阵转置的乘法就是这样一种运算。
这种运算用于通过卷积层反向传播误差的导数,所以它在训练多于一个隐藏层的CNN时是必要的。
如果我们想要从隐藏层单元重构可视化单元时,同样的运算也是需要的。
重构可视化单元是本书第\ref{part:deep_learning_research}部分的模型广泛用到的一种运算,这些模型包括稀疏自编码器,限制玻尔兹曼机和稀疏编码等等。


构建这些模型的卷积化的版本都要用到转置化卷积。
就像核梯度的运算,这种输入梯度运算在某些情况下可以用卷积来实现,但在一般情况下需要用到第三种运算来实现。
必须非常小心地来使这种转置运算和前向传播过程相协调。
转置运算返回的输出的大小取决于三个方面:零填充的策略、前向传播运算的步长和前向传播的输出映射的大小。
在一些情况下,不同大小的输入通过前向传播过程能够得到相同大小的输出映射,所以必须明确地告知转置运算原始输入的大小。

这三种运算——卷积、从输出到权重的反向传播和从输出到输入的反向传播——对于训练任意深度的前馈卷积网络,以及训练带有(基于卷积的转置的)重构函数的卷积网络,这三种运算都足以计算它们所需的所有梯度。

对于完全一般的多维、多样例情况下的公式,完整的推导可以参见\cite{Goodfellow-TR2010}。 
为了直观说明这些公式是如何起作用的,我们这里给出一个二维单个样例的版本。
 

 
假设我们想要训练这样一个CNN,它包含步长为$s$的卷积,该卷积的核为$\TSK$,作用于多通道的图像$\TSV$,表示为$c(\TSK, \TSV, s)$,就像公式\ref{eq:9.8}中一样。
假设我们想要优化某个损失函数$J(\TSV, \TSK)$。
在前向传播过程中,我们需要用$c$本身来输出$\TSZ$,然后$\TSZ$传递到网络的其余部分并且被用来计算损失函数$J$。
在反向传播过程中,我们会收到一个张量$\TSG$表示为$\TEG_{i, j, k} = \frac{\partial}{\partial \TEZ_{i, j, k}} J(\TSV, \TSK)$。

为了训练网络,我们需要对卷积核中的权重求导。
为了实现这个目的,我们使用一个函数

\begin{equation}
g(\TSG, \TSV, s)_{i, j, k, l} = \frac{\partial}{\partial \TEK_{i, j, k, l}} J(\TSV, \TSK) = \sum_{m, n} \TEG_{i, m, n} \TEV_{j, (m-1)\times s+k, (n-1)\times s+l}.
\end{equation}

如果这一层不是网络的最后一层,我们需要对$\TSV$求梯度来使得误差进一步反向传播。
我们可以使用如下的函数

\begin{eqnarray}
h(\TSK, \TSG, s)_{i, j, k} &=& \frac{\partial }{\partial \TEV_{i, j, k}} J(\TSV, \TSK)\\
&=& \sum_{\substack{l, m\\
                  \text{s.t.}\\
                  (l-1)\times s+m = j}} \sum_{\substack{n, p\\
                                                            \text{s.t.}\\
                                                            (n-1)\times s +p = k}}
            \sum_q \TEK_{q,i,m,p} \TEG_{q, l, n}.
\end{eqnarray}


第\ref{chap:autoencoders}章描述的稀疏自编码器网络,是一些训练成把输入复制到输出的前馈网络。
一个简单的例子是PCA(主成分分析)算法,将输入$\bm{x}$拷贝到一个近似的重构值$\bm{r}$,通过函数$\bm{W}^\top \bm{Wx}$来实现。
使用权重矩阵转置的乘法,就像PCA算法这种,在一般的AE(稀疏自编码器)中是很常见的。
为了使这些模型卷积化,我们可以用函数$h$来实现卷积运算的转置。
假定我们有和$\TSZ$相同格式的隐藏单元$\TSH$,并且我们定义一种重构运算

\begin{equation}
\TSR = h(\TSK, \TSH, s).
\end{equation}

为了训练AE,我们会收到关于$\TSR$的梯度,表示为一个张量$\TSE$。
为了训练解码器,我们需要获得对于$\TSK$的梯度,通过$g(\TSH, \TSE, s)$来得到。
为了训练编码器,我们需要获得对于$\TSH$的梯度,通过$c(\TSK, \TSE, s)$来得到。
也可能通过用$c$和$h$对$g$求微分得到,但这些运算对于任何标准神经网络上的反向传播算法来说都是不需要的。
 

 
一般来说,在卷积层从输入到输出的变换中我们不仅仅只用线性运算。
我们一般也会在进行非线性运算前,对每个输出加入一些偏置项。
这样就产生了如何在偏置项中共享参数的问题。
对于局部连接层,很自然地对每个单元都给定它特有的偏置,对于tiled convolution,也很自然地用与核一样的拼接模式来共享参数。
对于卷积层来说,通常的做法是在输出的每一个通道上都设置一个偏置,这个偏置在每个卷积映射的所有位置上共享。
然而,如果输入是已知的固定大小,也可以在输出映射的每个位置学习一个单独的偏置。
分离这些偏置可能会稍稍降低模型的统计效率,但同时也允许模型来校正图像中不同位置的统计差异。
例如,当使用隐含的零填充时,图像边缘的探测单元接收到较少的输入,因此需要较大的偏置。


%%%%%%%%%%%%%%%%%%%%%%%%%%%%%%%%%%%%%%%%%%%%%%%%%%%%%%%%%
%%%%%%%%%%%%%%%%%%% author: iWeisskohl %%%%%%%%%%%%%%%%%%
%%%%%%%%%%%%%%%%%%% part:9.7-9.11      %%%%%%%%%%%%%%%%%%
%%%%%%%%%%%%%%%%%%%%%%%%%%%%%%%%%%%%%%%%%%%%%%%%%%%%%%%%%

\section{9.6}
\label{sec:9.6}
%%%%%%%%%%%%%%%%%%%%%%%%%%%%%%%%%%%%%%%%%%%%%%%%%%%%%%%%%
%%%%%%%%%%%%%%%%% author:dormir_yin %%%%%%%%%%%%%%%%%%%%%%%%%%%%%%
%%%%%%%%%%%%%%%%%%%%%%%%%%%%%%%%%%%%%%%%%%%%%%%%%%%%%%%%%

\chapter{序列模型:循环网络与递归网络}
\label{chap:10}
循环神经网络(Recurrent neural networks,简称RNN),是一系列处理序列数据(sequential data)的神经网络的统称。前面的章节我们介绍的卷积神经网络是专门来处理格状结构的数据X的,比如说图像。而循环神经网络是用来处理序列的,序列一般表示为:$x^{(1)}...x^{(\tau)}$\footnote{译者注:序列数据可以是时间序列,可以是空间序列,当然也可以是其他形式的的序列。为了便于理解我们假设我们处理的是时间序列,上标对应的是不同的时间点。每个时间点对应的输入向量是$x^{(t)}$}。我们已经知道卷积神经网络在处理长宽都很大的图像和处理小图像模型并没有很大的不同,甚至有的卷积神经网络模型可以处理大小不固定的输入图像。递归神经网络也有类似能力,当递归神经网络处理序列时,并不需要针对不同长度的序列分别设计一个模型。大部分递归神经网络模型可以处理长度变化的输入序列。

在我们介绍循环神经网络之前,首先向大家介绍一下在八十年代机器学习和统计模型领域提出的一个观点:在模型的不同部分共享参数(sharing parameters)。参数共享可以帮助我们扩展模型,让模型去处理不同形式(在这里主要是不同长度的序列)的输入数据。处理序列数据时,如果我们对序列中每个不同的时间点对应的输入元素使用不同的参数,我们就不能处理长度未知的序列,训练的过程中也不能享有统计优势(statistical strength)\footnote{译者注;感觉作者想要表达的意思是序列中每个元素有可能都不是相互独立的,我们如果使用参数共享,可以从整体来提取一些共性的东西。举个例子,如果数据集是从某个分布采样得到的,如果我们对这个数据的整体来处理,反而更容易得到这个分布的信息}。当某个有用的信息在序列的不同位置都有可能出现时,参数共享就变得尤为重要。举个例子:比较两个句子“I went to Nepal in 2009”  和 “In 2009,I went to Nepal.”,如果我们把这两个句子输入机器学习模型,让模型把叙述者去Nepal的年份给提取出来,我们期望模型识别2009作为有用的信息,尽管这个词在这个两个句子中出现的位置不一样:一个是第六个单词,一个是第二个单词。假设我们用前馈神经网络处理句子。一个传统的全连接前馈神经网络只能处理固定长度的句子序列,对序列不同位置或者不同时间点的输入的特征都对应不同的参数,所以它需要学习句子中每个位置所对应的语法规则。与传统的全连接网络不同,循环神经网络在序列的每个时间点所对应的权值参数是一样的,也就是说序列不同位置的输入特征会共享参数。

其实当我们在在1维时间序列(1-D temporal sequence)中使用卷积的时候,也使用了参数共享。序列1维卷积的方法也是时间延迟神经网络(time-delay neural networks)的基础(Lang and Hinton, 1988; Waibel et al., 1989; Lang et al., 1990)。卷积操作允许神经网络在不同的时间上共享权值。但是这个网络特别浅。序列经过卷积之后,它的输出依然是一个序列。输出序列的每个成员是对应位置输入与周围相邻的输入经过某个函数的映射输出。在每个时间点我们都使用了相同的卷积核,这也正体现了前面所提的参数共享的观点。在循环神经网络中,参数共享是以另一种方式实现的。1维卷积只考虑对应位置附近的特征,而在RNN中每个输出序列里的成员都是之前所有的输出经过一个函数映射得到的。也就是说在产生当前时间点输出的时候,我们要考虑之前所有的输出。在计算输出序列每个成员的时候会考虑所有的历史输出,而所采取的映射规则是一致的。如果用计算图来展示这个递归模式,会发现是在一个非常深的计算图中做参数共享。

为了便于理解,我们记RNN处理的序列数据为一系列向量$\bm{x}^{(t)}$。$t$表示向量在序列中的位置,我们可以理解为时间点,范围是从$1$到$\tau$。在实际应用中,RNN通常分批次(minibatch)处理序列,每个minibatch里序列的长度$\tau$可能不一样。为了简化,我们没有在下标中标注批次。实际上,序列中向量的下标不一定要与现实生活中的时间对应起来,有的时候它仅仅表示该成员在序列中的位置。RNN有时也可以处理二维空间数据,比如说图像。在处理与时间有关的序列时,RNN还可以反向处理数据,或者说神经网络在处理序列之前已经将整个序列看了一遍。\footnote{这段现在解释起来有点难度,在看到双向循环神经网络就会理解了。} 

这章会对我们之前学过的计算图进行扩展,我们会看到带有环状结构的计算图。这个环表示当前时间点会作为一个变量影响未来时间点的值。这类计算图可以帮助我们定义循环神经网络。我们随后会展示用不同的方式构建,训练和使用循环神经网络。

如果你想了解更多和循环神经网络有关的内容,可以看一下Graves (2012)。

\section{计算图的展开}
\label{sec:10.1}
计算图可以形象的展示一系列计算操作,在神经网络中计算题可以清晰的展示如何将输入和模型参数映射成为输出和损失。可以从\ref{sec:6.5.1} 看一下更详细的介绍。在这一节我们将会解释如何将递归或者循环计算展开成具有重复结构的计算图,这个展开的计算题是一个链式结构。在展开过程中我们或看到如何在一个深的网络结构中使用参数共享。
举个例子,考虑一下典型的动态系统:
\begin{equation}
\bm{s}^{(t)} = f(\bm{s}^{(t-1)};\theta)
\label{form:10.1}
\end{equation}

其中$\bm{s}^{(t)}$表示当前系统状态。
等式\ref{form:10.1}是递归的因为$t$时刻状态$\bm{s}^{(t)}$的定义会使用到$t-1$时刻的状态$\bm{s}^{(t-1)}$定义,状态定义的方式是一样的,所以这个式子是递归的。

对于一个有限长的序列,假设他的长度是$\tau$,我们只需要重复使用$\tau-1$次上面的式子就可以把计算图展开。比如说我们要展开式子$\tau =3$,我们可以得到
\begin{eqnarray}
\bm{s}^{(3)} & = & f(\bm{s}^{(2)};\theta) \\
& = & f(f(\bm{s}^{(1)};\theta);\theta)
\end{eqnarray}
通过重复的调用等式\ref{form:10.1}来展开这个等式,我们可以得到一个不包含递归部分的表达式。这个表达式就可以用传统无环计算图来表示。式子\ref{form:10.1}对应展开的计算图可以看 
图\ref{fig:10_1}。
\begin{figure}[htbp] %  figure placement: here, top, bottom, or page
   \centering
   \includegraphics[width=6in]{fig/chap10/10_1.PNG} 
   \caption{这是一个展开的计算图,描述的是等式 定义的一个动态系统。每个节点代表不同时刻$t$的状态,函数$f$将$t$时刻的状态映射到$t+1$时刻的状态,每个$f$所使用的参数是一样的。}
   \label{fig:10_1}
\end{figure}

我们现在看另一个例子,我们考虑另一个动态系统,这个系统每次会接收一个外部的信号$\bm{x}^{(t)}$,
\begin{equation}
\bm{s}^{(t)} = f(\bm{s}^{(t-1)},\bm{x}^{(t)};\theta)
\label{form:10.4}
\end{equation}
我们可以看到当前状态包含所有之前时间点的历史信息。

循环神经网络可以用不同的方式来构建。我们所使用的函数几乎都可以看作是前馈神经网络(输入经过函数映射到输出),如果在函数外面再引入循环结构,那么就可以被看作循环神经网络。

很多神经网络用等式\ref{form:10.5} 或者类似式子来定义它们隐藏单元的值。为了这个状态是神经网络的隐藏单元,我们对式子\ref{form:10.4}进行了修改,用变量$h$来表示这个状态:
\begin{equation}
\bm{h}^{(t)} = f(\bm{h}^{(t-1)},\bm{x}^{(t)};\theta)
\label{form:10.5}
\end{equation}
这个式子对应的图是\ref{fig:10_2}典型的RNN会增加一个额外的结构,比如说输出层,从隐藏状态$h$中提取信息来作出预测。
\begin{figure}[htbp] %  figure placement: here, top, bottom, or page
   \centering
   \includegraphics[width=6in]{fig/chap10/10_2.PNG} 
   \caption{一个没有输出的循环神经网络结构。网络会从输入$x$中提取信息,然后整合到隐藏状态$h$中,$h$会被传递下去作为下一个时间点输入的一部分。(左边)未展开的带环计算图。黑色的方块表示一个单位时间的延迟。(右边) 展开的计算图,本质和右边带环的图是一样的,每个节点对应一个特定的时间点。}
   \label{fig:10_2}
\end{figure}

等式\ref{form:10.5}可以用两种不同的方式来理解,我们用计算图来来帮助我们理解这两种思路。第一种可以看作是未展开的计算图,只包含有一个节点,我们甚至可以在现实中用物理实现这个模型。参考生物意义上的神经网络。在这个情况下,图中包含有一个环,它会实时的对网络进行操作。就像图\ref{fig:10_2} 左边的那个图一样。在本章中,我们在环中加了一个黑色的小方块来表示某个交互操作延后了一个时间单位,从$t$时刻的状态延后到$t+1$\footnote{译者注;这个地方不太好翻译,大家对照着图\ref{fig:10_2}去理解。}。另一种是用一个展开的计算图来描述RNN。图中有很多的节点,每个节点表示一个变量,这样每个时间点都对应一个变量来表示在那个时间点的状态。就像图\ref{fig:10_2}右边的那样。展开就是把图\ref{fig:10_2}左边带环的图映射到右边有重复结构的计算图的形式。展开的图的大小取决于序列的长度。

我们可以用函数$g^{(t)}$来表示经过$t$步展开之后的式子:
\begin{eqnarray}
\bm{h}^{(t)} & = & g^{(t)}(\bm{x}^{(t)},\bm{x}^{(t-1)},...,\bm{x}^{(2)},\bm{x}^{(1)}) \\
& = & f(\bm{h}^{(t-1)},\bm{x}^{(t)};\theta)
\end{eqnarray}

函数$g^{(t)}$把整个历史序列$(\bm{x}^{(t)},\bm{x}^{(t-1)},...,\bm{x}^{(2)},\bm{x}^{(1)})$都作为输入,输出为当前状态。但是展开的循环结构允许我们将函数用$f$来展开,我们只要递归调用函数$f$就可以得到函数$bm{h}$。展开过程有两个主要的优势:
\begin{enumerate}
\item 无论序列有多长,模型的输入变量的维度是一致的,因为从一个状态转到另一个状态的映射是固定的。而不是说把一个个长度不定的历史序列映射到一个个状态。
\item 我们每一步都可以使用一样的转换函数$f$,函数$f$的参数也可以共享。
\end{enumerate}

正因为这两个优势,我们可以只训练一个模型$f$, 这样我们不需要考虑序列的长度,每一步我们使用相同的$f$,不需要对所有可能的时间点都学习一个模型$g^{(t)}$。使用一个可以共享参数的模型能够允许我们将模型泛化到任意长度的序列中,即使某个序列的长度没有在训练集中出现。相比较不使用参数共享的模型,我们的模型在训练的时候只需要少量的训练数据。

循环图和展开图有不同的用处。循环图是简洁的。展开的计算图可以向我们展示每一步的计算,也可以展示信息流动的路径:如何沿着时间方向流动的(计算输出和损失),如何反着流动(计算梯度)。

\section{循环神经网络}
\label{sec:10.2}
在\ref{sec:10.1}我们介绍了计算图的展开和参数共享的思想。现在我们可以设计其他更复杂的递归神经网络模型。
\begin{figure}[htbp] %  figure placement: here, top, bottom, or page
   \centering
   \includegraphics[width=6in]{fig/chap10/10_3.PNG} 
   \caption{这个计算图是用来展示如何计算循环神经网络训练过程中损失函数的值。这个循环神经网络会将输入序列$\bm{x}$映射到一个输出序列$\bm{o}$。损失函数$L$会在训练过程计算预测序列和对应的真实序列之间的距离。当我们使用softmax作为输出,我们假设$\bm{o}$是未归一化的对数概率。损失函数会首先计算$\hat{y}=softmax(\bm{o})$,然后将它和真实值$\bm{y}$进行比较。RNN使用权重矩阵$U$构建的映射关系,将输入映射到隐藏状态,隐藏状态到隐藏状态的映射是权重矩阵$W$来构建的,隐藏状态到输出状态的映射是有权重矩阵$V$来构建的。等式\ref{form:10.8}定义了这个模型的前向传播表达式。(左边)RNN模型和损失函数计算。(右边)将左边的计算图展开了,每个节点对应着一个特定的时间节点。}
   \label{fig:10_3}
\end{figure}

\begin{figure}[htbp] %  figure placement: here, top, bottom, or page
   \centering
   \includegraphics[width=6in]{fig/chap10/10_4.PNG} 
   \caption{在这个递归神经网络模型中,递归部分是输出指向下个时间点对应的隐藏状态的反馈。在每个时间点$t$, 输入是$\Vx_t$,隐藏层的节点是$\Vh^{(t)}$,输出是$\Vo^{(t)}$,真实值是$\Vy^{(t)}$,损失是$L^{(t)}$。(左边)带环的计算图。(右边)展开的计算图。这个模型和图\ref{fig:10_3}表示的RNN模型相比有一些局限(只能模拟很少的一部分函数),图\ref{fig:10_3}代表的模型可以将任意的历史信息存储到隐藏状态$\Vh$中,让后将$\Vh$传给之后的隐藏状态。这个图中的RNN需要把一个特制的输出值放进$\Vo$中,只有$\Vo$会被传递给之后的节点。无法直接从$\Vh$传递信息给后面的节点。之前的隐藏节点$\Vh$不能直接和当前的隐藏节点相连,只能通过它产生的输出来连接下一个隐藏节点。除非$\Vo$是一个维度非常大的向量,可以存储很多的信息,否则$\Vo$或丢失掉很多重要的历史信息。这就造成了这个模型有一些局限,但是这个模型很容易训练,因为每一步可以独立的进行训练,所以在训练过程中我们可以使用并行计算,这个我们在\ref{sec:10.2.1}中进行了介绍。}
   \label{fig:10_4}
\end{figure}

下面我们介绍了典型循环神经网络一些重要设计模式的例子:
\begin{itemize}
\item 循环神经网络每一时间点产生一个输出,隐藏单元之间互相联系,如图\ref{fig:10_3}。
\item 循环神经网络每一时间点产生一个输出,隐藏单元之和前一个时间点的输出有联系,如图\ref{fig:10_4}。
\item 循环神经网络的隐藏单元互相联系,在处理完整个序列之后常数一个输出,如图\ref{fig:10_5}。

\end{itemize}
本章主要介绍图\ref{fig:10_3}展示的模型。

图\ref{fig:10_3}和\ref{form:10.8}展示的循环神经网络模型是图灵完备的,任何可以被图灵机计算的函数也能够被一个有限大小的循环神经网络来计算。在一些时间步之后,计算RNN的输出的算法复杂度与和图灵机用的时间步的长度是线性渐进关系,和输入的长度也是线性渐进关系(Siegelmann and Sontag, 1991; Siegelmann, 1995; Siegelmann and Sontag, 1995;Hyotyniemi, 1996)。图灵机能计算的函数是离散的,所以这些结果注重精确的实现函数,而不是近似。当我们把RNN作为一个图灵机使用,输入为二进制序列,输出必须先离散化来使输出变成二进制。用一个单独特制有限大小的RNN在这个设置下计算所有可以模拟图灵机的函数是可行的(Siegelmann andSontag (1995) use 886 units)。图灵机的输入是一个特制的需要计算函数,所以一个能够模拟图灵机的RNN可以解决所有问题。理论上的RNN可以模拟无线堆栈,只要将他的激活值和权重由无限精度的有理数来表示。\footnote{这段感觉主要讲RNN是图灵完备的,那几篇论文主要证明了RNN和图灵机的等价。如果没学过计算复杂性理论我觉得没必要看了。后面讲了无限堆栈,不知道和数据结构里的堆栈有啥联系,不是很懂。大家这段可以略过,不影响之后的内容。}

我们现在已经得到了图\ref{fig:10_3}描述的RNN前向传播的公式。但是这个图并没有表明隐藏单元使用的激活函数。这里我们假设他使用的激活函数是是双曲正切(tanh)。这个图也没有展示输出和损失函数的形式。这儿我们假设输出是离散的,因为RNN经常被用来预测单词和字母。一个很自然表示离散变量的方式是把输出$\Vo$看作是一个对于离散变量,每个可能的值赋予一个未被归一化的对数概率。然后我们就可以通过softmax操作作为来处理刚刚的数据,这样就可以得到归一化之后的输出概率$\hat \Vy$。前向传播从初始值$\Vh^{(0)}$开始。然后接下来每一步,从$t= 1$到$t = \tau$我们使用下面的公式递归来完成:
\begin{align}
\label{eq:108a}
 \Va^{(t)} &= \Vb + \MW \Vh^{(t-1)} + \MU \Vx^{(t)}, \\
  \Vh^{(t)} &= \tanh(\Va^{(t)} ), \\
  \Vo^{(t)} &= \Vc + \MV \Vh^{(t)}, \\
  \hat \Vy^{(t)} &= \text{softmax}(\Vo^{(t)}),
\end{align}
这些公式里面的参数有偏置向量$\Vb$和$\Vc$,以及权重矩阵$\MU$、$\MV$和$\MW$,这些权重矩阵分别对应三个函数映射。输入到隐藏状态,隐藏到输出状态,隐藏到隐藏状态。这个例子展示了循环神经网络如何将输入序列映射到相同长度的输出序列。计算损失时我们得根据输入序列$\Vx$,和对应的目标序列$\Vy$,然后我们把每个时间点对应的损失都加起来。比如说,如果损失$L^{(t)}$我们选给定 $\Vx^{(1)}, \dots, \Vx^{(t)}$情况下$\Vy^{(t)}$的下负对数似然(negativelog-likelihood),那么:
\begin{align} \label{eq:1012L}
 & L\big( \{ \Vx^{(1)}, \dots, \Vx^{(\tau)} \}, \{ \Vy^{(1)}, \dots, \Vy^{(\tau)}  \} \big) \\
 & = \sum_t L^{(t)} \\
 & = - \sum_t \log p_{\text{model}} \big(  y^{(t)} |  \{ \Vx^{(1)}, \dots, \Vx^{(t)} \} \big) ,
\end{align}
其中$p_{\text{model}} \big(  y^{(t)} |  \{ \Vx^{(1)}, \dots, \Vx^{(t)} \} \big) $是需要我们计算模型的输出$\hat \Vy^{(t)}$中对应目标值$y^{(t)}$的信息\footnote{译者注: 此处不确定,感觉原文有误,entry应该为entropy,希望在之后的校对中解决这个问题}。计算参数在该损失函数中的梯度计算很复杂。计算梯度的时候我们需要先从左到右使用前向传播算法,正如展开图\ref{fig:10_3}一样,接着从右到左使用后向传播算法。时间复杂度是$\CalO(\tau)$, 而且这个过程不能通过并行计算来提速。因为前向计算时是有顺序的。必须要先计算好一个然后再计算下一个。前向计算的所有结果必须先储存起来,因为在后向计算的时候会被用到,后向计算用完之后就可以销毁了。所以算法的存储复杂度也是$\CalO(\tau)$,当我们把后向传播算法应用搭配RNN的展开图中的时候,复杂度为$\CalO(\tau)$,我们叫这个算法的名字是随着时间后向传播算法(back-propagation through time)简称 BPTT。这个算法的细节会在\ref{sec:10.2.2}讲。这个在隐藏状态之间存在递归的网络是非常强大的,但是训练需要很多时间。有没有可以替代的方案呢?

\subsection{老师驱动(Teacher Forcing)以及有循环输出的神经网络}
\label{sec:10.2.1}
如果神经网络的递归部分存在于每一步的输出和下一步的隐藏单元的连接中,网络的能力会比较小。因为它缺少隐藏状态之间的递归联系。举个例子,它不能模拟一个通用图灵机。由于网络结构缺少隐藏层之间的联系,它需要输出单元能够将所有的历史信息都保存起来,然后网络可以用来预测未来信息。因为输出单元主要是用来匹配训练集的目标的,而不是用来从历史输入中提取有用的信息。除非使用者已经知道如何描述系统的所有状态,然后把他作为训练集目标输出的一部分,这个时候存储有用信息和输出尽量与目标相等这两个任务就重合了。把隐藏单元到隐藏单元的递归关系去掉的好处是: 如果损失函数是基于比较$t$时刻的预测值和训练目标的时候,每个时间点对应的部分都是互不影响的。所以训练是可以并行进行,每一步的梯度是独立求解的。没有必要先去计算前面步骤的输出,因为训练集提供了我们需要输出的理想值\footnote{译者注:此时目标$\Vy$中提供了所有有用的信息,而我们的预测$\hat{\Vy}$肯定不会有目标$\Vy$这么完美}。
\begin{figure}[htbp] %  figure placement: here, top, bottom, or page
   \centering
   \includegraphics[width=6in]{fig/chap10/10_5.PNG} 
   \caption{这个展开的计算图只有在序列的末尾才有一个的输出。这类神经网络模型主要用来对序列进行提取抽象特征,将之表示为一个固定维度的向量(embedding)用于之后的处理。在这个模型中可能每个序列会有一个目标向量来和这个输出进行比较,也有可能使用后向传播算法的时候有梯度信息从之后的模块中流过来。}
   \label{fig:10_5}
\end{figure}

具有从输出到模型递归联系的的模型,可以通过教师驱动(teacher forcing)来训练\footnote{译者注:有的论文中也喜欢叫oracle}。教师驱动是从最大似然准则衍生出来的,在训练过程中,模型会接受到输出的真实值$y^{(t)}$作为$t+1$的输入。我们可以通过下面只有两个时间步序列的例子来了解一下这个方法。条件最大似然准则是:
\begin{align}
 &\log p(\Vy^{(1)},\Vy^{(2)} ~|~ \Vx^{(1)}, \Vx^{(2)} ) \\
 &= \log  p(\Vy^{(2)} ~|~\Vy^{(1)}, \Vx^{(1)}, \Vx^{(2)} )  + \log p(\Vy^{(1)} ~|~ \Vx^{(1)}, \Vx^{(2)}) .
\end{align}

\begin{figure}[htbp] %  figure placement: here, top, bottom, or page
   \centering
   \includegraphics[width=6in]{fig/chap10/10_6.PNG} 
   \caption{ 这个图展示了教师驱动。教师驱动是一种训练技巧,常常用于输出与下一个时间点的隐藏状态有连接的循环神经网络。(左边)在训练过程中我们将$t$时刻对应的真实值$\Vy^{(t)}$ 作为输入传递给下一个隐藏节点$\Vh^{(t+1)}$。(右边)当我们真正使用这个模型或者测试的时候,真实值是不知道的,我们就会用我们模型自己做的预测值$\Vo^{(t)}$来近似代替$\Vy^{(t)}$传递给下一个隐藏状态。这里就形成了一个反馈}
   \label{fig:10_6}
\end{figure}
在这个例子中,我们看到在$t=2$的时刻,模型希望通过训练来实现在给定当前的序列$\Vx$和之前的训练集的输出$\Vy$条件下,最大化$\Vy^{(2)}$的条件概率。最大化似然是在训练过程中实现的,而不是将输出反馈给模型来实现的。我们在训练过程中不会将输出反馈给模型。我们需要传递给模型目标值,来告诉模型正确的输出是什么样的。这在图\ref{fig:10_6}做了说明。

这个模型虽然不是很强大,但是它没有隐藏到隐藏之间的联系,我们算梯度的时候就不需要像传统的RNN一样,根据时间一步一步进行后向运算。教师指导在模型拥存在隐藏节点到隐藏节点关系的时候也会被应用到模型中,只要模型当前时间点的输出与后面某个需要计算值之间存在联系我们就可以使用。但是,只要隐藏单元是历史信息的函数,我们就需要使用BPTT算法。一些模型会同时用教师指导和BPTT来进行训练。

如果之后模型会在\textbf{开环}(open-loop)模式下使用,此时网络输出(如果输出是一个概率分布或者服从某个概率分布,也有可能是输出分布样本)会被作为输入重新反馈到模型中,如果我们使用严格的教师指导来进行训练,这个时候会产生一些问题。在这种情况下,网络在训练过程中看到的输入的种类会和测试的时候看到的输入的分布会非常不一样\footnote{译者注 原文是kind,本人感觉应该是输入分布的不同,校对者来做最终的决定}。一种缓和这个问题的方式是同时使用教师指导下的输入(目标值)和模型自己产生的输出作为输入,比如说通过展开的输入到输出联系预测未来几个时间点正确的目标值。在这个方式下,网络可以学习那些没有在训练过程中见到过输入(比如说那些通过自由运转模式自己生成的输出作为的输入)以及学习如何将这个状态映射到是能够使网络在未来的几步产生合适输出。另一个方法(Bengioet al., 2015b)来解决训练过程中的输入和测试集输入不同的方式是随机选择生成的输出作为输入或者用目标值作为输入。这个方法使用了一个课程学习策略(curriculum learning strategy),它会逐渐的使用更多的生成的值来取代目标值作为输入。

\subsection{计算循环神经网络的梯度}
\label{sec:10.2.2}
计算循环神经网络的梯度是很直接的。只需要将\ref{sec:6.5.6}介绍的后向算法稍微修改一下就可以在展开图中使用,来计算梯度。通过后向算法得到的梯度随后可以被用于任何基于梯度的优化方法来训练RNN。

为了方便大家理解BPTT算法如何运作,我们提供了一个例子来展示如何通过BPTT来给RNN计算梯度(等式\ref{eq:108a}和\ref{eq:1012L})。我们计算图的节点包括参数$\MU,\MV,\MW, \Vb$和$\Vc$,以及以$t$为索引的序列节点$\Vx^{(t)}, \Vh^{(t)},\Vo^{(t)}$和$L^{(t)}$。对于每一个节点$\TSN$,我们需要递归地计算梯度$\nabla_{\TSN} L$,计算过的梯度会传递给计算图中前面的节点,在计算前面节点的梯度的时候会用到后面节点对应的梯度。递归从损失函数对应的那个节点开始:
\begin{align}
 \frac{\partial L}{\partial L^{(t)}} = 1.
\end{align}
在这个导数表达中,为了是最终的预测值$\hat{\Vy}$为归一化的概率向量,我们假定输出$\Vo^{(t)}$ 被传递给softmax函数。我们还假设损失函数为给定输入之后对目标值的负对数似然估计。对$t$时刻输出在损失函数上的梯度$\nabla_{\Vo^{(t)}} L$,它的第$i$项可以这样计算:
\begin{align}
 (\nabla_{\Vo^{(t)}} L)_i =  \frac{\partial L}{\partial o_i^{(t)}} 
 =  \frac{\partial L}{\partial L^{(t)}}  \frac{\partial L^{(t)}}{\partial o_i^{(t)}}  
 = \hat y_i^{(t)} - \mathbf{1}_{i,y^{(t)}}.
\end{align}
之后我们可以从序列的最后一项开始,逐步向前进行后向算法。最后的时间步$\tau$对应的项 $\Vh^{(\tau)}$只有$\Vo^{(\tau)}$作为后续节点,因此这个序列计算起来很简单:
\begin{align}
 \nabla_{\Vh_{(\tau)}} L = \MV^\top \nabla_{\Vo^{(\tau)}} L.
\end{align}
我们接着可以迭代地从时刻$t=\tau-1$到$t=1$进行后向计算,来随着时间后向传播梯度。$\Vh^{(t)}(t < \tau)$同时具有$\Vo^{(t)}$和$\Vh^{(t+1)}$两个后继节点。所以它的梯度用如下的方式进行计算:
\begin{align}
  \nabla_{\Vh_{(\tau)}} L = \Big( \frac{\partial \Vh^{(t+1)}}{ \Vh^{(t)}}  \Big)^\top(\nabla_{\Vh^{(t+1)}} L) 
  + \Big( \frac{\partial \Vo^{(t)}}{ \Vh^{(t)}}  \Big)^\top (\nabla_{\Vo^{(t)}} L) \\
  = \MW^\top (\nabla_{\Vh^{(t+1)}} L) \text{diag} \Big( 1 - (\Vh^{(t+1)})^2 \Big) 
  + \MV^\top \nabla_{\Vo^{(\tau)}} L,
\end{align}
其中$\text{diag} \Big( 1 - (\Vh^{(t+1)})^2 \Big) $ 表示一个包含元素$1 - (h_i^{(t+1)})^2$的对角矩阵。这是时刻$t+1$对应隐藏单元$i$(激活函数为双曲正切)对应的雅可比矩阵(Jacobian)。

一旦获得了计算图中的内部节点的梯度,我们可以获得参数节点的梯度。因为这些参数是在所有的时间节点是被共享的。在对这些参数进行求导计算的时候是要多加小心。我们需要根据\ref{sec:6.5.6}中介绍的bprop实现的方式,来计算计算图中每条边对梯度的贡献。但是,微积分中的$\nabla_{\MW} f$算子,计算$\MW$对于$f$的贡献时将计算图中\emph{所有}边都考虑进去了。为了消除这个歧义,我们引入哑变量$\MW^{(t)}$,他是$\MW$的副本,但是只在$t$时刻使用。我们可能用$\nabla_{\MW^{(t)}}$来表示,$t$时刻对应的权重对梯度的贡献。

通过这些符号,其他参数的梯度可以这样计算:
\begin{align}
 \nabla_{\Vc} L &=  \sum_t \Big( \frac{\partial \Vo^{(t)}}{\partial \Vc} \Big)^\top \nabla_{\Vo^{(t)}} L 
 = \sum_t \nabla_{\Vo^{(t)}} L ,\\
 \nabla_{\Vb} L &= \sum_t \Big( \frac{\partial \Vh^{(t)}}{\partial \Vb^{(t)}} \Big)^\top \nabla_{\Vh^{(t)}} L 
 = \sum_t \text{diag} \Big( 1 - \big( \Vh^{(t)} \big)^2 \Big)  \nabla_{\Vh^{(t)}} L  ,\\
 \nabla_{\MV} L &= \sum_t \sum_i \Big( \frac{\partial L} {\partial o_i^{(t)}}\Big) \nabla_{\MV} o_i^{(t)} 
 = \sum_t (\nabla_{\Vo^{(t)}} L) \Vh^{(t)^\top},\\
 \nabla_{\MW} L &= \sum_t \sum_i \Big( \frac{\partial L} {\partial h_i^{(t)}}\Big) 
 \nabla_{\MW^{(t)}} h_i^{(t)} \\
&= \sum_t \text{diag} \Big( 1 - \big( \Vh^{(t)} \big)^2 \Big) ( \nabla_{\Vh^{(t)}} L) \Vh^{(t-1)^\top} ,\\
 \nabla_{\MU} L &= \sum_t \sum_i \Big( \frac{\partial L} {\partial h_i^{(t)}}\Big) 
 \nabla_{\MU^{(t)}} h_i^{(t)} \\
&= \sum_t \text{diag} \Big( 1 - \big( \Vh^{(t)} \big)^2 \Big) ( \nabla_{\Vh^{(t)}} L) \Vx^{(t)^\top} 
\end{align}

我们不需要在训练的时候计算损失函数对输入$\Vx^{(t)}$的梯度,因为梯度是用于更新参数的,而输入节点的后面没有参数节点了,所以没有必要计算他们对应的梯度。

\subsection{把循环神经网络看作有向图模型}
我们目前介绍的循环神经网络的例子中,损失函数$L^{(t)}$是训练目标$\Vy^{(t)}$和输出$\Vo^{(t)}$之间的交叉熵。循环神经网络也是前馈网络的一种,所以我们其实可以使用任意的的损失函数。我们需要根据任务来寻找合适的损失函数。作为一个前馈网络,我们通常希望RNN的输出可以被看作是概率分布,随后我们可以根据输出分布来与目标之间的交叉熵来定义损失。举个例子,前馈神经网络中,如果输出是高斯单元的话,均方误差就是交叉熵损失。

但我们使用一个预测目标时的对数似然训练过程中的目标函数,比如说等式\ref{eq:1012L},我们在训练RNN的时候,在给定前面的序列元素的条件之下,估计下一个序列元素的条件分布。这个可以看作最大化对数似然
\begin{align}
 \log p(\Vy^{(t)} \mid \Vx^{(1)},\dots, \Vx^{(t)}),
\end{align}
或者,如果模型包含从某一结点输出到下一节点的联系时,
\begin{align}
 \log p(\Vy^{(t)} \mid \Vx^{(1)},\dots, \Vx^{(t)},\Vy^{(1)},\dots, \Vy^{(t-1)} ).
\end{align}
把整个输出序列$\Vy$的联合概率分解为基于一系列单步的概率预测相乘的形式,可以帮助我们获得整个序列的联合概率分布。当我们不再把$\Vy$值作为预测下一个元素的条件时,这个有向图模型不再包含任何从过去$\Vy^{(i)}$到当前$\Vy^{(t)}$的边。在这个情况下,当给定序列$\Vx$的值的作为条件的时候,输出$\Vy$里面的元素是条件独立的。但我们反馈给网络当前真实的$\Vy$值(不是对应的预测值,而是真正观测到或生成的值),那么有向图模型包含了所有的从过去$\Vy^{(i)}$到当前$\Vy^{(t)}$的边。

举一个简单的例子,我们想象一个RNN模型只处理一个标量随机变量序列$ \SetY = \{\RSy^{(1)},\dots,\RSy^{(\tau)}\}$,没有额外的输入$\RSx$。里面的每个值是随机变量,没有其他额外的输出x。$t$时刻的输入时$t-1$时刻的输出。RNN为这$\RSy$个变量定义了一个有向图模型。我们使用链式法来利用条件概率参数化这些观测值的联合分布:
\begin{align}
 P(\SetY) = P(\RVy^{(1)},\dots,\RVy^{(\tau)}) = \prod_{t=1}^{\tau}P(\RVy^{(t)} \mid \RVy^{(t-1)},\RVy^{(t-2)},
 \dots,\RVy^{(1)}),
\end{align}
当$t=1$时,竖杠右侧显然为空。因此,根据这样一个模型,这一系列值$\{y^{(1)},\dots,y^{(\tau)} \}$的负对数似然为
\begin{align}
 L = \sum_{t} L^{(t)},
\end{align}
其中, 
\begin{align}
 L^{(t)} = -\log P(\RSy^{(t)} = y^{(t)} \mid y^{(t-1)},y^{(t-2)}, \dots, y^{(1)}).
\end{align}

\begin{figure}[htbp] %  figure placement: here, top, bottom, or page
   \centering
   \includegraphics[width=6in]{fig/chap10/10_7.PNG} 
   \caption{ 序列$\RSy^{(1)},\RSy^{(2)},\dots,\RSy^{(t)},\dots$对应的全连接的图模型。之前的观测值$\RSy^{(i)}$,会影响未来的观察值$\RSy^{(t)}(t>i)$的条件分布。直接参数化这个图模型会有点麻烦,因为随着输入的增加参数也会大量增加。RNN也是一个全连接结构,但是它使用了较少的参数,可以看一下图\ref{fig:10_8}来进行一下对比。}
   \label{fig:10_7}
\end{figure}

\begin{figure}[htbp] %  figure placement: here, top, bottom, or page
   \centering
   \includegraphics[width=6in]{fig/chap10/10_8.PNG} 
   \caption{ 展示了RNN代表的图模型的状态变量,尽管输入到状态变量的映射是一个固定的函数,但是它可以帮助我们使用较少的参数来参数化\ref{form:10.5}我们的模型。序列中的每一个步骤都使用了相同的结构(每个节点具有相同的输入维度),并且这些结构每一步共享它们的参数。}
   \label{fig:10_8}
\end{figure}

图模型中的边表示的是变量之间的依赖关系。很多图模型期望可以通过省略一些没有很强交互性的边来换取统计和计算的效率。举个例子,通常我们会假设图模型具有马尔科夫性。这个时候图中只包含从$\{ \RSy^{(t-k)}, \dots, \RSy^{(t-1)}\}$到$\RSy^{(t)}$的边,而不是包含所有从历史节点到当前结点的边。但是在有的情况下,我们相信所有的历史的输入对序列的下一个元素都有影响。当我们认为$\RSy^{(t)}$的分布可能取决于过去的某个节点$y^{(i)}$,且无法通过$y^{(t-1)}$获得$y^{(i)}$的信息时,RNN就可以大显身手。

当我们用图模型的解释RNN的时候,我们可以把RNN看作是一个全连接的图模型。这是表示任何一对的$y$之间都有依赖关系。图\ref{fig:10_8}是$y$值之间具有全连接结构的图模型。
我们在学习这个RNN代表的完全图的时候通过边缘化\footnote{译者注:概率统计有介绍,把某个随机变量的所有值累加起来}的方式消去了模型中的隐藏单元$\Vh^{(t)}$。

当我们把RNN看作是一个图模型的时候,我们可以把隐藏单元$\Vh^{(t)}$看作是随机变量\footnote{给变量的父节点之后,那些变量的条件分布就确定了。
尽管设计具有这样隐藏单元具有确定分布的图模型是很少见的,但这是完全合理的。}。
在RNN对应的图模型中包括隐藏单元,说明我们可以非常有效的参数化观测的联合分布。
假设我们用表格(tabular representation)来表示某些离散变量上任意的联合分布,在表格中,任意的离散变量的组合都对应着一个小格子,里面的值代表这个组合发生的概率。
如果$y$可以取$k$个不同的值,表格将会有$\CalO(k^\tau)$个参数。
但是RNN由于参数共享,它的参数数目为$\CalO(1)$,是序列长度的函数\footnote{译者注:此处存疑,感觉是作者笔误,复杂度应该为常数}
我们可以调节参数数量来控制RNN的模型容量,但参数数目不会随着序列长度增加而增加。
等式\ref{eq:105h}展示了RNN通过递归掉用相同的函数$f$以及在每个时间点的使用相同参数$\Vtheta$,有效地参数化的变量之间的联系,即使这些变量之间的距离可能很长。
可以通过图\ref{fig:10_8}加深我们对RNN所代表的图模型的理解。
在图模型中,如果我们给定$\Vh^{(t)}$节点,这个节点是过去和未来的中间节点,这个时候,过去和未来就相互独立了。
过去的某个变量$y^{(i)}$可以通过$\Vh$来影响未来的某个变量$y^{(t)}$。
在这个图模型中,我们对每个时间步都使用相同的条件概率分布,这样就可以高效地参数化模型,并且当观察到全部变量时,可以高效地评估联合分布分配给所有变量的概率\footnote{译者注:此处不是很清晰}。

虽然我们可以高效地参数化图模型,某些操作在计算上依然具有挑战性。
比如说,难以预测在序列中间缺少的值。

虽然循环神经网络的参数数目并不多,但是参数\emph{优化}可能会变困难。

我们之所以在循环神经网络中使用的参数共享,是因为我们假设相同参数可用于序列不同的位置。
同样的,这个也是假设在给定时刻$t$的变量的条件下,在时刻$t +1$变量的条件概率分布是固定的stationary,这意味着序列不同位置的变量之间的关系并不依赖于位置$t$。
原则上,可以使用$t$作为每个时间步的额外输入,并让学习器在发现任何时间依赖性的同时,在不同的时间步之间尽可能多地共享这个信息。
相比在每个位置$t$使用不同的条件概率分布,这个策略已经好很多了,但网络将必须在面对新$t$时进行推断更新。

为了更全面的吧RNN用图模型来解释,我们必须描述如何从这个模型中进行采样。
我们要做是每个不同的时间步的条件分布中进行采样。
然而,这儿有一个小麻烦。RNN必须有某种机制来确定序列的长度。当然,这可以通过很多方式来实现。

在当输出是属于某个词汇表中的符号时,我们可以在词汇表中添加一个特殊符号来表示序列终止(Schmidhuber, 2012)。
当这个符号出现时,采样过程停止。
在训练集中,我们将该符号作为一个额外成员插入序列的末尾,即紧跟每个训练样本$\Vx^{(\tau)}$之后。

另一种方法是在模型中引入一个额外的Bernoulli输出,表示在每个时间步之后决定继续生成还是终止。
这种方法比向词汇表增加一个额外符号的方法使用的更广泛,因为它可以适用于任何RNN模型,而不仅仅用于输出符号序列的RNN模型。
例如,它可以应用于一个产生实数序列的RNN模型。
新的输出单元通常使用sigmoid单元,损失函数则为交叉熵。
在这种方法中,sigmoid被训练为最大化正确预测的对数概率,即在每个时间步决定序列是否结束\footnote{译者注:此处不是很懂}。

确定序列长度$\tau$的另一种方法是将一个额外的输出添加到模型用来预测$\tau$本身。
模型可以采样出一个$\tau$的值,然后采样$\tau$步有价值的数据。
这种方法需要在每个时间步的循环更新的时候增加一个额外输入,使得循环在更新的时候知道它是否是靠近所产生序列的末尾\footnote{译者注:更新应该指的是产生条件概率}。
这种额外的输入可以是$\tau$的值,也可以是$\tau - t$,即剩下时间步的数量。
如果没有这个额外的输入,RNN可能会突然结束序列的产生,例如在产生句子时,可能句子还没有完整,生成就结束了。
这个方法可以用下面这个式子来解释:
\begin{align}
 P(\Vx^{(1)},\dots, \Vx^{(\tau)}) = P(\tau) P(\Vx^{(1)},\dots,\Vx^{(\tau)} \mid \tau) .
\end{align}
直接预测$\tau$的例子请参考 Goodfellow et al.(2014d)。

\subsection{通过RNN基于上下文对序列进行建模}
在上一小节,我们描述了如何从概率图模型的角度来看待RNN,还举了一个例子,在这个例子里没有输入$\Vx$,直接对随机变量序列$y^{(t)}$进行建模。当然,传统的RNN如等式\ref{eq:108a}所示,包含一个输入序列$\Vx^{(1)},\Vx^{(2)},\dots,\Vx^{(\tau)}$。一般情况下,我们在用图模型解释RNN的时候,不仅可以表示$y$变量的联合分布也能表示给定$\Vx$条件下,$y$条件分布。正如我们在\ref{sec:6.2.1.1}中介绍的前向网络一样,任何可以表示为$P(\Vy;\Vtheta)$的模型都能被重新写成条件分布$P(\Vy \mid \Vomega)$的形式,其中$\Vomega=\Vtheta$。
同样的,我们可以通过使用$P(\Vy \mid \Vomega)$代表分布$P(\Vy \mid \Vx)$来扩展这样的模型,但要令$\Vomega$是关于$\Vx$的函数。在RNN中,这个条件可以用多种不同的方式来实现。此处我们回顾一下最常用最有效的一些方法。

之前,我们已经讨论了将的向量$\Vx^{(t)}$序列作为输入传递给RNN,其中$t =1, \dots, \tau$。RNN也可以接受单个向量$\Vx$作为输入。
当$\Vx$是一个固定大小的向量,我们可以简单地将其看作产生$\Vy$序列RNN的一个额外输入。
将额外输入传递给RNN的一些常见方法是:
\begin{enumerate}
 \item 在每个时刻作为一个额外输入,或
 \item 把$\Vx$作为初始状态$\Vh^{(0)}$,或
 \item 两种方式同时使用。
\end{enumerate}

第一个也是最常用的方法在图\ref{fig:10_9}中进行了说明。
输入$\Vx$和每个隐藏单元向量$\Vh^{(t)}$之间的联系通过新引入的权重矩阵$\MR$来进行参数化的,这个权重矩阵在只包含$y$序列的模型中是没有的。
同样的,$\Vx^\top\MR$在每个时间步会作为隐藏单元的一个额外输入。
我们可以认为在确定$\Vx^\top\MR$值的时候,$\Vx$的选择可以看作是每个隐藏单元的一个偏置参数。
权重与输入保持独立。
我们可以把这种模型看作是由参数$\Vtheta$决定的非条件模型,我们可以把$\Vtheta$改成$\Vomega$,这时$\Vomega$里面的偏置参数是输入的函数。
\begin{figure}[htbp] %  figure placement: here, top, bottom, or page
   \centering
   \includegraphics[width=6in]{fig/chap10/10_9.PNG} 
   \caption{ 在这个图中,RNN将一个固定长度的向量$\Vx$映射为一个序列$Y$的分布。在做图像说明(image captioning)任务的时候我们常常使用这类模型。此时,图像就作为模型的输入,模型需要输出一个单词序列来描述这个图片。每个已观察到的输出序列里的元素$\Vy^{(t)}$会作为输入(给当前的时间步),同时它也是目标值(针对的是之前时间步)。}
   \label{fig:10_9}
\end{figure}

除了接收单个的一个向量作为输入以外,RNN也可以接收向量序列$\Vx^{(t)}$作为输入。等式\ref{eq:108a}描述的RNN对应的条件分布是$P(\Vy^{(1)}, \dots, \Vy^{(\tau)} ~|~ \Vx^{(1)}, \dots, \Vx^{(\tau)})$,根据条件独立性假设\footnote{译者注:在给定历史的输入$\Vy^{(t)} \mid \Vx^{(1)}, \dots, \Vx^{(t)}$的条件之后,$\Vy^{(t)}$直接相互条件独立}这个分布可以分解为
\begin{align}
 \prod_t P(\Vy^{(t)} \mid \Vx^{(1)}, \dots, \Vx^{(t)}).
\end{align}
为消除条件独立的假设,我们可以给在时刻$t$的输出到时间$t+1$的隐藏单元之间添加连接,如图\ref{fig:10_10}所示。
该模型就可以表示$\Vy$序列的任意概率分布。
这种给定一个序列的条件下表示另一个序列分布的模型的还是有一个限制,就是这给定序列和要表示的序列长度必须相等。在\ref{sec:10.4}我们介绍了如何取消这个限制条件。
\begin{figure}[htbp] %  figure placement: here, top, bottom, or page
   \centering
   \includegraphics[width=6in]{fig/chap10/10_10.PNG} 
   \caption{ 这是一个条件循环神经网络(conditional recurrent neural network),长度不定的输入序列$\Vx$映射成为一个和输入具有相同长度的序列$\Vy$的分布。和图\ref{fig:10_3}相比,这个模型中多了输出到下一状态节点之间的联系。这些联系可以帮助RNN给定序列$\Vx$的条件下,对$\Vy$的任意分布进行建模。而图\ref{fig:10_3}对应的模型在对$\Vy$进行建模时,必须保证给定$\Vx$的条件下,$\Vy$条件独立。}
   \label{fig:10_10}
\end{figure}
\section{双向RNN(Bidirectional RNNs)}
\label{sec:10.3}
目前为止我们考虑的所有RNN模型的例子都是“因果”结构的,意味着在时刻$t$的状态只能从历史序列$\Vx^{(1)},\dots,\Vx^{(t-1)}$以及当前的输入$\Vx^{(t)}$中获得信息。
在一些例子中,我们还讨论了在训练过程中,使用过去的目标值$\Vy$来影响当前状态。

但是,在许多应用中,我们要输出的预测$\Vy^{(t)}$可能依赖于整个输入序列。举个例子,在语音识别中,由于协同发音(co-articulation),语音会被分解成音素(phoneme),当前的语音所对应的音素的可能取决于未来几个音素,甚至由于相邻的单词之间存在语义依赖,该音素可能取决于未来的几个单词:如果当前的词有两种声学上合理的解释,我们可能要在更远的未来(和过去)来寻找信息从两个中选择一个正确的。
这在手写识别和许多其他序列到序列的学习任务中也是如此,将会在下一节对这类问题进行描述。

发明双向递归神经网络(bidirectional RNNs)的发明就是为了解决上一段提到的那些问题(Schuster and Paliwal, 1997)。当需要处理的任务有这种需求时,双向循环神经网络获得了非常大的成功 (Graves, 2012),比如说手写识别 (Graves et al., 2008; Graves and Schmidhuber, 2009), 语音识别 (Graves and Schmidhuber, 2005; Graves et al., 2013) 以及生物信息学 (Baldiet al., 1999)。

顾名思义,双向循环神经网络由两个循环神经网络组成,一个是前向(forward)循环神经网络:从序列的开始向后处理数据。另一个事后向(backward)循环神经网络:从序列的的末尾向前处理数据。
图\ref{fig:10_11}展示了典型的双向循环神经网络的结构,其中$\Vh^{(t)}$是前向循环神经网络的状态节点,$\Vg^{(t)}$是后向循环神经网络的状态节点。
这时输出单元$\Vo^{(t)}$就可以同时考虑过去和未来的信息。当然$\Vo^{(t)}$的值对时刻$t$的输入值是很敏感的。此时我们不必像前馈神经网络,卷积神经网络或者是一个单向的具有固定大小的先行缓存器( look-ahead buffer)的RNN一样,在$t$周围指定一个固定大小的窗口。模型可以获得这个窗口里面输入的信息。

\begin{figure}[htbp] %  figure placement: here, top, bottom, or page
   \centering
   \includegraphics[width=6in]{fig/chap10/10_11.PNG} 
   \caption{一个典型的双向循环神经网络计算图,学习任务是将输入序列$\Vx$映射到目标序列$\Vy$,每一步的损失为$L^{(t)}$,$\Vh$的信息循环向前传播(从左往右),$\Vg$的信息循环向后传播(从右往左)。所以$\Vh^{(t)}$中包含了过去的输入信息而$\Vg^{(t)}$中包含了未来的输入信息,输出单元$\Vo^{(t)}$会同时参考$\Vh^{(t)}$和$\Vg^{(t)}$,那么也就可以同时考虑过去和未来的输入信息}
   \label{fig:10_11}
\end{figure}

这个想法可以扩展到2维的输入数据,比如说图像,这时我们就需要\emph{四个}RNN,沿着四个方向(上、下、左、右)来对数据进行处理。
如果循环神经网络能够学习到更多的上下文信息,那在计算2维网格上的每个点$(i, j)$的输出$O_{i,j}$就能可以不仅仅只考虑局部信息,还能够考虑距离位置$(i, j)$更远的信息。
相比于卷积神经网络,在图像上使用RNN通常更复杂,但RNN就允许特征图的特征之间存在长期横向的相互作用\footnote{译者注:卷积神经网络只可以考虑到局部的联系,即只能考虑到那个固定窗口里面的信息,而循环神经网络则可以获得更广的上下文信息,不过四向RNN应用的很少}(Visin et al., 2015; Kalchbrenner et al., 2015)。
实际上,对于这样的循环神经网络模型,前向传播公式可以写成卷积的形式,在包含横向联系的特征图上进行递归计算之前,在每一层自底向上对输入进行处理\footnote{译者注:原文表达不清晰,不知道怎么翻译}。

\section{基于编码-解码的序列到序列架构}
\label{sec:10.4}
我们已经在图\ref{fig:10_5}看到RNN如何将输入序列映射成固定大小的向量,在图\ref{fig:10_9}中展示了RNN如何将固定大小的向量映射成为一个序列,在图\ref{fig:10_3}、\ref{fig:10_4}、\ref{fig:10_10}和\ref{fig:10_11}中看到RNN如何将一个输入序列映射到与输入等长的输出序列。

本节我们讨论如何训练RNN,使其将输入序列映射到任意长度的输出序列。
这在许多机器学习任务中都有使用,如语音识别、机器翻译或问答系统,其中训练集的输入和输出序列的长度通常不相同(虽然它们的长度可能有联系)。

\begin{figure}[htbp] %  figure placement: here, top, bottom, or page
   \centering
   \includegraphics[width=4in]{fig/chap10/10_12.PNG} 
   \caption{基于编码-解码的序列到序列架构的一个实例。根据输入序列$(\Vx^{(1)},\Vx^{(2)} \dots, \Vx^{(n_x)})$产生输出序列$(\Vy^{(1)},\Vy^{(2)} \dots, \Vy^{(n_y)})$。这个模型有两部分RNN组成,第一个编码RNN用来读入序列,第二个解码RNN用来产生输出序列(或者说计算输出序列的分布)。编码RNN的最后一个隐藏状态可以用固定大小的变量$C$(contex)来表示,$C$里面存储这对输入序列语义的总结信息。$C$也会作为输入传递给解码RNN用来生成输出序列。}
   \label{fig:10_12}
\end{figure}

我们通常较RNN的输入为\textbf{上下文}(contex),我们通常用$C$来表示上下文信息。
这个上下文$C$可能是对输入序列$\MX=(\Vx^{(1)},\dots,\Vx^{(n_x)})$信息的概括。他可以是向量或者向量序列。

用于将长度不定的序列映射到另一一个长度不定序列的RNN架构最初由 Cho et al. (2014a)提出,不久之后 Sutskever et al. (2014)对这个架构进行了发展,并把这种方法应用到机器翻译翻译中,取得了较好结果。这个系统是对另一个机器翻译系统产生的翻译结果进行评分,而后者使用一个传统的递归神经网络来生成翻译结果。
这些作者对这个架构\ref{fig:10_12}不同的称呼,通常叫编码-解码或序列到序列架构。
这个架构的本质非常简单:(1)用\textbf{编码}或\textbf{读取器}(reader)或\textbf{输入}(input)RNN接收输入序列。
这个编码RNN最后会输出上下文$C$(通常是对最后一个隐藏状态进行处理输出)。
(2)\textbf{解码}或\textbf{写入器}(writer)或\textbf{输出}(output)RNN则接收固定长度的向量$C$(如图\ref{fig:10_9})然后生成输出序列$\MY=(\Vy^{(1)}, \dots, \Vy^{(n_y)})$。
这种架构的创新之处在于输入输出序列的长度$n_x$和$n_y$可以彼此不同,而前面介绍的RNN的架构基本都要求输入序列长度等于输出序列长度$n_x = n_y = \tau$。
在这个序列到序列的架构中,我们需要同时训练这两个RNN来最大化训练集中所有$\Vx$和$\Vy$对数条件概率$\log P( \Vy^{(1)}, \dots, \Vy^{(n_y)} ~|~ \Vx^{(1)},\dots,\Vx^{(n_x)} )$。
编码RNN的最后一个隐藏状态$\Vh_{n_x}$通常被当作输入的上下文$C$,或者是输入序列信息的总结,并作为解码RNN的输入。

如果上下文$C$是一个向量,则解码RNN的结构和我们在\ref{sec:10.2.4}节介绍的一样,是一个将向量映射到序列RNN结构。
正如之前介绍的那样,将向量映射到序列的RNN至少有两种方式来接受输入向量。
输入可以作为RNN的初始状态,或也可以作为输入传给每一个时间步中隐藏单元。这两种方式也可以结合起来使用。

这个结构并不限制编码RNN和解码RNN具有相同大小的隐藏层。

但是这个架构有一个明显的不足:如果编码RNN输出的上下文$C$的维度太小,它不能将输入序列的有用的信息全部存进这个向量$C$。
这是由 Bahdanau et al. (2015)在用这个架构做机器翻译的时候观察到的。
他们提出让$C$成为长度可变的序列,而不是一个固定大小的向量。
此外,他们还介绍了注意力机制(attention mechanism),将序列$C$的元素和输出序列的元素相关联起来。更多的细节会在\ref{sec:12.4.5.1}进行介绍。

\section{深度RNN结构}
\label{sec:10.5}
大多数RNN的计算可以分为三个模块,当然也对应着三组参数和三个变换映射:
\begin{enumerate}
 \item 从输入到隐藏状态,
 \item 从前一隐藏状态到下一隐藏状态,
 \item 从隐藏状态到输出。
\end{enumerate}
\ref{fig:10_3}是传统的RNN的结构,这三个模块每个都有一个与之相关的权重矩阵。或者说,当我们把RNN展开时,每个模块对应一个简单的变换,而这个变化是可以用单层或者多层的MLP来表示的。通常这个转换分为两步,显示一个仿射变换,紧接着把仿射变换的结果进行非线性变换\footnote{sigmoid,tanh}。

那么如果我们使用更复杂的变换,或者原来只有一层的MLP,我们把它变成多层,这样会不会让我们的模型有一个更好的表现呢?
 (Graves et al., 2013; Pascanu et al., 2014a)里的实验证明了这一点。我们需要足够深的网络结构来获得理想的变换映射。 一些早期深度RNN的介绍可以参考 Schmidhuber (1992),El Hihi and Bengio (1996), 或者 Jaeger (2007a)。
 
 Graves et al.(2013) 首先发现了像\ref{fig:10_13}(左)一样,将RNN的状态分为多层可以给模型的表现带来显著的提升,如。
我们可以认为,在\ref{fig:10_13}(a)所示在多层结构中最开始的那些层主要将原始输入进行抽象转换,输出一个对隐藏状态更合适,更高层次的表示。
Pascanu et al. (2014a)进一步提出在上述三个模块中分布使用一个单独的MLP结构(可以是多层的),如\ref{fig:10_13}(b)所示。
考虑到模型的表示能力,我们建议在这三个模块合理分配足够的容量,但当我们增加模型的深度可能会造成优化困难而损害学习效果。
在一般情况下,浅层的架构更容易优化,如果我们像\ref{fig:10_13}(b)加入额外的层会导致从时间点$t$的变量到时间点$t+1$的最短路径变长。
例如,如果单隐藏层的MLP被用于状态到状态的转换,那么我们在\ref{fig:10_13}使用的结构任何两个不同时间点变量之间最短路径的长度会加倍。
然而Pascanu et al(2014a)认为,在隐藏状态到隐藏状态的路径中引入跨越连接(skip connection)如\ref{fig:10_13}(c)所示,可以缓和这个问题。

 \begin{figure}[htbp] %  figure placement: here, top, bottom, or page
   \centering
   \includegraphics[width=4in]{fig/chap10/10_13.PNG} 
   \caption{一个循环神经网络可以通过不同的方式来加深网络的层数(Pascanu et al, 2014a)。(a)RNN中的隐藏状态可以被分成多组层次结构。(b)在输入到隐藏状态,隐藏到隐藏状态和隐藏到输入状态中加深结构(比如说吧MLP由一层转为多层)。这会延迟不同时间步节点之间的最短路径的长度。(c)路径延长的副作用可以通过跨越连接来缓和。}
   \label{fig:10_13}
\end{figure}


\section{递归神经网络(Recursive Neural Networks)}
\label{sec:10.6}
递归神经网络\footnote{我们建议不要将递归神经网络缩写为RNN,以免与循环神经网络发生混淆。}是循环神经网络的另一个扩展,它计算图为树状结构而不是传统循环神经网络的链状结构。
递归网络的典型的计算图结构如\ref{fig:10_14}所示。
递归神经网络由Pollack (1990)引入,而Bottou (2011)描述了递归神经网络的潜在用途——学习推论。
自然语言处理(Socher et al., 2011a,c, 2013a)领域和计算机视觉(Socher et al., 2011b)领域,递归网络已成功地应用于输入为\emph{数据结构}的神经网络(Frasconi et al., 1997,1998)。

递归网络的一个明显优势是,对于长度为$\tau$的序列,深度(通过非线性操作的组合数量来衡量)可以急剧地从$\tau$减小为$\CalO(\log \tau)$,这可能有助于解决下一小节提到的长时依赖(long-term dependencies)问题。
一个还没有解决的问题是如何以最佳的方式构造树。一种选择是使用不依赖于数据的树结构,如平衡二叉树。
在某些应用领域,外部方法可以帮助我们选择适当的树结构。
例如,处理自然语言中的句子时,可以用自然语言语法分析器对句子构造一个语法分析树,作为递归网络的树结构 (Socher et al., 2011a, 2013a)。
理想的情况下,人们希望学习器根据任意输入自行发现和推断合适的树结构,如 Bottou (2011)所建议的那样。
 
\begin{figure}[htbp] %  figure placement: here, top, bottom, or page
   \centering
   \includegraphics[width=4in]{fig/chap10/10_14.PNG} 
   \caption{递归神经网络的计算图有传统循环神经网络的链式结构变成了树状结构。长度不定的序列$\Vx^{(1)},\Vx^{(2)},\dots,\Vx^{(t)}$可以通过固定的参数集合(权重矩阵$\MU$、$\MV$、$\MW$)被映射为一个固定大小的表示(输出向量$\Vo$)。上图展示了一个监督学习实例,每个序列对应着一个目标$\Vy$。}
   \label{fig:10_14}
\end{figure}
 
有很多对递归神经网络的变种。举个例子,Frasconiet al. (1997) 和 Frasconi et al. (1998)把数据和一个树结构联系起来,输入和目标值都是树结构中的单独的节点。每个树节点进行的计算和传统的人工神经网络的计算方式不一样(通常是将输入向量进行仿射变换,然后将变换后的结果通过一个单调的非线性单元。)例如, Socher et al. (2013a)提出了使用张量运算和双线性形式来对概念(concept)之间的联系进行建模。 (Weston et al., 2010; Bordes et al., 2012)已经证明了当概念是有一个里面元素为连续值得向量来表示(embeddings)的时候,张量运算和双线性模式对概念间的联系建模很有用。


\section{长时依赖(Long-Term Dependencies)的挑战}
\label{sec:10.7}
递归神经网络在虚席过程中会遇到长时依赖的问题,这个问题的细节我们在\ref{sec:8.2.5}进行了介绍。产生这个问题的根本原因是梯度经过多步传播之后可能会产生弥散(大部分情况)或者爆炸(很少,但是会对优化过程产生不好的影响),尽管我们假定我们设置参数来使RNN保持(拥有记忆功能,可以存储历史信息,梯度不会爆炸),但是当参数的值很小,而序列的长度很长而且相关联的信息之间的距离也很长,这个时候就很容易产生长时依赖问题\footnote{译者注:这儿没有例子,读者理解起来可能会比较困难,建议加个例子。}。 (Hochreiter, 1991; Doya, 1993; Bengio et al., 1994:Pascam et al,2013)对这个问题进行了更深层次的讨论。在这一小结,我们对这个问题进行了仔细的描述。在本章剩余的章节中我们介绍一些方法来解决这个问题。

循环神经网络会在每一个时间步使用相同的函数。这些函数多次重复的组合之后就会造成极端的非线性现象\footnote{ 译者注:这部分用中文表达略有困难,主要的意思就是对数据连续进行多次非线性函数映射,可能函数的形式是一样的,参数不同,会产生比较奇怪的非线性现象。}。如图\ref{fig:10_15}所示。

\begin{figure}[htbp] %  figure placement: here, top, bottom, or page
   \centering
   \includegraphics[width=4in]{fig/chap10/10_15.PNG} 
   \caption{当我们把多个非线性函数(想图中展示的tanh层)组合到一起,会产生较强的非线性。尤其是当其中的大部分值得倒数比较小,有一部分值得倒数比较大,而且增加减小交替变换非线性会变得更严重。在这幅图中,我们将一个100维的隐藏状态投影(主要为了降维)到1维空间并在$y$坐标中展示,$x$坐标轴表示的是初始100维的隐藏状态随机的一个维度方向。我们可以把这曲线看作是高维函数的一个线性的横截面。不同的曲线代表不同的时间步,每个事件不得转换函数都会把之前的那些转换函数累积起来。}
   \label{fig:10_15}
\end{figure}

我们循环神经网络中进行的函数组合有点像连续的矩阵乘法,我们考虑一个简化版的循环神经网络,递推关系
\begin{align}
 \Vh^{(t)} = \MW^\top \Vh^{(t-1)}
\end{align}
里面没有非线性激活函数和输入$\Vx$。如在\ref{sec:8.2.5}中描述的那样,递推关系的本质就是矩阵的幂乘。我们可以把它简化为:
\begin{align}
 \Vh^{(t)} = (\MW^t)^\top \Vh^{(0)},
\end{align}
当$\MW$可以特征分解成下面的形式
\begin{align}
 \MW = \MQ \VLambda \MQ^\top,
\end{align}
其中$\MQ$为正交矩阵,递推关系式也可以进一步简化为
\begin{align}
 \Vh^{(t)} = \MQ^\top \VLambda^t \MQ \Vh^{(0)}.
\end{align}
特征值进行$t$次幂乘之后,幅值不到一的特征值会衰减到零,而幅值大于一的就会激增。
分解之后,$\Vh^{(0)}$中任何不被投影到最大的特征值对应的那个维度上的部分将最终将被丢弃\footnote{译者注:结合奇异值分解,主成分分析法来理解这句话}。

这个问题主要在循环神经网络中比较严重。在标量的情况下,考虑我们对权重$w$做多次幂乘,$w^t$要么消失($w$的绝对值小于1)要么激增($w$的绝对值大于1)。但是如果我们考虑一个非循环神经网络,在每个时间步使用不同权重$w^{(t)}$,那么情况就不同了。
如果初始状态为$1$,那么时刻$t$的状态可以由$\prod_t w^{(t)}$给出。
假设$w^{(t)}$的值是随机生成的,各自独立,均值为$0$,方差为$v$。
$\prod_t w^{(t)}$的方差就为$\CalO(v^n)$。
为了获得某些我们所期望的方差$v^*$,我们可以分别选择不同的权重,并让那些权重的方差为$v=\sqrt[n]{v^*}$。

非常深的前馈神经网络可以通过精心设计权重来避免梯度消失和梯度爆炸。在Sussillo (2014)中做了介绍。

梯度消失和梯度爆炸问题是游不同的研究者 (Hochreiter, 1991; Bengio et al., 1993, 1994)分别发现的。有的人可能希望如果我们只在某个固定的区域空间选择权重来避免梯度消失和梯度爆炸。不幸的是为了储存历史信息,并能够承受小扰动,RNN在训练过程中必须让参数进入会造成梯度消失的一些区域 (Bengio et al., 1993,1994)。尤其是我们希望模型能够表示长时依赖的时候,长时交互的梯度和短视交互的梯度相比会呈指数级减小。当然这不是说我们不能进行训练学习了,只是我们需要跟多的时间来学习长时依赖,因为长时依赖关系的信息会被短期依赖信息的一些小的波动来取代掉。实际上, Bengio et al. (1994)的实验表明,当我们增加了需要获得的依赖关系信息的跨度,基于梯度的优化会变得越来越困难,当我们在在长度仅为10或20的序列上用SGD对RNN进行训练时,成功的概率迅速变为0。

如果想从动态系统的观点来看待RNN,可以阅读一下 Doya(1993), Bengio et al. (1994) , Siegelmann and Sontag (1995), 和一篇综述: Pascanu et al. (2013).
本章的其余部分将讨论目前已经提出的缓和长时依赖问题的方法(在某些情况下,允许一个RNN模型学习横跨数百步的依赖关系),但长时依赖问题仍是深度学习中的一个主要挑战。

\section{回声状态网络(Echo State Networks)}
\label{sec:10.8}
在循环神经网络中,将$\Vh^{(t-1)}$映射到$\Vh^{(t)}$的权重矩阵和映将$\Vx^{(t)}$映射到$\Vh^{(t)}$的权重矩阵是最难进行优化学习的。

Jaeger, 2003; Maass et al., 2002; Jaeger and Haas, 2004;Jaeger, 2007b提出避免这种困难的方法是:设置合适的递归权重使得隐藏单元可以很好地记录历史输入信息,这样我们\emph{只需要学习输出权重}。这个想法分别被用于回声状态网络(echo state networks 或者 ESN) (Jaeger and Haas, 2004; Jaeger, 2007b) 和液态机 (liquid state machines)(Maass et al., 2002). 这两个结构比较像,只不过液态机使用的脉冲神经元(输出为二进制)而回声状态网络的隐藏单元的值是连续的。回声状态网络和液态机都被称作容器计算结构(reservoir computing)(Lukoševičius and Jaeger, 2009)。因为他们连个结构的隐藏单元变成了存储时许特征的容器,时序特征是不同方面的历史输入信息。

这类容器计算结构的神经网络可以看作是核机器(kernel machines)\footnote{回想一下支持向量机}:他们将一个任意长度的序列(到时刻$t$的历史输入)映射成为一个固定长度的向量(循环状态$\Vh^{(t)}$),之后将向量通过一个线性预测器(通常是线性回归)来解决我们的问题。
训练的损失函数可以很容易地设计为输出权重的凸函数。
例如,如果输出是隐藏单元到输出目标的一个线性回归问题,训练的损失函数就可以是均方误差。这是一个凸函数,一样我们可以很容易的找到最优解 (Jaeger, 2003)。

现在我们所面临的问题是:如何设置我们的输入和递归权重使我们循环神经网络的隐藏状态可以存储序列的历史信息?这个答案在介绍容器计算的文章中已经给出了。我们可以把循环网络看作是一个动态系统。设置输入和循环的权重是这个动态系统接近稳定。

最初的想法是使神经网络中状态到状态转换函数的雅可比矩阵的特征值接近1。如在\ref{sec:8.2.5}中介绍的那样,循环神经网络的一个重要特征就是雅可比矩阵的特征值谱$\MJ^{(t)} = \frac{\partial s^{(t)}}{\partial s^{(t-1)}}$。其中$\MJ^{(t)}$的谱半径是特征值绝对值的最大值。

为了理解一下谱半径的影响,我们举一个简单的例子。假设我们在做后向传播算法的时候,雅可比矩阵$\MJ$不随$t$改变(如果网络状态的转移函数是纯线性的,那么可能出现这种情况)。

假设$\MJ$特征值$\lambda$对应的特征向量为$\Vv$。
考虑当我们把梯度向量后向传播时会发生什么。
如果最开始的梯度向量为$\Vg$,然后经过一步的后向传播之后,梯度向量变为$\MJ \Vg$,在$n$步之后,梯度向量为$\MJ^n \Vg$。
现在考虑一下我们在后向传播$\Vg$的时候加一个小的噪声发生什么。
如果最开始为$\Vg + \delta \Vv$,一步之后,我们会得到$\MJ(\Vg + \delta \Vv)$。
$n$步之后,我们将得到$\MJ^n(\Vg + \delta \Vv)$。
由此我们可以看出,由$\Vg+\delta \Vv$的结果比由$\Vg$开始的结果,$n$步之后偏离了$\delta \MJ^n \Vv$。
如果$\Vv$选择为$\MJ$特征值$\lambda$对应的一个单位特征向量,那么在每一步乘雅可比矩阵,只是对原梯度向量进行简单地缩放。
后向传播的两次执行偏离的距离为$\delta | \lambda |^n$。
当$\Vv$为绝对值$|\lambda|$最大的特征值对应的特征向量,噪声$\delta$会造成最大的偏离量。

当$ | \lambda | > 1$,偏差$\delta | \lambda |^n$就会呈指数级增长。
当$ | \lambda | < 1$,偏差就会变呈指数级减小。

当然,在这个例子中我们假设每一步的雅可比矩阵是不变的,这个情况只在线性循环神经网络中才会出现。
当网络中有非线性映射是,非线性函数的导数将在多步的后向传播之后后接近零,这有助于防止因过大的谱半径而造成梯度爆炸。
事实上,回声状态网络的最近工作提倡使用远大于1的谱半径 (Yildiz et al., 2012; Jaeger, 2012)。

我们之前提到过在后向传播的时候使用矩阵的幂乘,如果网络是线性的,这也可以被用到前向传播过程中。状态$\Vh^{(t+1)} = \Vh^{(t)\top} \MW $

当我们把$\Vh$经过一个线性映射$\MW^\top$之后,它的$L^2$范数总是缩小,那么我们说这个映射是收缩(contractive)的。
当谱半径小于一,则从$\Vh^{(t)}$到$\Vh^{(t+1)}$的映射是收缩的,因此小变化在每步之后后变得更小。
当我们使用有限精度(如32位整数)来存储状态向量时,必然会使得网络忘掉过去的信息。

雅可比矩阵告诉我们$\Vh^{(t)}$一个微小的搅动是如何一步步前向传播的,以及,$\Vh^{(t+1)}$的梯度如何一步步后向传播的。
需要注意的是,$\MW$和$\MJ$都不需要是对称矩阵(尽管它们是实方阵),因此它们的特征值和特征向量里的元素可以是负数,其中虚数分量对应于潜在的振荡行为(如果每一步的雅可比矩阵是一样的话)。
即使$\Vh^{(t)}$或$\Vh^{(t)}$上的一个搅动都是实数。在后向传播之后可能会造成复数的偏差。
我们所关注的是,当我们用这些矩阵来乘向量,这些复数偏差的幅值(复数的模值)会发生什么变化。
如果特征值的幅值大于一,那么回造成梯度放大(如果我们连续的乘以这个矩阵,会造成指数级增长)。或收缩(如果我们连续的乘以这个矩阵,会造成指数级减小)\footnote{译者注:此处存疑,未讨论幅值小于一的情况,而且是不是幅值大于1才会造成振荡?}。

如果系统里面有非线性的映射函数,那么雅可比矩阵在每个时间步都有可能变化。这个动态系统会变得更复杂。但是,初始值的一个小的搅动经几部之后依然会被放大为一个大的偏差。和线性的动态系统不同的是,如果我们的非线性单元是类似tanh那种压缩非线性单元\footnote{译者注:任何值都会被映射到一个有界的区域。},那么这个循环动态系统状态的值会有界。注意,即使前向传播由于压缩非线性单元会变得有界,在做后向传播时,是有可能没有这个界的。例如,在系统中那个非线性单元为$\tanh$,序列中的元素传入这个非线性单元之前会由谱半径大于1的权重矩阵将之映射为一个值。
然而,所有$\tanh$单元同时位于它们的线性激活点是非常罕见的\footnote{译者注:此处不理解}。

回声状态网络的策略是固定权重为拥有类似于3的谱半径的矩阵,信息通过时间前向传播,但会由于容易饱和的非线性单元(如$\tanh$)的稳定作用而不会产生信息爆炸。

最近的研究表明,这些设置回声状态网络权重的技术可以用于循环神经网络的初始化(如果这个网络的隐藏到隐藏状态的权重可以通过后向传播算法来进行训练的话),可以帮助我们学习长时依赖信息 (Sutskever, 2012; Sutskeveret al., 2013)。在设置的时候,如果初始化权重的谱半径是1.2可以获得一个好的效果。同时我们还可以采取\ref{sec:8.4}节介绍的稀疏初始化的策略。

\section{Leaky单元和其他处理多个时间尺度的策略}
\label{sec:10.9}
一种解决长时依赖问题的方法是设计一个模型用来处理不同的时间尺度的依赖问题。所以模型的一部分用来处理细微的时间尺度并能处理一些小的细节。模型的其他部分可以处理大的时间尺度并能高效地把遥远的历史信息传递给当前的状态。可以用不同的策略来实现这两个功能。我们可以在时间轴上增加跳跃连接(skip connection),Leaky单元可以整合某些固定时间点的信号,去除一些用于细微时间尺度的建模的连接。

\subsection{时间轴上建立跳跃连接}
\label{sec:10.9.1}
一种处理大的时间尺度上依赖的方法是直接在过去的某个变量和当前变量之间建立直接联系。这个建立跳跃连接的想法醉在是由 Lin et al. (1996)提出的,后来在 (Lang and Hinton, 1988)中,在前馈神经网络中引入了单位延迟单元。这样就产生了循环神经网络,从时刻$t$的单元连接到时刻$t+1$单元构建一个循环连接。我们也可以用一个更长的延迟单元来建立循环神经网络 (Bengio, 1991)。

我们在\ref{sec:8.2.5}中看到,随着序列长度的增加,梯度会发生消失或者爆炸现象。 Lin et al. (1996)中介绍通过引入$d$时延的循环连接来缓解这个问题。
现在导数指数减小的速度是$\frac{\tau}{d}$而不是$\tau$的函数。
既然同时存在延迟和单步连接,梯度仍可能呈$\tau$指数爆炸。
这时学习算法获得一些更长的依赖信息,但不是所有的长时依赖信息可以通过这种方式完美的获得。

\subsection{Leaky单元和和一系列不同时间尺度}
\label{sec:10.9.2}
使某条路径上导数乘积接近1的另一方式是设置\emph{线性}自连接单元,并且这些连接的权重值接近1。

当我们用增量计算的方法\footnote{译者注:可以了解一下增量学习,在线学习}来计算$v^{(t)}$的均值$\mu^{(t)}$,更新公式为$\mu^{(t)} \gets \alpha \mu^{(t-1)} + (1-\alpha) v^{(t)}$,其中$\alpha$是一个从$ \mu^{(t-1)}$到$ \mu^{(t)}$线性自连接的例子。 
当$\alpha$接近1时,这个动态的平均值能记住过去很长一段时间的信息,而当$\alpha$接近0,历史信息会被迅速丢弃。拥有线性自连接结构的隐藏单元和增量均值类似。这种隐藏单元我们叫它Leaky单元。

$d$时间步的跳跃连接可以确保当前的单元总能被之前$d$个时间步之前的值所影响。
使用权重接近1的线性自连接是另一种方法确保该单元可以获得过去值得信息。
和对跳跃连接调整步长的方式相比,线性自连接可以通过调节$\alpha$来平滑灵活地调整影响。

这个想法是由Mozer (1992) and by El Hihi and Bengio (1996)提出的。在回声状态网络中,Leaky单元也被发现很有用(Jaeger et al., 2007)。

有两种基本策略来设置Leaky单元使用的时间常数。一种是手动的把他们的值设置为一个常数。例如在初始化时从某些分布进行采样一些值来设置它们。
另一种策略是使把时间常数作为变量,并学习出来。
在不同时间尺度使用Leaky单元似乎能帮助我们解决长时依赖问题 (Mozer, 1992; Pascanu et al., 2013)。


\subsection{连接移除}
\label{sec:10.9.3}
另一种解决长时依赖问题的方法是合理组织不同时间尺度上RNN的状态 (El Hihi and Bengio, 1996)。当时间尺度变慢的时候(把一些连接给去了),信息更容易流动。

这和我们前面提到的跨越连接不一样,因为它会主动\emph{删除}一些长度为1的连接,并用更长的连接来替换它们。我们通过这种方式来修改单元可以帮助我们操作一个更长的时间尺度。跳跃连接会在单元间增加一些边。这些新的连接所连接的单元又可以能会从新的长的时间尺度去获取信息,也有可能它们只专注于那些短的连接所传递过来的信息。

让一组循环单元处理不同时间尺度的数据有不同的方式。一个是使用Leaky单元,但是不同组的单元对应不同的固定的时间尺度。这个最初在 Mozer (1992)提出并在Pascanuet al. (2013)中成功进行了应用。

另一种方式是使显式离散地在不同的时间步进行更新,不同组的单元对应不同的更新频率。这个方法由El Hihi and Bengio (1996) and Koutnik et al. (2014)提出,并在一些数据集中获得了较好的效果。

\section{长短时记忆网络(LSTM)和其它有门结构的RNN(Gated RNN)}
\label{sec:10.10}
在纂写本书的时候,实际应用中最有效的序列模型是带门结构的RNN(Gated RNN)。它们包括长短时记忆网络和基于门随机单元(gated recurrent unit 或 GRU)的网络。

和Leaky单元一样,Gated RNN也是重新在时间轴上生成路径,来避免导数消失和导数爆炸。
在Leaky单元里,我们可以通过手动选择常数作为连接权重或把它作为一个需要学习的参数来达到这一目的。
Gated RNN将其发展为在每个时间步的连接权重都可能发生改变。

Leaky单元允许网络累积较长一段时间的信息(信息主要是指某些特定特征或类的信息)。但是,一旦这些信息被使用之后,我们有时可以把这些旧的状态的信息给丢掉。比如说,
如果一个序列是由子序列组成,我们希望Leaky单元能在每个子序列内部累积信息,我们需要建立一个机制来忘记的状态,通常我们将状态设置为0就表示忘记了该状态。
我们希望神经网络学会决定何时丢弃一些状态信息,而不是手动决定。
这就是Gated RNN要做的事。

\subsection{LSTM}
\label{sec:10.10.1}
在网络结构中引入自循环来产生一条供梯度长时间流动且不会消失的路径是最初长短时记忆网络(long short-term memory 或 LSTM)的核心思想 (Hochreiter and Schmidhuber, 1997)。一个重要的变化是自循环上的权重是有上下文决定的,而不是固定的 (Gers et al., 2000)。在这个自循环的的权重上加一个门限(gate)(这个门限是由另一个隐藏单元决定的),不同时间尺度累积的信息可以动态地改变。
在这种情况下,即使是具有固定参数的LSTM,不同时间尺度上累积的信息也可以根据输入序列来改变,因为时间常数是模型本身的输出。LSTM在很多应用用取得了重大的成功。比如说无约束手写识别(Graveset al., 2009),语音识别(Graves et al., 2013; Graves and Jaitly, 2014),手写文字生成(Graves, 2013), 机器翻译(Sutskever et al., 2014),图像说明(Kiros et al., 2014b; Vinyals et al., 2014b; Xu et al., 2015) 和语义分析 (Vinyals et al., 2014a)。
\begin{figure}[htbp] %  figure placement: here, top, bottom, or page
   \centering
   \includegraphics[width=4in]{fig/chap10/10_16.PNG} 
   \caption{这是LSTM循环神经网络的cell结构。Cell之间是循环联系的,LSTM就是把传统单隐藏层的循环神经网络的隐藏单元换成了Cell。输入特征是由一个普通的人工神经元单元计算得到的。如果输入门(input gate)允许,输入信息可以传递给cell的状态。cell的状态有一个线性自连接结构,其中的权重是有忘记门(forget gate)来控制的。cell的输出由输出门来控制。所有的门,都有一个sigmoid函数,输入门可以有任意的压缩非线性。状态单元的值也可以作为门单元的一个额外的输入。黑色的方块表示的是一个时间单位的时延}
   \label{fig:10_16}
\end{figure}

图\ref{fig:10_16}是LSTM的模块结构示意图。我们在下面给出了浅层循环神经网络的前向传播公式。在(Graveset al., 2013; Pascanu et al., 2014a)中成功的使用了更深的结构。和传统的循环神经网络不同,LSTM不仅仅是将循环单元\footnote{译者注:隐藏状态}和输入合并的向量进行仿射变换然后再进行一个非线性系统,它有一个LSTM cell结构,在LSTM cell里面有一个内部循环(自连接循环)。这个cell的输入输出和传统的循环神经网络一样,但是cell里面有更多的参数,同时还有门单元来控制信息的流动。这里面最重要的部分是cell状态单元$s_i^{(t)}$,它与前一节讨论的Leaky单元类似,有线性自环。
然而,此处自环的权重(或相关联的时间常数)由记忆门(forget gate)$f_i^{(t)}$控制($t$为时间步,$i$为cell状态向量对应的下标),记忆门中的非线性sigmoid单元会将权重设置为0和1之间的值:
\begin{align}
 f_i^{(t)} = \sigma \Big( b_i^f + \sum_j U_{i,j}^f x_j^{(t)} + \sum_j W_{i,j}^f h_j^{(t-1)} \Big),
\end{align}
其中$\Vx^{(t)}$是当前输入向量,$\Vh^{t}$是当前隐藏层向量,$\Vh^{t}$包含所有LSTM cell的输出信息。 
$\Vb^f, \MU^f, \MW^f$分别是偏置参数、输入权重和记忆门的循环权重。
因此LSTM cell状态以如下方式更新,其中有一个由记忆门产生的自环权重$f_i^{(t)}$来控制历史信息:
\begin{align}
 s_i^{(t)} = f_i^{(t)}  s_i^{(t-1)} +  g_i^{(t)}
 \sigma \Big( b_i + \sum_j U_{i,j} x_j^{(t)} + \sum_j W_{i,j} h_j^{(t-1)} \Big),
\end{align}
其中$\Vb, \MU, \MW$分别是LSTM cell中的偏置、输入权重和记忆门的循环权重。
\textbf{外部输入门}(external input gate)单元$g_i^{(t)}$和记忆门(使用一个sigmoid函数来获得0和1之间的值)的更新方式类似,但只不过使用了不同的参数:
\begin{align}
 g_i^{(t)} = \sigma \Big( b_i^g + \sum_j U_{i,j}^g x_j^{(t)} + \sum_j W_{i,j}^g h_j^{(t-1)} \Big).
\end{align}
LSTM cell的输出$h_i^{(t)}$也可以由\textbf{输出门}(output gate) $q_i^{(t)}$来控制,它也使用sigmoid单元作为门限:
\begin{align}
 h_i^{(t)} &= \text{tanh}\big( s_i^{(t)} \big) q_i^{(t)}, \\
 q_i^{(t)} &= \sigma \Big( b_i^o + \sum_j U_{i,j}^o x_j^{(t)} + \sum_j W_{i,j}^o h_j^{(t-1)} \Big),
\end{align}
其中$\Vb^o, \MU^o, \MW^o$分别是其偏置、输入权重和记忆门的循环权重。
在LSTM的一些变体中,可以选择使用cell状态$s_i^{(t)}$作为额外的输入,输入到第$i$个单元的三个门\footnote{译者注:之前gate的输入为当前输入,上一个输出,这个时候就变成当前输入,上一个输出,上一个cell状态},如\ref{fig:10_16}所示。
这是我们将需要三个额外的参数\footnote{译者注:因为有三个门}。

LSTM神经网络比传统的神经网络更容易获得长时依赖信息。它也被成功应用于不同的任务。首先它被用于在人工设计的数据集中测试能否缓解长时依赖问题(Bengio et al., 1994; Hochreiterand Schmidhuber, 1997; Hochreiter et al., 2001)。接着人们用它来进行序列处理任务是获得了较好的结果(Graves, 2012;Graves et al., 2013; Sutskever et al., 2014)。 LSTM的其它一些变种模型我们在下一小节进行讨论。


\subsection{其它带门结构的RNN}
\label{sec:10.10.2}
LSTM中哪些结构是真正所必须的?还有哪些其他的结构可以允许网络动态的的控制时间尺度和不同单元的遗忘行为?

最近研究的带门结构的RNN给出了这些问题的一些答案。在这些结构中它们包含了门限循环单元(gated recurrent units 或 GRUs) (Cho et al., 2014b;Chung et al., 2014, 2015a; Jozefowicz et al., 2015; Chrupala et al., 2015)。它和LSTM最大的不同是:它只用一个门就同时控制状态单元决定忘记哪些信息,更新哪些信息。更新等式如下:
\begin{align}
 h_i^{(t)} = u_i^{(t-1)} h_i^{(t-1)} + (1 - u_i^{(t-1)}) \sigma 
 \Big( b_i + \sum_j U_{i,j} x_j^{(t)} + \sum_j W_{i,j} r_j^{(t-1)} h_j^{(t-1)} \Big),
\end{align}
其中$\Vu$代表更新门,$\Vr$代表复位门。
它们的值定义如下:
\begin{align}
 u_i^{(t)} = \sigma \Big( b_i^u + \sum_j U_{i,j}^u x_j^{(t)} + \sum_j W_{i,j}^u h_j^{(t)} \Big),
\end{align}
和
\begin{align}
 r_i^{(t)} = \sigma \Big( b_i^r + \sum_j U_{i,j}^r x_j^{(t)} + \sum_j W_{i,j}^r h_j^{(t)} \Big).
\end{align}
复位和更新门能分别地在每个时间步"忽略"状态向量的一部分。
更新门像条件Leaky累积器一样,可以给任意的维度加一个线性门限,从而选择将它完全保存下来(sigmoid值取0)或完全由新的"目标状态"值(朝Leaky累积器的收敛方向)替换并完全忽略它(在另一个极端,sigmoid值取1)。
复位门控制状态的哪一部分用于计算下一个目标状态,在过去状态和未来状态的关系之间引入了新的非线性效应。

我们可以设计很多围绕这个主题的变种模型结构。例如,复位门(或记忆门)的输出可以在多个隐藏单元之间共享。
另外,全局门(覆盖一整组的单元,例如一整层)和一个局部门(针对每单元)的乘积可以把全局控制和局部控制结合起来。但是经过调查研究发现,其它的一些变种结构在很多任务上都比不过LSTM和GRU (Greff et al., 2015; Jozefowicz et al., 2015)。 Greffet al. (2015)发现一个关键因素是记忆门。 Jozefowiczet al. (2015)发现在LSTM的记忆门中加入一个偏置1,这个也有 Gers et al. (2000) 所提倡,可以使LSTM比所有已知的变种结构都厉害。

\section{长时依赖的优化}
\label{sec:10.11}
在\ref{sec:8.2.5}和\ref{sec:10.7}中我们介绍了当我们优化一个具有较长序列的RNN的时候,梯度会消失或者爆炸。

 Martens and Sutskever (2011)提出了一个非常有趣的想法:二阶导数可能在一阶导数消失的同时消失。
二阶优化算法可以简单的理解为将一阶导数除以二阶导数(在更高维数,由梯度乘以海森矩阵的逆)。如果二阶导数与一阶导数以类似的速率收缩,那么这两者的比率可保持相对恒定。不过基于二阶导数的优化方法有很多缺点,包括高计算成本,得适应较大的minibatch,有可能使结果陷入鞍点。 Martens and Sutskever (2011)发现使用二阶导数可以保证我们获得好的结果。 Sutskever et al. (2013)发现使用简单的方法也可以达到类似的效果,比如说小心初始化后的Nesterov动量法。更多细节请参考 Sutskever (2012)。在实际应用中,这两种方法都被SGD(甚至不带动量)所取代。
这是机器学习中一个将会持续讨论的主题,设计一个易于优化模型通常比设计出强大的优化算法容易。


\subsection{裁剪梯度}
\label{sec:10.11.1}
在\ref{sec:8.2.4}中,我们讨论了循环神经网络中强非线性函数经过多次递归计算之后,参数的梯度会趋向于爆炸或者消失。这个我们在图\ref{fig:8.3}和图\ref{fig:10.17}中进行了说明,在图\ref{fig:10.17}中我们可以看到目标函数(待学习参数的函数)所表示的三维图像中间有一个悬崖,原来宽阔平坦的区域被目标函数变化快的小区域隔开了,这个小区域形成了悬崖。

这导致的困难是,当参数梯度非常大时,我们用梯度下降法来更新参数时会把参数抛出很远,进入目标函数值较大的区域,我们之前做的工作都白费了。
梯度告诉我们,围绕当前参数值附近的区域(围绕该值的无穷小区域)内最速下降的方向。在这个无穷小区域之外,损失函数的值可能开始增加了,我们得取一个较小的学习速率来避免我们的参数直接进入损失函数值上升的区域。在实际应用中我们通常使用衰减的学习率来进行优化,衰减的速率要足够的慢,使相邻的步骤的学习率大致相同。
如果每个学习速率在目标函数相对线性的部分比较适合,那么当到达比较陡的部分的时候,同样的学习速率就会使得我们的要优化的参数进入目标函数较大的位置。


\begin{figure}[htbp] %  figure placement: here, top, bottom, or page
   \centering
   \includegraphics[width=4in]{fig/chap10/10_17.PNG} 
   \caption{这是一个在循环神经网络上进行梯度裁剪的例子,在循环神经网络上有两个参数$\Vw$和$\Vb$。梯度裁剪可以使梯度下降法在损失函数陡峭的部分表现得更理智。这些陡峭的悬崖通常出现在循环神经网络的的损失函数近似线性部分的附近。循环神经网络在在多个时间步之后悬崖会变得越来越陡峭,因为权重矩阵在每个时间步都要乘以一下自己。(左边)未使用梯度裁剪的梯度下降法,直接越过了这个小山谷的底部,到达了悬崖对面一个梯度较大的的区域,这个区域已经超出了我们的坐标范围。(右边)基于梯度裁剪的梯度下降法对待陡峭的悬崖比较温和。它不会直接把参数更新到悬崖对面,它会限制学习率,所以它会把参数更新到离解附近的地方。该图取自 Pascanu et al. (2013),使用已经获得了作者的许可}
   \label{fig:10_17}
\end{figure}


一个简单的解决方案已被大家使用多年:梯度截断。
这个想法衍生出不同的实例 (Mikolov, 2012;Pascanu et al., 2013)。
一种选择是在参数更新之前,将通过minibatch产生的参数的梯度的emph{每个元素}进行裁剪(Mikolov, 2012)。
另一种是在参数更新之前裁剪梯度$\Vg$的emph{范数}$\norm{ \Vg }$(Pascanu et al., 2013):
\begin{align}
 \text{if}~ \norm{\Vg} &> v \\
 \Vg &\gets \frac{\Vg v}{\norm{g}},
\end{align}
其中$v$是范数上界,$\Vg$用来更新参数。
因为所有参数(包括不同的参数组,如权重和偏置)的梯度同时被被一个缩放因子规范化,所以第二个选择保证了每步更新方向仍在梯度方向上,但实验表明两个选择效果差不多。
虽然参数更新的方向与真实梯度的方向相同,经过梯度范数裁剪,参数更新的向量现在变得有界。
这种有界参数能避免在梯度爆炸时参数更新发生问题。
事实上,当梯度大小高于阈值时,即使使用简单的\emph{随机}更新参数,往往也可以取得一样好的效果。
如果爆炸非常严重,梯度为Inf或Nan(无穷大或不是一个数字),则可以随机选择大小为$v$的一步来更新参数,通常会参数离开数值不稳定的状态。
截断每个minibatch的梯度范数不会改变单个minibatch的梯度方向。
然而,经许多minibatch计算出梯度,然后使用范数裁剪,再把结果求平均值,这个均值不等于裁剪真实梯度(通过所有的实例计算梯度)的范数的结果。
如果某些样本计算出的梯度范数较大,以及与这些样本处于同一个minibatch的样本,他们对最终优化方向的贡献将消失。
和传统基于minibatch的梯度下降法不同,传统方法中,真实的梯度方向是等于所有minibatch梯度的平均。
换句话说,传统的随机梯度下降法使用梯度的的无偏估计,而与使用范数裁剪的梯度下降法中,我们通过经验引入了一个启发式偏置。
通过逐元素裁剪,更新的方向与真实梯度或基于minibatch的梯度不再一样,但是它仍然是一个使损失函数值下降的方向。
(Graves, 2013)还有提出在后向传播算法时截断梯度(相对于隐藏单元),但是没有人把这些变种梯度下降法进行比较。 我们推测,所有这些方法都具有类似的表现。


\subsection{通过正则化来引导信息流动}
\label{sec:10.11.2}
梯度裁剪可以帮助我们解决梯度下降法时遇到梯度爆炸的问题,但它不能帮助我们解决梯度消失问题。
为了解决梯度消失问题和更好地获得长时依赖信息,我们讨论了在循环结构展开的计算图中,创建一条路径,在路径上梯度累积的乘积等于1。
在\ref{sec:10.10}中已经提到过,实现这一点的一种方法是使用LSTM以及其他自循环和门限机制。
另一个方法是通过正则化来约束参数,然后来引导信息流动。
特别是,我们希望梯度向量$\nabla_{\Vh^{(t)}} L$在进行后向传播时可以维持它的幅度,即使损失函数只对序列最后部分的输出作出惩罚。
公式化这个想法,我们要使
\begin{align}
 (\nabla_{\Vh^{(t)}} L) \frac{\partial \Vh^{(t)}}{\partial \Vh^{(t-1)}}
\end{align}
与
\begin{align}
\nabla_{\Vh^{(t)}} L 
\end{align}
一样大。
在这个目标下, Pascanu et al. (2013)提出了下面的正则项:
\begin{align}
 \Omega = \sum_t \Bigg(  \frac{
 \norm{ (\nabla_{\Vh^{(t)}} L) \frac{\partial \Vh^{(t)}}{\partial \Vh^{(t-1)}}}}
 {\norm{\nabla_{\Vh^{(t)}} L}} -1 \Bigg)^2.
\end{align}
计算这一正则项的梯度可能会出现困难,但 Pascanu et al. (2013)提出可以将后向传播向量$\nabla_{\Vh^{(t)}} L$近似为一个常数(为了实现正则化,没有必要把它们向后传播)。
使用该正则项的实验表明,如果与范数启发式裁剪(处理梯度爆炸)相结合,这个正则项可以显著地增加RNN处理长时依赖的能力。
梯度裁剪特别重要,因为它保证了在发生梯度爆炸的时候RNN依然可以进行学习优化。
如果没有梯度裁剪,发生梯度爆炸后模型将无法进行后续的学习。

这种方法的一个主要弱点是,在处理含有冗余数据的任务时,如语言模型时,它并不像LSTM一样高效。



%%%%%%%%%%%%%%%%%%%%%%%%%%%%%%%%%%%%%%%%%%%%%%%%%%%%%%%%%
%%%%%%%%%%%%%%%%%%%author:rickymf4%%%%%%%%%%%%%%%%%%%%%%%
%%%%%%%%%%%%%%%%%%%%%%%%%%%%%%%%%%%%%%%%%%%%%%%%%%%%%%%%%
\chapter{实战方法}
\label{chap:11}

成功地应用深度学习技术远远不仅仅需要知道有哪些算法和这些算法的原理。一个好的机器学习实践者亦需要知道如何针对特定的应用选择算法,如何监控实验并对获得的反馈做出回应,从而优化该机器学习系统。在机器学习系统的日常开发中,实践者们需要决定是否需要搜集更多的数据,是增加还是减少模型的容量,是加上还是去掉规则化特征,是改进模型的优化,还是改进模型中的近似推理,亦或是对模型的实现程序进行调试。所有这些操作都是非常耗时的尝试,所以能够确定正确的行动方针是很重要的,而不是盲目猜测它。

这本书的大部分内容是关于各种机器学习模型、训练算法,和目标函数。这可能会造成这样的印象:成为机器学习专家最重要的本领是要了解各种机器学习技术并擅长不同类型的数学。 然而在实践中,正确地使用一个常见的算法比草率地使用一个晦涩的算法要好。算法的正确应用取决于掌握一些相当简单的方法。本章节中许多建议改编自。

我们建议如下的实际设计过程:
\begin{itemize}
\item 确定你的目标——要使用的误差度量和相应的目标价值。这些目标和误差度量应该由应用要解决的问题来驱动。
\item 尽快建立一个可以工作的端到端的工作流程,包括适当的性能指标。
\item 较好地给系统装上“仪表”来确定其性能瓶颈。诊断哪些部分的性能低于预期,以及是否是由于过度拟合、欠拟合,或数据或程序中存在着缺陷。
\item 基于来自“仪表”中特定的发现,重复地进行逐步改更,如收集新数据,调整超参或更改算法。
\end{itemize}

我们将采用街景地址号码转录系统作为一个运行的例子。 这个应用的目的是将建筑添加到谷歌地图中。街景车对建筑进行拍摄并记录照片对应的GPS坐标。通过卷积网络来识别每张照片中的地址号码,使得谷歌地图数据库可以将该地址加入到正确的位置上。关于如何开发这个商业应用程序的故事给出了如何遵循我们倡导的设计方法的一个例子。我们现在来描述这个过程中的每个步骤。

\section{性能度量}
\label{sec:11.1}
确定目标(即采用什么样误差度量)是必要的第一步,因为你的误差度量将指导你以后的所有操作。你还应该知道你想要什么级别的性能。

请记住,对于大多数应用,不可能实现绝对零误差。即使你有无限训练数据,并且可以恢复真实的概率分布,而贝叶斯误差定义了你能实现的最小错误率。这是因为你的输入特征可能不包含有关输出变量的完整信息,或者因为系统本质上可能是随机的。你也将受限于有限数量的训练数据。

由于各种原因,训练数据的量被限制了。当你的目标是构建一个一流的实际产品或服务时,你通常可以收集更多数据,但必须确定进一步减少错误率的价值,并将其与收集更多数据的成本进行权衡。数据收集可能需要时间、金钱,或给人们大带来痛苦(例如,如果你的数据收集过程涉及侵入性的医学测试)。 而当你的目标是回答在固定的基准上哪个算法的表现更好时,通常是不允许收集更多数据作为基准的训练集的。

如何给性能水平确定一个合理的期望?通常,在做学术时,我们基于以前发布的基准测试结果,可以做一些错误率的估计。而在做实际项目时,我们考虑的错误率是要让我们的应用是安全的,成本效益的,或是吸引消费者的。一旦确定了实际期望的错误率,那么为了达到此错误率将指导你的设计决策。

除了性能指标的目标值之外的另一个重要考虑是选择使用哪个指标。可以使用几种不同的性能指标来测量一个完整的包含各种机器学习组件的应用的有效性。这些性能指标通常不同于用于训练模型的损失函数。 如\secref{sec:5.1.2}所述,我们通常测量系统的准确率或错误率。

然而,许多应用需要更高级的指标。

有时一种错误的代价会比另一种更高。例如,电子邮件垃圾检测系统可能产生两种错误:错误地将合法的邮件分类为垃圾邮件,以及错误地允许垃圾邮件在收件箱中显示。

阻止合法消息比允许可疑消息通过更糟糕。 我们不是度量垃圾邮件分类器的错误率,我们可能希望用某种形式来衡量总体代价,认为阻止合法邮件的代价高于允许垃圾邮件。

有时我们想要训练一个二元分类器,用于检测一些异常事件。例如,我们将要为一个罕见的疾病设计一个医学测试。假设每百万人中只有一人患有这种疾病。我们通过简单地将分类器结果固定为无疾病,可以很容易地实现$99.9999\%$的检测精度。显然,用准确性来表征这种系统的性能的是不行的。 解决这个问题的一种方法是改为采用精度(precision)和召回(recall)的统计方法。精度是模型的检测结果中正确的比例,而召回是真实事件被模型检测到的比例。当一个检测器说没有人患这种疾病时将有完美的精度,但召回是零。当一个检测器说每个人都有疾病时将实现完全召回,但精度等于患有该疾病的人的百分比(在我们的例子中,一百万人中有一人的比例是$0.0001\%$)。当使用精度和召回时,我们通常绘制PR曲线,y轴表示精度,x轴上表示召回。当待检测的事件发生时,分类器将输出较高的分数。 例如,一个检测疾病的前馈网络的输出$\hat{y} = P(y=1\mid\Vx)$,表示医疗结果具有特征x的人的患病概率。当该输出概率分数大于某个阈值时,则认为检测到了。通过改变阈值,我们可以权衡在某召回下的精度。在许多情况下,我们希望用单个数字而不是曲线来总结分类器的性能。为此,我们将精度$p$和召回$r$转换为一个$F$分数:

\begin{equation}
        F = \frac{2pr}{p+r}.
\end{equation}

另一个方法是统计位于PR曲线下方的总面积。

在一些应用中,机器学习系统可以拒绝做决定。当机器学习算法可以估计一个决策的可信度时这是有用的,特别是如果错误的决策可能带来危害并且操作人员能够偶尔接管的话。街景转录系统就是这样一个例子。 它的任务是从照片中转录地址号码,以便将拍摄照片的位置与地图中的正确地址相关联。因为如果地图不准确,地图的价值会大大降低,所以只有在转录正确的情况下才添加地址非常重要。如果机器学习系统比人类更难获得正确的转录,那么最好的办法就是让人来替代。当然,如果机器学习系统能够大大减少运营人员必须处理的照片数量,那么它是有用的。在这种情况下使用覆盖率来作为性能指标。覆盖率是机器学习系统能产生响应的样本的比率。 可以将覆盖率和准确率进行权衡。 如果我们决绝了所有样本,则能获得$100\%$的准确率,但覆盖率减少到$0\%$。对于街景转录这个任务,目标是在保持$95\%$的覆盖率下达到人类水平的转录准确率。在这个任务上,人类水平为$98\%$。

还有可能是其它指标。 例如,我们可以衡量点击率,收集用户满意度调查等。许多专业应用领域也有应用特定的标准。

重要的是确定要先提高哪个性能指标,然后集中精力提高这个指标。没有明确的目标,很难判断机器学习系统的优化是否取得进展。


\section{默认基准模型}
\label{sec:11.2}

选择性能指标和目标后,下一步,对于任何的实际应用都是要尽快建立一个合理的端到端系统。 在本节中,我们提供了在各种情况下用哪个算法作为第一个基准方法的建议。请记住,深度学习研究进展是很快的,所以本书出版后可能会有更好的默认算法。

根据你的问题的复杂性,你甚至可能不想使用深度学习。如果你的问题可以通过正确地选择一些线性权重就能解决,你可能从一个简单统计模型开始做,如逻辑回归。

如果你知道你的问题属于“AI-完全的”类别,如物体识别,语音识别,机器翻译等等,那么你很可能通过一个恰当的深度学习模型来做出比较好的效果。

首先,根据你的数据的结构,选择一个类型的模型。如果你想做有监督学习,而且有固定大小的向量作为输入,那么使用带全连接层的前馈网络。如果已知输入的拓扑结构(例如,输入是一幅图像),那么使用卷积网络。在这些情况下,你应该从使用一些分段线性单元(ReLUs 或其推广如Leaky ReLUs,PreLus和MAXOUT)开始。如果你的输入或输出是一个序列,那么使用门控递归神经网络(LSTM或GRU)。

优化算法的合理选择是带动量的随机梯度下降法(SGD),加上带衰减的学习率(流行的衰减方法在不同的问题上表现有好有坏,包括:线性地衰减到一个固定的最小值,指数衰减,或每次遇到验证错误率趋于平稳时将学习效率降低2到10倍)。另一个非常合理的选择是Adam方法。批量归一化对优化性能有很大的影响,尤其是卷积网络和带sigmoidal 的非线性的网络。虽然你刚开始的第一个基线不用批量归一化是合理的,但如果优化似乎是有问题的,应该马上使用它。

除非你的训练集包含了数千万的例子或更多,否则你应该从开始就引入一些轻度的正规化。提前终止被普遍使用。Dropout是一种不错的正则化方法,它易于实现,并与许多模型和训练算法兼容。批量归一化也往往能降低泛化误差,而不用dropout,这是因为在用于规范化每个变量的统计估计中存在噪声。 

如果你的任务和另一个被广泛研究的任务相似的话,你可以先拷贝那些已经被证明在先前研究任务上表现最好的模型和算法,从而获得比较好的结果。你甚至可以从该任务复制一个训练好的模型。例如,一个是常见的做法是使用基于在ImageNet上训练得到的卷积网络的特征,来解决其他计算机视觉任务。

一个常见的问题是,是否要使用无监督学习,我们会在第三部分进一步介绍。这是和一些特定领域的有关的。有一些领域,如自然语言处理,都大大受益于无监督学习技术,如学习无监督词嵌入。在其他领域,如计算机视觉,目前的无监督学习技术没有优势,除了基于少量标记样本的半监督的学习。如果无监督学习对你的应用重要时,那么你可以引入它作为你第一个端到端学习的基线。否则,只有当你想解决的任务是无监督的时候,你才用无监督学习。还有,如果你发现你的初始基线过拟合了,你总是可以在后期尝试引入无监督学习。


\section{确定是否收集更多的数据}
\label{sec:11.3}

在建立了第一个端到端的系统之后,接下来便是测量我们的算法性能,并确定如何改进它。许多机器学习新手试图通过尝试许多不同的算法来进行改进。 然而,收集更多的数据往往要比改进学习算法更有效。

如何决定是否收集更多的数据? 首先,确定在训练集上的性能是否是可接受的。如果在训练集上的性能差,那么学习算法并没有利用已有的训练数据,因此没有理由收集更多数据。相反,得尝试通过添加更多的层或向每个层中添加更多隐藏单元来增加模型的大小。 此外,还得尝试改进学习算法,例如通过调整学习率等超参。如果你用的是大模型,并且对其进行了精心的调整优化,却不能很好地工作,则可能是训练数据质量的问题。数据可能含有太多噪声,或可能不包括待预测的目标输出的所需的正确输入。这时建议你从头开始,收集更干净的数据或收集更丰富的特征集合。

如果在训练集上的性能可接受,那么对测试集上的性能进行测试。如果测试集上的性能也是可接受的,那么没有什么可以做的了。如果测试集上的性能比训练集上的性能糟糕得多,那么最有效的解决方案之一便是收集更多的数据。要考虑的关键因素是收集更多数据的成本和可行性,通过其他方式降低测试集错误率的成本和可行性,以及预期能显着提高测试集性能的所需的数据量。在拥有数百万或数十亿用户的大型互联网公司,收集大型数据集是可行的,这样做的费用可能远低于其他替代方案,因此答案几乎总是收集更多的训练数据。例如,构建大规模的标记数据集是解决目标识别的最重要的因素之一。而在其他情况下,例如医疗应用,收集更多数据可能是昂贵的或不可行的。一个简单的替代收集更多数据的方法是通过调整超参(例如权重衰减系数)或通过添加正则化策略(例如dropout)来减小模型的大小或改进正则化。如果你发现即使调整正则化超参后,训练和测试性能之间的差距仍然不可接受,那么建议你收集更多的数据。

在决定要收集更多数据时,还需要决定收集多少。我们可以绘制训练集大小和泛化误差之间的关系曲线,如图\ref{fig:5_4}所示。 通过外推这样曲线,可以预测要达到某个性能水平所需要的额外训练数据量。通常,添加相对总量很小的一部分样本数量不会对泛化错误产生显着影响。因此建议你以对数标量来设置训练集的大小进行实验,例如使相邻两次实验之间的数据量加倍。

如果收集更多的数据是不可行的,唯一一个其它的方法是改进学习算法本身来改进泛化误差。这变成是一个研究领域,而不是一个单单给应用实践者一些建议的范畴。

\section{超参的选择}
\label{sec:11.4}

大多数的深度学习算法有许多超参,它们控制着算法的诸多方面。有的超参影响算法运行的时间和内存的消耗,有的则影响着模型的质量,以及在新的输入上推断出正确结果的能力。

有两种选择超参的基本方法:即人工选择和自动选择。人工选择超参需要理解超参的作用以及是如何帮助机器学习模型达到好的泛化的。自动选择超参可以大大降低对这些思想理解的需要,但它们往往有更大计算代价。


\subsection{手动超参调整}
\label{sec:11.4.1}

为了手动设置超参,你必须知道超参、训练误差、泛化误差和计算资源(内存和运行时间)之间的关系。这着这意味着建立一个坚实的基础,是在\ref{chap:5}中提到的关于学习算法有效容量的基本思想。

手动超参数搜索的目标通常是在一定的运行时间和内存的预算条件下找到最低的泛化误差。这里我们不讨论如何确定各种超参对运行时间和内存的影响,因为这是平台相关的。

手动超参数搜索的主要目的是调整模型的有效性容量来匹配任务的复杂性。 模型的有效容量受三个因素的约束:模型的表达能力,学习算法的最小化代价函数的能力,以及代价函数和训练过程使模型正则化的程度。含有更多层和每层更多隐藏单元的模型具有更高的表达能力(其能够表达更复杂的函数)。如果某些函数不能很好地最小化训练代价,或者诸如权重衰减的正则化项排除了这些函数,则训练算法不一定实际上学习所有这些函数。

泛化误差关于某个超参的关系函数绘制后通常成U形的曲线,如图\ref{fig:5_3}所示。曲线上的一个极端情况是,超参数值对应于低容量模型,使得训练误差高,而导致泛化误差高。 这是欠拟合。 另一个极端情况是,超参数值对应于高容量模型,使得训练和测试误差之间的差距大,而导致泛化误差高。 两个极端的中间的某个地方存在着最佳的模型容量,其通过结合中等大小的训练误差中和中等大小的泛化误差,来实现最低可能的泛化误差。

对于某些超参,当其值很大时会发生过拟合。其中一个例子是一层中的隐藏单元数量,因为随着隐藏单元数量的增加,模型容量会增加。而对于某些超参,当其值很小时会发生过拟合。例如,权重衰减系数的最小值0对应于学习算法的最大有效容量。

不是每个超参数都能够连续遍历整个U形曲线。许多超参数是离散的,例如一层中的单元数或maxout单元中的线性段数,因此只能沿着曲线访问几个点。有些超参是二值的。这些超参往往是一些指定是否要使用学习算法的一些可选组件的开关,比如对输入特征进行减均值和除以标准方差这些归一化的预处理步骤。这些超参数只能遍历曲线上的两个点。其他超参数则具有一些最小或最大值约束,限制它们访问曲线的某些部分。例如,权重衰减系数的最小值为零。这意味着如果模型在权重衰减值为零时欠拟合,我们不能通过修改权重衰减系数来进入过拟合区域。换句话说,一些超参数只能减掉容量。

学习率也许是最重要的超参了。如果你只有调整一个超参的时间,那么就去调整学习率。它用比其他超参更复杂的方式控制模型的有效容量(针对优化问题,学习率设置正确时,而不是特别大或特别小时,模型的有效容量达到最高)。学习率关于训练误差的是成U形曲线的,如图\ref{fig:11_1}所示。当学习率太大时,梯度下降可能无意中增加而不是减少训练误差。在理想化的二次情况下,当学习率大于等于其最优值的两倍时,就会出现这种情况。 当学习速率太小时,训练不仅更慢,而且可能永久地陷入在较高的训练误差中。我们对这种影响的理解是不足的(对于凸的损失函数这并不会发生)。

\begin{figure}[htbp] %  figure placement: here, top, bottom, or page
   \centering
   \includegraphics[width=6in]{fig/chap11/11_1.png} 
   \caption{学习率和训练误差之间的典型关系。 可以看到当学习率高于一个最佳值时误差急剧上升。这里的训练时间是固定的,因为较小的学习率有时可能仅仅减慢了训练,且减慢的速度和学习率减小量成比例。泛化误差也可以遵循该曲线,或者由于具有过大或过小的学习率而引起的正则化效应而变得复杂,因为较差的优化可以在一定程度上减少或防止过拟合,并且甚至在等效的训练误差下具有不同的泛化误差。}
   \label{fig:11_1}
\end{figure}

调整学习率以外的参数需要同时监控训练误差和测试误差,以诊断你的模型是过度拟合还是欠拟合,然后适当调整其容量。

如果你在训练集上的错误率高于你的目标错误率,除了增加模型容量没有别的选择。如果你不使用正则化,并且你确信自己的优化算法正确地执行了,则必须向网络中添加更多层或添加更多隐藏单元。但这增加了模型的计算成本。

如果你在测试集上的错误率高于目标错误率,你可以采取两种操作。测试误差是训练误差和训练与测试误差之间差距的和。通过权衡这些量可以找到最佳测试误差。神经网络通常在训练误差非常低(则容量高时)时表现得最好,并且测试误差主要由其与训练误差之间的差距来驱动。你的目标是快速地减少这个差距,而并不增加训练误差。为了减少这个差距,你要改变用于正则化的超参,以减少有效模型容量,例如添加dropout或权重衰减。最佳的性能往往来自于大模型,而且它是较好地正则化的,如使用dropout。

大多数超参可以通过推断它们是增加还是降低模型容量和设置,一些例子如表\ref{tab:11.1}所示。

\begin{table}[!hbp]
    \begin{tabular}{|c|c|c|c|}
        \hline
        \hline
        超参 & 何时可使增大容量 & 原因 & 注意点 \\
        \hline
        隐藏单元数量 & 增加 & 增加隐藏单元数量可以增加模型的表达能力 & 增加隐藏单元数量会对模型的每个操作都增加计算和内存消耗 \\
        \hline
        学习率 & 调整到最优 & 一个过大或过小的不当的学习率会导致优化失败而使得模型有效容量低 & \\
        \hline
        卷积核宽度 & 增大 & 增大卷积核宽度会增加模型的参数个数 & 较宽核的输出较窄,除非使用0填充来减小影响,否则会降低模型容量。宽核需要更多的内存来存储参数,并且计算量增加,但更窄的输出可以减少内存消耗。 \\
        \hline
        隐式0填充 & 增加 & 在卷积前加入隐式0填充可以保持较大的表达尺寸 & 对于大部分的操作都增加了时间和内容的消耗 \\
        \hline
        权重衰减系数 & 减小 & 减小权重衰减系数会让模型的参数自由而变大 &  \\
        \hline
        Dropout比例 & 减小 & Dropout单元会减少各个单元之间“合作”的机会来拟合训练集 &  \\
        \hline
        \hline
    \end{tabular}
    \caption{各种超参对模型容量的影响}
    \label{tab:11.1}
\end{table}

在手动调整超参时,不要忽视你的最终目标,即在测试集上获得良好的性能。添加正则化只是实现这一目标的一种方法。只要你具有低的训练误差,你始终可以通过收集更多的训练数据来减少泛化误差。实际中保证成功的一个暴力的方法是不断增加模型容量和训练集大小,直到任务被解决。这种方法必然增加了训练和预测的计算成本,因此只有在适当的资源下才是可行的。理论上这种方法可能由于优化困难而失败,但对于许多问题,只要选择合适的模型,优化似乎不是重要的障碍。

\subsection{自动超参优化算法}
\label{sec:11.4.2}

理想的学习算法只需要一个数据集然后输出一个函数,而不需要手动调节超参。一些比较受欢迎的算法,如逻辑回归和支持向量机源于它们本身的能力,只需调节用一、两个超参就可获得良好效果。神经网络有时可以只调节少数的超参而表现不错,但是通常需要调节四十个或更多个的超参来获得大幅度的提升。对于手动调整超参,当使用者有一个良好的起始条件,例如重用了别人的在类似的应用和架构上的初始值,或者当使用者具有类似任务的数月或多年的超参探索经验时可以获得较好的效果。然而,对于大部分应用,往往没有这样的起始条件。在这些情况下,自动算法可以找到有效的超参值。

当我们考虑一个学习算法使用者在搜索一个好的超参时所使用的方式,意识到这正是在进行优化:即我们试图找到超参的值来优化目标函数,例如验证误差,而且有时它是被一定条件约束的(例如训练耗时、内存消耗或识别耗时)。因此,理论上我们可以开发一个学习算法的选择超参的优化算法,从而将超参对使用者进行隐藏。不幸的是,超参优化算法通常有它们自己的超参,例如每个学习算法的超参应该有取值范围。然而,这些辅助超参往往更容易选择, 某种意义上,使用同样辅助超参的任务中的大部分可以获得可接受的效果。

\subsection{网格搜索}
\label{sec:11.4.3}

当存在三个或更少的超参时,通常的做法是网格搜索。对于每个超参,选择少量值的集合进行尝试。网格搜索算法然后针对超参值的集合的笛卡尔乘积后的联合值训练模型。实验后我们选择获得了最好校验误差对应的超参。如图\ref{fig:11_2}的左边,展示了一个超参值的网格。

\begin{figure}[htbp] %  figure placement: here, top, bottom, or page
   \centering
   \includegraphics[width=6in]{fig/chap11/11_2.png} 
   \caption{网格搜索和随机搜索的比较。为了演示,我们展示了含有2个超参的情况,当然我们通常关注有更多超参的情况。(左)为了执行网格搜索,我们为每个超参提供一组值。搜索算法对这些集合中的每个联合超参组合运行训练。(右)为了执行随机搜索,我们提供联合超参的概率分布。通常,这些超参中的大多数彼此独立。单个超参数的常见分布包括均匀分布和对数均匀分布(对数均匀分布的采样,即从均匀分布中抽取样本进行指数运算)。搜索算法随后对联合超参组合进行随机采样,并对其中的每个组合进行训练。网格搜索和随机搜索都进行验证集错误情况的评估,并返回最佳组合。该图展示了典型情况,其中只有一些超参对结果具有显着影响。在该图中,只有水平轴上的超参具有显着的效果。网格搜索浪费了计算量,它在无影响的超参的数量上是指数级的;而随机搜索几乎在每个试验上都测试了对结果有影响的每一个超参的一个值。}
   \label{fig:11_2}
\end{figure}

应该如何选择待搜索的值列表呢?在超参是数值(有序)的情况下,根据以往在类似实验上的经验,保守地选择列表中最小和最大的值,以确保最佳值在这个范围能被选到。比较典型的是,网格搜索采用在对数尺度上近似地取值,比如在$\{0.1,0.01,10^{-3},10^{-4},10^{-5}\}$这个集合上选取学习率,或者在$\{50,100,200,500,1000,2000\}$这个集合上选取隐含单元的个数。

不断重复地执行的网络搜索往往表现得最好。比如,假设我们在超参$\alpha$的取值为$\{-1,0,1\}$上执行网格搜索。如果最佳值是1,那么我们低估了最佳$\alpha$的取值范围,进而我们应该搜索其他的取值范围,如$\{1,2,3\}$。如果我们找到的$\alpha$最佳值为0,那么我们可能希望通过放大\{-0.1,0,0.1\}来进行更加精细的网格搜索。

网格搜索的一个明显的问题是,他的计算成本随着超参的个数指数地增长。如果有$m$个超参,每个超参最多有$n$个取值,那么训练和评估试验的次数以$O(n^m)$的方式增长。试验可以平行地跑,并利用松散的并行(搜索几乎无须不同机器之间的通信)。然而,由于网格搜索的指数级的成本,即使是并行也无法提供满意的搜索规模。

\subsection{随机搜索}
\label{sec:11.4.4}

幸运的是,有一个替代网格搜索的方案,它容易编程,使用更方便,并能更快地收敛到好的超参值:随机搜索。
随机搜索按照如下步骤进行。首先我们为每个超参定义边缘分布,例如,用于二值或离散超参的伯努利分布或多项分布,或对于正实数值超参的对数均匀分布。 例如,
\begin{align}
        \texttt{log\_learning\_rate} &\sim u(-1, -5), \\
        \texttt{learning\_rate} &= 10^{\texttt{log\_learning\_rate}},
\end{align}
其中,$u(a,b)$表示区间$(a,b)$上均匀采样的样本。
同样地,$\texttt{log\_number\_of\_hidden\_units}$可从$u(\log(50), \log(2000))$上采样获得。

和网格搜索不同,我们不需要对超参进行二值化或离散化。这使得我们可以搜索更多的值而不需要增加额外的计算成本。事实上,如图\ref{fig:11_2}所示,当存在一些对性能影响不十分大的超参时,随机搜索将指数地有效于网格搜索。这在被详细地研究,他们发现随机搜索比网格搜索能更快地减少校验集误差,即每种方法更少的试验次数。
与网格搜索一样,人们往往想要运行随机搜索的多个重复版本,以基于第一次运行的结果来精细化搜索。

随机搜索能比网格搜索能更快获得好的解决方案的主要原因是它没有浪费的实验,不像网格搜索那样一个超参的两个值(给定其他超参的值)会获得相同的结果。在网格搜索的情况下,这两次运行的其他超参会具有相同的值,而对于随机搜索,它们通常具有不同的值。因此,如果这两个值的变化在校验集误差上并没有多少差别,网格搜索会不必要地重复进行两次等效实验,而随机搜索仍将给出其他超参的两次独立探索。

\subsection{基于模型的超参优化}
\label{sec:11.4.5}

可以将搜索好的超参作为一个优化问题。决策变量是这些超参。待优化的代价是用这些超参训练获得的结果在校验集上的误差。在简化的情况下,计算一些关于超参可微的校验集误差度量的梯度,我们可以简单地遵循这些梯度。然而,在大多数实际情况下,这个梯度是不可用的,因为其较高的计算和内存消耗,或由于超参本质上是在校验集误差上不可微的,比如离散值超参的情况。

了弥补这种缺乏梯度,我们可以构建一个校验集误差的模型,然后通过在这个模型下执行优化并提出一些新的超参猜测。大部分基于模型的超参搜索算法采用贝叶斯回归模型来估计每个超参的校验集误差的期望值以及期望值周围的不确定性。由此,优化需要在探索(指提出具有高度不确定性的超参,这可能导致大幅度的改进,但也可能表现不佳)和利用(指提出模型确定的超参,与目前任何超参表现一样好-通常这些超参非常类似于之前的超参)之间做权衡。当前超参优化方法包括Spearmint,TPE和SMAC。

目前我们不能明确地推荐贝叶斯超参优化法作为一个成熟工具来达到更好的深度学习结果,或者用更少的工作获得那些结果。 贝叶斯超参优化法有时表现得和人类专家相当,有时更好,但在其他问题上完败。在一个特殊问题上是值得尝试看它是否工作,但还不够成熟或可靠。那是说,超参优化是一个重要的研究领域,往往主要由深度学习驱动,而它具有贡献的潜力,不仅仅是在机器学习领域,而且是在整个工程学科。

\subsection{调试技巧}
\label{sec:11.5}

当机器学习系统表现不佳时,通常难以分辨性能不佳是算法本身固有的,还是算法实现中存在的错误。由于各种各样的原因,机器学习系统难以调试。

在大多数情况下,我们预先不知道算法的预期表现是什么。事实上,使用机器学习的整个出发点是它会的发现有用的行为,而这些是我们自己无法指定的行为。如果我们在一个新的分类任务上训练神经网络,并且它达到$5\%$的测试误差,我们没有直接的方法了解这是一个预期表现,还是一个次优的表现。

另一个困难是大多数机器学习模型具有多个部件,而且每个都是自适应的。如果一个部件坏了,其他部件仍然可以适应并达到大致可接受的性能。例如,假设我们训练具有由权重$\MW$和偏置$b$的参数的多层神经网络。其次假设我们对每个参数分别手动实现了梯度下降规则,我们做了一个错误的偏置的更新:
\begin{equation}
    \Vb \leftarrow \Vb - \alpha,
\end{equation}
其中$\alpha$是学习率。这个错误的更新没有用到梯度。这使得偏置随着训练不断地变为负数。通过检查模型的输出可能无法发现这个错误。根据输入的分布,权重可能能够适应并补偿这个负偏置。

神经网络的大多数调试策略旨在绕过这一个或这两个困难。或者我们设计一个用例可以简单地预测正确的行为,或者我们设计一个测试,单独执行神经网络的一部分。

一些重要的调试测试包括:

emph{模型执行可视化}:
当训练模型用来检测图像中的目标物体时,可以将模型的检测结果显示在图像上来观察。当训练语音的生成模型时,可以听其产生的一些语音样本。这可能看起来很明显,但是实践中很容易落入仅考虑诸如准确度或对数似然的定量性能度量。直接观察机器学习模型如何执行任务,将帮助你确定其实现的定量性能数值是否合理。对错误的评估却可能是最具破坏性的错误,因为它们可能误导你相信你系统性能是良好的,但事实上该系统性能却不好。

emph{最差样本可视化}:
大多数模型可以为所执行的任务输出某种可信度量。比如,基于softmax的分类器将一个概率值赋给每个类别。因此,分配给最可信的类所对应的概率值给出了分类器决策的置信度的估计。通常,最大似然训练导致这些值是被高估的,而不是对正确预测的一个准确概率,但它们一定程度上是有用的,因为那些实际上不大会被正确地标注的样本在模型下的概率值更小。通过查看那些对模型来说最难的训练集样本集,人们可以经常发现数据处理或标注的方法问题。比如,街景转录系统最初有一个问题,即地址号码检测系统会因将图像裁剪过小而漏掉一些数字。转录网络然后给正确答案的图像分配了将非常低的概率。由此对图像进行排序,确定了一些最可信的错误,表明裁剪出现了系统性的问题。修改检测系统以裁剪更宽的图像使得整个系统获得更好的性能,即使这会导致转录网络需要能够处理有更大变化位置和比例的地址编号。

emph{用训练和测试误差来论证软件}:
通常很难确定底层软件是否正确地实现。可以从训练和测试中获得一些线索。如果训练误差低但测试误差高,则训练过程可能是正确的,而由于基本的算法原因导致模型过拟合。另一种可能性是,训练后保存的模型被重新加载用于对测试集的评估,或者测试数据的准备与训练数据不同,这些问题导致测试误差被错误地度量。如果训练和测试误差都很高,则难以确定是否存在软件缺陷,或者是否由于基本算法原因导致模型欠拟合。这种情况需要进一步的测试,将下面描述。

emph{拟合一个小数据集}:
如果你的训练误差较大,请确定是否它是由于真正的欠拟合或由于软件缺陷。通常甚至是小模型也能保证可以拟合足够的小数据集。例如,仅通过正确地设置输出层的偏置就可以拟合具有一个样本的分类数据集。通常,如果你不能通过训练一个分类器来对单个样例进行标记,不能训练一个自动编码器来成功重现高保真的单个样例,或不能训练一个生成式模型来一直生成类似单个样例的样本,那么你的软件存在缺陷,阻碍了对训练的成功优化。这个测试可以扩展到一个只有少数几个样例的数据集上。

emph{比较反向传播的导数和数值导数}:
如果你正在使用一个需要你自己实现梯度计算的软件框架,或者如果你添加一个新的操作到求导计算库中,并必须定义它的反向传播(bprob)方法,那么常见的错误来源于梯度表达的实现错误。一个验证求导正确性的方法是比较你自己实现的自动求导所计算出的导数和有限差分计算出的导数。因为
\begin{equation}
    f'(x) = \lim_{\epsilon \to 0} \frac{f(x+\epsilon) - f(x)}{\epsilon},
\end{equation}
我们可以通过一个小的,有限的$\epsilon$来近似导数
\begin{equation}
    f'(x) \approx \frac{f(x+\epsilon) - f(x)}{\epsilon}.
\end{equation}
我们可以通过中心差分来提升近似精度:
\begin{equation}
    f'(x) \approx \frac{ f(x+\frac{1}{2}\epsilon) - f(x-\frac{1}{2}\epsilon) }{\epsilon}.
\end{equation}
扰动大小$\epsilon$必须足够大,以确保扰动不会由于有限精度的数值计算而过分舍入。
通常,我们要测试矢量值函数$g:\SetR^m \to \SetR^n$的梯度或雅可比矩阵。不幸的是,有限差分只允许我们一次求一个导数。我们既可以运行有限差分$mn$次以评估$g$的所有偏导数,又可以将测试应用于一个在$g$的输入和输出处都采用随机投影的新函数。例如,我们可以将我们的导数的实现的测试应用于$f(x) = \Vu^T g(\Vv x)$,其中$\Vu$和$\Vv$是随机向量。
正确计算$f'(x)$要求能够正确地通过$g$反向传播,但是使用有限差分来算是有效的,因为f只有单个输入和单个输出。对于有多个u和v值的情况重复这种测试通常是个不错的主意,以减少测试忽略与随机投影正交的几率。

如果可以在复数上进行数值计算,那么通过使用复数作为函数的输入能非常高效地用数值方法来估计梯度\citep{Squire+Trapp-1998}。
该方法基于如下观察
\begin{align}
    f(x + i\epsilon) &= f(x) + i\epsilon f'(x) + O(\epsilon^2) ,\\
    \text{real}( f(x+i\epsilon) ) &= f(x) + O(\epsilon^2), \quad \text{image}( \frac{f(x+i\epsilon)}{ \epsilon } ) = f'(x) + O(\epsilon^2),
\end{align}
其中$i=\sqrt{-1}$。
与上面的实值情况不同,由于取不同点处的$f$值之间的差分而没有抵消影响。这允许使用很小的$\epsilon$值,比如$\epsilon = 10^{-150}$,这使得误差对于所有实际使用的目标微不足道。

emph{监控激活和梯度的直方图}:
在大量训练迭代后(也许是一轮迭代)搜集神经网络的激活和梯度并进行统计可视化往往是非常有用的。隐藏单元的预激活值能告诉我们该单元是否饱和,或它们多久出现一次。比如,对于镇流器,他们多久关一次?是否存在某些一直关闭着的单元?对于双曲正切(tanh)单元,预激活绝对值的平均值告诉我们该单元的饱和程度。在深度网络中传播的梯度快速增长或快速消失,优化可能受到阻碍。最后,将参数梯度的幅值和参数自身的值进行比较是有用的。建议,参数在一个小批量上的更新幅度希望是参数值的$1\%$,而不是$50\%$或$0.001\%$(这会导致参数移动过慢)。也有可能是一些参数以良好的速度移动,而其它却停滞了。当数据是稀疏的(像自然语言),一些参数可能很少更新,在监控他们变化时应该记住这点。

最后,很多深度学习算法为每步产生的结果提供了一定的担保。比如,在第3节,我们会看到一些基于代数解决优化问题的近似推理算法。这些往往可以通过测试它们每个担保来调试。优化算法提供的担保包括:目标函数在算法的每步从不增长,关于某些变量子集的梯度会在算法每步变成0,以及关于所有变量的梯度会在收敛时变成0。常常,由于舍入误差,这些条件无法在数值计算中完全成立,所以调试测试应该包含一些可容忍误差的参数。

\section{示例:多位数字识别}
\label{sec:11.6}

为了端到端地描述如何应用我们的设计方法到实践中,我们从设计深度学习这个组件的角度,简单地展示下街景转录系统。显然,该系统的其他许多组件,例如街景车,数据库设施等等,也是很重要的。

从机器学习的视角出发,这个过程从数据收集开始。街景车采集了原始数据,人工操作员提供标注。转录任务开始前有大量的数据处理工作,包括使用其他机器学习技术来检测房屋号码。

转录项目开始时需要选择性能度量和相应的预期值。一个重要的总原则是使度量的选择适应该项目的业务目标。因为地图只有当它准确度高时才有用,所以为该项目设置一个高准确度的需求非常重要。具体地,以获得人类$98\%$准确度为目标。这个准确度水平可能并不总能达到。为了达到这个水平的准确度,街景转录系统牺牲了覆盖率。由此为了保持$98\%$的准确度,覆盖率变成了项目中优化的主要性能度量。随着卷积网络的改进,可以降低网络拒绝转录输入的置信度阈值,最终超出了$95\%$覆盖率的目标。

在选择量化目标后,在我们建议的方法中下一步是快速建立一个合理的基线系统。对于视觉任务而言就是一个带修正线性单元的卷积网络。转录系统从就这个模型开始。那个时候,卷积网络输出一个预测序列并不常见。为了从一个尽可能简单的基线开始,模型输出层的第一个实现由n个不同的softmax单元组成,来预测$n$个字符序列。softmax单元的训练和分类任务时的训练完全一样,每个softmax单元被单独训练。

我们推荐的方法是不断地精细化这个基线,对每个改变测试其是否带来改进。街景转录系统的第一个改变的动机来源于对覆盖率度量的理论理解和数据的结构。具体地,当输出序列的概率$p(\Vy\mid\Vx) < t$,即低于某个阈值$t$时,网络拒绝对输入$\Vx$进行分类。$p(\Vy\mid\Vx)$最初被定义为ad-hoc,即将所有softmax输出简单地相乘在一起。这促使开发一个特定输出层和代价函数,用来真正计算出合理的对数似然。这个方法使得样本拒绝机制工作得更有效。

至此,覆盖率依低于$90\%$,该方法显然没有理论问题了。因此我们的方法建议考察训练和测试集的性能,以确定问题是欠拟合或过拟合。在这种情况下,训练和测试的误差几乎是一样的。事实上,该项目进展得如此顺利的主要原因是有数以千万的标注样本可供使用。因为训练和测试误差如此相近,这表明要么欠拟合了,要么训练数据问题。我们建议的一个调试的策略是将模型最差的错误进行可视化。在这种情况下,对训练集上模型给出最高置信度的却是错误的转录进行可视化。结果显示这些大部分是由于输入图片被过度裁剪,导致一些地址的数字被删掉了。举个例子,一幅``1849''的地址图片可能被裁剪得太紧使得只有``849''可见。这个问题可以通过花费几周来提高负责确定裁剪区域的地址号码检测系统的准确性来解决。相反,这个团队采取了一个更实际的决定,即简单地扩大裁剪区域的宽度使其系统地比地址号码检测系统预测更宽。这个单一的改变使转录系统的覆盖率提高了十个百分点。

最后,性能提升的最后几个百分点来自对超参的调整。这大多数包括使模型更大同时保持一些在计算代价上的限制。因为训练和测试误差保持大致相等,所以总是清楚的是,任何性能缺陷是由于欠拟合造成的,还有一些余下的数据集本身的问题。

总之,转录系统是一个成功的项目,使得可以比人工更快速、低成本地转录数亿的地址。

我们希望本章所讲述的设计原则能引导更多其他类似的成功。


\input{chap12_part_1.tex}
%%%%%%%%%%%%%%%%%%%%%%%%%%%%%%%%%%%%%%%%%%%%%%%%%%%%%%%%%
%%%%%%%%%%%%%%%%%% author:chaocraig %%%%%%%%%%%%%%%%%%%%%
%%%%%%%%%%%%%%%%%% part:12.4-12.6   %%%%%%%%%%%%%%%%%%%%%
%%%%%%%%%%%%%%%%%%%%%%%%%%%%%%%%%%%%%%%%%%%%%%%%%%%%%%%%%

\section{自然语言处理}


自然语言处理(NLP)是让计算机使用人类的语言,包括英语或法语等。特殊设计的计算机程序通常读取和发出专门的语言,目的在允许通过简单的程序进行高效和明确的解析。使用的语言越自然,就越模糊与违反正规的描述。自然语言处理包括的应用,如:机器翻译,学习者一定要读入一句人类语言,然后使用另一个人类语言发出相等的句子。许多自然语言处理的应用,是基于自然语言中,定义文字(words)、字元(characters)或位元组(bytes)序列分布的机率分布的语言模型来发展的。


就像在本章中,所讨论的其他应用,非常通用的神经网路技术,可以很成功地应用的自然语言处理中。然而,为了达到优异的效能,以及规模化到大型应用中,一些领域特定的策略变得非常重要。为了建立一个有效率的自然语言模型,我们经常要使用特别为了处理序列资料的技术。在很多案例中,我们选择将自然语言是唯一序列的字,而不是一序列独立的字元或位元组。因为可能的字的总数很大,以字为基础的语言模型,一定要在一个极端高维度且稀疏分散的空间上操作。已经发展了好几种策略来让这样的空间模式,能在计算与统计意义上有效。


\subsection{n-grams}

一个语言的模型定义了一个在自然语言连续符号上的机率分布,根据这个模式如何背设计,一个符号可能是一个字、一个字元、或甚至一个位元组,符号总是离散的实体。最早成功的语言模型,是基于固定长度序列的符号之模型,称为 n-grams,一个 n-grams 是一串连续的 n 个符号。基于 n-grams 的模型,定义了在给定前面 n-1 个符号的第 n 个符号的条件机率。这个模型使用了这些条件分布的乘积,来定义在更长序列上的条件机率。

\begin{align}
\label{eq:necessary}
P(\bm{x}_{1},...\bm{x}_{T}) 
 = P(\bm{x}_{1},...\bm{x}_{n-1}) \prod_{t=n}^{T}P(\bm{x}_{t} | \bm{x}_{t-n+1},...\bm{x}_{t-1}) 
\end{align}


这个分解可以由概率的链式法来加以证明,而初始序列上的概率分布,则可以经由带有较小 n 值的不同模型来建模。


训练 n-gram 模型是很直接的,因为最大似然估计可以简单地计算,在训练集中,每个可能的 n gram 发生几次而得到估计值。基于 n-grams 的模型,已经有好几十年是统计语言模型的核心模块 (Jelinek and Mercer, 1980; Katz, 1987; Chen and Goodman, 1999)。


对于小的 n 值,模型有其特定的名称: 当 n = 1 时,称为一元语法(unigram);当 n = 2 时,称为二元语法(bigram);及当 n = 3 时,称为三元语法(trigram)。这些名称源于相应数字的拉丁前缀,同时,希腊后缀 “-gram" 表示所写的一些东西。通常我们同时训练 n-gram 模型和 n−1 gram 模型,这使得下列式子很容易计算概率:


\begin{align}
\label{eq:necessary}
P(\bm{x}_{t} | \bm{x}_{t-n+1},...\bm{x}_{t}) 
= \frac{P_n(\bm{x}_{t-n+1},...\bm{x}_{t}) }{P_{n-1}(\bm{x}_{t-n+1},...\bm{x}_{t-1}) }
\end{align}


简单地查找两个存好的概率值就能计算,为了在 Pn 中精确地再产生推论,当我们训练 \({P_{n-1}}\) 时,必须省略每个序列最后的字元。

举一个例子,我们演示用三元模型如何计算句子 "THE DOG RAN AWAY." 的概率。因为句子的开头没有上下文,所以句子的第一个词,不能通过上述默认的条件概率公式计算。替代方式是在句子的开头,我们必须使用词的边缘概率。因此,我们计算 \({P_{3}}\)(THE DOG RAN),最后,是一个典型情况,可以使用条件分布 P(AWAY|DOG RAN)()来预测最后一个词。将算法这与公式 (12.6) 放在一起,我们得到:

\begin{align}
\label{eq:necessary}
P(\textrm{THE DOG RAN AWAY}) = {P_{3}}(\textrm{DOG RAN}){P_{3}}(\textmd{RAN AWAY})/{P_{2}}(\textmd{RAN})
\end{align}


对于 n-gram 模型之最大似然的基本限制,是即使元组(tuple) \({X_{t-n+1}}\), ..., \({X_{t}}\) 可能出现在测试集中,在许多情况下从训练集计数估计的 Pn ,在很多案例情形很可能为零,这可能会导致两种不同的灾难性后果。当 \({P_{n-1}}\) 为零时,该比率是未定义的,因此模型甚至不能产生有意义的输出。当 \({P_{n-1}}\) 非零而 Pn 为零时,测试样本的对数似然值为 − \(\infty\) 。为了要避免这种灾难性的后果,大多数 n-gram 模型采用某种形式的平滑(smoothing),平滑技术将概率质量从观察到的元组,移转到类似的未观察到的元组,这些可以进一步参考 Chen and Goodman (1999) 的文献回顾和实证比较。其中一种基本技巧,是向所有可能的下一个符号值加入非零概率质量,这个方法可以被证明,是在具有均匀或狄利克雷(Dirichlet)分布的计数参数上的先验贝叶斯推断。另一个非常流行的想法,是形成包含高阶和低阶 n-gram 模型的混合模型, 其中高阶模型提供更多的容纳量,而低阶模型尽可能地避免计数值为零。如果上下文 \({X_{t-n+k}}\), ..., \({X_{t-1}}\) 的出现频率太小,而不能使用高阶模型,回退方法 (back-off methods) 就查找低阶 n-gram 的值。更正式地来说,它们通过使用上下文 \({X_{t-n+k}}\), ..., \({X_{t-1}}\) 来估计  \({X_{t}}\) 上的分布,并增加 k 直到找到足够可靠的估计值。


经典的 n-gram 模型特别容易引起维度灾难(the curse of dimensionality)。总共会有 \(|\mathbb{V}|^n\) 个可能的 n-grams ,且 \(|\mathbb{V}|\) 经常很大。即使有大量训练数据和适当的 n, 大多数的 n-gram 也不会出现在训练集中。经典的 n-gram 模型的一种观点,是视为执行最近邻查询。换句话说,它可以被视为局部非参数预测器,类似于 k-最近邻(k-nearest neighbors)。这些极端的局部预测器面临的统计问题,描述在 5.11.2 节之中。语言模型的问题,甚至比普通模型更严重,因为任何两个不同的词在独热(one-hot)向量空间中,具有彼此相同的距离。因此,难以大量利用来自任意“邻居”的信息——只有重复相同上下文的训练样本,对局部泛化有用。为了克服这些问题,语言模型必须能够在一个词和其他语义相似的词之间共享知识。

为了提高 n-gram 模型的统计效率,基于类的语言模型(Brown et al., 1992; Ney and Kneser, 1993; Niesler et al., 1998)引入词类别的概念,然后属于同一类别的词与词之间的统计强度可以共享。这个想法使用聚类算法,基于它们与其他词同时出现的频率,将该组词分成集群或类。之后,模型可以在条件棒的右侧使用词类 ID 而不是单个词的 ID,复合模型透过混合(或回退)词模型和类模型也是可能的。尽管词类提供了在序列之间泛化的方式,但是其中一些词被相同类的另一个词替换,使得这样的呈现方式,会丢失了很多信息。



\subsection{神經語言模型}


神经语言模型 (Neural Language Model, NLM) 是一種设计用来克服维度灾难的语言模型,它使用词的分布式表示对自然语言序列建模 (Bengio et al., 2001b)。和基于类 n-gram 模型不同,神经语言模型能夠识别两个词是相似的,而不丧失将每个词编码为互相不同的能力。神经语言模型共享一个词(及其上下文)和其他类似词和上下文之间的统计强度。模型学习到每个词的分布式表示,允许模型視具有共同特征的词是相似的,来实现这种共享。例如,如果词 dog 和词 cat 映射到共享许多属性的表示,则包含词 cat 的句子,其資訊有助於模型对包含词 dog 的句子做出预测,反之亦然。因为有很多这样的属性,所以可以發生许多泛化的方式,将信息从每个训练语句,传送到指数数量的语义相关语句。维度灾难需要模型泛化到句子长度是指数的一些句子。该模型透过将每个训练句子与指数数量的类似句子進行相关联,来克服这个维度灾难的问题。






\part{深度学习研究}
\label{part:3}

\chapter{线性模型}
\label{chap:13}

%%%%%%%%%%%%%%%%%%%%%%%%%%%%%%%%%%%%%%%%%%%%%%%%%%%%%%%%%
%%%%%%%%%%%%%%%%%%%author:shanong%%%%%%%%%%%%%%%%%%%%%%%%
%%%%%%%%%%%%%%%%%%%%%%%%%%%%%%%%%%%%%%%%%%%%%%%%%%%%%%%%%
%%%%%%%%%%%%%%%%%%%%%%%%%%%%%%%%%%%%%%%%%%%%%%%%%%%%%%%%%
%%%%%%%%%%%%%%%%%%% author:hijeffery %%%%%%%%%%%%%%%%%%%%
%%%%%%%%%%%%%%%%%%% part:14.0-14.6   %%%%%%%%%%%%%%%%%%%%
%%%%%%%%%%%%%%%%%%%%%%%%%%%%%%%%%%%%%%%%%%%%%%%%%%%%%%%%%

\chapter{自编码器}
\label{chap:14}

\emph{自编码器}是神经网络的一种,用以训练来实现将输入复制到输出的目的。在其内部,有一个用编码表示输入的隐层$h$。自编码器可以看作由两部分组成:编码函数$h = f(x)$ 和进行信号重建的解码函数$r = g(h)$。 具体结构如图\ref{fig:14.1}所示。如果自编码器学到的结果仅仅是处处将$g(f(x)) = x$,则其没有起到任何作用。相反,自编码器被设计成了不能够完美的复制输入到输出的工作方式。通常他们被限定在只能够近似的复制,并且只复制能够与训练数据相像的输入。鉴于模型被强制执行输入的某些部分应当被复制,所以通常来讲,自编码器能够学习到数据的有用的特性。
\begin{figure}[htbp] %  figure placement: here, top, bottom, or page
   \centering
   \includegraphics[width=1in]{fig/chap14/14_1.png} 
   \caption{自编码器的基本结构,将输入信号$x$通过一个内部表征或者编码$h$ 映射到输出(也叫重建) $r$。 自编码器有两个子部分:编码器(将$x$映射到$h$)与解码器(将$h$映射到$r$)。}
   \label{fig:14.1}
\end{figure}

现代自编码器已经从执行特定的函数映射扩展到了执行随机映射$p_{encoder}(h|x)$ 和 $p_{decoder}(x|h)$。

自编码器的思想已经在神经网络研究领域存在了几十年。传统上来讲,自编码器是用来执行数据降维与特征学习的。近来,自编码器与隐变量模型的联系使得自编码器成为生成模型的研究前沿,详情见本书第\ref{chap:20}章介绍。自编码器可以视为前馈网络的一种特殊形式,并且可以用与其相同的方式进行训练,如以子集沿着反向传播计算的梯度下降方向求解。同一般的前馈网络不同的是,自编码器也可以用\emph{再循环}的方式进行训练,即对比原始输入的网络相应与重建信号作为输入的网络相应的差别。再循环技术被视为比反向传播算法更为接近生物特性的方法,但是其很少被出现在其他机器学习应用中。

\section{不完备自编码器}
\label{sec:14.1}
将输入复制到输出听起来似乎无特别作用,但其实我们也不关心解码器的输出。事实上,我们期望通过训练自编码器实现复制输入的任务能够产生有实用意义的$h$。

一个从自编码器获取有效的特征的方法是将$h$限定到一个比$x$更低的维度上。特征维度比输入维度低的自编码器是\emph{不完备}的。学习一个不完备的特征迫使自编码器捕获到训练数据中的最为突出的特征信息。

学习过程可以简单的表述为最小化一个随时函数的形式:
\begin{equation}
	L(x,g(f(x)))
\end{equation}
其中,损失函数 $L$ 通过如最小化均方误差等限定$g(f(x))$ 与 $x$ 尽量相近。 

当解码器是线性函数,并且$L$ 是均方误差时,不完备自编码器学习到的生成子空间与PCA一致。此时,被用来执行复制任务的自编码器事实上附加学到了训练数据的主子空间。

因此,具有非线性编码函数$f$和非线性解码函数$g$的自编码器可以学习到更强有力的非线性泛化PCA。遗憾的是,如果编码与解码部分被给予太强能力的化,自编码器仍然仅仅完成复制输入到输出的功能而忽略了抽取数据分布等有效信息的能力。理论上说,我们可以设想一个只有一维编码的自编码器,其编码器具有强大的能力将每一个训练数据$x^{(i)}$表示成编码$i$。此外,解码器能够学习并将每一个整型值映射回特定的训练数值上。这种特殊情形在现实中不会出现,但其足够说明,一个被训练用来进行数据复制的自编码器,如果被赋予了过与强大的能力,并不会学习到任何与用的信息。

\section{正则化自编码器}
\label{sec:14.2}
编码维数比输入维数低的自编码器能够学习到数据分布的最为显著的信息。我们也提到,如果编码与解码部分被赋予了太强的能力,会导致自编码器学习不到任何有用的东西。

相似的问题也会发生在编码维数同输入维数相同的自编码器中,以及隐层编码维数比输入数据维数大的\emph{过完备}自编码器中。在这种情况时,甚至线性的编码器与解码器也会产生直接复制输入数据到输出数据而不获取任何数据分布有效信息的问题。

理想来说,我们只要根据待建模分布的复杂度选取编码的维数以及编码器解码器的容量,我们就可以成功的训练任何结构的自编码器。正则化自编码器就提供了这种功能。正则化自编码器不是通过控制编码器解码器的层数与编码长度来达到限定模型容量的目的,而是通过损失函数来促使模型具有复制输入到输出以外的特性的。这些特性包括描述特征的稀疏度,缩小特征的偏导数以及对于噪声和数据确实的鲁棒性等。正则化自编码器可以是非线性的或者过完备的,但是,即便有足够容量可以取巧只学习简单的复制函数,其仍可以学习到数据分布的一些有用的东西。

除了本处列举的几个可以被很自然的理解为正则化自编码器的方法之外,几乎任何有隐变量以及推理过程(从给定输入推算隐变量表示)的生成模型都可以被视为一种具有特殊形式的自编码器。有两种同自编码器有此种高度关联的生成模型如变分自编码器和随机生成网络,他们都是Helmholtz机的子类。这些模型会自然的学习高容量,过完备的编码方式,并且不需要归一化处理才能使这些编码具有实用性。这些编码之所以有效,是因为模型是被训练来近似最大化训练数据的概率分布,而不是去执行复制输入输出的操作。

\subsection{稀疏自编码器}
\label{sec:14.2.1}
稀疏自编码器是在自编码器的基础上,简单的引入稀疏惩罚项$\Omega(h)$加到编码层上,并同重建误差项一起训练:
\begin{equation}
L(x,g(f(x))) + \Omega(h) 
\end{equation}
其中,$g(h)$ 是解码器的输出,并且通常编码器的输出为$h = f(x)$。

稀稀疏自编码器通常被用于其他任务中,如分类。稀疏泛化的自编码器必须能够对其针对的训练数据的特定统计特性做出响应,而不能是简单的复制操作。此时,训练执行复制任务,并且有一定的稀疏惩罚项能够得到具有附加特征表示能力的模型。

我们可以把惩罚项$\Omega(h)$ 简单的看作添加到前馈网络上的正则项,该网络主要是实现复制输入到输出的任务(非监督学习目标函数),或者根据这些稀疏特征执行一些有监督的任务(监督学习目标函数)。
同其他正则项如权值衰减等不同,改该正则项并没有直观的贝叶斯解释。如\ref{sec:5.6.1}中所描述,包含权值衰减和其他正则化惩罚项的训练过程可以被视为贝叶斯推断的最大后验近似,其中加入的正则惩罚项可以视为参数模型的先验概率。在这个角度,最大化的最大似然同最大化$p(\theta|x)$相关,有等效于最大化$log p(x|\theta) + log p(\theta)$。$log p(x|\theta)$是常规的数据对数似然项,对数先验项$log p(\theta)$包含对于$\theta$的某些特定值的偏好。这部分内容在第\ref{sec:5.6}节中有介绍。正则化自编码器则不赞同这种解释,因为正则项本身依赖于数据,因此语义上来讲,其定义并不是严格的先验。

与其将稀疏惩罚项视为复制任务的正则项,我们不妨将整个稀疏自编码框架看过是有隐变量的生成模型最大似然训练。假设我们有一个模型,其观测变量为$\bm{x}$隐变量为$\bm{h}$,并且其联合概率分布为$p_{model}(\bm{x},\bm{h}) = p_{model}(\bm{x}|\bm{h})$。我们定义$p_{model}(\bm{h})$为模型对于隐变量的先验概率,表示模型在观察变量$\bm{x}$前的置信度。同我们之前使用“先验”这个词有些不同,这里是指,在我们观察到训练数据之前,$p(\bm{\theta})$ 代表了我们对于模型参数的置信度。其对数似然可以分解为
\begin{equation}
	log p_{model}(\bm{x}) = log \sum_{\bm{h}}p_{model}(\bm{h},\bm{x})
\end{equation}
自编码器可以视为用$\bm{h}$的一个最相似的点对上述求和的近似。这同稀疏表示生成模型(\ref{sec:13.4}节)很像,但$\bm{h}$是参数化编码器的输出而不是推测最相似$\bm{sh}$的优化结果。从这个角度,给定$\bm{h}$后,须最大化
\begin{equation}
	log p_{model}(\bm{h},\bm{x}) = log p_{model}(\bm{h}) + log p_{model}(\bm{x}|\bm{h})
\end{equation}
其中,$log p_{model}(\bm{h})$ 可以是稀疏诱导项。比如,拉普拉斯先验,
\begin{equation}
	p_{model}(h_i) = \frac{\lambda}{2}e^{- \lambda |h_i|},
\end{equation}
对应着一个绝对值的稀疏惩罚项。将对数先验表示为绝对值惩罚项,则
\begin{equation}
	\Omega(\bm{h}) = \lambda \sum_i | h_i |
\end{equation}
\begin{equation}
	-log p_{model}(\bm(h)) = \sum_i (\lambda|h_i| - log \frac{\lambda}{2}) = \Omega(\bm{h}) + const
\end{equation}
其中,常数项只与$\lambda$有关,与$\bm{h}$无关。我们通常将$\lambda$视为超参数并忽略常数项,因为其对于参数学习没有影响。其他的先验函数如student-t分布,同样可以引入稀疏性。因为稀疏性是在用$p_{model}(\bm{h})$近似最大似然估计学习的过程中引入的,因此,从这个角度来看,稀疏惩罚项根本不能算作正则项。它仅仅是一系列模型隐变量分布的结果。这个思路给自编码器训练了一个不同的动机:其是训练生成模型的一个近似方法。同时,也给为什么自编码器学习到的特征有用提供了新的原因:他们描述了能够解释输入的隐变量。

稀疏自编码器的早期工作介绍了多种形式的稀疏,并提出了稀疏惩罚项同$log Z$之间的关系,$log Z$是在用最大似然估计无向图模型$p(x) = \frac{1}{Z}\hat{p}(x)$时产生的。其基本思路是,最小化$log Z$能够防止图模型在各处产生很高的概率,而对自编码器引入稀疏性能够防止在任意地方产生极低的重建误差。这种情况下,二者的联系存在于一个比较直观的普通机制理解层面上而不是仅仅有数学联系。在有向图模型$p_{model}(\bm{h}p_{model}(\bm{x}|\bm{h})$中,与稀疏项有关的$log p_{model}(\bm{h})$则具有更好的数学直观解释。

Glorot等人给出了一个实现稀疏(降噪)自编码器隐层编码$\bm{h}$中\emph{绝对零值}的方法。其思路是用ReLU来生成编码层。有了能够使描述逼近零值的先验(比如绝对值惩罚项),我们可以用间接的手段来控制特征表示中的零值的平均个数。

\subsection{降噪自编码器}
除了给损失函数添加惩罚项$\Omega$外,我们还可以通过改变损失函数项来实现训练自编码器学习有效信息的方法。

传统的自编码器的损失函数为
\begin{equation}
L(x,g(f(x)))
\end{equation}
其中,$L$是限定$g(f(x))$同 $x$相似度的损失函数,如对比二者的$L^2$ 范数损失。这使得$g\circ f$ 如果有能力的化就会学习到一个复制函数。

而\emph{降噪自编码器}则是最小化
\begin{equation}
L(x,g(f(\hat{x})))
\end{equation}
其中,$\hat{x}$是$x$ 被某种噪声干扰后的数据。降噪自编码器必须设法处理这种干扰而不是仅仅复制输入数据。

降噪训练的过程迫使$f$和$g$去学习潜在的$p_{data}(x)$的结构。降噪自编码器再次印证了在最小化重建误差的过程中,会有很多有用的性质作为副产品出现。
他们也可以作为过完备、高容量模型可以用来作为自编码器的例子,只要这些模型能够加入防止其学习到恒等函数的设计即可。      
降噪自编码器在第\ref{sec:14.5}节有更为详细的介绍。

\subsection{梯度惩罚正则化}
另一个正则化自编码器的方法是通过借用稀疏自编码器中的惩罚项$\Omega$, 
\begin{equation}
L(x,g(f(x))) + \Omega(h,x),
\end{equation}
但是,此处用了一个不同形式的$\Omega$
\begin{equation}
\Omega(h,x) = \lambda \sum_i \| \nabla_x h_i\|^2.
\end{equation}
这使得模型能够学到一个当$x$发生轻微变化的时候不会改变太多的函数。由于这个惩罚项仅仅作用于训练数据,其迫使自编码器能够捕获到训练数据的分布信息。

此种方式泛化的自编码器被成为\emph{收缩自编码器}。这种方法同降噪自编码器,流行学习以及概率模型具有很强的理论联系。更多的细节参看\ref{sec:14.7}节。

\section{表征能力,层级及深度}
\label{sec:14.3}
自编码器的训练通常只包含一层编码器和一层解码器。然而,这不是绝对的。事实上,用多层编码器和解码器能够提供非常多的好处。

我们可以回想第\ref{sec:6.4.1}节所述,用多层网络对于前馈型神经网络有很多的好处。因为自编码器是前馈型神经网络,这些优势也同样适用。此外,自编码器的编码器和解码器本身也都是前馈网络,所以这两部分都可以单独从网络深度获益。

网络深度的不平凡之处在于通用近似理论确保了对于有至少一层隐藏层的前馈神经网络,只要其具有足够的隐藏节点,就能够以任意精度近似任何函数(在一个很广的分类限定内)。这意味着具有一个隐藏层的自编码器就可以在数据域内足够好的近似恒等函数。但是,从输入到特征编码的映射是浅层的。这意味着我们无法满足随意的限定,如编码必须是稀疏的等。一个深度的自编码器,在编码器内部有至少一层隐藏层的情况下,如果有足够的隐藏节点,就可以任意好的近似从输入到编码的任意映射函数。

网络深度可以以指数级的优势降低某些方程的计算复杂度。网络深度也可以以指数级降低训练某些函数所需要的训练样本的数量。请看参看\ref{sec:6.4.1}获取更多的网络深度对于前馈型网络的优势的信息。

实验效果上来说,深度自编码器比传统的浅层或者线性自编码器具有更好的特征压缩性能。

训练深度自编码网络的一个常见手法是逐层训练一个堆叠的浅层自编码器,所以,即便我们的最终目标是训练一个深度自编码器,我们其实常常还是会遇到浅层自编码器。

\section{随机编码与解码}
\label{sec:14.4}
自编码器就是前馈型神经网络的一种。传统前馈神经网络的损失函数和输出单元都可以用到自编码器中。

如第\ref{sec:6.2.2.4}节所描述,设计前馈神经网络输出单元与损失函数的一个常用的策略就是定义一个输出分布$p(y| x)$ 并且最小化其负指数似然函数$-log p(y|x)$。在这种情况下,$y$是目标向量,如分类的类别标记。

在自编码器的情况下,$\bm{x}$既是输入也是输出。但是,我们仍然可以使用以前的学习机制。给定一个隐层编码$\bm{h}$,我们可以把解码器视为条件概率分布$p_{decoder}(\bm{x}|\bm{h})$。我们可以通过最小化$-log p_{decoder}(\bm{x}|\bm{h})$来训练自编码器损失函数的具体形式会根据$p_{decoder}$而发生改变。同传统前馈神经网络一样,如果$\bm{x}$是实数值的话,我们通常用线性输出单元来参数化高斯分布的均值。此时,付对数似然等效于最小均方误差。相似的,二进制$\bm{x}$值对应与伯努利分布,其参数由sigmoid函数给出;离散$\bm{x}$值对应于softmax分布等等。一般来讲,输出被视为独立于隐层单元$\bm{h}$,如此,概率分布的计算相对容易求解。但是有些方法如混合密度输出允许借助协方差来连续建模。

为了比前面介绍的前馈神经网络有一个更根本的转变,我们可以把\emph{编码函数}$f(\bm{x})$表示为\emph{编码分布}$p_{encoder}(\bm{h}|\bm{x})$,如图\ref{fig:14.2}所示。
\begin{figure}[htbp] %  figure placement: here, top, bottom, or page
   \centering
   \includegraphics[width=2in]{fig/chap14/14_2.png} 
   \caption{随机自编码器结构图。其中编码器与解码器都不是简单的函数而是包含了一些噪声,这意味着他们的输出可以看作是从要给概率分布中抽样获得的,编码的分布为$p_{encoder(\bm{h}|\bm{x})}$,解码的分布为$p_{decoder}(\bm{x}|\bm{h})$。}
   \label{fig:14.2}
\end{figure}

任何隐变量模型$p_{model}(h,x)$都可以用来定义随机编码器
\begin{equation}
	p_{encoder}(h|x) = p_{model}(h|x)
\end{equation}
和随机解码器
\begin{equation}
	p_{decoder}(x|h) = p_{model}(x|h)
\end{equation}
通常来讲,编码器和解码器并不一定是同联合概率分布$p_{model}(x,h)$吻合的条件概率分布。Alain指出作为降噪自编码训练编码器和解码器能够使他们渐进兼容(在有足够容量和样本的情况下)。

\section{降噪自编码器}
\label{sec:14.5}
降噪自编码器是在给定的输入包含噪声的情况下训练以预测其未被干扰的输入的自编码器。

降噪自编码器额训练过程在图\ref{fig:14.3}中给出。我们介绍加噪处理$C(\hat{x}|x)$,其代表在给出训练样本$x$时的$\hat{x}$的条件概率分布。自编码器从训练样本对$(x,\hat{x})$中学习到\emph{重建分布}$p_{reconstruction}(x|\hat{x})$,过程如下:

1. 从训练样本中抽取一个样本$x$。

2. 从分布$C(\hat{x}|x)$中抽取带有噪声的样本$\hat{x}$。

3. 将$(x,\hat{x})$作为训练样本去预测自编码器的重建分布$p_{reconstruction}(x|\hat{x}) = p_{decoder}(x|h)$,其中,$\bm{h}$是编码器$f(\hat{x})$的输出,$p_{decoder}$一般是由解码器$g(\bm{h})$给出。

一般来讲,我们可以对负对数似然$-log p_{decoder}(x|h)$做基于梯度的近似最小化(如小批次梯度下降)。只要编码器是确定性的,降噪自编码器就是前馈神经网络,并且可以用其他前馈网络的训练方法进行训练。
\begin{figure}[htbp] %  figure placement: here, top, bottom, or page
   \centering
   \includegraphics[width=2in]{fig/chap14/14_3.png} 
   \caption{降噪自编码器的损失计算图,即训练其使得能够从带噪声的样本$\hat{x}$中恢复出干净的样本$x$来。训练是通过最小化损失$L = -log p_{decoder}(x|h = f(\hat{x}))$得到的,其中$\hat{x}$是从加噪函数$C(\hat{x}|x)$得到的含有噪声版本的$x$。通常$p_{decoder}$是一个因子分布,其参数均值由前向网络$g$给出。}
   \label{fig:14.3}
\end{figure}

我们可以把降噪自编码器看作是针对下面期望的随机梯度下降:
\begin{equation}
 -\mathbb{E}_{x\sim \hat{p}_{data}(x)}\mathbb{E}_{\hat{x} \sim C(\hat{x}|x)}log p_{decoder}(x| h = f(\hat{x}))
\end{equation}
其中,$\hat{p}_{data}(x)$是训练样本分布。

\subsection{估算评分}
\label{sec:14.5.1}
评分估计是最大似然估计的一种替代方法。它能够通过促使模型在每一个样本点上同数据分布有一样的评分来给出概率分布的一个一致的估计值。此处,这个评分是一个特别的梯度域:
\begin{equation}
	\nabla_x log p(\bm{x})
\end{equation}
评分匹配在第\ref{sec:18.4}节中有更为细致的讲解。就当前对自编码器的讨论来讲,了解学习梯度域$log p_{data}$ 是学习$p_{data}$结构的一种方法就足够了。

降噪自编码的一个非常重要的性质就是他们的训练准则(用条件高斯$p(\bm{x}|\bm{h})$)使得自编码器学习到了一个计算数据分布评分的向量域$(g(f(x)) - x)$。请参考图\ref{fig:14.4}。
\begin{figure}[htbp] %  figure placement: here, top, bottom, or page
   \centering
   \includegraphics[width=4.5in]{fig/chap14/14_4.png} 
   \caption{降噪自编码器被训练用以将含噪数据$\hat{\bm{x}}$映射回原始数据$\bm{x}$。原始数据用红色X号表示,其处于黑色实线表示的一维流面上。加噪函数$C(\bm{\hat{x}},\bm{x})$用灰色圆圈表示其各向同性。灰色箭头表示了样本是如和从流面被引入噪声映射到别处的。当降噪自编码器被训练来降低均方误差$\|g(f(\hat{x})) -x \|^2$时,重建函数$g(f(\hat{x}))$计算了期望$\mathbb{E}_{x,\hat{x} \sim p_{data} C(\bm{\hat{x}}, \bm{x})}[\bm{x}|\bm{\hat{x}}]$。向量$g(f(\hat{\bm{x}})) - \hat{\bm{x}}$指向了流面上最近的点,这是因为$g(f(\hat{\bm{x}}))$估算的可能产生噪点$\bm{\hat{x}}$的纯净数据$\bm{x}$的重心。因此自编码器学习到了一个向量空间$g(f(\hat{\bm{x}})) - \bm{x}$,图中绿线所示。这个向量域用一系列乘数因子计算评分$\nabla x \log p_{data}(\bm{x})$,这些因子是重建均方误差的根值平均数。}
   \label{fig:14.4}
\end{figure}

用高斯噪声和均方差作为重建损失来训练一个特定的降噪自编码器(隐层为sigmoid单元,重建为线性单元)等效于训练一种特定的具有高斯可观察单元的称为RBM的无向图模型。这种模型将会在第\ref{SEC:20.5.1}节进行详细介绍;此处仅需知道他是一种能够提供明确函数$p_{model}(\bm{x};\bm{\theta})$的模型即可。当RBM被用\emph{降噪评分匹配}来进行训练的时候,其学习算法等同于和它对应的自编码器的降噪训练。给定特定的噪声值,泛化评分匹配并不是一个一致的估算;它其实恢复了一个模糊化的分布。但是,如果噪声水平接近于零,并且训练样本趋近于无穷,此时一致性又得到了恢复。降噪评分匹配在第\ref{sec:18.5}节有详细讨论。

RBM同自编码器还有其他的联系。评分匹配应用到RBM上产生的损失函数等同于重建误差加上一个同收缩自编码中的收缩项相似的正则化项。Bengio等人指出自编码器的导数为RBM的contrastive divergance计算提供了近似。

对于连续值$\bm{x}$,包含高斯噪声和重建分布的降噪标准为常规编码器和解码器的参数化评分计算提供了依据。这意味着生成编码-解码结构通过训练后可以被用来计算评分,训练用的均方误差标准为
\begin{equation}
	\|g(f(\bm{\hat{x}})) - \bm{x}\|^2
\end{equation}
以及噪声
\begin{equation}
	C(\hat{x} = \bm{\hat{x}} | \bm{x}) = \mathcal{N}(\bm{\hat{x}}; \mu = \bm{x}, \Sigma = \sigma^2I)
\end{equation}
其中,$\sigma^2$为噪声方差。关于具体如何运作,请参考图\ref{fig:14.5}。
\begin{figure}[htbp] %  figure placement: here, top, bottom, or page
   \centering
   \includegraphics[width=4.5in]{fig/chap14/14_5.png} 
   \caption{降噪自编码器学习到的二维空间中的一维曲面流面附近的向量域。每一个箭头都与重建向量减去输入向量的值成一定比例并且箭头指向预测到的概率分布的较高概率值。向量域在估算的密度函数的极大值(在数据流面上)与极小值处取值为零。例如,螺旋的臂构成了相互连接的有局部极大值的一维流面。局部极小值在两个臂之间的区域的中间位置。当重建误差的模(以箭头的长度表示)较大的时候,意味着沿着箭头的方向移动的话概率值可以得到很大的提升,反之亦然。自编码器将这些低概率的点映射为高概率的重建值。在概率取得极大值的地方,箭头缩短,因为此时的重建值更加精确。}
   \label{fig:14.5}
\end{figure}

通常来讲,并没有任何保证说重建$g(f(x))$ 减去输入$\bm{x}$对应于任何方程的导数,更别提评分了。如此可以解释为什么早期的结果只能应对特定的参数,即$g(f(x)) - x$ 仅可以通过另一个方程的导数来获得。Kamyshanska通过定义一族浅层自编码器使$g(f(x)) - x$对族内所有成员都有对应的评分,如此进一步优化了Vincent的结果。

至此我们仅讨论了降噪自编码器是如何学习表示一个概率分布的。更一般来讲,我们可能会想用自编码器作为生成模型并从其分布中抽取样本。这些将在后面第\ref{sec:20.11}节讨论。

\subsubsection{历史回顾}
\label{sec:14.5.1.1}

\section{用自编码器进行流行学习}
\label{sec:14.6}

同其他很多机器学习方法一样,自编码器同样假设数据集中在一个低维流面或者一个这种流面的小集合上,详见\ref{sec:5.11.3}节。一些机器学习方法效果有限,尽管他们可能可以学习到一个在流面上正常的函数,但是对于不在流面上的输入,可能会产生不正常的表现。
自编码器进一步拓展了这种思想,并设法去学习流行的结构。

为了理解为什么自编码器能够有这样的效果,我们需要介绍流行的几个重要的特性。

其中一个很重要的特性是其\emph{正切平面}集合。对于一个d-维流面上的点$x$, 其切面是由能够张成流面上局部方向变化的d-维向量给出。如图\ref{fig:14.6}所示,这些局部方向指出了我们能如何对$x$在流面上进行无限小的位置改变。
\begin{figure}[htbp] %  figure placement: here, top, bottom, or page
   \centering
   \includegraphics[width=4in]{fig/chap14/14_6.jpg} 
   \caption{超切平面概念示意图。这里我们给出一个784-维空间内的一维流面。我们选取MNIST数据集中的一幅784像素的图像,并沿着竖直方向移动它。这样的一组竖直移动数据定义了沿一维流面方向的坐标,并在图像空间中划出一条曲线。图中给出了在这个流面上的几个点。为了可视效果,我们用PCA把流面投影到了二维空间中。一个n维流面在任一点都有一个n维切面。切面在流面上的该点处完美贴合,并且与该点的表面平行。该切面定义了使得数据保持在流面上的可以移动的方向空间。图中的一维流面有一个单独的切线。我们在图中给出了一个切线示例,以及在图像空间中沿切线移动时所产生的变化。灰色的像素是在移动过程这个不发生变化的点,白色代表颜色变亮,黑色代表颜色加深。}
   \label{fig:14.6}
\end{figure}

所有的自编码器的训练过程是两个限定因素的折衷:

1. 学习一个训练样本$x$的特征表示$h$,使得$x$能够通过解码器从$h$中大致恢复出来。$x$是从训练样本抽取的这个事实使得问题有点儿麻烦,因为这意味着自编码器没有必要对与不在数据生成分布下的输入进行很好的重建。

2. 满足限定或者泛化惩罚。这个可以是结构上限定自编码器的容量的设定,或者是添加到重建误差后的一项泛化项。这些技术通常倾向于对于输入数据不敏感的解决方案。

显然,不管是从输入到输出的复制本身,或者是直接忽略输入,二者任何之一单独出现都没有太大作用。相反,二者同时出现,能够保证方法有效,因为他们迫使隐藏特征表示去捕获能够描述数据生成分布的内在结构。重要的一点就是自编码器可以只去表示重建训练数据所必须的那些变化。如果生成数据的分布本身集中在一个低维流面上,这诱使特征描述仅仅表现这个流面上的一个局部的坐标系统:即只有在$x$附近的与流面相切的变化可以联系到$h = f(x)$的改变上。所以,自编码器学到的从输入空间$x$到表示空间的映射,是一个仅仅对于在流面方向上的变化敏感但对于垂直于流面方向上的变化不敏感的映射。

图\ref{fig:14.7} 给出了一个一维的例子,该图表明通过限定重建函数对于数据点周围扰动的敏感性,我们的自编码器可以重建流面结构。
\begin{figure}[htbp] %  figure placement: here, top, bottom, or page
   \centering
   \includegraphics[width=4in]{fig/chap14/14_7.png} 
   \caption{如果自编码器学习到了一个能够对在数据点周围的扰动具有不变性的重建方程,则其学习到了数据本身的流行结构。图中给出的是0-维流面的流行结构。虚线表示用以重建的恒等函数。最有重建函数在有数据的地方同恒等函数重合。底部的箭头表示重建误差向量$r(x)-x$,在输入空间,这些箭头总是指向其最近的“流面”(在一维的情况下,是一个孤立点)。降噪自编码器直接设法使重建函数$r(x)$的偏导数在数据点周围尽量小。收缩自编码器对于编码器进行了同样的操作。尽管$r(x)$的导数在数据点周围尽量小,在不同数据点之间,它可以变得很大。数据点之间的空间同流面之间的区域直接相关,在这些地方,重建函数的导数必须很大,才能将被破坏的点重新映射回流面上。}
   \label{fig:14.7}
\end{figure}

为了更好的理解为什么自编码器可以进行流行学习,我们有必要将其与其他方法进行比较 。对于流行的特性最常用的研究方法是点在流面上或者流面附近的\emph{特征表示}。对于某个样本的特征表示也被称作其嵌入。特征表示通常用一个低维的向量来表示,即比作为一个低维子集存在的流面的原始外部空间的维数要低。一些算法直接对每一个训练样本学习一个嵌入表示(如下面导论到的非参数流行学习方法),与此同时,其他的一些方法则设法学习到一个更为通用的映射,该映射将周围空间(输入空间)映射到嵌入空间。有时这种映射被叫做自编码器或者表示方程式。 

流行学习通常关注于非监督学习的方法来得到这些流面。早期的大部分学习流面的研究集中在基于最近邻图模型的非参数方法上。在图模型中,每个样本表示为一个节点,并通过边将最近邻的样本连接起来。如图\ref{fig:14.8}所示,这些方法将每个节点表示为切平面,这些切平面描述了样本同其邻域样本的变化方向。
\begin{figure}[htbp] %  figure placement: here, top, bottom, or page
   \centering
   \includegraphics[width=4in]{fig/chap14/14_8.png} 
   \caption{非参数流行学习方法构建了一个最近邻图模型,其中,节点表示训练样本,有向边表明最近邻关系。如此,通过各种不同的方法可以得到切平面以及与其相对应的邻域图模型,此外,还有一个将每一个样本同一个实值向量或嵌入对应的坐标系统。如此我们可以通过一定的改动使特征表示适应新的样本。只要样本数量足够大能够覆盖流面的弯曲与褶皱,这些方法都可以很好的工作。图片来源为QMUL多视人脸数据库。}
   \label{fig:14.8}
\end{figure}

我们之后可通过优化算法或者求解线性系统来获取一个全局坐标系统。图\ref{fig:14.9}给出了一个流面视如何平铺成大量的局部线性类高斯碎片的(或者叫做圆饼,因为高斯函数在切线方向是平坦的)。
\begin{figure}[htbp] %  figure placement: here, top, bottom, or page
   \centering
   \includegraphics[width=4in]{fig/chap14/14_9.png} 
   \caption{如果切平面(参考图\ref{fig:14.6})在处处都是已知的,则他们可以平铺成一个全局的坐标系统或者密度方程。每一个局部分块都可以被看成是一个局部的欧氏坐标系统或者局部扁平高斯或者圆盘,他们在垂直于圆盘的方向变化很小,在定义圆盘坐标系统的方向上变化很大。这样的一些高斯函数组合成了在流面高斯窗算法或者其基于非局部神经网络的变种中的一个预测密度函数。}
   \label{fig:14.9}
\end{figure}

但是,用这些局部非参数的方法在进行流行学习的时候有一个基本的难题:如果流面本身非常不平整(有许多峰值、低谷、褶皱),我们可能需要大量的样本来覆盖这些变量,从而降低了对未知样本的泛化能力。事实上,这些方法仅仅能够通过邻域信息的插值来泛化流面的形状。不幸的是,在人工智能问题中涉及到的流行往往具有非常复杂的结构,很难从局部插值中获取所有信息。比如说图\ref{fig:14.6}中的通过平移变换产生的流面,如果我们仅仅观察输入向量的一个坐标$x_i$的话,当图像移动时,我们可以看到,每当图像在遇到灰度的峰值或者低谷的时候,我们的坐标就会遇到一个峰值或者低谷。换言之,待处理图像的灰度明暗的复杂度决定了对于像进行简单的平移操作产生的流面的复杂度。这些促使了通过分布式描述和深度学习进行流行结构学习的需求。
%%%%%%%%%%%%%%%%%%%%%%%%%%%%%%%%%%%%%%%%%%%%%%%%%%%%%%%%%
%%%%%%%%%%%%%%%%%%%author:Euniceu%%%%%%%%%%%%%%%%%%%%%%%%
%%%%%%%%%%%%%%%%%%%part:14.7-14.9%%%%%%%%%%%%%%%%%%%%%%%%
%%%%%%%%%%%%%%%%%%%%%%%%%%%%%%%%%%%%%%%%%%%%%%%%%%%%%%%%%

\section{14.7}

%%%%%%%%%%%%%%%%%%%%%%%%%%%%%%%%%%%%%%%%%%%%%%%%%%%%%%%%%
%%%%%%%%%%%%%%%%%%%  author:tangzhenyu %%%%%%%%%%%%%%%%%%
%%%%%%%%%%%%%%%%%%%  part:15.0-15.3    %%%%%%%%%%%%%%%%%%
%%%%%%%%%%%%%%%%%%%%%%%%%%%%%%%%%%%%%%%%%%%%%%%%%%%%%%%%%

\chapter{表征学习}
\label{chap:15}
\section{15.1}
\label{sec:15.1}
%%%%%%%%%%%%%%%%%%%%%%%%%%%%%%%%%%%%%%%%%%%%%%%%%%%%%%%%%
%%%%%%%%%%%%%%%%%  author:chongruo  %%%%%%%%%%%%%%%%%%%%%
%%%%%%%%%%%%%%%%%  part:15.4-15.6   %%%%%%%%%%%%%%%%%%%%%
%%%%%%%%%%%%%%%%%%%%%%%%%%%%%%%%%%%%%%%%%%%%%%%%%%%%%%%%%

\section{15.4}

\chapter{深度学习的结构化概率模型}
\label{chap:16}
%%%%%%%%%%%%%%%%%%%%%%%%%%%%%%%%%%%%%%%%%%%%%%%%%%%%%%%%%
%%%%%%%%%%%%%%% author:YisenWang  %%%%%%%%%%%%%%%%%%%%%%%%%
%%%%%%%%%%%%%%% part: 16.0-16.2.5 %%%%%%%%%%%%%%%%%%%%%%%%%%%
%%%%%%%%%%%%%%%%%%%%%%%%%%%%%%%%%%%%%%%%%%%%%%%%%%%%%%%%%
深度学习借鉴了很多模型形式方面的内容来帮助研究者指导他们的设计理论和算法描述。这其中之一就是有结构的概率模型。我们在前面3.14节的时候,简单的讨论过一些有结构的概率模型。那些简短的介绍已经足够用来理解怎么使用有结构的概率模型来描述第二部分的一些算法。在第三部分,有结构的概率模型是深度学习里很多重要的研究方向的关键成分。为了后续讨论这些研究点,在这一章,我们会详细介绍有结构的概率模型。不过,读者不用担心,这一章是自我完备的,在开始学习这一章之前,读者不需要复习之前的介绍。

有结构的概率模型是描述概率分布的一种方式,它直接通过图的形式来描述概率分布中随机变量之间的关系。这里,我们沿用了图理论里面图的概念,即,节点之间通过边来连接。因为模型的结构是由图来定义的,因此,这些模型也经常被称作图模型。

图模型的研究领域很大,也已经发展了很多不同的模型、训练算法和推断算法。在这一章,我们主要讲一些图模型里最核心的思想,并把重点放在那些对深度学习的领域很有用的一些概念上。如果你已经有很强的图模型背景,你可以跳过本章的大部分内容。但是,即使是一个图模型方面的专家,他也可能会从这一章的最后一部分(16.7节)收益,因为,在16.7节,我们会高亮一些特有的图模型可以用在深度学习里的方法。相比于图模型的研究者,深度学习的参与者更倾向于用非常不同的模型结构、学习算法和推断过程。在这一章,我们指出了他们在偏好上的不同,并解释其中的原因。

在这一章,我们首先描述了建立大规模概率模型的挑战。接着,我们描述了怎么用图来描述一个概率分布的结构。虽然这种方法允许我们克服许多挑战,但它不是没有自己的复杂性。在图模型里,一个最主要的困难是理解在一个图里,哪些变量之间需要直接相连,也就是,对于一个给定的问题,哪种图结构是最合适的。我们在16.5节简介了两种方式来解决这个问题。最后,我们通过讨论深度学习与图模型之间的关系来结束本章。

\section{无结构模型的挑战}
深度学习的目标是延伸机器学习到解决人工智能所面临的各种挑战,这意味着深度学习能够理解具体很丰富结构的高维数据。比如说,我们希望人工智能算法能够理解自然图像,声波表示的语音,以及包含多个单词和标点符号的文档。

分类算法能够从很高维的分布中取出一个输入并用一个类别标签总结他,比如照片里面是个什么物体,语音说的是哪个词,文档是关于哪个话题的。分类的过程忽略了输入里的大部分信息,只产生了一个输出(或那个单一输出值的概率分布)。一个分类器也经常忽略输入的很多部分。比如,当识别图片中一个物体,经常会忽略图片中的背景。

让概率模型做很多其他的任务也是可能的。这些任务通常比分类更昂贵,比如,其中一些可能需要输出多个值,而且大部分要求一个对输入有一个完整的理解,不能忽略其中的任何一部分。这些任务包括:
\begin{itemize}
\item 密度估计:给定一个输入$x$, 机器学习系统返回一个在数据生成分布下的真实密度$p(x)$的估计。这虽然只要求一个输出,但它需要对整个输入有一个完整的理解。即使是输入向量中只有一个元素是不寻常的,该系统也必须给它分配一个很低的概率。
\item 去噪:给定一个损坏的或不正确的观测输入$\hat{x}$,机器学习系统返回一个原始或正确$x$的估计。比如,机器学习系统可能被要求从旧照片去除灰尘或划痕。这需要多个输出(估计的干净样本的每个元素)和对整个输入的理解(因为一个损坏的区域将仍然导致最终的估计是损坏的)。
\item 缺失值的插补:给定$x$的一些元素的观察,要求模型返回在一些或所有未观察到的$x$元素的估计或概率分布。这需要多个输出。因为模型可以被要求恢复$x$的任何元素,它必须理解整个输入
\item 采样:该模型从分布$p(x)$生成新样本。应用包括语音合成,即产生像自然人类语音一样的新波形。这需要多个输出值和整个输入的模型。如果样本甚至有一个元素来自错误的分布,那么抽样过程是错误的。
\end{itemize}

比如一个用很多小的自然图像采样的一个例子,如图16.1所示。

在数千或数百万的随机变量上建模丰富的分布是计算和统计学上的挑战性任务。假设我们只想模拟二进制变量。这是最简单的可能情况,但已经是压倒性的。对于小的$32×32$像素的彩色(RGB)图像,存在$2^3072$个这种形式的可能的二进制图像。这个数字比宇宙中估计的原子数多$10^800$多倍。

一般来说,如果我们希望对一个包含$n$个离散变量且每个变量能取$k$个值的随机向量$x$建模分布,则通过存储具有每个可能结果的一个概率值的查找表来表示$P(x)$的朴素方法需要$k^n$参数!

由以下几个原因导致这样做不可行:
\begin{itemize}
\item 内存:存储表示的成本:对于除了非常小的$n$和$k$之外的所有值,作为表示分布的表将需要太多的值来存储。
\item 统计效率:随着模型中参数数量的增加,使用统计估计器来选择参数值所需的训练数据量也增加。因为基于表的模型具有天文数量的参数,所以需要天文大的训练集来精确拟合。任何这样的模型将在训练集上有非常严重地过拟合,除非作出连接表中差异的附加假设(例如,像在back-off或平滑的n-gram模型中,第12.4.1节)
\item 运行时间:推断成本:假设我们要执行一个推断任务,我们使用联合分布$P(x)$来计算一些其他分布,例如边际分布$P(x_1)$或条件分布$P(X_2|x_1)$。计算这些分布将需要对整个表进行求和,因此这些操作的运行时间与存储模型的不可计算存储成本一样高。
\item 运行时:抽样成本:同样,假设我们要从模型中抽取一个样本。 这样做的天真的方法是抽样一些值$u〜U(0,1$,然后通过表迭代,添加概率值,直到它们超过并返回对应于表中该位置的结果。这需要读取整个表 在最坏的情况下,所以它具有与其他操作相同的指数成本。
\end{itemize}

基于表的方法的问题是我们明确地建模每个可能的变量子集之间可能的交互。我们在实际任务中遇到的概率分布比这更简单。通常,大多数变量仅间接地相互影响。

例如,考虑建模一个团队在接力赛中的完成时间。假设团队由三个跑步者组成:爱丽丝,鲍勃和卡罗尔。在比赛开始的时候,爱丽丝拿着一根接力棒,开始在赛道上跑。在完成她的一圈之后,她将接力棒交给鲍勃。鲍勃然后跑他自己的那一圈,并将指挥棒交给卡罗尔,卡罗尔跑最后一圈。我们可以将他们中的每一个人的完成时间建模为一个连续随机变量。爱丽丝的完成时间不依赖于任何人,因为她走在第一。鲍勃的完成时间取决于爱丽丝的,因为鲍勃没有机会开始跑他的那一圈,直到爱丽丝完成她的那一圈。如果爱丽丝更快,鲍勃会更快地完成,反之亦然。最后,卡罗尔的完成时间取决于她的队友。如果爱丽丝很慢,鲍勃也可能会结束的晚。因此,卡罗尔会有相当的晚的启动时间,因此很可能有一个晚的完成时间。然而,卡罗尔的完成时间仅仅间接地通过鲍勃的完成时间依赖于爱丽丝。如果我们已经知道鲍勃的完成时间,我们将无法通过确定爱丽丝的完成时间来更好地估计卡罗尔的完成时间。这意味着我们可以仅使用两个交互来建模接力赛:爱丽丝对鲍勃的影响和鲍勃对卡罗尔的影响。我们可以省略我们模型中的爱丽丝与卡罗尔的间接相互作用。

结构化概率模型提供了一种用于仅对随机变量之间的直接相互作用进行建模的形式框架。这允许模型具有较少的参数,因此可以从较少的数据可靠地估计。这些较小的模型还显着降低了存储模型,在模型中执行推理,以及从模型中采样的计算成本。

\section{用图来描述模型结构}
结构化概率模型使用图(在图论意义上由边连接的“节点”或“顶点”)来表示随机变量之间的交互。每个节点表示随机变量。每个边表示直接交互。这些直接交互意味着其他的间接交互,但是只有直接交互需要被明确地建模。

通过使用图,有多种方法来描述概率分布中的交互。在下面的章节中,我们描述了一些最受欢迎和有用的方法。图形模型可以大致分为两类:基于有向无环图的模型和基于无向图的模型。

\subsection{有向模型}
一种结构化概率模型是定向图形模型,或者称为置信网络或贝叶斯网络。

定向图形模型被称为“定向”,因为它们的边缘是定向的,也就是说,它们从一个顶点指向另一个顶点。该方向在图中用箭头表示。箭头的方向表示哪个变量的概率分布是根据对方的定义。绘制一个从a到b的箭头,意味着,我们通过条件分布来定义b的概率分布,即a在条件栏的右侧的变量。换句话说,b上的分布取决于a的值。

继续来看第16.1节的接力赛的示例,假设我们命名爱丽丝的完成时间为$t_0$,鲍勃的完成时间为$t_1$,卡罗尔的完成时间为$t_2$。如我们前面所看到的,$t_1$的估计依赖于$t_0$。$t_2$的估计直接依赖于$t_1$但只间接依赖于$t_0$。我们可以在有向图模型中描述这种关系,如图16.2所示。

形式上,由定向无环图定义的变量x定义的有向图形模型。
%%%%%%%%%%%%%%%%%%%%%%%%%%%%%%%%%%%%%%%%%%%%%%%%%%%%%%%%%
%%%%%%%%%%%%%%% author:heailong2013 %%%%%%%%%%%%%%%%%%%%%%%%%%%
%%%%%%%%%%%%%%% part: 16.2.6-16.6       %%%%%%%%%%%%%%%%%%%%%%%%%%%
%%%%%%%%%%%%%%%%%%%%%%%%%%%%%%%%%%%%%%%%%%%%%%%%%%%%%%%%%

\chapter{蒙特卡洛方法}
\label{chap:17}
%%%%%%%%%%%%%%%%%%%%%%%%%%%%%%%%%%%%%%%%%%%%%%%%%%%%%%%%%
%%%%%%%%%%%%%%%%%%% author:kiseliu  %%%%%%%%%%%%%%%%%%%%%
%%%%%%%%%%%%%%%%%%% part17.0-17.3   %%%%%%%%%%%%%%%%%%%%%
%%%%%%%%%%%%%%%%%%%%%%%%%%%%%%%%%%%%%%%%%%%%%%%%%%%%%%%%%

\section{采样和蒙特卡罗方法}
随机化算法可以分成大致两类:拉斯维加斯算法和蒙特卡罗算法。拉斯维加斯算法总是准确地返回正确答案(或者报告失败)。这类算法假设随机量的资源,通常是内存或者时间。相反,蒙特卡罗算法返回答案时会带有随机量的错误。错误量通常可以通过扩展更多的资源(通常是运行时间和内存)被减少。对任意固定的计算方案,蒙特卡罗都可以给出一个近似回答。

机器学习领域中的许多问题都很困难,除了精确的确定性算法和拉斯维加斯算法,我们不能期待获得这些困难问题的精确答案。相反,我们必须使用确定性的近似算法或者蒙特卡罗近似,这两种方法在机器学习中是广泛存在的。本章,我们将讨论蒙特卡罗方法。

\subsection{采样和蒙特卡罗方法}
许多用于实现机器学习目标的重要技术都是基于从一些概率分布中抽取样本,然后用这些样本进行一定理想数量的蒙特卡罗估计。

\subsection{为什么采样?}
我们想要从概率分布中抽取样本有许多原因。采样用更少的代价,提供了一种灵活的方法来近似许多加和和积分。有时候我们使用采样来加速代价昂贵但是容易处理的加和,比如我们使用minibatches对整个训练代价子采样的情况。在其他情况,我们的学习算法需要逼近一个不容易处理的加和或者积分,比如对数模型的对数配分函数(log partition function)的梯度。在许多其他情况下,在我们想训练一个可以从训练分布中采样的模型的意义上,采样确实是我们的目标。


\subsection{蒙特卡罗采样的基本知识}
 当加和或者积分不能准确地被计算出来(比如,该加和具有指数级别数量的项,并且没有精确的简化),我们经常会使用蒙特卡罗采样来逼近它。其思想是将加和或者积分看作是某种分布下的期望,然后通过相应的平均来近似该期望。令
 $$ s = \sum _{ x }^{  }{ p(\bm{x})f(\bm{x}) ={ E }_{ p }[f(\textbf{x})] }\eqno{(17.1)} $$
或者
 $$ s=\int { p(\bm{x})f(\bm{x})d\bm{x}= } { E }_{ p }[f(\textbf{x})]\eqno{(17.2)} $$
 是将要估计的加和或者积分,我们把它重新写成一个期望,这里$p$是随机变量\(\textbf{x}\)的概率分布(对加和而言)或者概率密度(对积分而言)。
 
 我们可以通过从$p$中抽取$n$个样本 $ { \bm{x} }^{ (1) },...,{ \bm{x} }^{ (n) }$,然后构建下面的经验平均来估计$s$:
 $${ \hat { s }  }_{ n } =\frac { 1 }{ n } \sum _{ i=1 }^{ n }{ f({ \bm{x} }^{ (i) }) } .\eqno{(17.3)}$$
这种近似可以由几个不同的性质来证明。首先,我们通过观察可以发现\(\hat { { s } } \)是无偏的,因此
$$\mathbb{E}[\hat{s}_n]=\frac{1}{n}\sum_{i=1}^{n}\mathbb{E}[f(\bm{x}^{(i)})]=\frac{1}{n}\sum_{i=1}^{n}s=s \eqno{(17.4)}$$
但是除此之外,大数定律告诉我们如果样本\({ \bm{x} }^{ (i) }\)是独立同分布,那么几乎可以肯定该平均收敛于期望值:
$$\lim _{ n\rightarrow \infty  }{ { \hat { s }  }_{ n }= } s, \eqno{(17.5)}$$
条件是每一项的方差\(Var[f({ \bm{x} }^{ (i) })]\)是有界的。为了看的更清楚,考虑\(n\)递增时\({\hat { s }  }_{ n }\)的方差。只要
\(Var[f({ \textbf{x} }^{ (i) })]<\infty \),方差\(Var[{ \hat { s }  }_{ n }]\)递减且收敛于0:
$$ Var[{ \hat { s } }_{ n }]=\frac { 1 }{ { n }^{ 2 } } \sum _{ i=1 }^{ n }{ Var[f(\textbf{x})] }  \eqno{(17.6)}$$
$$=\frac { Var[f(\textbf{x})] }{ n } . \eqno{(17.7)}$$
这个方便的结论也告诉了我们如何用蒙特卡罗平均估计不确定或者等价地蒙特卡罗近似的预期误差量。我们计算\(f({\bm{x}}^{ (i) })\)的经验平均\footnote{更常说方差的无偏估计量,其中平方误差和除以n-1而不是n。}和经验方差,然后用样本数量\(n\)来除估计的方差获得\(Var[{\hat { s }  }_{ n }]\)的估计。中心极限定理告诉我们分布的均值\({\hat { s }  }_{ n }\)收敛到均值为\(s\),方差为\(\frac { Var[f(\bf{x})] }{ n } \)的正态分布。这使得我们能够使用正态密度的累计分布来估计\({\hat { s }  }_{ n }\)的置信区间。

然而所有这些依赖于我们能否很容易地从基本分布\(p(\bf{x})\)进行采样,但是我们可能并不能总是这样做。如果从\(p\)中采样不合理时,另一种方法是使用重要性采样,我们会在17.2节介绍它。一种更常用的方法是构造收敛到感兴趣的分布的估计序列,这种方法是蒙特卡罗马尔可夫链(见17.3节)。


\section{重要性采样}

在方程17.2中,蒙特卡罗方法使用的被积函数(或被加数)分解中,有一步很重要,那就是决定被积函数的哪部分应该作为概率分布\(p(\bm{x})\),被积函数的哪部分应该作为其期望值(在概率分布下)被估计的数量函数\(f(\bm{x})\)。由于\(p(\bm{x})f(\bm{x})\)总是可以被写成
$$p(\bm{x})f(\bm{x})=q(\bm{x})\frac { p(\bm{x})f(\bm{x}) }{ q(\bm{x}) } ,\eqno{(17.8)}$$
所以被积函数没有唯一的分解。这里我们从\(q\)和均值\(\frac{pf}{q}\)进行抽样。在许多情况下,我们希望对于给定的\(p\)和\(f\)计算期望,并且从开始指定问题作为期望的事实表明这个\(p\)和\(f\)将是一种自然分解。但是就为了得到给定精度的准确率所需要的样本数量而言,问题的原始规范可能并不是最优的选择。幸运地是,最优选择\({ q }^{ * }\)的形式可以很容易地被推导出来。最优的\({ q }^{ * }\)对应于最优的重要性采样。

由于方程17.8中等式,任何蒙特卡罗估计:
$${ \hat { s }  }_{ p }=\frac { 1 }{ n } \sum _{ i=1,{ \bf{x} }^{ (i) }\sim  p }^{ n }{ f({ \bm{x} }^{ (i) }) } \eqno{(17.9)}$$
都可以被转换为重要性采样估计:
$${ \hat { s }  }_{ q }=\frac { 1 }{ n } \sum _{ i=1,{ \bf{x}}^{ (i) } \sim  q }^{ n }{ \frac { p({\bm{x}}^{ (i) })f({\bm{x}}^{ (i) }) }{ q({\bm{x}}^{ (i) }) }  } . \eqno{(17.10)}$$

我们可以很容易地看到估计的期望值不依赖于\(q\):
$$\mathbb{E}_{ q }\left[ \hat { { s }_{ q } }  \right] =\mathbb{E}_{ q }\left[ \hat { { s }_{ p } }  \right] =s. \eqno{(17.11)}$$
但是,重要性采样估计的方差对\(q\)的选择非常敏感。方差由下式给出:
$$Var[\hat { { s }_{ q } } ]=Var[\frac { p(\bf{x})f(\bf{x}) }{ q(\bf{x}) } ]/n. \eqno{(17.12)}$$
当\(q\)取值如下的时候,可以使得方差最小:
$${ q }^{ * }(\bm{x})=\frac { p(\bm{x})|f(\bm{x})| }{ Z } , \eqno{(17.13)}$$
这里Z是正则化常数,它被用来使得\({ q }^{ * }(\bm{x})\)加和或者积分等于1。越好的重要性采样分布,在被积函数越大的地方,赋予的权重就越大。事实上,当\(f(\bm{x})\)不改变符号,\(Var[{ \hat { s }  }_{ { q }^{ * } }]=0\),这意味着当我们使用最优分布时,单样本就足够了。当然,这只是因为\(q^{*}\)的计算基本上解决了原始问题,从最优分布中抽取一个单样本通常是不切实际的。
(从获得正确期望值的意义上来说)采样分布\(q\)的任何选择都是有效的,并且(从获得最小方差的意义上来说)\(q^{*}\)是最优的一个。从\(q^{*}\)中采样通常是不可行的,但是当我们继续减少方差,\(q\)的其他选择是可行的。

另一种方法是使用有偏的重要性采样,它的一个优势是不需要正则化的\(p\)或者\(q\)。在离散变量的情况下,有偏的重要性采样估计由下式给出:
$${ \hat { s }  }_{ BIS }=\frac { \sum _{ i=1 }^{ n }{ \frac { p({\bm{x}}^{ (i) }) }{ q({\bm{x}}^{ (i) }) }  } f({\bm{x}}^{ (i) }) }{ \sum _{ i=1 }^{ n }{ \frac { p({\bm{x}}^{ (i) }) }{ q({\bm{x}}^{ (i) }) }  }  } \eqno{(17.14)}$$
$$=\frac { \sum _{ i=1 }^{ n }{ \frac { p({\bm{x}}^{ (i) }) }{ \tilde { q } ({\bm{x}}^{ (i) }) }  } f({\bm{x}}^{ (i) }) }{ \sum _{ i=1 }^{ n }{ \frac { p({\bm{x}}^{ (i) }) }{ \tilde { q } ({\bm{x}}^{ (i) }) }  }  } \eqno{(17.15)}$$
$$=\frac { \sum _{ i=1 }^{ n }{ \frac { \tilde{p}({\bm{x}}^{ (i) }) }{ \tilde { q } ({\bm{x}}^{ (i) }) }  } f({\bm{x}}^{ (i) }) }{ \sum _{ i=1 }^{ n }{ \frac {\tilde {p}({\bm{x}}^{ (i) }) }{ \tilde { q } ({\bm{x}}^{ (i) }) }  }  }, \eqno{(17.16)}$$
这里\(\tilde {p}\)和\(\tilde {q}\)是\(p\)和\(q\)的无正则化形式,\({\bm{x}}^{ (i) }\)是\(q\)中的样本。因为\(\mathbb{E}\left[ \hat { { s }_{ BIS } }  \right] \neq s\),该估计是有偏的,除了当\( n\rightarrow \infty\),和方程17.14的分母收敛于1。因此这种估计被叫做渐进无偏的。

虽然\(q\)的一个好的选择可以极大地提高蒙特卡罗估计的效率,但是一个不好的选择会使得效率更差。回到方程17.12,我们可以看到如果\( \frac { p(\bm{x})|f(\bm{x})| }{ q(\bm{x}) } \)中的\(q\)的样本很大,那么估计的方差会变得非常大。这可能发生在\(q(\bm{x})\)很小,而\(p(\bm{x})\)和\(f(\bm{x})\)都不够小来抵消它时。\(q\)分布通常被选择为非常简单的分布以使得容易进行抽样。当\(\bm{x}\)是高维的,\(q\)的简单性会使得它对\(p\)或者\(p|f|\)的匹配非常糟糕。当\(q({\bm{x}}^{ (i) })\gg p({\bm{x}}^{ (i) })|f({\bm{x}}^{ (i) })\),重要性采样会得到一些无用的样本(总和微小的数字或零)。另一方面,当\(q({\bm{x}}^{ (i) })\ll p({\bm{x}}^{ (i) })|f({\bm{x}}^{ (i) })\),尽管这种情况更少发生,但是无用的样本比例会很大。由于后者的情况是极少的,它们可能不会出现在典型的样本中,产生s的典型低估,很少被大量过高估计补偿。所谓典型的,当\(\bm{x}\)是高维的,非常大或者非常小的数字就是典型的,因为在高维情况下,联合概率的动态范围是非常大的。

尽管存在上述危险,但是重要性采样和它的变种们在许多机器学习算法中都非常有用,包括深度学习算法。比如,使用重要性采样可以加速具有很大词汇表的神经语言模型,或者具有很多输出的其他神经网络的训练。另请参考18.7章节重要性采样是如何被用来估计配分函数(概率分布的正则化常数)的,和20.10.3章节重要性采样是如何被用来估计深度有向模型,比如变分自编码器的对数似然的。重要性采样可以被用来改进代价函数的梯度估计,该代价函数被用来训练使用随机梯度下降的模型的参数,特别是对于诸如分类器的模型,其中代价函数的大部分总值来自很少量的被分错的样本。更频繁地抽样更困难的例子可以减少这种情况下梯度的方差(Hinton,2006)。

\section{马尔可夫蒙特卡罗方法}
在许多情况下,我们想要使用蒙特卡罗技术,但是没有容易的方法从分布\({ p }_{ model }(\bf{x})\)或者从一个好的(低方差)重要性采样分布\(q(\bf{x})\)中抽取精确的样本。在深度学习中,这种情况经常发生在\({ p }_{ model }(\bf{x})\)被一个无向模型表示时。在这些情况下,我们引入一种数学工具叫马尔可夫链来近似地从\({ p }_{ model }(\bf{x})\)中采样。使用马尔可夫链来进行蒙特卡罗估计的一类算法叫做马尔可夫蒙特卡罗方法(MCMC)。在Koller and Friedman (2009)中,用了很大篇幅来描述机器学习中的马尔可夫蒙特卡罗方法。
MCMC技术最标准且通用的保证仅适用于模型不对任何状态指定零概率的情况。因此,把这些技术表示成从如在16.2.4中描述的基于能量的模型(EBM) \(p(\bm{x})\propto exp(-E(\bm{x}))\)中采样是最方便的。在EBM公式中,每个状态都被确保具有非零概率。实际上,MCMC方法更广泛地适用,并且可以被与包含零概率状态的许多概率分布一起使用。但是,关于MCMC方法表现的理论保证必须在不同类型的分布下逐个证明。在深度学习的背景下,依赖于能够应用所有基于能量的模型的最一般的理论保证是最常见的。

为了理解为什么从基于能量的模型抽取样本是困难的,让我们考虑只有两个变量的EBM模型,定义一个分布\(p(a,b)\)。为了采样\(a\),我们必须从\(p(a|b)\)中抽取\(a\),为了采样\(b\),我们必须从\(p(b|a)\)中抽取\(b\)。它看起来是个不容易处理的鸡生蛋,蛋生鸡的问题。有向模型可以避免这个问题,因为它们的图是有向无环的。要进行祖先采样,只需要按照拓扑顺序对每个变量进行采样,对每个变量的父节点进行调节,并保证父节点已经被采样(见第16.3节)。祖先采样定义了一种获得样本的有效的单程方法。

在EBM模型中,我们可以通过使用马尔可夫链来抽样,从而避免这个鸡生蛋,蛋生鸡的问题。马尔可夫链的核心思想是有一个以任意值开始的状态,随着时间,我们不断地随机更新\(\bm{x}\)。最终\(\bm{x}\)(几乎接近)变成\(p(\bm{x})\)中的一个真实的样本。正式来说,马尔可夫链被定义为随机状态\(\bm{x}\),和转移分布\(T({ \bm{x} }^{ \prime  }|\bm{x})\),该分布给出了从状态\(\bm{x}\)开始,随机更新到状态\({ \bm{x} }^{ \prime  }\)的概率。运行马尔可夫链意味着不断地更新状态\(\bm{x}\)到从\(T({ \bm{x} }^{ \prime  }|\bm{x})\)抽样出来的值\({ \bm{x} }^{ \prime }\)。

得到一些关于MCMC方法如何工作的理论理解,对于再参数化该问题是很有用。首先,我们把问题限制在随机变量\(\bf{x}\)只有有限种状态的情况。在这种情况,我们可以把状态表示成一个正整数\(x\)。x的不同的整数值对应原始问题中不同的状态\(\bm{x}\)。

现在让我们考虑下,当我们无限次并行地运行许多马尔可夫链时会发生什么。不同的马尔可夫链的所有状态都是从相同的分布\({ q }^{ (t) }(x)\)中抽取的,这里\(t\)表示已经过去的时间步长的数量。在开始时,\({ q }^{ (0) }\)是一些用来对每个马尔可夫链任意地初始化\(x\)的分布。随后,\({ q }^{ (t) }\)被目前已经运行过的所有的马尔可夫链步骤所影响。我们的目标是让\({ q }^{ (t) }(x)\)收敛到\(p(x)\)。

因为我们已经就正整数\(x\)重新参数化了该问题,因为我们可以用一个向量\(\bm{v}\)来描述概率分布\(q\),如下式:
 $$ { q }(x=i)={ v }_{ i }. \eqno{(17.17)} $$
 
 考虑下当我们更新单独一个马尔可夫链的状态\(x\)到一个新的状态\({x}^{\prime}\)时,会发生什么。一个单独的状态落到状态\({x}^{\prime}\)的概率由下式给出:
$${ q }^{ (t+1) }({ x }^{ \prime  })=\sum _{ x }^{  }{ { q }^{ (t) }({ x }) } T({ x }^{ \prime  }|x).  \eqno{(17.18)}  $$

使用我们的整数参数化,我们可以用矩阵\(\bm{A}\)表示转移操作\(T\)的影响。我们定义\(\bm{A}\)为:
$${ A }_{ i,j }=T({ \bf{x} }^{ \prime  }=i|\bf{x} = j). \eqno{(17.19)}  $$

使用这个定义,我们现在可以重写方程17.18。代替书写该公式使用\(q\)和\(T\)来理解单个状态是如何被更新的,我们可能可以用\(\bm{v}\)和\(\bm{A}\)把基于所有不同马尔可夫链的整个分布是如何并行运行的过程描述为我们进行的一次更新:
$${\bm{v}}^{ (t) }=\bm{A}{\bm{v}}^{ (t-1) }. \eqno{(17.20)}$$

应用马尔可夫链不断地进行更新,对应于不断地乘以矩阵\(\bm{A}\)。也就是说,我们可以把该过程看作对矩阵\(\bm{A}\)取幂:
$${ \bm{v} }^{ (t) }=\bm{A}{ \bm{v} }^{ (0) }  \eqno{(17.21)}$$

矩阵\(\bm{A}\)有特殊的结构,它的每一列表示一个概率分布。这样的矩阵也叫做随机矩阵。如果在某次幂\(t\),从任意状态\(x\)转移到其他状态\({x}^{ \prime }\)的概率都非零,那么Perron-Frobenius定理(Perron, 1907; Frobenius, 1908)保证,最大的特征值是实数,并且等于1。随着时间,我们可以看到所有的特征值都可以写成指数形式:
$${\bm{v}}^{(t)}={(\bm{V}diag(\bm{\lambda}){\bm{V}}^{-1})}^{t}{\bm{v}}^{(0) }=\bm{V}diag(\bm{\lambda}){\bm{V}}^{ -1 }{\bm{v}}^{ (0) } \eqno{(17.22)}$$

该过程会使得所有不等于1的特征值衰减到0。在其他宽松的条件下,A被确保只有一个特征值为1的特征向量。因此该过程收敛到稳定分布,有时也称为平衡分布。收敛时,
$${ \bm{v} }^{ \prime  }=\bm{A}\bm{v}=\bm{v}, \eqno{(17.23)}$$
这个相同的条件对每个额外的步骤都成立。这是一个特征向量方程。为了成为一个稳定点,\(\bm{v}\)必须是对应于特征值为1的特征向量。这个条件确保一旦到达稳定分布,转移抽样过程的重复应用不改变基于各种各样的马尔可夫链的状态的分布(尽管转移操作本来就不改变每个个体状态)。

如果我们正确地选择了\(T\),那么稳定分布\(q\)会等于我们希望抽样的分布\(p\)。我们会在17.4章节简要地描述如何选择\(T\)。

具有可数状态的马尔可夫链的大多数性质可以推广到连续变量的马尔可夫链。在这种情况下,一些作者叫马尔可夫链为Harris链,但是我们使用马尔可夫链这个术语来描述这两种情况。通常在宽松的条件下,具有转移操作\(T\)的马尔可夫链会收敛到一个稳定点,该点可以用下述方程来描述的:
$$  {q}^{\prime}({\bf{x}^{\prime})={\mathbb{E}}_{\bf{x}\sim q}T({\bf{x}^{\prime}|\bf{x})  \eqno{(17.24)} $$
 离散的情况,我们刚才已经重写成方程17.23了。当\(\bf{x}\)是离散的,期望对应于加和,当\(\bf{x}\)是连续的,期望对应于积分。

不管状态是连续的还是离散的,所有的马尔可夫链方法都包含重复地应用随机更新的过程,直到最终状态开始从平稳分布产生样本。运行马尔可夫链直到它达到平稳分布的过程叫做“burning in”马尔可夫链。在达到了平稳分布,非常多的样本序列可以从平稳分布中抽取出来。它们被相同地分布,但是任意两个成功的样本之间是高度相关的。因此有限的样本序列可能不能表示平稳分布。减轻这个问题的一种方法是仅返回每\(n\)个连续样本,这会使得我们对平衡分布的统计的估计不会由于MCMC采样和接下来的几个样本之间的相关性而有偏差。因此使用马尔可夫链的代价很高,因为需要时间来构建平稳分布,以及在达到平稳后需要时间从一个样本转换到另一个合理地去除了相关性的样本。如果想得到真正独立的样本,可以并行地运行多个马尔可夫链。这种方法使用了额外的并行代价来消除延迟。只使用一个单独的马尔可夫链来产生所有的样本的策略,和对每一个理想的样本都使用一个马尔可夫链的策略是两种极端;深度学习从业人员通常使用与小批量中的样本数量相似的多个链,然后从这个固定的马尔科夫链中绘制所需的多个样本。\(A\)常用的马尔可夫链的数目是100。

另一个困难是我们不能提前知道,在达到平稳分布前,马尔可夫链要运行多少步。该时间长度称为混合时间。测试马尔可夫链是否达到平衡也是非常困难的。暂时没有足够准确的理论来指导我们回答这一问题。理论只是告诉我们,马尔可夫链会收敛,没有更多的信息了。如果我们从作用于概率向量\(\bm{v}\)的矩阵\(\bm{A}\)的角度来分析马尔可夫链,那么我们知道当\({ \bm{A} }^{ t }\)已经有效地丢失了来自\(\bm{A}\)的除了唯一的特征值1之外的所有特征值时,链条混合。这意味着第二大的特征值的量级将决定混合时间。但是实际上,我们不能用矩阵表示出马尔可夫链。我们的概率模型可以访问的状态数量对于变量的数量是指数级大的,所以表示\(\bm{x}\),\(\bm{A}\)或者\(\bm{A}\)的特征值是不合理的。由于这些阻碍和其他的阻碍,我们通常不知道马尔可夫链是否已经混合。相反,我们简单地运行马尔可夫链一段时间,然后大致地估计是否足够,并且使用启发式方法来确定链是否混合。这些启发式方法包括人工检查样本,或者测量成功抽取的样本之间的相关性。

\section{吉布斯采样}
\label{sec:17.4}

到目前为止,本章已经介绍了如何通过迭代更新$x\rightarrow x' \sim T (x'|x) $在概率分布$q(x)$上采样的方法.
但是,还没有说明怎样确保$q(x)$是有效的概率分布.
本书主要考虑两种基本方法.第一种方法根据给定的通过学习得到的$p_{model}$推导出$T$,下文将会详细介绍从EBMs(energy-based model)采样的例子.
第二种方法是直接参数化$T$并学习,使其平稳分布可以隐式地定义$p_{model}$的兴趣点.
第二种方法的例子将在章节"\textcolor{red}{20.12}"和"\textcolor{red}{20.13}"阐述.

在深度学习中,通常使用马尔科夫链从以基于能量的模型定义的概率分布$p_{model}$中采样.
此例中,马尔科夫链所需的$q(x)$就是$p_{model}$.
为得到满足需求的$q(x)$,必须选择合适的$T(x'|x)$.

为了构建马尔科夫链,一种概念上简单有效的方法就是使用\textit{吉布斯采样}从$p_{model}$中采样.
在吉布斯采样中,通过选择一个变量$x_i$实现从$T(x'|x)$中采样,$x_i$的选择方法取决于它在基于能量模型结构的无向图$\mathcal{G}$上的邻居变量.
只要给定的所有邻居节点都条件独立,也可以同时对多个变量采样.
在16.7.1中的RBM(受限玻尔兹曼机)例子中,RBM的所有隐含层单元可以被同时采样,是因为他们都条件独立于其他的给定可视层单元.
同样的,因为对于给定的隐含层单元,所有的可视层单元都是条件独立的,因此所有的可视层单元也可以同时被采样.
以这种方式同时更新多个变量的吉布斯采样方法,被称为块吉布斯(block gibbs)采样.

从$p_{model}$中采样来设计马尔科夫链的代替方法是可行的.例如,Metropolis-Hastings算法就广泛应用于其他学科领域.
在深度学习领域实现无向建模,除吉布斯采样外很少使用其他方法.
改进采样技术是一个潜在的研究领域.

\section{分离状态混合的挑战}
\label{sec:17.5}

$MCMC$方法的主要问题是混合效果比较差.理想情况下,从$P(x)$分布设计的马尔科夫链中得到连续采样之间将会是完全相互独立的,
这些采样在$x$空间下访问不同的区域可能性与他们的概率成正比.
然而,尤其是在高维情况下,MCMC样本之间会强相关.
我们称这种行为是缓慢混合,甚至是失败混合.
相对于链(随机变量被采样)的状态,$MCMC$方法的缓慢混合可以被看作是在能量函数上随意地执行了类似嘈杂梯度下降(noisy gradient descent)操作,
或者等效地在概率函数上执行嘈杂爬山(noisy hill climbing)操作.
马尔科夫链往往使用渐进策略(在马尔科夫链状态空间中),从组态$x_{(t-1)}$到$x_{(t)}$,
能量$E(x_{(t)})$往往低于或约等于能量$E(x_{(t-1)})$,偏好会产生较低能量组态的移动.
当从相对不可能的组态(比来自$p(x)$的典型构型具有更高的能量)开始时,链趋向于逐步降低能量的状态仅仅偶尔移动到另一种状态.
一旦链找到了低能量的区域(例如,假设变量是图片中的像素点,则低能量的区域可以是同一对象的图像中的连通的多边形),我们称其之为一个状态,链将会尝试在状态上游走(类似于随机游走).
一旦跳出那个状态,一般会返回原状态或者(如果找到了逃离路线)会移动到另一个状态.
问题是成功的逃离路线很少有很多有趣的分布,所以马尔科夫链将会在同一状态而不是它应有的状态上连续采样.

在使用吉布斯采样算法时这一点是显而易见的$(Sec. 17.4)$.
在这种情况下,在特定的步骤内从一个状态转移到邻近状态的概率取决于状态之间"能量势垒"(energy barrier)的形状.
被高能量势垒(低概率区域)分隔的两个状态之间是不太可能发生转移的(在能量势垒的高度方面).
如图$17.1$所示.
当多个高概率状态被低概率区域分隔时会出现问题,特别是当吉布斯采样每步仅仅更新一小部分变量,而这部分变量的值主要由其它变量确定时.

举个简单的例子,假设基于能量模型的两个变量$a$和$b$,他们是带符号的二进制,取值为-1和1.
对于模型$E(a,b)=-wab$,如果$w$是一个大正数,那么这个模型中的$a$和$b$极有可能具有相同的符号.
如果使用$a=1$的吉布斯采样步骤来更新$b$,那么$b$的条件分布则是由$P(b=1|a=1)=\sigma(w)$确定的.
如果$w$很大,$b$也赋值为1的概率接近于1.
同理,如果$a=-1$,则将$b$赋值为-1的概率接近1.
根据$P_{model}(a,b)$模型,两个变量的符号可能相同.
也就是说,吉布斯采样不会改变变量的符号.

实际应用场景更具有挑战性,因为不仅要关注两个状态之间的转移,而且在更具一般性的真实模型中可能更关心多个状态之间的转移.因为状态混合困难性,获取覆盖大多数状态的可靠样本集合花费的代价巨大,而且到达马尔科夫链稳定分布状态的收敛速度非常慢等诸多原因,这样的多个状态之间的转移是很难的.

有时这个问题可以通过找到含有高度依赖性单元的群体并按块同时更新所有这些单元来解决.
然而不幸的是,这种依赖性是复杂的,从这样的群体中采样是难以计算的.
毕竟,马尔科夫链最初被引入来解决的问题就是怎样从一组变量中采样.

\begin{figure}[htbp]
	\centering
	\includegraphics[width=5in]{fig/chap17/chap17.1.png}
	\caption{三种分布的吉布斯采样路径,在这两种情况下,马尔科夫链以同样的模式初始化.
	$($左图$)$两个条件独立变量的多变量正态分布.因为变量之间是条件独立的,所以吉布斯采样混合的很好.
	$($中图$)$变量高度相关的多变量正态分布.变量之间的相关性使得马尔科夫链很难混合.因为每个变量的更新都要取决于其他变量,相关性降低了马尔科夫链从起始点移动到其他状态的概率.
	$($右图$)$广泛分离模式的高斯分布的混合是非轴对称的.因为当一次仅改变一个变量时,马尔科夫链难以改变状态,所以吉布斯采样混合的很慢.}
	\label{fig:chap17.1.png}
\end{figure}

在含有隐含变量的模型中,定义了联合分布$p_{model}(x|h)$,常常通过交替在$p_{model}(x|h)$和$p_{model}(h|x)$中对$x$进行采样.
从快速混合的目的出发,我们希望$p_{model}(h|x)$有更高的熵.
但是,从学习到有用的$h$的表达式的目的出发,我们又希望$h$有足够的信息来编码并重建$x$,这意味着$h$和$x$应该有很高的交互信息.
这个两个目标是相互矛盾的.
常见的生成模型可以非常精确地将$x$编码为$h$,但是往往不能混合地很好.
这种情况在玻尔兹曼机中很常见-玻尔兹曼机学习到的分布越尖锐,马尔科夫链从分布模型中的采样就越难以混合良好.
这个问题在图$17.2$中说明.

当兴趣分布具有多种结构且每一类有单独的样本,所有上述问题都可能导致$MCMC$方法在兴趣分布上失效:
概率分布集中在许多状态周围,这些状态被高能量区域分隔开.
这类分布就是我们在许多分类问题中期望的,状态之间的较差混合导致$MCMC$方法收敛地非常缓慢.

\begin{figure}[htbp]
	\centering
	\includegraphics[width=5in]{fig/chap17/chap17.2.png}
	\caption{在深度概率模型中漫混合问题的例子.每个图像应该从左到右,从上到下阅读.
	$($左图$)$是来自吉布斯采样的连续样本应用于MNIST数据集上训练的深度玻尔兹曼机.连续样本之间是很相似的.
	因为吉布斯采样是在深度图模型中执行的,所以这种相似性更多地基于语义而不是原始视觉特征,
	但是吉布斯链仍然难以从分布的一种状态转换到另一种状态,例如通过改变数字标识.
	$($右图$)$是来自生成对抗网络的连续祖代样本.因为祖代采样独立于其他样本而生成每个样本,所以没有混合问题.}
	\label{fig:chap17.2.png}
\end{figure}

\section{17.5.1 基于回火的状态混合}
\label{sec:17.5.1}

当一个分布显示出的形态是低概率区域围绕着高概率的“尖峰”时候,这个分布的不同状态很难混合。很多快速混合技术都基于构建的可以替代目标分布的版本,在新构建的分布中峰没有目标分布高,周围的谷也没有目标分布低。基于能量的模型提供了一个特别简单的方法来实现。到目前为止,我们已经描述了一个基于能量的模型来作为概率分布的定义,如下式:
$p(x)\propto exp(-E(x))$                   (17.25)
基于能量的模型可以增加一个额外的参数$\beta$来控制这个分布快速达到峰值。
$p_{\beta}(x)\propto exp(-\beta E(x))$              (17.26)
参数$\beta$通常被描述为温度的倒数,这反映了基于能量的模型在统计物理学中的起源。当温度下降到零并且$\beta$趋于无穷大时,基于能量的模型变得稳定。当温度趋于无穷大并且$\beta$下降到零时,离散$x$的分布变得均匀。

通常来说,可以训练一个模型来得到$\beta=1$时的估计值。然而,我们可以使用其他温度,特别是$\beta<1$处的温度。回火是通过描绘$\beta<1$的采样来实现$p_{1}$的多个状态之间快速混合的一个一般性策略。

为了混合不同的模式,基于回火转换的马尔可夫链(Neal在1994年提出)暂时从更高的温度分布采样,然后再恢复到从单位温度分布上采样。
这些技术已经被应用到一些模型上,比如RBMs(Salakhutdinov在2010年提出)。
另一种方法是使用并行回火(Iba在2001年提出),其中马尔可夫链在不同温度下,并行模拟了许多不同的状态。
最高温度状态下混合缓慢,但是最低温度状态在温度为1时提供了来自模型的精确样本。
转换操作包括在两种不同温度水平下随机切换状态,因此从高温时隙中获得的一个高概率样本可以跳到一个较低温度的时隙中。
这个方法也被应用到RBMs中(Desjardins等在2010年提出;Cho等在2010年提出)。
虽然回火是一种很有使用前景的方法,但是现在研究人员还没能在从复杂EBMs采样的问题上取得突破性进展。
一个可能的原因是为了使得回火有效临界温度周围的温度转换必须是非常缓慢的(因为温度是逐渐降低的)。

\section{17.5.2 深度会促进混合}
\label{sec:17.5.2}

当从一个隐变量模型$p(h,x)$采样时,我们看到,如果$p(h|x)$对$x$编码太长,则从$p(x|h)$采样时$x$变化很小并且混合效果差。
解决这个问题的一个方法是,将$h$作为一个深度表示,在这个深度表示中,将x编码到h中来使得在$h$区间中的马尔科夫链更容易混合。
许多表征学习算法,如自动编码器和$RBMs$,更容易产生一个在$h$上的边缘分布,其比在$x$上的原始数据分布更统一和更单调。
可以说这是由于在使用所有存在的表征空间来试图最小化重建误差时,因为不同的训练样本在$h$区间上更容易划区分,在训练样本上最小化重建误差将实现的更好,并因此获得更好的相互独立性。
Bengio等人(2013a)观察到,在顶层h区间上,正则化过的自动编码器或$RBMs$的深栈堆产生边缘分布,其出现更加广阔和均匀,对于不同模式之间具有较小的间隙(在实验中)。
在较高层的空间中训练$RBM$可以使Gibbs采样在模式之间更快地混合。
然而,它仍然不清楚如何利用这种观察以便更好地训练和从深度生成模型中抽样。

尽管混合比较困难,但蒙特卡洛方法是有效的并且经常被认为是可用的最好的工具。
实际上,它是用于面对无向模型的难处理配分函数的主要工具,将在接下来进行讨论。

\chapter{处理配分函数}
\label{chap:18}
% Partition Function 配分函数的定义见 wiki:https://www.wikiwand.com/zh/%E9%85%8D%E5%88%86%E5%87%BD%E6%95%B0

%%%%%%%%%%%%%%%%%%%%%%%%%%%%%%%%%%%%%%%%%%%%%%%%%%%%%%%%%
%%%%%%%%%%%%%%%%%%% author:quxiaofeng %%%%%%%%%%%%%%%%%%%
%%%%%%%%%%%%%%%%%%%%%%%%%%%%%%%%%%%%%%%%%%%%%%%%%%%%%%%%%

如 \remove{16.2.2 节}中所见,\add{很多概率模型(一般称为无向图模型)}是用\consider{(非正态/未标准化/unnormalized)概率分布 \mbox{ \(\widetilde{p}(\bm{x};\bm{\theta})\)} }定义的。用配分函数\footnote{译者注:配分函数(Partition Function)的定义见 wiki:\url{https://www.wikiwand.com/zh/\%E9\%85\%8D\%E5\%88\%86\%E5\%87\%BD\%E6\%95\%B0}}\(Z(\bm{\theta})\)去除\(\widetilde{p}\)来得到一个概率分布:
\[p(\bm{x};\bm{\theta})=\frac{1}{Z(\bm{\theta})}\widetilde{p}(\bm{x};\bm{\theta}).\]
\chapter{近似推理}
\label{chap:19}
%%%%%%%%%%%%%%%%%%%%%%%%%%%%%%%%%%%%%%%%%%%%%%%%%%%%%%%%%
%%%%%%%%%%%%%%%%% author:caigaojiang@gmail.com %%%%%%%%%%
%%%%%%%%%%%%%%%%%%%%%%%%%%%%%%%%%%%%%%%%%%%%%%%%%%%%%%%%%
\chapter{深度生成式模型}
\label{chap:20}



\end{document}