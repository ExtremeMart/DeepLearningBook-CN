\chapter{蒙特卡洛方法}
\label{chap:17}
%%%%%%%%%%%%%%%%%%%%%%%%%%%%%%%%%%%%%%%%%%%%%%%%%%%%%%%%%
%%%%%%%%%%%%%%%%%%% author:kiseliu  %%%%%%%%%%%%%%%%%%%%%
%%%%%%%%%%%%%%%%%%% part17.0-17.3   %%%%%%%%%%%%%%%%%%%%%
%%%%%%%%%%%%%%%%%%%%%%%%%%%%%%%%%%%%%%%%%%%%%%%%%%%%%%%%%

\section{采样和蒙特卡罗方法}
随机化算法可以分成大致两类:拉斯维加斯算法和蒙特卡罗算法。拉斯维加斯算法总是准确地返回正确答案(或者报告失败)。这类算法假设随机量的资源,通常是内存或者时间。相反,蒙特卡罗算法返回答案时会带有随机量的错误。错误量通常可以通过扩展更多的资源(通常是运行时间和内存)被减少。对任意固定的计算方案,蒙特卡罗都可以给出一个近似回答。

机器学习领域中的许多问题都很困难,除了精确的确定性算法和拉斯维加斯算法,我们不能期待获得这些困难问题的精确答案。相反,我们必须使用确定性的近似算法或者蒙特卡罗近似,这两种方法在机器学习中是广泛存在的。本章,我们将讨论蒙特卡罗方法。

\subsection{采样和蒙特卡罗方法}
许多用于实现机器学习目标的重要技术都是基于从一些概率分布中抽取样本,然后用这些样本进行一定理想数量的蒙特卡罗估计。

\subsection{为什么采样?}
我们想要从概率分布中抽取样本有许多原因。采样用更少的代价,提供了一种灵活的方法来近似许多加和和积分。有时候我们使用采样来加速代价昂贵但是容易处理的加和,比如我们使用minibatches对整个训练代价子采样的情况。在其他情况,我们的学习算法需要逼近一个不容易处理的加和或者积分,比如对数模型的对数配分函数(log partition function)的梯度。在许多其他情况下,在我们想训练一个可以从训练分布中采样的模型的意义上,采样确实是我们的目标。


\subsection{蒙特卡罗采样的基本知识}
 当加和或者积分不能准确地被计算出来(比如,该加和具有指数级别数量的项,并且没有精确的简化),我们经常会使用蒙特卡罗采样来逼近它。其思想是将加和或者积分看作是某种分布下的期望,然后通过相应的平均来近似该期望。令
 $$ s = \sum _{ x }^{  }{ p(\bm{x})f(\bm{x}) ={ E }_{ p }[f(\textbf{x})] }\eqno{(17.1)} $$
或者
 $$ s=\int { p(\bm{x})f(\bm{x})d\bm{x}= } { E }_{ p }[f(\textbf{x})]\eqno{(17.2)} $$
 是将要估计的加和或者积分,我们把它重新写成一个期望,这里$p$是随机变量\(\textbf{x}\)的概率分布(对加和而言)或者概率密度(对积分而言)。
 
 我们可以通过从$p$中抽取$n$个样本 $ { \bm{x} }^{ (1) },...,{ \bm{x} }^{ (n) }$,然后构建下面的经验平均来估计$s$:
 $${ \hat { s }  }_{ n } =\frac { 1 }{ n } \sum _{ i=1 }^{ n }{ f({ \bm{x} }^{ (i) }) } .\eqno{(17.3)}$$
这种近似可以由几个不同的性质来证明。首先,我们通过观察可以发现\(\hat { { s } } \)是无偏的,因此
$$\mathbb{E}[\hat{s}_n]=\frac{1}{n}\sum_{i=1}^{n}\mathbb{E}[f(\bm{x}^{(i)})]=\frac{1}{n}\sum_{i=1}^{n}s=s \eqno{(17.4)}$$
但是除此之外,大数定律告诉我们如果样本\({ \bm{x} }^{ (i) }\)是独立同分布,那么几乎可以肯定该平均收敛于期望值:
$$\lim _{ n\rightarrow \infty  }{ { \hat { s }  }_{ n }= } s, \eqno{(17.5)}$$
条件是每一项的方差\(Var[f({ \bm{x} }^{ (i) })]\)是有界的。为了看的更清楚,考虑\(n\)递增时\({\hat { s }  }_{ n }\)的方差。只要
\(Var[f({ \textbf{x} }^{ (i) })]<\infty \),方差\(Var[{ \hat { s }  }_{ n }]\)递减且收敛于0:
$$ Var[{ \hat { s } }_{ n }]=\frac { 1 }{ { n }^{ 2 } } \sum _{ i=1 }^{ n }{ Var[f(\textbf{x})] }  \eqno{(17.6)}$$
$$=\frac { Var[f(\textbf{x})] }{ n } . \eqno{(17.7)}$$
这个方便的结论也告诉了我们如何用蒙特卡罗平均估计不确定或者等价地蒙特卡罗近似的预期误差量。我们计算\(f({\bm{x}}^{ (i) })\)的经验平均\footnote{更常说方差的无偏估计量,其中平方误差和除以n-1而不是n。}和经验方差,然后用样本数量\(n\)来除估计的方差获得\(Var[{\hat { s }  }_{ n }]\)的估计。中心极限定理告诉我们分布的均值\({\hat { s }  }_{ n }\)收敛到均值为\(s\),方差为\(\frac { Var[f(\bf{x})] }{ n } \)的正态分布。这使得我们能够使用正态密度的累计分布来估计\({\hat { s }  }_{ n }\)的置信区间。

然而所有这些依赖于我们能否很容易地从基本分布\(p(\bf{x})\)进行采样,但是我们可能并不能总是这样做。如果从\(p\)中采样不合理时,另一种方法是使用重要性采样,我们会在17.2节介绍它。一种更常用的方法是构造收敛到感兴趣的分布的估计序列,这种方法是蒙特卡罗马尔可夫链(见17.3节)。


\section{重要性采样}

在方程17.2中,蒙特卡罗方法使用的被积函数(或被加数)分解中,有一步很重要,那就是决定被积函数的哪部分应该作为概率分布\(p(\bm{x})\),被积函数的哪部分应该作为其期望值(在概率分布下)被估计的数量函数\(f(\bm{x})\)。由于\(p(\bm{x})f(\bm{x})\)总是可以被写成
$$p(\bm{x})f(\bm{x})=q(\bm{x})\frac { p(\bm{x})f(\bm{x}) }{ q(\bm{x}) } ,\eqno{(17.8)}$$
所以被积函数没有唯一的分解。这里我们从\(q\)和均值\(\frac{pf}{q}\)进行抽样。在许多情况下,我们希望对于给定的\(p\)和\(f\)计算期望,并且从开始指定问题作为期望的事实表明这个\(p\)和\(f\)将是一种自然分解。但是就为了得到给定精度的准确率所需要的样本数量而言,问题的原始规范可能并不是最优的选择。幸运地是,最优选择\({ q }^{ * }\)的形式可以很容易地被推导出来。最优的\({ q }^{ * }\)对应于最优的重要性采样。

由于方程17.8中等式,任何蒙特卡罗估计:
$${ \hat { s }  }_{ p }=\frac { 1 }{ n } \sum _{ i=1,{ \bf{x} }^{ (i) }\sim  p }^{ n }{ f({ \bm{x} }^{ (i) }) } \eqno{(17.9)}$$
都可以被转换为重要性采样估计:
$${ \hat { s }  }_{ q }=\frac { 1 }{ n } \sum _{ i=1,{ \bf{x}}^{ (i) } \sim  q }^{ n }{ \frac { p({\bm{x}}^{ (i) })f({\bm{x}}^{ (i) }) }{ q({\bm{x}}^{ (i) }) }  } . \eqno{(17.10)}$$

我们可以很容易地看到估计的期望值不依赖于\(q\):
$$\mathbb{E}_{ q }\left[ \hat { { s }_{ q } }  \right] =\mathbb{E}_{ q }\left[ \hat { { s }_{ p } }  \right] =s. \eqno{(17.11)}$$
但是,重要性采样估计的方差对\(q\)的选择非常敏感。方差由下式给出:
$$Var[\hat { { s }_{ q } } ]=Var[\frac { p(\bf{x})f(\bf{x}) }{ q(\bf{x}) } ]/n. \eqno{(17.12)}$$
当\(q\)取值如下的时候,可以使得方差最小:
$${ q }^{ * }(\bm{x})=\frac { p(\bm{x})|f(\bm{x})| }{ Z } , \eqno{(17.13)}$$
这里Z是正则化常数,它被用来使得\({ q }^{ * }(\bm{x})\)加和或者积分等于1。越好的重要性采样分布,在被积函数越大的地方,赋予的权重就越大。事实上,当\(f(\bm{x})\)不改变符号,\(Var[{ \hat { s }  }_{ { q }^{ * } }]=0\),这意味着当我们使用最优分布时,单样本就足够了。当然,这只是因为\(q^{*}\)的计算基本上解决了原始问题,从最优分布中抽取一个单样本通常是不切实际的。
(从获得正确期望值的意义上来说)采样分布\(q\)的任何选择都是有效的,并且(从获得最小方差的意义上来说)\(q^{*}\)是最优的一个。从\(q^{*}\)中采样通常是不可行的,但是当我们继续减少方差,\(q\)的其他选择是可行的。

另一种方法是使用有偏的重要性采样,它的一个优势是不需要正则化的\(p\)或者\(q\)。在离散变量的情况下,有偏的重要性采样估计由下式给出:
$${ \hat { s }  }_{ BIS }=\frac { \sum _{ i=1 }^{ n }{ \frac { p({\bm{x}}^{ (i) }) }{ q({\bm{x}}^{ (i) }) }  } f({\bm{x}}^{ (i) }) }{ \sum _{ i=1 }^{ n }{ \frac { p({\bm{x}}^{ (i) }) }{ q({\bm{x}}^{ (i) }) }  }  } \eqno{(17.14)}$$
$$=\frac { \sum _{ i=1 }^{ n }{ \frac { p({\bm{x}}^{ (i) }) }{ \tilde { q } ({\bm{x}}^{ (i) }) }  } f({\bm{x}}^{ (i) }) }{ \sum _{ i=1 }^{ n }{ \frac { p({\bm{x}}^{ (i) }) }{ \tilde { q } ({\bm{x}}^{ (i) }) }  }  } \eqno{(17.15)}$$
$$=\frac { \sum _{ i=1 }^{ n }{ \frac { \tilde{p}({\bm{x}}^{ (i) }) }{ \tilde { q } ({\bm{x}}^{ (i) }) }  } f({\bm{x}}^{ (i) }) }{ \sum _{ i=1 }^{ n }{ \frac {\tilde {p}({\bm{x}}^{ (i) }) }{ \tilde { q } ({\bm{x}}^{ (i) }) }  }  }, \eqno{(17.16)}$$
这里\(\tilde {p}\)和\(\tilde {q}\)是\(p\)和\(q\)的无正则化形式,\({\bm{x}}^{ (i) }\)是\(q\)中的样本。因为\(\mathbb{E}\left[ \hat { { s }_{ BIS } }  \right] \neq s\),该估计是有偏的,除了当\( n\rightarrow \infty\),和方程17.14的分母收敛于1。因此这种估计被叫做渐进无偏的。

虽然\(q\)的一个好的选择可以极大地提高蒙特卡罗估计的效率,但是一个不好的选择会使得效率更差。回到方程17.12,我们可以看到如果\( \frac { p(\bm{x})|f(\bm{x})| }{ q(\bm{x}) } \)中的\(q\)的样本很大,那么估计的方差会变得非常大。这可能发生在\(q(\bm{x})\)很小,而\(p(\bm{x})\)和\(f(\bm{x})\)都不够小来抵消它时。\(q\)分布通常被选择为非常简单的分布以使得容易进行抽样。当\(\bm{x}\)是高维的,\(q\)的简单性会使得它对\(p\)或者\(p|f|\)的匹配非常糟糕。当\(q({\bm{x}}^{ (i) })\gg p({\bm{x}}^{ (i) })|f({\bm{x}}^{ (i) })\),重要性采样会得到一些无用的样本(总和微小的数字或零)。另一方面,当\(q({\bm{x}}^{ (i) })\ll p({\bm{x}}^{ (i) })|f({\bm{x}}^{ (i) })\),尽管这种情况更少发生,但是无用的样本比例会很大。由于后者的情况是极少的,它们可能不会出现在典型的样本中,产生s的典型低估,很少被大量过高估计补偿。所谓典型的,当\(\bm{x}\)是高维的,非常大或者非常小的数字就是典型的,因为在高维情况下,联合概率的动态范围是非常大的。

尽管存在上述危险,但是重要性采样和它的变种们在许多机器学习算法中都非常有用,包括深度学习算法。比如,使用重要性采样可以加速具有很大词汇表的神经语言模型,或者具有很多输出的其他神经网络的训练。另请参考18.7章节重要性采样是如何被用来估计配分函数(概率分布的正则化常数)的,和20.10.3章节重要性采样是如何被用来估计深度有向模型,比如变分自编码器的对数似然的。重要性采样可以被用来改进代价函数的梯度估计,该代价函数被用来训练使用随机梯度下降的模型的参数,特别是对于诸如分类器的模型,其中代价函数的大部分总值来自很少量的被分错的样本。更频繁地抽样更困难的例子可以减少这种情况下梯度的方差(Hinton,2006)。

\section{马尔可夫蒙特卡罗方法}
在许多情况下,我们想要使用蒙特卡罗技术,但是没有容易的方法从分布\({ p }_{ model }(\bf{x})\)或者从一个好的(低方差)重要性采样分布\(q(\bf{x})\)中抽取精确的样本。在深度学习中,这种情况经常发生在\({ p }_{ model }(\bf{x})\)被一个无向模型表示时。在这些情况下,我们引入一种数学工具叫马尔可夫链来近似地从\({ p }_{ model }(\bf{x})\)中采样。使用马尔可夫链来进行蒙特卡罗估计的一类算法叫做马尔可夫蒙特卡罗方法(MCMC)。在Koller and Friedman (2009)中,用了很大篇幅来描述机器学习中的马尔可夫蒙特卡罗方法。
MCMC技术最标准且通用的保证仅适用于模型不对任何状态指定零概率的情况。因此,把这些技术表示成从如在16.2.4中描述的基于能量的模型(EBM) \(p(\bm{x})\propto exp(-E(\bm{x}))\)中采样是最方便的。在EBM公式中,每个状态都被确保具有非零概率。实际上,MCMC方法更广泛地适用,并且可以被与包含零概率状态的许多概率分布一起使用。但是,关于MCMC方法表现的理论保证必须在不同类型的分布下逐个证明。在深度学习的背景下,依赖于能够应用所有基于能量的模型的最一般的理论保证是最常见的。

为了理解为什么从基于能量的模型抽取样本是困难的,让我们考虑只有两个变量的EBM模型,定义一个分布\(p(a,b)\)。为了采样\(a\),我们必须从\(p(a|b)\)中抽取\(a\),为了采样\(b\),我们必须从\(p(b|a)\)中抽取\(b\)。它看起来是个不容易处理的鸡生蛋,蛋生鸡的问题。有向模型可以避免这个问题,因为它们的图是有向无环的。要进行祖先采样,只需要按照拓扑顺序对每个变量进行采样,对每个变量的父节点进行调节,并保证父节点已经被采样(见第16.3节)。祖先采样定义了一种获得样本的有效的单程方法。

在EBM模型中,我们可以通过使用马尔可夫链来抽样,从而避免这个鸡生蛋,蛋生鸡的问题。马尔可夫链的核心思想是有一个以任意值开始的状态,随着时间,我们不断地随机更新\(\bm{x}\)。最终\(\bm{x}\)(几乎接近)变成\(p(\bm{x})\)中的一个真实的样本。正式来说,马尔可夫链被定义为随机状态\(\bm{x}\),和转移分布\(T({ \bm{x} }^{ \prime  }|\bm{x})\),该分布给出了从状态\(\bm{x}\)开始,随机更新到状态\({ \bm{x} }^{ \prime  }\)的概率。运行马尔可夫链意味着不断地更新状态\(\bm{x}\)到从\(T({ \bm{x} }^{ \prime  }|\bm{x})\)抽样出来的值\({ \bm{x} }^{ \prime }\)。

得到一些关于MCMC方法如何工作的理论理解,对于再参数化该问题是很有用。首先,我们把问题限制在随机变量\(\bf{x}\)只有有限种状态的情况。在这种情况,我们可以把状态表示成一个正整数\(x\)。x的不同的整数值对应原始问题中不同的状态\(\bm{x}\)。

现在让我们考虑下,当我们无限次并行地运行许多马尔可夫链时会发生什么。不同的马尔可夫链的所有状态都是从相同的分布\({ q }^{ (t) }(x)\)中抽取的,这里\(t\)表示已经过去的时间步长的数量。在开始时,\({ q }^{ (0) }\)是一些用来对每个马尔可夫链任意地初始化\(x\)的分布。随后,\({ q }^{ (t) }\)被目前已经运行过的所有的马尔可夫链步骤所影响。我们的目标是让\({ q }^{ (t) }(x)\)收敛到\(p(x)\)。

因为我们已经就正整数\(x\)重新参数化了该问题,因为我们可以用一个向量\(\bm{v}\)来描述概率分布\(q\),如下式:
 $$ { q }(x=i)={ v }_{ i }. \eqno{(17.17)} $$
 
 考虑下当我们更新单独一个马尔可夫链的状态\(x\)到一个新的状态\({x}^{\prime}\)时,会发生什么。一个单独的状态落到状态\({x}^{\prime}\)的概率由下式给出:
$${ q }^{ (t+1) }({ x }^{ \prime  })=\sum _{ x }^{  }{ { q }^{ (t) }({ x }) } T({ x }^{ \prime  }|x).  \eqno{(17.18)}  $$

使用我们的整数参数化,我们可以用矩阵\(\bm{A}\)表示转移操作\(T\)的影响。我们定义\(\bm{A}\)为:
$${ A }_{ i,j }=T({ \bf{x} }^{ \prime  }=i|\bf{x} = j). \eqno{(17.19)}  $$

使用这个定义,我们现在可以重写方程17.18。代替书写该公式使用\(q\)和\(T\)来理解单个状态是如何被更新的,我们可能可以用\(\bm{v}\)和\(\bm{A}\)把基于所有不同马尔可夫链的整个分布是如何并行运行的过程描述为我们进行的一次更新:
$${\bm{v}}^{ (t) }=\bm{A}{\bm{v}}^{ (t-1) }. \eqno{(17.20)}$$

应用马尔可夫链不断地进行更新,对应于不断地乘以矩阵\(\bm{A}\)。也就是说,我们可以把该过程看作对矩阵\(\bm{A}\)取幂:
$${ \bm{v} }^{ (t) }=\bm{A}{ \bm{v} }^{ (0) }  \eqno{(17.21)}$$

矩阵\(\bm{A}\)有特殊的结构,它的每一列表示一个概率分布。这样的矩阵也叫做随机矩阵。如果在某次幂\(t\),从任意状态\(x\)转移到其他状态\({x}^{ \prime }\)的概率都非零,那么Perron-Frobenius定理(Perron, 1907; Frobenius, 1908)保证,最大的特征值是实数,并且等于1。随着时间,我们可以看到所有的特征值都可以写成指数形式:
$${\bm{v}}^{(t)}={(\bm{V}diag(\bm{\lambda}){\bm{V}}^{-1})}^{t}{\bm{v}}^{(0) }=\bm{V}diag(\bm{\lambda}){\bm{V}}^{ -1 }{\bm{v}}^{ (0) } \eqno{(17.22)}$$

该过程会使得所有不等于1的特征值衰减到0。在其他宽松的条件下,A被确保只有一个特征值为1的特征向量。因此该过程收敛到稳定分布,有时也称为平衡分布。收敛时,
$${ \bm{v} }^{ \prime  }=\bm{A}\bm{v}=\bm{v}, \eqno{(17.23)}$$
这个相同的条件对每个额外的步骤都成立。这是一个特征向量方程。为了成为一个稳定点,\(\bm{v}\)必须是对应于特征值为1的特征向量。这个条件确保一旦到达稳定分布,转移抽样过程的重复应用不改变基于各种各样的马尔可夫链的状态的分布(尽管转移操作本来就不改变每个个体状态)。

如果我们正确地选择了\(T\),那么稳定分布\(q\)会等于我们希望抽样的分布\(p\)。我们会在17.4章节简要地描述如何选择\(T\)。

具有可数状态的马尔可夫链的大多数性质可以推广到连续变量的马尔可夫链。在这种情况下,一些作者叫马尔可夫链为Harris链,但是我们使用马尔可夫链这个术语来描述这两种情况。通常在宽松的条件下,具有转移操作\(T\)的马尔可夫链会收敛到一个稳定点,该点可以用下述方程来描述的:
$$  {q}^{\prime}({\bf{x}^{\prime})={\mathbb{E}}_{\bf{x}\sim q}T({\bf{x}^{\prime}|\bf{x})  \eqno{(17.24)} $$
 离散的情况,我们刚才已经重写成方程17.23了。当\(\bf{x}\)是离散的,期望对应于加和,当\(\bf{x}\)是连续的,期望对应于积分。

不管状态是连续的还是离散的,所有的马尔可夫链方法都包含重复地应用随机更新的过程,直到最终状态开始从平稳分布产生样本。运行马尔可夫链直到它达到平稳分布的过程叫做“burning in”马尔可夫链。在达到了平稳分布,非常多的样本序列可以从平稳分布中抽取出来。它们被相同地分布,但是任意两个成功的样本之间是高度相关的。因此有限的样本序列可能不能表示平稳分布。减轻这个问题的一种方法是仅返回每\(n\)个连续样本,这会使得我们对平衡分布的统计的估计不会由于MCMC采样和接下来的几个样本之间的相关性而有偏差。因此使用马尔可夫链的代价很高,因为需要时间来构建平稳分布,以及在达到平稳后需要时间从一个样本转换到另一个合理地去除了相关性的样本。如果想得到真正独立的样本,可以并行地运行多个马尔可夫链。这种方法使用了额外的并行代价来消除延迟。只使用一个单独的马尔可夫链来产生所有的样本的策略,和对每一个理想的样本都使用一个马尔可夫链的策略是两种极端;深度学习从业人员通常使用与小批量中的样本数量相似的多个链,然后从这个固定的马尔科夫链中绘制所需的多个样本。\(A\)常用的马尔可夫链的数目是100。

另一个困难是我们不能提前知道,在达到平稳分布前,马尔可夫链要运行多少步。该时间长度称为混合时间。测试马尔可夫链是否达到平衡也是非常困难的。暂时没有足够准确的理论来指导我们回答这一问题。理论只是告诉我们,马尔可夫链会收敛,没有更多的信息了。如果我们从作用于概率向量\(\bm{v}\)的矩阵\(\bm{A}\)的角度来分析马尔可夫链,那么我们知道当\({ \bm{A} }^{ t }\)已经有效地丢失了来自\(\bm{A}\)的除了唯一的特征值1之外的所有特征值时,链条混合。这意味着第二大的特征值的量级将决定混合时间。但是实际上,我们不能用矩阵表示出马尔可夫链。我们的概率模型可以访问的状态数量对于变量的数量是指数级大的,所以表示\(\bm{x}\),\(\bm{A}\)或者\(\bm{A}\)的特征值是不合理的。由于这些阻碍和其他的阻碍,我们通常不知道马尔可夫链是否已经混合。相反,我们简单地运行马尔可夫链一段时间,然后大致地估计是否足够,并且使用启发式方法来确定链是否混合。这些启发式方法包括人工检查样本,或者测量成功抽取的样本之间的相关性。
