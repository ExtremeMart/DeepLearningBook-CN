\chapter{深度学习的结构化概率模型}
\label{chap:16}
%%%%%%%%%%%%%%%%%%%%%%%%%%%%%%%%%%%%%%%%%%%%%%%%%%%%%%%%%
%%%%%%%%%%%%%%% author:YisenWang %%%%%%%%%%%%%%%%%%%%%%%%%%%
%%%%%%%%%%%%%%%%%%%%%%%%%%%%%%%%%%%%%%%%%%%%%%%%%%%%%%%%%
深度学习借鉴了很多模型形式方面的内容来帮助研究者指导他们的设计理论和算法描述。这其中之一就是有结构的概率模型。我们在前面3.14节的时候,简单的讨论过一些有结构的概率模型。那些简短的介绍已经足够用来理解怎么使用有结构的概率模型来描述第二部分的一些算法。在第三部分,有结构的概率模型是深度学习里很多重要的研究方向的关键成分。为了后续讨论这些研究点,在这一章,我们会详细介绍有结构的概率模型。不过,读者不用担心,这一章是自我完备的,在开始学习这一章之前,读者不需要复习之前的介绍。

有结构的概率模型是描述概率分布的一种方式,它直接通过图的形式来描述概率分布中随机变量之间的关系。这里,我们沿用了图理论里面图的概念,即,节点之间通过边来连接。因为模型的结构是由图来定义的,因此,这些模型也经常被称作图模型。

图模型的研究领域很大,也已经发展了很多不同的模型、训练算法和推断算法。在这一章,我们主要讲一些图模型里最核心的思想,并把重点放在那些对深度学习的领域很有用的一些概念上。如果你已经有很强的图模型背景,你可以跳过本章的大部分内容。但是,即使是一个图模型方面的专家,他也可能会从这一章的最后一部分(16.7节)收益,因为,在16.7节,我们会高亮一些特有的图模型可以用在深度学习里的方法。相比于图模型的研究者,深度学习的参与者更倾向于用非常不同的模型结构、学习算法和推断过程。在这一章,我们指出了他们在偏好上的不同,并解释其中的原因。

在这一章,我们首先描述了建立大规模概率模型的挑战。接着,我们描述了怎么用图来描述一个概率分布的结构。虽然这种方法允许我们克服许多挑战,但它不是没有自己的复杂性。在图模型里,一个最主要的困难是理解在一个图里,哪些变量之间需要直接相连,也就是,对于一个给定的问题,哪种图结构是最合适的。我们在16.5节简介了两种方式来解决这个问题。最后,我们通过讨论深度学习与图模型之间的关系来结束本章。

\section{无结构模型的挑战}
深度学习的目标是延伸机器学习到解决人工智能所面临的各种挑战,这意味着深度学习能够理解具体很丰富结构的高维数据。比如说,我们希望人工智能算法能够理解自然图像,声波表示的语音,以及包含多个单词和标点符号的文档。

分类算法能够从很高维的分布中取出一个输入并用一个类别标签总结他,比如照片里面是个什么物体,语音说的是哪个词,文档是关于哪个话题的。分类的过程忽略了输入里的大部分信息,只产生了一个输出(或那个单一输出值的概率分布)。一个分类器也经常忽略输入的很多部分。比如,当识别图片中一个物体,经常会忽略图片中的背景。

让概率模型做很多其他的任务也是可能的。这些任务通常比分类更昂贵,比如,其中一些可能需要输出多个值,而且大部分要求一个对输入有一个完整的理解,不能忽略其中的任何一部分。这些任务包括:
\begin{itemize}
\item 密度估计
\item 去噪
\item 缺失值的插补
\item 采样
\end{itemize}
比如一个用很多小的自然图像采样的一个例子,如图16.1所示。






